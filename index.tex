% Options for packages loaded elsewhere
\PassOptionsToPackage{unicode}{hyperref}
\PassOptionsToPackage{hyphens}{url}
\PassOptionsToPackage{dvipsnames,svgnames,x11names}{xcolor}
%
\documentclass[
  letterpaper,
  DIV=11,
  numbers=noendperiod]{scrreprt}

\usepackage{amsmath,amssymb}
\usepackage{iftex}
\ifPDFTeX
  \usepackage[T1]{fontenc}
  \usepackage[utf8]{inputenc}
  \usepackage{textcomp} % provide euro and other symbols
\else % if luatex or xetex
  \usepackage{unicode-math}
  \defaultfontfeatures{Scale=MatchLowercase}
  \defaultfontfeatures[\rmfamily]{Ligatures=TeX,Scale=1}
\fi
\usepackage{lmodern}
\ifPDFTeX\else  
    % xetex/luatex font selection
\fi
% Use upquote if available, for straight quotes in verbatim environments
\IfFileExists{upquote.sty}{\usepackage{upquote}}{}
\IfFileExists{microtype.sty}{% use microtype if available
  \usepackage[]{microtype}
  \UseMicrotypeSet[protrusion]{basicmath} % disable protrusion for tt fonts
}{}
\makeatletter
\@ifundefined{KOMAClassName}{% if non-KOMA class
  \IfFileExists{parskip.sty}{%
    \usepackage{parskip}
  }{% else
    \setlength{\parindent}{0pt}
    \setlength{\parskip}{6pt plus 2pt minus 1pt}}
}{% if KOMA class
  \KOMAoptions{parskip=half}}
\makeatother
\usepackage{xcolor}
\setlength{\emergencystretch}{3em} % prevent overfull lines
\setcounter{secnumdepth}{5}
% Make \paragraph and \subparagraph free-standing
\makeatletter
\ifx\paragraph\undefined\else
  \let\oldparagraph\paragraph
  \renewcommand{\paragraph}{
    \@ifstar
      \xxxParagraphStar
      \xxxParagraphNoStar
  }
  \newcommand{\xxxParagraphStar}[1]{\oldparagraph*{#1}\mbox{}}
  \newcommand{\xxxParagraphNoStar}[1]{\oldparagraph{#1}\mbox{}}
\fi
\ifx\subparagraph\undefined\else
  \let\oldsubparagraph\subparagraph
  \renewcommand{\subparagraph}{
    \@ifstar
      \xxxSubParagraphStar
      \xxxSubParagraphNoStar
  }
  \newcommand{\xxxSubParagraphStar}[1]{\oldsubparagraph*{#1}\mbox{}}
  \newcommand{\xxxSubParagraphNoStar}[1]{\oldsubparagraph{#1}\mbox{}}
\fi
\makeatother

\usepackage{color}
\usepackage{fancyvrb}
\newcommand{\VerbBar}{|}
\newcommand{\VERB}{\Verb[commandchars=\\\{\}]}
\DefineVerbatimEnvironment{Highlighting}{Verbatim}{commandchars=\\\{\}}
% Add ',fontsize=\small' for more characters per line
\usepackage{framed}
\definecolor{shadecolor}{RGB}{241,243,245}
\newenvironment{Shaded}{\begin{snugshade}}{\end{snugshade}}
\newcommand{\AlertTok}[1]{\textcolor[rgb]{0.68,0.00,0.00}{#1}}
\newcommand{\AnnotationTok}[1]{\textcolor[rgb]{0.37,0.37,0.37}{#1}}
\newcommand{\AttributeTok}[1]{\textcolor[rgb]{0.40,0.45,0.13}{#1}}
\newcommand{\BaseNTok}[1]{\textcolor[rgb]{0.68,0.00,0.00}{#1}}
\newcommand{\BuiltInTok}[1]{\textcolor[rgb]{0.00,0.23,0.31}{#1}}
\newcommand{\CharTok}[1]{\textcolor[rgb]{0.13,0.47,0.30}{#1}}
\newcommand{\CommentTok}[1]{\textcolor[rgb]{0.37,0.37,0.37}{#1}}
\newcommand{\CommentVarTok}[1]{\textcolor[rgb]{0.37,0.37,0.37}{\textit{#1}}}
\newcommand{\ConstantTok}[1]{\textcolor[rgb]{0.56,0.35,0.01}{#1}}
\newcommand{\ControlFlowTok}[1]{\textcolor[rgb]{0.00,0.23,0.31}{\textbf{#1}}}
\newcommand{\DataTypeTok}[1]{\textcolor[rgb]{0.68,0.00,0.00}{#1}}
\newcommand{\DecValTok}[1]{\textcolor[rgb]{0.68,0.00,0.00}{#1}}
\newcommand{\DocumentationTok}[1]{\textcolor[rgb]{0.37,0.37,0.37}{\textit{#1}}}
\newcommand{\ErrorTok}[1]{\textcolor[rgb]{0.68,0.00,0.00}{#1}}
\newcommand{\ExtensionTok}[1]{\textcolor[rgb]{0.00,0.23,0.31}{#1}}
\newcommand{\FloatTok}[1]{\textcolor[rgb]{0.68,0.00,0.00}{#1}}
\newcommand{\FunctionTok}[1]{\textcolor[rgb]{0.28,0.35,0.67}{#1}}
\newcommand{\ImportTok}[1]{\textcolor[rgb]{0.00,0.46,0.62}{#1}}
\newcommand{\InformationTok}[1]{\textcolor[rgb]{0.37,0.37,0.37}{#1}}
\newcommand{\KeywordTok}[1]{\textcolor[rgb]{0.00,0.23,0.31}{\textbf{#1}}}
\newcommand{\NormalTok}[1]{\textcolor[rgb]{0.00,0.23,0.31}{#1}}
\newcommand{\OperatorTok}[1]{\textcolor[rgb]{0.37,0.37,0.37}{#1}}
\newcommand{\OtherTok}[1]{\textcolor[rgb]{0.00,0.23,0.31}{#1}}
\newcommand{\PreprocessorTok}[1]{\textcolor[rgb]{0.68,0.00,0.00}{#1}}
\newcommand{\RegionMarkerTok}[1]{\textcolor[rgb]{0.00,0.23,0.31}{#1}}
\newcommand{\SpecialCharTok}[1]{\textcolor[rgb]{0.37,0.37,0.37}{#1}}
\newcommand{\SpecialStringTok}[1]{\textcolor[rgb]{0.13,0.47,0.30}{#1}}
\newcommand{\StringTok}[1]{\textcolor[rgb]{0.13,0.47,0.30}{#1}}
\newcommand{\VariableTok}[1]{\textcolor[rgb]{0.07,0.07,0.07}{#1}}
\newcommand{\VerbatimStringTok}[1]{\textcolor[rgb]{0.13,0.47,0.30}{#1}}
\newcommand{\WarningTok}[1]{\textcolor[rgb]{0.37,0.37,0.37}{\textit{#1}}}

\providecommand{\tightlist}{%
  \setlength{\itemsep}{0pt}\setlength{\parskip}{0pt}}\usepackage{longtable,booktabs,array}
\usepackage{calc} % for calculating minipage widths
% Correct order of tables after \paragraph or \subparagraph
\usepackage{etoolbox}
\makeatletter
\patchcmd\longtable{\par}{\if@noskipsec\mbox{}\fi\par}{}{}
\makeatother
% Allow footnotes in longtable head/foot
\IfFileExists{footnotehyper.sty}{\usepackage{footnotehyper}}{\usepackage{footnote}}
\makesavenoteenv{longtable}
\usepackage{graphicx}
\makeatletter
\def\maxwidth{\ifdim\Gin@nat@width>\linewidth\linewidth\else\Gin@nat@width\fi}
\def\maxheight{\ifdim\Gin@nat@height>\textheight\textheight\else\Gin@nat@height\fi}
\makeatother
% Scale images if necessary, so that they will not overflow the page
% margins by default, and it is still possible to overwrite the defaults
% using explicit options in \includegraphics[width, height, ...]{}
\setkeys{Gin}{width=\maxwidth,height=\maxheight,keepaspectratio}
% Set default figure placement to htbp
\makeatletter
\def\fps@figure{htbp}
\makeatother


% load packages
\usepackage{geometry}
\usepackage{xcolor}
\usepackage{eso-pic}
\usepackage{fancyhdr}
\usepackage{sectsty}
\usepackage{fontspec}
\usepackage{titlesec}


% %% Set page size with a wider right margin
\geometry{a4paper, total={170mm,257mm}}

%% Let's define some colours
\definecolor{light}{HTML}{edf7fa}
\definecolor{highlight}{HTML}{2c6d7d}
\definecolor{dark}{HTML}{336882}

% 
% % Custom command for logo
% \newcommand{\logoinclude}[2][]{%
% \includegraphics[#1]{#2}%
% }
% 
% % Let's add the border on the right-hand side
% \AddToShipoutPicture{%
% \AtPageLowerLeft{%
% \put(\LenToUnit{\dimexpr\paperwidth-1.75cm},0){%
% \color{light}\rule{3cm}{\LenToUnit{\paperheight}}%
% }%
% }%
% % logo
% \AtPageLowerLeft{% start the bar at the bottom right of the page
% \put(\LenToUnit{\dimexpr\paperwidth-1cm},27.2cm){% move it to the top right
% \logoinclude[width=1.2cm]{_extensions/nrennie/PrettyPDF/logo.png}%
% }%
% }%
% }



%% Style the page number
\fancypagestyle{mystyle}{
  % \fancyhf{}
  % \renewcommand\headrulewidth{0pt}
  % Empty style
}

\pagestyle{mystyle}


% Center align chapter titles
\usepackage{titlesec}
\titleformat{\chapter}[display]
  {\normalfont\huge\bfseries\centering\color{dark}}{\chaptertitlename\ \thechapter}{20pt}{\Huge}


% add a border to images
% \let\originalincludegraphics\includegraphics
% \renewcommand{\includegraphics}[2][]{%
%   \fcolorbox{light}{white}{\originalincludegraphics[#1]{#2}}%
% }

\let\originalincludegraphics\includegraphics
\renewcommand{\includegraphics}[2][]{%
  \setlength{\fboxrule}{4pt} % Set border thickness to 2pt
  \fcolorbox{light}{white}{\originalincludegraphics[#1]{#2}}%
}



%% Use some custom fonts
\setsansfont{Ubuntu}[
    Path=_extensions/nrennie/PrettyPDF/Ubuntu/,
    Scale=0.9,
    Extension = .ttf,
    UprightFont=*-Regular,
    BoldFont=*-Bold,
    ItalicFont=*-Italic,
    ]

\setmainfont{Ubuntu}[
    Path=_extensions/nrennie/PrettyPDF/Ubuntu/,
    Scale=0.9,
    Extension = .ttf,
    UprightFont=*-Regular,
    BoldFont=*-Bold,
    ItalicFont=*-Italic,
    ]
\KOMAoption{captions}{tableheading}
\makeatletter
\@ifpackageloaded{tcolorbox}{}{\usepackage[skins,breakable]{tcolorbox}}
\@ifpackageloaded{fontawesome5}{}{\usepackage{fontawesome5}}
\definecolor{quarto-callout-color}{HTML}{909090}
\definecolor{quarto-callout-note-color}{HTML}{0758E5}
\definecolor{quarto-callout-important-color}{HTML}{CC1914}
\definecolor{quarto-callout-warning-color}{HTML}{EB9113}
\definecolor{quarto-callout-tip-color}{HTML}{00A047}
\definecolor{quarto-callout-caution-color}{HTML}{FC5300}
\definecolor{quarto-callout-color-frame}{HTML}{acacac}
\definecolor{quarto-callout-note-color-frame}{HTML}{4582ec}
\definecolor{quarto-callout-important-color-frame}{HTML}{d9534f}
\definecolor{quarto-callout-warning-color-frame}{HTML}{f0ad4e}
\definecolor{quarto-callout-tip-color-frame}{HTML}{02b875}
\definecolor{quarto-callout-caution-color-frame}{HTML}{fd7e14}
\makeatother
\makeatletter
\@ifpackageloaded{bookmark}{}{\usepackage{bookmark}}
\makeatother
\makeatletter
\@ifpackageloaded{caption}{}{\usepackage{caption}}
\AtBeginDocument{%
\ifdefined\contentsname
  \renewcommand*\contentsname{Table of contents}
\else
  \newcommand\contentsname{Table of contents}
\fi
\ifdefined\listfigurename
  \renewcommand*\listfigurename{List of Figures}
\else
  \newcommand\listfigurename{List of Figures}
\fi
\ifdefined\listtablename
  \renewcommand*\listtablename{List of Tables}
\else
  \newcommand\listtablename{List of Tables}
\fi
\ifdefined\figurename
  \renewcommand*\figurename{Figure}
\else
  \newcommand\figurename{Figure}
\fi
\ifdefined\tablename
  \renewcommand*\tablename{Table}
\else
  \newcommand\tablename{Table}
\fi
}
\@ifpackageloaded{float}{}{\usepackage{float}}
\floatstyle{ruled}
\@ifundefined{c@chapter}{\newfloat{codelisting}{h}{lop}}{\newfloat{codelisting}{h}{lop}[chapter]}
\floatname{codelisting}{Listing}
\newcommand*\listoflistings{\listof{codelisting}{List of Listings}}
\makeatother
\makeatletter
\makeatother
\makeatletter
\@ifpackageloaded{caption}{}{\usepackage{caption}}
\@ifpackageloaded{subcaption}{}{\usepackage{subcaption}}
\makeatother
\makeatletter
\@ifpackageloaded{tcolorbox}{}{\usepackage[skins,breakable]{tcolorbox}}
\makeatother
\makeatletter
\@ifundefined{shadecolor}{\definecolor{shadecolor}{rgb}{.97, .97, .97}}{}
\makeatother
\makeatletter
\@ifundefined{codebgcolor}{\definecolor{codebgcolor}{named}{light}}{}
\makeatother
\makeatletter
\ifdefined\Shaded\renewenvironment{Shaded}{\begin{tcolorbox}[boxrule=0pt, colback={codebgcolor}, frame hidden, sharp corners, enhanced, breakable]}{\end{tcolorbox}}\fi
\makeatother

\ifLuaTeX
  \usepackage{selnolig}  % disable illegal ligatures
\fi
\usepackage{bookmark}

\IfFileExists{xurl.sty}{\usepackage{xurl}}{} % add URL line breaks if available
\urlstyle{same} % disable monospaced font for URLs
\hypersetup{
  pdftitle={Introduction to Data Science with Python},
  pdfauthor={The GRAPH Courses},
  colorlinks=true,
  linkcolor={highlight},
  filecolor={Maroon},
  citecolor={Blue},
  urlcolor={highlight},
  pdfcreator={LaTeX via pandoc}}


\title{Introduction to Data Science with Python}
\usepackage{etoolbox}
\makeatletter
\providecommand{\subtitle}[1]{% add subtitle to \maketitle
  \apptocmd{\@title}{\par {\large #1 \par}}{}{}
}
\makeatother
\subtitle{With Real-World Examples from Research, Business and Industry}
\author{The GRAPH Courses}
\date{}

\begin{document}
\maketitle

\pagestyle{mystyle}

\renewcommand*\contentsname{Table of contents}
{
\hypersetup{linkcolor=}
\setcounter{tocdepth}{2}
\tableofcontents
}

\bookmarksetup{startatroot}

\chapter*{Introduction}\label{introduction}
\addcontentsline{toc}{chapter}{Introduction}

\markboth{Introduction}{Introduction}

This book is a compilation of lesson notes for a 3-month online course
offered by The GRAPH Courses. To access the lesson videos, exercise
files, and online quizzes, please visit our website,
\href{https://thegraphcourses.org}{thegraphcourses.org}.

The GRAPH Courses is a project of the Global Research and Analyses for
Public Health (GRAPH) Network, a non-profit organization dedicated to
making code and data skills accessible through affordable live bootcamps
and free self-paced courses.

\section*{Contributors}\label{contributors}
\addcontentsline{toc}{section}{Contributors}

\markright{Contributors}

We are extremely grateful to the following individuals who have
contributed to the development of these materials over several years:

Amanda McKinley, Andree Valle Campos, Aziza Merzouki, Benedict Nguimbis,
Bennour Hsin, Camille Beatrice Valera, Daniel Camara, Eduardo Araujo,
Elton Mukonda, Guy Wafeu, Imad El Badisy, Imane Bensouda Korachi, Joy
Vaz, Kene David Nwosu, Lameck Agasa, Laure Nguemo, Laure
Vancauwenberghe, Matteo Franza, Michal Shrestha, Olivia Keiser, Sabina
Rodriguez Velasquez, Sara Botero Mesa.

\section*{Partners \& Funders}\label{partners-funders}
\addcontentsline{toc}{section}{Partners \& Funders}

\markright{Partners \& Funders}

\begin{itemize}
\tightlist
\item
  University of Geneva
\item
  University of Oxford
\item
  World Health Organization
\item
  Global Fund
\item
  Ernst Goehner Foundation
\end{itemize}

\part{Foundations}

\chapter{Introduction to Google
Colab}\label{introduction-to-google-colab}

\section{Learning objectives}\label{learning-objectives}

\begin{enumerate}
\def\labelenumi{\arabic{enumi}.}
\tightlist
\item
  Understand what Google Colab is and its advantages for data science
  and AI
\item
  Learn how to access and navigate Google Colab
\item
  Create and manage notebooks in Google Colab
\item
  Run Python code in Colab cells
\item
  Use text cells for explanations and formatting
\item
  Import and use pre-installed libraries for data analysis
\item
  Import and use data to perform analysis
\item
  Share Colab notebooks
\end{enumerate}

\section{Introduction}\label{introduction-1}

Google Collaboratory, or Colab for short, is a free online platform that
allows you to work with Python or R code in your browser. It's a great
way to get started with Python, as you don't have to install anything on
your computer.

Some limitations if you're running heavy workload though. Can get
timeout. But for beginner data analysts, it's perfect and free.

\section{Getting Started with Colab}\label{getting-started-with-colab}

\begin{enumerate}
\def\labelenumi{\arabic{enumi}.}
\tightlist
\item
  Search for ``Google Colab'' in your browser
\item
  Usually the first option. Currently, it's colab.research.google.com,
  but it may change in the future.
\item
  Sign in with your Google account (create a Gmail account if you don't
  have one)
\end{enumerate}

\section{Creating and Managing
Notebooks}\label{creating-and-managing-notebooks}

\begin{itemize}
\tightlist
\item
  Notebooks are the main way to organize work in Colab. They contain
  code cells and text cells.
\item
  Create a new notebook: File \textgreater{} New Notebook
\item
  Rename your notebook for better organization
\end{itemize}

\section{Working with Code Cells}\label{working-with-code-cells}

\begin{itemize}
\item
  Code cells are where you write and execute Python code
\item
  Type 1 + 1 in a cell then run it
\item
  Run a cell by clicking the play button or using keyboard shortcuts:

  \begin{itemize}
  \tightlist
  \item
    Command + Enter: Run the current cell
  \item
    Shift + Enter: Run the current cell and create a new one below
  \end{itemize}
\item
  Try to get comfortable with keyboard shortcuts
\item
  May take a while to run the first time. See it's using Python
\item
  Can change runtime to R actually
\item
  When you run a cell, the output is displayed below the cell
\item
  To see multiple outputs, explicitly print them
\end{itemize}

\section{Text Cells}\label{text-cells}

\begin{itemize}
\tightlist
\item
  Use text cells for explanations and titles
\item
  The toolbar makes formatting easy, but pay attention to the generated
  markdown
\end{itemize}

\section{Example of working with
data}\label{example-of-working-with-data}

\begin{enumerate}
\def\labelenumi{\arabic{enumi}.}
\tightlist
\item
  Click on the files tab to see the sample\_data folder
\item
  Import the California housing test dataset:
\end{enumerate}

\begin{Shaded}
\begin{Highlighting}[]
\ImportTok{import}\NormalTok{ pandas}
\NormalTok{housing\_data }\OperatorTok{=}\NormalTok{ pandas.read\_csv(}\StringTok{"/content/sample\_data/california\_housing\_test.csv"}\NormalTok{)}
\end{Highlighting}
\end{Shaded}

\begin{enumerate}
\def\labelenumi{\arabic{enumi}.}
\setcounter{enumi}{2}
\tightlist
\item
  View the dataset by typing \texttt{housing\_data} in a cell and
  running it
\end{enumerate}

\section{Practice}\label{practice}

Import the train dataset and repeat the process

\section{Getting data from your
Drive}\label{getting-data-from-your-drive}

\begin{enumerate}
\def\labelenumi{\arabic{enumi}.}
\tightlist
\item
  Search for ``{[}your city{]} housing filetype:csv'' on Google
\item
  In the files tab, click the button to mount your drive
\item
  Create a folder and upload the downloaded file
\item
  Import it with pandas as before
\end{enumerate}

\section{Where is your notebook
saved?}\label{where-is-your-notebook-saved}

\begin{itemize}
\tightlist
\item
  All work is automatically saved to your Google Drive
\item
  Access your notebooks at drive.google.com in the ``Colab Notebooks''
  folder
\end{itemize}

\section{Sharing and Collaborating}\label{sharing-and-collaborating}

\begin{itemize}
\tightlist
\item
  Share notebooks with a link, giving viewer or editor access
\item
  Access notebooks later from your Google Drive
\item
  Download notebooks in various formats (ipynb, py)
\end{itemize}

\section{Conclusion}\label{conclusion}

Google Colab provides a powerful, accessible platform for data science
and AI projects. Its pre-configured environment, free access to hardware
accelerators, and easy sharing features make it an excellent choice for
beginners and experienced practitioners alike.

\chapter{Coding basics}\label{coding-basics}

\section{Learning objectives}\label{learning-objectives-1}

\begin{enumerate}
\def\labelenumi{\arabic{enumi}.}
\tightlist
\item
  You can write and use comments in Python (single-line and multi-line).
\item
  You know how to use Python as a calculator for basic arithmetic
  operations and understand the order of operations.
\item
  You can use the math library for more complex mathematical operations.
\item
  You understand how to use proper spacing in Python code to improve
  readability.
\item
  You can create, manipulate, and reassign variables of different types
  (string, int, float).
\item
  You can get user input and perform calculations with it.
\item
  You understand the basic rules and best practices for naming variables
  in Python.
\item
  You can identify and fix common errors related to variable usage and
  naming.
\end{enumerate}

\section{Introduction}\label{introduction-2}

In this lesson, you will learn the basics of using Python.

To get started, open your preferred Python environment (e.g., Jupyter
Notebook, VS Code, or PyCharm), and create a new Python file or
notebook.

Next, \textbf{save the file} with a name like ``coding\_basics.py'' or
``coding\_basics.ipynb'' depending on your environment.

You should now type all the code from this lesson into that file.

\section{Comments}\label{comments}

Comments are text that is ignored by Python. They are used to explain
what the code is doing.

You use the symbol \texttt{\#}, pronounced ``hash'' or ``pound'', to
start a comment. Anything after the \texttt{\#} on the same line is
ignored. For example:

\begin{Shaded}
\begin{Highlighting}[]
\CommentTok{\# Addition}
\DecValTok{2} \OperatorTok{+} \DecValTok{2}
\end{Highlighting}
\end{Shaded}

\begin{verbatim}
4
\end{verbatim}

If we just tried to write Addition above the code, it would cause an
error:

\begin{Shaded}
\begin{Highlighting}[]
\NormalTok{Addition}
\DecValTok{2} \OperatorTok{+} \DecValTok{2}
\end{Highlighting}
\end{Shaded}

\begin{verbatim}
NameError: name 'Addition' is not defined
\end{verbatim}

We can put the comment on the same line as the code, but it needs to
come after the code.

\begin{Shaded}
\begin{Highlighting}[]
\DecValTok{2} \OperatorTok{+} \DecValTok{2}  \CommentTok{\# Addition}
\end{Highlighting}
\end{Shaded}

\begin{verbatim}
4
\end{verbatim}

To write multiple lines of comments, you can either add more \texttt{\#}
symbols:

\begin{Shaded}
\begin{Highlighting}[]
\CommentTok{\# Addition}
\CommentTok{\# Add two numbers}
\DecValTok{2} \OperatorTok{+} \DecValTok{2}
\end{Highlighting}
\end{Shaded}

\begin{verbatim}
4
\end{verbatim}

Or you can use triple quotes
\texttt{\textquotesingle{}\textquotesingle{}\textquotesingle{}} or
\texttt{"""}:

\begin{Shaded}
\begin{Highlighting}[]
\CommentTok{\textquotesingle{}\textquotesingle{}\textquotesingle{}}
\CommentTok{Addition:}
\CommentTok{Below we add two numbers}
\CommentTok{\textquotesingle{}\textquotesingle{}\textquotesingle{}}
\DecValTok{2} \OperatorTok{+} \DecValTok{2}
\end{Highlighting}
\end{Shaded}

\begin{verbatim}
4
\end{verbatim}

Or:

\begin{Shaded}
\begin{Highlighting}[]
\CommentTok{"""}
\CommentTok{Addition:}
\CommentTok{Below we add two numbers}
\CommentTok{"""}
\DecValTok{2} \OperatorTok{+} \DecValTok{2}
\end{Highlighting}
\end{Shaded}

\begin{verbatim}
4
\end{verbatim}

\begin{tcolorbox}[enhanced jigsaw, colframe=quarto-callout-note-color-frame, opacityback=0, titlerule=0mm, bottomrule=.15mm, breakable, leftrule=.75mm, colbacktitle=quarto-callout-note-color!10!white, title=\textcolor{quarto-callout-note-color}{\faInfo}\hspace{0.5em}{Vocab}, rightrule=.15mm, coltitle=black, opacitybacktitle=0.6, colback=white, left=2mm, arc=.35mm, toptitle=1mm, bottomtitle=1mm, toprule=.15mm]

\textbf{Comment}: A piece of text in your code that is ignored by
Python. Comments are used to explain what the code is doing and are
meant for human readers.

\end{tcolorbox}

\begin{tcolorbox}[enhanced jigsaw, colframe=quarto-callout-tip-color-frame, opacityback=0, titlerule=0mm, bottomrule=.15mm, breakable, leftrule=.75mm, colbacktitle=quarto-callout-tip-color!10!white, title=\textcolor{quarto-callout-tip-color}{\faLightbulb}\hspace{0.5em}{Practice}, rightrule=.15mm, coltitle=black, opacitybacktitle=0.6, colback=white, left=2mm, arc=.35mm, toptitle=1mm, bottomtitle=1mm, toprule=.15mm]

\subsection{Q: Commenting in Python}\label{q-commenting-in-python}

Which of the following code chunks are valid ways to comment code in
Python?

\begin{verbatim}
# add two numbers
2 + 2
\end{verbatim}

\begin{verbatim}
2 + 2 # add two numbers
\end{verbatim}

\begin{verbatim}
''' add two numbers
2 + 2
\end{verbatim}

\begin{verbatim}
# add two numbers 2 + 2
\end{verbatim}

Check your answer by trying to run each code chunk.

\end{tcolorbox}

\section{Python as a calculator}\label{python-as-a-calculator}

As you have already seen, Python works as a calculator in standard ways.

Below are some other examples of basic arithmetic operations:

\begin{Shaded}
\begin{Highlighting}[]
\DecValTok{2} \OperatorTok{{-}} \DecValTok{2} \CommentTok{\# two minus two}
\end{Highlighting}
\end{Shaded}

\begin{verbatim}
0
\end{verbatim}

\begin{Shaded}
\begin{Highlighting}[]
\DecValTok{2} \OperatorTok{*} \DecValTok{2}  \CommentTok{\# two times two }
\end{Highlighting}
\end{Shaded}

\begin{verbatim}
4
\end{verbatim}

\begin{Shaded}
\begin{Highlighting}[]
\DecValTok{2} \OperatorTok{/} \DecValTok{2}  \CommentTok{\# two divided by two}
\end{Highlighting}
\end{Shaded}

\begin{verbatim}
1.0
\end{verbatim}

\begin{Shaded}
\begin{Highlighting}[]
\DecValTok{2} \OperatorTok{**} \DecValTok{2}  \CommentTok{\# two raised to the power of two}
\end{Highlighting}
\end{Shaded}

\begin{verbatim}
4
\end{verbatim}

There are a few other operators you may come across. For example,
\texttt{\%} is the modulo operator, which returns the remainder of the
division.

\begin{Shaded}
\begin{Highlighting}[]
\DecValTok{10} \OperatorTok{\%} \DecValTok{3}  \CommentTok{\# ten modulo three}
\end{Highlighting}
\end{Shaded}

\begin{verbatim}
1
\end{verbatim}

\texttt{//} is the floor division operator, which divides then rounds
down to the nearest whole number.

\begin{Shaded}
\begin{Highlighting}[]
\DecValTok{10} \OperatorTok{//} \DecValTok{3}  \CommentTok{\# ten floor division three}
\end{Highlighting}
\end{Shaded}

\begin{verbatim}
3
\end{verbatim}

\begin{tcolorbox}[enhanced jigsaw, colframe=quarto-callout-tip-color-frame, opacityback=0, titlerule=0mm, bottomrule=.15mm, breakable, leftrule=.75mm, colbacktitle=quarto-callout-tip-color!10!white, title=\textcolor{quarto-callout-tip-color}{\faLightbulb}\hspace{0.5em}{Practice}, rightrule=.15mm, coltitle=black, opacitybacktitle=0.6, colback=white, left=2mm, arc=.35mm, toptitle=1mm, bottomtitle=1mm, toprule=.15mm]

\subsection{Q: Modulo and floor
division}\label{q-modulo-and-floor-division}

Guess the result of the following code chunks then run them to check
your answer:

\begin{Shaded}
\begin{Highlighting}[]
\DecValTok{5} \OperatorTok{\%} \DecValTok{4}
\end{Highlighting}
\end{Shaded}

\begin{Shaded}
\begin{Highlighting}[]
\DecValTok{5} \OperatorTok{//} \DecValTok{4}
\end{Highlighting}
\end{Shaded}

\end{tcolorbox}

\subsection{Order of operations}\label{order-of-operations}

Python obeys the standard PEMDAS order of operations (Parentheses,
Exponents, Multiplication, Division, Addition, Subtraction).

For example, multiplication is evaluated before addition, so below the
result is \texttt{6}.

\begin{Shaded}
\begin{Highlighting}[]
\DecValTok{2} \OperatorTok{+} \DecValTok{2} \OperatorTok{*} \DecValTok{2}   
\end{Highlighting}
\end{Shaded}

\begin{verbatim}
6
\end{verbatim}

\begin{tcolorbox}[enhanced jigsaw, colframe=quarto-callout-tip-color-frame, opacityback=0, titlerule=0mm, bottomrule=.15mm, breakable, leftrule=.75mm, colbacktitle=quarto-callout-tip-color!10!white, title=\textcolor{quarto-callout-tip-color}{\faLightbulb}\hspace{0.5em}{Practice}, rightrule=.15mm, coltitle=black, opacitybacktitle=0.6, colback=white, left=2mm, arc=.35mm, toptitle=1mm, bottomtitle=1mm, toprule=.15mm]

\subsection{Q: Evaluating arithmetic
expressions}\label{q-evaluating-arithmetic-expressions}

Which, if any, of the following code chunks will evaluate to
\texttt{10}?

\begin{Shaded}
\begin{Highlighting}[]
\DecValTok{2} \OperatorTok{+} \DecValTok{2} \OperatorTok{*} \DecValTok{4}
\end{Highlighting}
\end{Shaded}

\begin{Shaded}
\begin{Highlighting}[]
\DecValTok{6} \OperatorTok{+} \DecValTok{2} \OperatorTok{**} \DecValTok{2}
\end{Highlighting}
\end{Shaded}

\end{tcolorbox}

\subsection{Using the math library}\label{using-the-math-library}

We can also use the \texttt{math} library to do more complex
mathematical operations. For example, we can use the \texttt{math.sqrt}
function to calculate the square root of a number.

\begin{Shaded}
\begin{Highlighting}[]
\ImportTok{import}\NormalTok{ math}
\NormalTok{math.sqrt(}\DecValTok{100}\NormalTok{)  }\CommentTok{\# square root}
\end{Highlighting}
\end{Shaded}

\begin{verbatim}
10.0
\end{verbatim}

Or we can use the \texttt{math.log} function to calculate the natural
logarithm of a number.

\begin{Shaded}
\begin{Highlighting}[]
\ImportTok{import}\NormalTok{ math}
\NormalTok{math.log(}\DecValTok{100}\NormalTok{)  }\CommentTok{\# logarithm}
\end{Highlighting}
\end{Shaded}

\begin{verbatim}
4.605170185988092
\end{verbatim}

\texttt{math.sqrt} and \texttt{math.log} are examples of Python
\emph{functions}, where an \emph{argument} (e.g., \texttt{100}) is
passed to the function to perform a calculation.

We will learn more about functions later.

\begin{tcolorbox}[enhanced jigsaw, colframe=quarto-callout-note-color-frame, opacityback=0, titlerule=0mm, bottomrule=.15mm, breakable, leftrule=.75mm, colbacktitle=quarto-callout-note-color!10!white, title=\textcolor{quarto-callout-note-color}{\faInfo}\hspace{0.5em}{Vocab}, rightrule=.15mm, coltitle=black, opacitybacktitle=0.6, colback=white, left=2mm, arc=.35mm, toptitle=1mm, bottomtitle=1mm, toprule=.15mm]

\textbf{Function}: A reusable block of code that performs a specific
task. Functions often take inputs (called arguments) and return outputs.

\end{tcolorbox}

\begin{tcolorbox}[enhanced jigsaw, colframe=quarto-callout-tip-color-frame, opacityback=0, titlerule=0mm, bottomrule=.15mm, breakable, leftrule=.75mm, colbacktitle=quarto-callout-tip-color!10!white, title=\textcolor{quarto-callout-tip-color}{\faLightbulb}\hspace{0.5em}{Practice}, rightrule=.15mm, coltitle=black, opacitybacktitle=0.6, colback=white, left=2mm, arc=.35mm, toptitle=1mm, bottomtitle=1mm, toprule=.15mm]

\subsection{Q: Using the math library}\label{q-using-the-math-library}

Using the \texttt{math} library, calculate the square root of 81.

Write your code below and run it to check your answers:

\begin{Shaded}
\begin{Highlighting}[]
\CommentTok{\# Your code here}
\end{Highlighting}
\end{Shaded}

\subsection{Q: Describing the use of the random
library}\label{q-describing-the-use-of-the-random-library}

Consider the following code, which generates a random number between 1
and 10:

\begin{Shaded}
\begin{Highlighting}[]
\ImportTok{import}\NormalTok{ random}
\NormalTok{random.randint(}\DecValTok{1}\NormalTok{, }\DecValTok{10}\NormalTok{)}
\end{Highlighting}
\end{Shaded}

\begin{verbatim}
1
\end{verbatim}

In that code, identify the library, the function, and the argument(s) to
the function.

\end{tcolorbox}

\section{Spacing in code}\label{spacing-in-code}

Good spacing makes your code easier to read. In Python, two simple
spacing practices can greatly improve your code's readability: using
blank lines and adding spaces around operators.

\subsection{Blank Lines}\label{blank-lines}

Use blank lines to separate different parts of your code:

For example, consider the following code chunk:

\begin{Shaded}
\begin{Highlighting}[]
\CommentTok{\# Set up numbers}
\NormalTok{x }\OperatorTok{=} \DecValTok{5}
\NormalTok{y }\OperatorTok{=} \DecValTok{10}
\CommentTok{\# Perform calculation}
\NormalTok{result }\OperatorTok{=}\NormalTok{ x }\OperatorTok{+}\NormalTok{ y}
\CommentTok{\# Display result}
\BuiltInTok{print}\NormalTok{(result)}
\end{Highlighting}
\end{Shaded}

\begin{verbatim}
15
\end{verbatim}

We can add blank lines to separate the different parts of the code:

\begin{Shaded}
\begin{Highlighting}[]
\CommentTok{\# Set up numbers}
\NormalTok{x }\OperatorTok{=} \DecValTok{5}
\NormalTok{y }\OperatorTok{=} \DecValTok{10}

\CommentTok{\# Perform calculation}
\NormalTok{result }\OperatorTok{=}\NormalTok{ x }\OperatorTok{+}\NormalTok{ y}

\CommentTok{\# Display result}
\BuiltInTok{print}\NormalTok{(result)}
\end{Highlighting}
\end{Shaded}

\begin{verbatim}
15
\end{verbatim}

Blank lines help organize your code into logical sections, similar to
paragraphs in writing.

\subsection{Spaces around operators}\label{spaces-around-operators}

Adding spaces around mathematical operators improves readability:

\begin{Shaded}
\begin{Highlighting}[]
\CommentTok{\# Hard to read}
\NormalTok{x}\OperatorTok{=}\DecValTok{5}\OperatorTok{+}\DecValTok{3}\OperatorTok{*}\DecValTok{2}

\CommentTok{\# Easy to read}
\NormalTok{x }\OperatorTok{=} \DecValTok{5} \OperatorTok{+} \DecValTok{3} \OperatorTok{*} \DecValTok{2}
\end{Highlighting}
\end{Shaded}

When listing items, add a space after each comma:

\begin{Shaded}
\begin{Highlighting}[]
\CommentTok{\# Hard to read}
\BuiltInTok{print}\NormalTok{(}\DecValTok{1}\NormalTok{,}\DecValTok{2}\NormalTok{,}\DecValTok{3}\NormalTok{)}

\CommentTok{\# Easy to read}
\BuiltInTok{print}\NormalTok{(}\DecValTok{1}\NormalTok{, }\DecValTok{2}\NormalTok{, }\DecValTok{3}\NormalTok{)}
\end{Highlighting}
\end{Shaded}

This practice follows the convention in written English, where we put a
space after a comma. It makes lists of items in your code easier to
read.

\section{Variables in Python}\label{variables-in-python}

\subsection{Create a variable}\label{create-a-variable}

As you have seen, to store a value for future use in Python, we assign
it to a \emph{variable} with the \emph{assignment operator}, \texttt{=}.

\begin{Shaded}
\begin{Highlighting}[]
\NormalTok{my\_var }\OperatorTok{=} \DecValTok{2} \OperatorTok{+} \DecValTok{2}  \CommentTok{\# assign the result of \textasciigrave{}2 + 2 \textasciigrave{} to the variable called \textasciigrave{}my\_var\textasciigrave{}}
\BuiltInTok{print}\NormalTok{(my\_var)  }\CommentTok{\# print my\_var}
\end{Highlighting}
\end{Shaded}

\begin{verbatim}
4
\end{verbatim}

Now that you've created the variable \texttt{my\_var}, Python knows
about it and will keep track of it during this Python session.

You can open your environment to see what variables you have created.
This looks different depending on your IDE.

So what exactly is a variable? Think of it as a named container that can
hold a value. When you run the code below:

\begin{Shaded}
\begin{Highlighting}[]
\NormalTok{my\_var }\OperatorTok{=} \DecValTok{20}
\end{Highlighting}
\end{Shaded}

you are telling Python, ``store the number 20 in a variable named
`my\_var'\,''.

Once the code is run, we would say, in Python terms, that ``the value of
variable \texttt{my\_var} is 20''.

Try to come up with a similar sentence for this code chunk:

\begin{Shaded}
\begin{Highlighting}[]
\NormalTok{first\_name }\OperatorTok{=} \StringTok{"Joanna"}
\end{Highlighting}
\end{Shaded}

After we run this code, we would say, in Python terms, that ``the value
of the \texttt{first\_name} variable is Joanna''.

\begin{tcolorbox}[enhanced jigsaw, colframe=quarto-callout-note-color-frame, opacityback=0, titlerule=0mm, bottomrule=.15mm, breakable, leftrule=.75mm, colbacktitle=quarto-callout-note-color!10!white, title=\textcolor{quarto-callout-note-color}{\faInfo}\hspace{0.5em}{Vocab}, rightrule=.15mm, coltitle=black, opacitybacktitle=0.6, colback=white, left=2mm, arc=.35mm, toptitle=1mm, bottomtitle=1mm, toprule=.15mm]

A text value like ``Joanna'' is called a \textbf{string}, while a number
like 20 is called an \textbf{integer}. If the number has a decimal
point, it is called a \textbf{float}, which is short for
``floating-point number''.

\end{tcolorbox}

\begin{tcolorbox}[enhanced jigsaw, colframe=quarto-callout-note-color-frame, opacityback=0, titlerule=0mm, bottomrule=.15mm, breakable, leftrule=.75mm, colbacktitle=quarto-callout-note-color!10!white, title=\textcolor{quarto-callout-note-color}{\faInfo}\hspace{0.5em}{Vocab}, rightrule=.15mm, coltitle=black, opacitybacktitle=0.6, colback=white, left=2mm, arc=.35mm, toptitle=1mm, bottomtitle=1mm, toprule=.15mm]

\textbf{Variable}: A named container that can hold a value. In Python,
variables can store different types of data, including numbers, strings,
and more complex objects.

\end{tcolorbox}

\subsection{Reassigning Variables}\label{reassigning-variables}

Reassigning a variable is like changing the contents of a container.

For example, previously we ran this code to store the value ``Joanna''
inside the \texttt{first\_name} variable:

\begin{Shaded}
\begin{Highlighting}[]
\NormalTok{first\_name }\OperatorTok{=} \StringTok{"Joanna"}
\end{Highlighting}
\end{Shaded}

To change this to a different value, simply run a new assignment
statement with a new value:

\begin{Shaded}
\begin{Highlighting}[]
\NormalTok{first\_name }\OperatorTok{=} \StringTok{"Luigi"}
\end{Highlighting}
\end{Shaded}

You can print the variable to observe the change:

\begin{Shaded}
\begin{Highlighting}[]
\NormalTok{first\_name}
\end{Highlighting}
\end{Shaded}

\begin{verbatim}
'Luigi'
\end{verbatim}

\subsection{Working with Variables}\label{working-with-variables}

Most of your time in Python will be spent manipulating variables. Let's
see some quick examples.

You can run simple commands on variables. For example, below we store
the value \texttt{100} in a variable and then take the square root of
the variable:

\begin{Shaded}
\begin{Highlighting}[]
\ImportTok{import}\NormalTok{ math}

\NormalTok{my\_number }\OperatorTok{=} \DecValTok{100}
\NormalTok{math.sqrt(my\_number)}
\end{Highlighting}
\end{Shaded}

\begin{verbatim}
10.0
\end{verbatim}

Python ``sees'' \texttt{my\_number} as the number 100, and so is able to
evaluate its square root.

\begin{center}\rule{0.5\linewidth}{0.5pt}\end{center}

You can also combine existing variables to create new variables. For
example, type out the code below to add \texttt{my\_number} to itself,
and store the result in a new variable called \texttt{my\_sum}:

\begin{Shaded}
\begin{Highlighting}[]
\NormalTok{my\_sum }\OperatorTok{=}\NormalTok{ my\_number }\OperatorTok{+}\NormalTok{ my\_number}
\NormalTok{my\_sum}
\end{Highlighting}
\end{Shaded}

\begin{verbatim}
200
\end{verbatim}

What should be the value of \texttt{my\_sum}? First take a guess, then
check it by printing it.

\begin{center}\rule{0.5\linewidth}{0.5pt}\end{center}

Python also allows us to concatenate strings with the \texttt{+}
operator. For example, we can concatenate the \texttt{first\_name} and
\texttt{last\_name} variables to create a new variable called
\texttt{full\_name}:

\begin{Shaded}
\begin{Highlighting}[]
\NormalTok{first\_name }\OperatorTok{=} \StringTok{"Joanna"}
\NormalTok{last\_name }\OperatorTok{=} \StringTok{"Luigi"}
\NormalTok{full\_name }\OperatorTok{=}\NormalTok{ first\_name }\OperatorTok{+} \StringTok{" "} \OperatorTok{+}\NormalTok{ last\_name}
\NormalTok{full\_name}
\end{Highlighting}
\end{Shaded}

\begin{verbatim}
'Joanna Luigi'
\end{verbatim}

\begin{tcolorbox}[enhanced jigsaw, colframe=quarto-callout-tip-color-frame, opacityback=0, titlerule=0mm, bottomrule=.15mm, breakable, leftrule=.75mm, colbacktitle=quarto-callout-tip-color!10!white, title=\textcolor{quarto-callout-tip-color}{\faLightbulb}\hspace{0.5em}{Practice}, rightrule=.15mm, coltitle=black, opacitybacktitle=0.6, colback=white, left=2mm, arc=.35mm, toptitle=1mm, bottomtitle=1mm, toprule=.15mm]

\subsection{Q: Variable assignment and
manipulation}\label{q-variable-assignment-and-manipulation}

Consider the code below. What is the value of the \texttt{answer}
variable? Think about it, then run the code to check your answer.

\begin{Shaded}
\begin{Highlighting}[]
\NormalTok{eight }\OperatorTok{=} \DecValTok{9}
\NormalTok{answer }\OperatorTok{=}\NormalTok{ eight }\OperatorTok{{-}} \DecValTok{8}
\NormalTok{answer}
\end{Highlighting}
\end{Shaded}

\end{tcolorbox}

\subsection{Getting User Input}\label{getting-user-input}

Though it's not used often in data analysis, the \texttt{input()}
function from Python is a cool Python feature that you should know
about. It allows you to get input from the user.

Here's a simple example. We can request user input and store it in a
variable called \texttt{name}.

\begin{Shaded}
\begin{Highlighting}[]
\NormalTok{name }\OperatorTok{=} \BuiltInTok{input}\NormalTok{()}
\end{Highlighting}
\end{Shaded}

And then we can print a greeting to the user.

\begin{Shaded}
\begin{Highlighting}[]
\BuiltInTok{print}\NormalTok{(}\StringTok{"Hello,"}\NormalTok{, name)}
\end{Highlighting}
\end{Shaded}

We can also include a question for the input prompt:

\begin{Shaded}
\begin{Highlighting}[]
\NormalTok{name }\OperatorTok{=} \BuiltInTok{input}\NormalTok{(}\StringTok{\textquotesingle{}What is your name? \textquotesingle{}}\NormalTok{)}
\BuiltInTok{print}\NormalTok{(}\StringTok{"Hello,"}\NormalTok{, name)}
\end{Highlighting}
\end{Shaded}

Let's see another example. We'll tell the user how many letters are in
their name.

\begin{Shaded}
\begin{Highlighting}[]
\NormalTok{name }\OperatorTok{=} \BuiltInTok{input}\NormalTok{(}\StringTok{\textquotesingle{}What is your name? \textquotesingle{}}\NormalTok{)}
\BuiltInTok{print}\NormalTok{(}\StringTok{"There are"}\NormalTok{, }\BuiltInTok{len}\NormalTok{(name), }\StringTok{"letters in your name"}\NormalTok{)}
\end{Highlighting}
\end{Shaded}

For instance, if you run this code and enter ``Kene'', you might see:

\begin{verbatim}
What is your name? Kene
There are 4 letters in your name
\end{verbatim}

\begin{tcolorbox}[enhanced jigsaw, colframe=quarto-callout-tip-color-frame, opacityback=0, titlerule=0mm, bottomrule=.15mm, breakable, leftrule=.75mm, colbacktitle=quarto-callout-tip-color!10!white, title=\textcolor{quarto-callout-tip-color}{\faLightbulb}\hspace{0.5em}{Practice}, rightrule=.15mm, coltitle=black, opacitybacktitle=0.6, colback=white, left=2mm, arc=.35mm, toptitle=1mm, bottomtitle=1mm, toprule=.15mm]

\subsection{Q: Using input()}\label{q-using-input}

Write a short program that asks the user for their favorite color and
then prints a message saying ``Your favorite color is {[}color{]}!''.
Test your program by running it and entering a color.

\end{tcolorbox}

\subsection{Common Error with
Variables}\label{common-error-with-variables}

One of the most common errors you'll encounter when working with
variables in Python is the \texttt{NameError}. This occurs when you try
to use a variable that hasn't been defined yet. For example:

\begin{Shaded}
\begin{Highlighting}[]
\NormalTok{my\_number }\OperatorTok{=} \DecValTok{48}  \CommentTok{\# define \textasciigrave{}my\_number\textasciigrave{}}
\NormalTok{My\_number }\OperatorTok{+} \DecValTok{2}  \CommentTok{\# attempt to add 2 to \textasciigrave{}my\_number\textasciigrave{}}
\end{Highlighting}
\end{Shaded}

If you run this code, you'll get an error message like this:

\begin{verbatim}
NameError: name 'My_number' is not defined
\end{verbatim}

Here, Python returns an error message because we haven't created (or
\emph{defined}) the variable \texttt{My\_number} yet. Recall that Python
is case-sensitive; we defined \texttt{my\_number} but tried to use
\texttt{My\_number}.

To fix this, make sure you're using the correct variable name:

\begin{Shaded}
\begin{Highlighting}[]
\NormalTok{my\_number }\OperatorTok{=} \DecValTok{48}
\NormalTok{my\_number }\OperatorTok{+} \DecValTok{2}  \CommentTok{\# This will work and return 50}
\end{Highlighting}
\end{Shaded}

\begin{verbatim}
50
\end{verbatim}

Always double-check your variable names to avoid this error. Remember,
in Python, \texttt{my\_number}, \texttt{My\_number}, and
\texttt{MY\_NUMBER} are all different variables.

\begin{center}\rule{0.5\linewidth}{0.5pt}\end{center}

When you first start learning Python, dealing with errors can be
frustrating. They're often difficult to understand.

But it's important to get used to reading and understanding errors,
because you'll get them a lot through your coding career.

Later, we will show you how to use Large Language Models (LLMs) like
ChatGPT to debug errors.

At the start though, it's good to try to spot and fix errors yourself.

\begin{tcolorbox}[enhanced jigsaw, colframe=quarto-callout-tip-color-frame, opacityback=0, titlerule=0mm, bottomrule=.15mm, breakable, leftrule=.75mm, colbacktitle=quarto-callout-tip-color!10!white, title=\textcolor{quarto-callout-tip-color}{\faLightbulb}\hspace{0.5em}{Practice}, rightrule=.15mm, coltitle=black, opacitybacktitle=0.6, colback=white, left=2mm, arc=.35mm, toptitle=1mm, bottomtitle=1mm, toprule=.15mm]

\subsection{Q: Debugging variable
errors}\label{q-debugging-variable-errors}

The code below returns an error. Why? (Look carefully)

\begin{Shaded}
\begin{Highlighting}[]
\NormalTok{my\_1st\_name }\OperatorTok{=} \StringTok{"Kene"}
\NormalTok{my\_last\_name }\OperatorTok{=} \StringTok{"Nwosu"}

\BuiltInTok{print}\NormalTok{(my\_Ist\_name, my\_last\_name)}
\end{Highlighting}
\end{Shaded}

Hint: look at the variable names. Are they consistent?

\end{tcolorbox}

\subsection{Naming Variables}\label{naming-variables}

\begin{quote}
There are only \textbf{\emph{two hard things}} in Computer Science:
cache invalidation and \textbf{\emph{naming things}}.

--- Phil Karlton.
\end{quote}

Because much of your work in Python involves interacting with variables
you have created, picking intelligent names for these variables is
important.

Naming variables is difficult because names should be both
\textbf{short} (so that you can type them quickly) and
\textbf{informative} (so that you can easily remember what the variable
contains), and these two goals are often in conflict.

So names that are too long, like the one below, are bad because they
take forever to type.

\begin{Shaded}
\begin{Highlighting}[]
\NormalTok{sample\_of\_the\_ebola\_outbreak\_dataset\_from\_sierra\_leone\_in\_2014}
\end{Highlighting}
\end{Shaded}

And a name like \texttt{data} is bad because it is not informative; the
name does not give a good idea of what the variable contains.

As you write more Python code, you will learn how to write short and
informative names.

\begin{center}\rule{0.5\linewidth}{0.5pt}\end{center}

For names with multiple words, there are a few conventions for how to
separate the words:

\begin{Shaded}
\begin{Highlighting}[]
\NormalTok{snake\_case }\OperatorTok{=} \StringTok{"Snake case uses underscores"}
\NormalTok{camelCase }\OperatorTok{=} \StringTok{"Camel case capitalizes new words (but not the first word)"}
\NormalTok{PascalCase }\OperatorTok{=} \StringTok{"Pascal case capitalizes all words including the first"}
\end{Highlighting}
\end{Shaded}

We recommend snake\_case, which uses all lower-case words, and separates
words with \texttt{\_}.

\begin{center}\rule{0.5\linewidth}{0.5pt}\end{center}

Note too that there are some limitations on variable names:

\begin{itemize}
\tightlist
\item
  Names must start with a letter or underscore. So \texttt{2014\_data}
  is not a valid name (because it starts with a number). Try running the
  code chunk below to see what error you get.
\end{itemize}

\begin{Shaded}
\begin{Highlighting}[]
\DecValTok{2014}\ErrorTok{\_data} \OperatorTok{=} \StringTok{"This is not a valid name"}
\end{Highlighting}
\end{Shaded}

\begin{itemize}
\tightlist
\item
  Names can only contain letters, numbers, and underscores
  (\texttt{\_}). So \texttt{ebola-data} or
  \texttt{ebola\textasciitilde{}data} or \texttt{ebola\ data} with a
  space are not valid names.
\end{itemize}

\begin{Shaded}
\begin{Highlighting}[]
\NormalTok{ebola}\OperatorTok{{-}}\NormalTok{data }\OperatorTok{=} \StringTok{"This is not a valid name"}
\end{Highlighting}
\end{Shaded}

\begin{Shaded}
\begin{Highlighting}[]
\NormalTok{ebola}\OperatorTok{\textasciitilde{}}\NormalTok{data }\OperatorTok{=} \StringTok{"This is not a valid name"}
\end{Highlighting}
\end{Shaded}

\begin{tcolorbox}[enhanced jigsaw, colframe=quarto-callout-note-color-frame, opacityback=0, titlerule=0mm, bottomrule=.15mm, breakable, leftrule=.75mm, colbacktitle=quarto-callout-note-color!10!white, title=\textcolor{quarto-callout-note-color}{\faInfo}\hspace{0.5em}{Side note}, rightrule=.15mm, coltitle=black, opacitybacktitle=0.6, colback=white, left=2mm, arc=.35mm, toptitle=1mm, bottomtitle=1mm, toprule=.15mm]

While we recommend snake\_case for variable names in Python, you might
see other conventions like camelCase or PascalCase, especially when
working with code from other languages or certain Python libraries. It's
important to be consistent within your own code and follow the
conventions of any project or team you're working with.

\end{tcolorbox}

\begin{tcolorbox}[enhanced jigsaw, colframe=quarto-callout-tip-color-frame, opacityback=0, titlerule=0mm, bottomrule=.15mm, breakable, leftrule=.75mm, colbacktitle=quarto-callout-tip-color!10!white, title=\textcolor{quarto-callout-tip-color}{\faLightbulb}\hspace{0.5em}{Practice}, rightrule=.15mm, coltitle=black, opacitybacktitle=0.6, colback=white, left=2mm, arc=.35mm, toptitle=1mm, bottomtitle=1mm, toprule=.15mm]

\subsection{Q: Valid variable naming
conventions}\label{q-valid-variable-naming-conventions}

Which of the following variable names are valid in Python? Try to
determine this without running the code, then check your answers by
attempting to run each line.

Then fix the invalid variable names.

\begin{Shaded}
\begin{Highlighting}[]
\DecValTok{1}\ErrorTok{st\_name} \OperatorTok{=} \StringTok{"John"}
\NormalTok{last\_name }\OperatorTok{=} \StringTok{"Doe"}
\NormalTok{full}\OperatorTok{{-}}\NormalTok{name }\OperatorTok{=} \StringTok{"John Doe"}
\NormalTok{age\_in\_years }\OperatorTok{=} \DecValTok{30}
\NormalTok{current}\OperatorTok{@}\NormalTok{job }\OperatorTok{=} \StringTok{"Developer"}
\NormalTok{PhoneNumber }\OperatorTok{=} \StringTok{"555{-}1234"}
\NormalTok{\_secret\_code }\OperatorTok{=} \DecValTok{42}
\end{Highlighting}
\end{Shaded}

\end{tcolorbox}

\section{Wrap-up}\label{wrap-up}

In this lesson, we've covered the fundamental building blocks of Python
programming:

\begin{enumerate}
\def\labelenumi{\arabic{enumi}.}
\tightlist
\item
  \textbf{Comments}: Using \texttt{\#} for single-line and triple quotes
  for multi-line comments.
\item
  \textbf{Basic Arithmetic}: Using Python as a calculator and
  understanding order of operations.
\item
  \textbf{Math Library}: Performing complex mathematical operations.
\item
  \textbf{Code Spacing}: Improving readability with proper spacing.
\item
  \textbf{Variables}: Creating, manipulating, and reassigning variables
  of different types.
\item
  \textbf{Getting User Input}: Using the \texttt{input()} function to
  get input from the user.
\item
  \textbf{Variable Naming}: Following rules and best practices for
  naming variables.
\item
  \textbf{Common Errors}: Identifying and fixing errors related to
  variables.
\end{enumerate}

These concepts form the foundation of Python programming. As you
continue your journey, you'll build upon these basics to create more
complex and powerful programs. Remember, practice is key to mastering
these concepts!

\chapter{Functions, Methods, and Libraries in
Python}\label{functions-methods-and-libraries-in-python}

\section{Learning objectives}\label{learning-objectives-2}

\begin{enumerate}
\def\labelenumi{\arabic{enumi}.}
\tightlist
\item
  You understand what functions and methods are in Python.
\item
  You can identify and use arguments (parameters) in functions and
  methods.
\item
  You know how to call built-in functions and methods on objects.
\item
  You understand what libraries are in Python and how to import them.
\item
  You know how to install a simple external library and use it in your
  code.
\end{enumerate}

\section{Introduction}\label{introduction-3}

In this lesson, you will learn about functions, methods, and libraries
in Python, building on the basics we covered in the previous lesson.

To get started, open your preferred Python environment (e.g., Jupyter
Notebook, VS Code, or PyCharm), and create a new Python file or
notebook.

Next, \textbf{save the file} with a name like
``functions\_and\_libraries.py'' or ``functions\_and\_libraries.ipynb''
depending on your environment.

You should now type all the code from this lesson into that file.

\section{Functions}\label{functions}

A function is a block of code that performs a specific task. It can take
inputs (arguments) and return outputs. Here's an example of a built-in
function with just one argument:

\begin{Shaded}
\begin{Highlighting}[]
\CommentTok{\# Using the len() function to get the length of a string}
\BuiltInTok{len}\NormalTok{(}\StringTok{"Python"}\NormalTok{)}
\end{Highlighting}
\end{Shaded}

\begin{verbatim}
6
\end{verbatim}

The \texttt{round()} function takes two arguments: the number to round
and the number of decimal places to round to.

\begin{Shaded}
\begin{Highlighting}[]
\CommentTok{\# Using the round() function to round a number}
\BuiltInTok{round}\NormalTok{(}\FloatTok{3.1415}\NormalTok{, }\DecValTok{2}\NormalTok{)}
\end{Highlighting}
\end{Shaded}

\begin{verbatim}
3.14
\end{verbatim}

\begin{tcolorbox}[enhanced jigsaw, colframe=quarto-callout-tip-color-frame, opacityback=0, titlerule=0mm, bottomrule=.15mm, breakable, leftrule=.75mm, colbacktitle=quarto-callout-tip-color!10!white, title=\textcolor{quarto-callout-tip-color}{\faLightbulb}\hspace{0.5em}{Practice}, rightrule=.15mm, coltitle=black, opacitybacktitle=0.6, colback=white, left=2mm, arc=.35mm, toptitle=1mm, bottomtitle=1mm, toprule=.15mm]

\subsection{Q: Using built-in
functions}\label{q-using-built-in-functions}

Use the \texttt{abs()} function to get the absolute value of -5.

Write your code below and run it to check your answer:

\begin{Shaded}
\begin{Highlighting}[]
\CommentTok{\# Your code here}
\end{Highlighting}
\end{Shaded}

\end{tcolorbox}

\section{Arguments (Parameters)}\label{arguments-parameters}

Arguments (also called parameters) are the values that you pass to a
function (or method) when you call it.

There are different ways to pass arguments to a function.

Consider again the \texttt{round()} function.

If we look at the documentation for the \texttt{round()} function, with
:

\begin{Shaded}
\begin{Highlighting}[]
\BuiltInTok{round}\NormalTok{?}
\end{Highlighting}
\end{Shaded}

We see that it takes two arguments:

\begin{itemize}
\tightlist
\item
  \texttt{number}: The number to round.
\item
  \texttt{ndigits}: The number of decimal places to round to.
\end{itemize}

There are two main ways to pass arguments to this function.

\begin{enumerate}
\def\labelenumi{\arabic{enumi}.}
\tightlist
\item
  Positional arguments: Passed in the order they are defined. Since the
  default order of the arguments is \texttt{number} then
  \texttt{ndigits}, we can pass the arguments in that order without
  specifying the argument names, as we did above.
\end{enumerate}

\begin{Shaded}
\begin{Highlighting}[]
\BuiltInTok{round}\NormalTok{(}\FloatTok{3.1415}\NormalTok{, }\DecValTok{2}\NormalTok{)}
\end{Highlighting}
\end{Shaded}

\begin{verbatim}
3.14
\end{verbatim}

If we swap the order of the arguments, we get an error:

\begin{Shaded}
\begin{Highlighting}[]
\BuiltInTok{round}\NormalTok{(}\DecValTok{2}\NormalTok{, }\FloatTok{3.1415}\NormalTok{)}
\end{Highlighting}
\end{Shaded}

\begin{enumerate}
\def\labelenumi{\arabic{enumi}.}
\setcounter{enumi}{1}
\tightlist
\item
  Keyword arguments: Passed by specifying the argument name followed by
  a \texttt{=} and the argument value.
\end{enumerate}

\begin{Shaded}
\begin{Highlighting}[]
\BuiltInTok{round}\NormalTok{(number}\OperatorTok{=}\FloatTok{3.1415}\NormalTok{, ndigits}\OperatorTok{=}\DecValTok{2}\NormalTok{)}
\end{Highlighting}
\end{Shaded}

\begin{verbatim}
3.14
\end{verbatim}

With this method, we can pass the arguments in any order, as long as we
use the argument names.

\begin{Shaded}
\begin{Highlighting}[]
\BuiltInTok{round}\NormalTok{(ndigits}\OperatorTok{=}\DecValTok{2}\NormalTok{, number}\OperatorTok{=}\FloatTok{3.1415}\NormalTok{)}
\end{Highlighting}
\end{Shaded}

\begin{verbatim}
3.14
\end{verbatim}

Specifying the keyword is usually recommended, except for simple
functions with very few arguments, or when the order of the arguments is
obvious from context.

\begin{tcolorbox}[enhanced jigsaw, colframe=quarto-callout-tip-color-frame, opacityback=0, titlerule=0mm, bottomrule=.15mm, breakable, leftrule=.75mm, colbacktitle=quarto-callout-tip-color!10!white, title=\textcolor{quarto-callout-tip-color}{\faLightbulb}\hspace{0.5em}{Practice}, rightrule=.15mm, coltitle=black, opacitybacktitle=0.6, colback=white, left=2mm, arc=.35mm, toptitle=1mm, bottomtitle=1mm, toprule=.15mm]

\subsection{\texorpdfstring{Q: Using Positional Arguments with
\texttt{pow()}}{Q: Using Positional Arguments with pow()}}\label{q-using-positional-arguments-with-pow}

Use the \texttt{pow()} function to calculate 2 raised to the power of 5
by passing positional arguments. You may need to consult the
documentation for the \texttt{pow()} function to see how it works.

Write your code below and run it to check your answer:

\begin{Shaded}
\begin{Highlighting}[]
\CommentTok{\# Your code here}
\end{Highlighting}
\end{Shaded}

\subsection{\texorpdfstring{Q: Using Keyword Arguments with
\texttt{round()}}{Q: Using Keyword Arguments with round()}}\label{q-using-keyword-arguments-with-round}

Use the \texttt{round()} function to round the number \texttt{9.8765} to
\texttt{3} decimal places by specifying keyword arguments.

Write your code below and run it to check your answer:

\begin{Shaded}
\begin{Highlighting}[]
\CommentTok{\# Your code here}
\end{Highlighting}
\end{Shaded}

\end{tcolorbox}

\section{Methods}\label{methods}

Methods are similar to functions, but they are associated with specific
objects or data types. They are called using dot notation.

For example, every string object comes with a range of built-in methods,
like \texttt{upper()} to convert to uppercase, \texttt{lower()} to
convert to lowercase, \texttt{replace()} to replace substrings, and many
more.

Let's see how to use these:

\begin{Shaded}
\begin{Highlighting}[]
\NormalTok{name }\OperatorTok{=} \StringTok{"python"}
\BuiltInTok{print}\NormalTok{(name.upper())}
\BuiltInTok{print}\NormalTok{(name.lower())}
\BuiltInTok{print}\NormalTok{(name.replace(}\StringTok{"p"}\NormalTok{, }\StringTok{"🐍"}\NormalTok{))}
\end{Highlighting}
\end{Shaded}

\begin{verbatim}
PYTHON
python
🐍ython
\end{verbatim}

We can also call the methods directly on the string object, without
assigning it to a variable:

\begin{Shaded}
\begin{Highlighting}[]
\CommentTok{\# Using the upper() method on a string}
\BuiltInTok{print}\NormalTok{(}\StringTok{"python"}\NormalTok{.upper())}
\BuiltInTok{print}\NormalTok{(}\StringTok{"PYTHON"}\NormalTok{.lower())}
\BuiltInTok{print}\NormalTok{(}\StringTok{"python"}\NormalTok{.replace(}\StringTok{"p"}\NormalTok{, }\StringTok{"🐍"}\NormalTok{))}
\end{Highlighting}
\end{Shaded}

\begin{verbatim}
PYTHON
python
🐍ython
\end{verbatim}

Similarly, numbers in Python come with some built-in methods. For
example, the \texttt{as\_integer\_ratio()} (added in Python 3.8) method
converts a decimalnumber to a ratio of two integers.

\begin{Shaded}
\begin{Highlighting}[]
\CommentTok{\# Using the as\_integer\_ratio() method on a float}
\NormalTok{example\_decimal }\OperatorTok{=} \FloatTok{1.5}
\NormalTok{example\_decimal.as\_integer\_ratio()}
\end{Highlighting}
\end{Shaded}

\begin{verbatim}
(3, 2)
\end{verbatim}

\begin{tcolorbox}[enhanced jigsaw, colframe=quarto-callout-note-color-frame, opacityback=0, titlerule=0mm, bottomrule=.15mm, breakable, leftrule=.75mm, colbacktitle=quarto-callout-note-color!10!white, title=\textcolor{quarto-callout-note-color}{\faInfo}\hspace{0.5em}{Practice}, rightrule=.15mm, coltitle=black, opacitybacktitle=0.6, colback=white, left=2mm, arc=.35mm, toptitle=1mm, bottomtitle=1mm, toprule=.15mm]

\subsection{Q: Definitions}\label{q-definitions}

Come up with simple definitions for the following terms that are clear
to YOU (even if not technically exactly accurate):

\begin{itemize}
\tightlist
\item
  Function
\item
  Method
\item
  Argument (Parameter)
\item
  Dot Notation
\end{itemize}

\end{tcolorbox}

\begin{tcolorbox}[enhanced jigsaw, colframe=quarto-callout-tip-color-frame, opacityback=0, titlerule=0mm, bottomrule=.15mm, breakable, leftrule=.75mm, colbacktitle=quarto-callout-tip-color!10!white, title=\textcolor{quarto-callout-tip-color}{\faLightbulb}\hspace{0.5em}{Practice}, rightrule=.15mm, coltitle=black, opacitybacktitle=0.6, colback=white, left=2mm, arc=.35mm, toptitle=1mm, bottomtitle=1mm, toprule=.15mm]

\subsection{Q: Using methods}\label{q-using-methods}

\begin{enumerate}
\def\labelenumi{\arabic{enumi}.}
\tightlist
\item
  Call the \texttt{replace()} method on the string ``Helo'' to replace
  the single l with two ls.
\item
  Call the \texttt{split()} method on the string ``Hello World'' to
  split the string into a list of words.
\end{enumerate}

\begin{Shaded}
\begin{Highlighting}[]
\CommentTok{\# Your code here}
\end{Highlighting}
\end{Shaded}

\end{tcolorbox}

\section{Libraries in Python}\label{libraries-in-python}

Libraries are collections of pre-written code that you can use in your
programs. They extend the functionality of Python by providing
additional functions and tools.

For example, the \texttt{math} library provides mathematical functions
like \texttt{sqrt()} for square roots and \texttt{sin()} for sine.

If we try to use the \texttt{sqrt()} function without importing the
\texttt{math} library, we get an error:

\begin{Shaded}
\begin{Highlighting}[]
\CommentTok{\# This will cause a NameError}
\NormalTok{sqrt(}\DecValTok{16}\NormalTok{)}
\end{Highlighting}
\end{Shaded}

We can import the \texttt{math} library and use the \texttt{sqrt()}
function like this:

\begin{Shaded}
\begin{Highlighting}[]
\CommentTok{\# Import the library}
\ImportTok{import}\NormalTok{ math}
\end{Highlighting}
\end{Shaded}

Then we can use the \texttt{sqrt()} function like this:

\begin{Shaded}
\begin{Highlighting}[]
\CommentTok{\# Use the sqrt() function}
\NormalTok{math.sqrt(}\DecValTok{16}\NormalTok{)}
\end{Highlighting}
\end{Shaded}

\begin{verbatim}
4.0
\end{verbatim}

We can get help on a function in a similar way, calling both the
function and the library it's in:

\begin{Shaded}
\begin{Highlighting}[]
\CommentTok{\# Get help on the sqrt() function}
\NormalTok{math.sqrt?}
\end{Highlighting}
\end{Shaded}

We can also import libraries with aliases. For example, we can import
the \texttt{math} library with the alias \texttt{m}:

\begin{Shaded}
\begin{Highlighting}[]
\CommentTok{\# Import the entire library with an alias}
\ImportTok{import}\NormalTok{ math }\ImportTok{as}\NormalTok{ m}
\CommentTok{\# Then we can use the alias to call the function}
\NormalTok{m.sqrt(}\DecValTok{16}\NormalTok{)}
\end{Highlighting}
\end{Shaded}

\begin{verbatim}
4.0
\end{verbatim}

Finally, if you want to skip the alias/library name, you can either
import the functions individually:

\begin{Shaded}
\begin{Highlighting}[]
\CommentTok{\# Import specific functions from a library}
\ImportTok{from}\NormalTok{ math }\ImportTok{import}\NormalTok{ sqrt, sin}
\CommentTok{\# Then we can use the function directly}
\NormalTok{sqrt(}\DecValTok{16}\NormalTok{)}
\NormalTok{sin(}\DecValTok{0}\NormalTok{)}
\end{Highlighting}
\end{Shaded}

\begin{verbatim}
0.0
\end{verbatim}

Or import everything from the library:

\begin{Shaded}
\begin{Highlighting}[]
\CommentTok{\# Import everything from the library}
\ImportTok{from}\NormalTok{ math }\ImportTok{import} \OperatorTok{*}
\CommentTok{\# Then we can all functions directly, such as sqrt() and sin()}
\NormalTok{sqrt(}\DecValTok{16}\NormalTok{)}
\NormalTok{cos(}\DecValTok{0}\NormalTok{)}
\NormalTok{tan(}\DecValTok{0}\NormalTok{)}
\NormalTok{sin(}\DecValTok{0}\NormalTok{)}
\end{Highlighting}
\end{Shaded}

\begin{verbatim}
0.0
\end{verbatim}

Phew that's a lot of ways to import libraries! You'll mostly see the
\texttt{import\ ...\ as\ ...} syntax, and sometimes the
\texttt{from\ ...\ import\ ...} syntax.

Note that we typically import all required libraries at the top of the
file, in a single code chunk. This is a good practice to follow.

\begin{tcolorbox}[enhanced jigsaw, colframe=quarto-callout-tip-color-frame, opacityback=0, titlerule=0mm, bottomrule=.15mm, breakable, leftrule=.75mm, colbacktitle=quarto-callout-tip-color!10!white, title=\textcolor{quarto-callout-tip-color}{\faLightbulb}\hspace{0.5em}{Practice}, rightrule=.15mm, coltitle=black, opacitybacktitle=0.6, colback=white, left=2mm, arc=.35mm, toptitle=1mm, bottomtitle=1mm, toprule=.15mm]

\subsection{Q: Definitions}\label{q-definitions-1}

Come up with simple definitions for the following terms that are clear
to YOU (even if not technically exactly accurate):

\begin{itemize}
\tightlist
\item
  Library (Module)
\item
  Import
\item
  Alias
\end{itemize}

\end{tcolorbox}

\begin{tcolorbox}[enhanced jigsaw, colframe=quarto-callout-tip-color-frame, opacityback=0, titlerule=0mm, bottomrule=.15mm, breakable, leftrule=.75mm, colbacktitle=quarto-callout-tip-color!10!white, title=\textcolor{quarto-callout-tip-color}{\faLightbulb}\hspace{0.5em}{Practice}, rightrule=.15mm, coltitle=black, opacitybacktitle=0.6, colback=white, left=2mm, arc=.35mm, toptitle=1mm, bottomtitle=1mm, toprule=.15mm]

\subsection{Q: Using functions from imported
libraries}\label{q-using-functions-from-imported-libraries}

\begin{itemize}
\tightlist
\item
  Import the \texttt{random} library and use the \texttt{randint()}
  function to generate a random integer between 1 and 10. You can use
  the \texttt{?} operator to get help on the function after importing
  it.
\end{itemize}

\begin{Shaded}
\begin{Highlighting}[]
\CommentTok{\# Your code here}
\end{Highlighting}
\end{Shaded}

\end{tcolorbox}

\section{Installing Libraries}\label{installing-libraries}

While Python comes with many built-in libraries, there are thousands of
additional libraries available that you can install to extend Python's
functionality even further. Let's look at how to install and use a
simple external library, with the \texttt{cowsay} library as an example.

If we try to import this library without first installing it, we get an
error:

\begin{Shaded}
\begin{Highlighting}[]
\ImportTok{import}\NormalTok{ cowsay}
\end{Highlighting}
\end{Shaded}

To install the library, you can use the \texttt{!pip\ install} command
in a code cell in Google Colab. For \texttt{cowsay}, you would run:

\begin{Shaded}
\begin{Highlighting}[]
\OperatorTok{!}\NormalTok{pip install cowsay}
\end{Highlighting}
\end{Shaded}

Pip installs packages from a remote repository called
\href{https://pypi.org/}{PyPI}. Anyone can create and upload a package
to PyPI. After a few checks, it's then available for anyone to install.

\begin{tcolorbox}[enhanced jigsaw, colframe=quarto-callout-note-color-frame, opacityback=0, titlerule=0mm, bottomrule=.15mm, breakable, leftrule=.75mm, colbacktitle=quarto-callout-note-color!10!white, title=\textcolor{quarto-callout-note-color}{\faInfo}\hspace{0.5em}{Side-note}, rightrule=.15mm, coltitle=black, opacitybacktitle=0.6, colback=white, left=2mm, arc=.35mm, toptitle=1mm, bottomtitle=1mm, toprule=.15mm]

For those working on local Python instances, you can install
\texttt{cowsay} using pip in your terminal:

\begin{verbatim}
pip install cowsay
\end{verbatim}

\end{tcolorbox}

Once installed, we can now import and use the \texttt{cowsay} library:

\begin{Shaded}
\begin{Highlighting}[]
\ImportTok{import}\NormalTok{ cowsay}

\CommentTok{\# Make the cow say something}
\NormalTok{cowsay.cow(}\StringTok{\textquotesingle{}Moo!\textquotesingle{}}\NormalTok{)}
\end{Highlighting}
\end{Shaded}

\begin{verbatim}
  ____
| Moo! |
  ====
    \
     \
       ^__^
       (oo)\_______
       (__)\       )\/\
           ||----w |
           ||     ||
\end{verbatim}

This should display an ASCII art cow saying ``Moo!''.

\begin{tcolorbox}[enhanced jigsaw, colframe=quarto-callout-tip-color-frame, opacityback=0, titlerule=0mm, bottomrule=.15mm, breakable, leftrule=.75mm, colbacktitle=quarto-callout-tip-color!10!white, title=\textcolor{quarto-callout-tip-color}{\faLightbulb}\hspace{0.5em}{Practice}, rightrule=.15mm, coltitle=black, opacitybacktitle=0.6, colback=white, left=2mm, arc=.35mm, toptitle=1mm, bottomtitle=1mm, toprule=.15mm]

\subsection{Q: Using the emoji library}\label{q-using-the-emoji-library}

\begin{enumerate}
\def\labelenumi{\arabic{enumi}.}
\tightlist
\item
  Install the \texttt{emoji} library.
\item
  Import the \texttt{emoji} library.
\item
  Consult the help for the \texttt{emojize()} function in the
  \texttt{emoji} library.
\item
  Use the \texttt{emojize()} function to display an emoji for ``thumbs
  up''.
\end{enumerate}

\begin{Shaded}
\begin{Highlighting}[]
\CommentTok{\# Your code here}
\end{Highlighting}
\end{Shaded}

\end{tcolorbox}

\section{Wrap-up}\label{wrap-up-1}

In this lesson, we've covered:

\begin{enumerate}
\def\labelenumi{\arabic{enumi}.}
\tightlist
\item
  Functions and methods in Python
\item
  Arguments (parameters) and how to use them
\item
  Importing and using libraries
\item
  Installing and using an external library
\end{enumerate}

These concepts are fundamental to Python programming and will be used
extensively as you continue to develop your skills. Practice using
different functions, methods, and libraries to become more comfortable
with these concepts.

\begin{Shaded}
\begin{Highlighting}[]
\ImportTok{import}\NormalTok{ os}
\ImportTok{from}\NormalTok{ PyPDF2 }\ImportTok{import}\NormalTok{ PdfReader, PdfWriter}
\ImportTok{from}\NormalTok{ reportlab.pdfgen }\ImportTok{import}\NormalTok{ canvas}
\ImportTok{from}\NormalTok{ reportlab.lib.pagesizes }\ImportTok{import}\NormalTok{ letter}

\CommentTok{\# Path to the input PDF file}
\NormalTok{input\_pdf }\OperatorTok{=} \StringTok{"/Users/kendavidn/Dropbox/Mac (2)/Downloads/Certificates 19Sep2024.pdf"}

\CommentTok{\# List of students and their information}
\NormalTok{students }\OperatorTok{=}\NormalTok{ [}
    \StringTok{"Adriana Gomes"}\NormalTok{,}
    \StringTok{"Loureen Valyne"}\NormalTok{,}
    \StringTok{"Nuneaton Ramesar"}\NormalTok{,}
    \StringTok{"Toheeb Olanrewaju Sarafadeen"}\NormalTok{,}
    \StringTok{"Rajkumar Rajendram"}\NormalTok{,}
    \StringTok{"Abdeldjalil Belkheir"}\NormalTok{,}
    \StringTok{"Eman Elafef"}\NormalTok{,}
    \StringTok{"Yopa Abraham"}\NormalTok{,}
    \StringTok{"Augustine Alie"}\NormalTok{,}
    \StringTok{"Sarah Macklin"}\NormalTok{,}
    \StringTok{"Ekumbo Botuli"}\NormalTok{,}
    \StringTok{"Roya Hosseini"}
\NormalTok{]}

\CommentTok{\# Certificate name (you can modify this as needed)}
\NormalTok{certificate\_name }\OperatorTok{=} \StringTok{"FOSSA Certificate"}

\CommentTok{\# Create output directory if it doesn\textquotesingle{}t exist}
\NormalTok{output\_dir }\OperatorTok{=} \StringTok{"/Users/kendavidn/Dropbox/Mac (2)/Downloads/split\_certificates"}
\NormalTok{os.makedirs(output\_dir, exist\_ok}\OperatorTok{=}\VariableTok{True}\NormalTok{)}

\CommentTok{\# Open the input PDF}
\NormalTok{pdf\_reader }\OperatorTok{=}\NormalTok{ PdfReader(input\_pdf)}

\CommentTok{\# Iterate through pages and save each as a separate file}
\ControlFlowTok{for}\NormalTok{ i, student }\KeywordTok{in} \BuiltInTok{enumerate}\NormalTok{(students, start}\OperatorTok{=}\DecValTok{1}\NormalTok{):}
\NormalTok{    pdf\_writer }\OperatorTok{=}\NormalTok{ PdfWriter()}
\NormalTok{    pdf\_writer.add\_page(pdf\_reader.pages[i}\OperatorTok{{-}}\DecValTok{1}\NormalTok{])}
    
    \CommentTok{\# Create output filename}
\NormalTok{    output\_filename }\OperatorTok{=} \SpecialStringTok{f"}\SpecialCharTok{\{}\NormalTok{student}\SpecialCharTok{\}}\SpecialStringTok{ {-} }\SpecialCharTok{\{}\NormalTok{certificate\_name}\SpecialCharTok{\}}\SpecialStringTok{.pdf"}
\NormalTok{    output\_path }\OperatorTok{=}\NormalTok{ os.path.join(output\_dir, output\_filename)}
    
    \CommentTok{\# Save the individual page}
    \ControlFlowTok{with} \BuiltInTok{open}\NormalTok{(output\_path, }\StringTok{"wb"}\NormalTok{) }\ImportTok{as}\NormalTok{ output\_file:}
\NormalTok{        pdf\_writer.write(output\_file)}
    
    \BuiltInTok{print}\NormalTok{(}\SpecialStringTok{f"Saved: }\SpecialCharTok{\{}\NormalTok{output\_filename}\SpecialCharTok{\}}\SpecialStringTok{"}\NormalTok{)}

\BuiltInTok{print}\NormalTok{(}\StringTok{"All certificates have been split and saved."}\NormalTok{)}
\end{Highlighting}
\end{Shaded}

\begin{verbatim}
Saved: Adriana Gomes - FOSSA Certificate.pdf
Saved: Loureen Valyne - FOSSA Certificate.pdf
Saved: Nuneaton Ramesar - FOSSA Certificate.pdf
Saved: Toheeb Olanrewaju Sarafadeen - FOSSA Certificate.pdf
Saved: Rajkumar Rajendram - FOSSA Certificate.pdf
Saved: Abdeldjalil Belkheir - FOSSA Certificate.pdf
Saved: Eman Elafef - FOSSA Certificate.pdf
Saved: Yopa Abraham - FOSSA Certificate.pdf
Saved: Augustine Alie - FOSSA Certificate.pdf
Saved: Sarah Macklin - FOSSA Certificate.pdf
Saved: Ekumbo Botuli - FOSSA Certificate.pdf
Saved: Roya Hosseini - FOSSA Certificate.pdf
All certificates have been split and saved.
\end{verbatim}

\chapter{Data Structures in Python}\label{data-structures-in-python}

\section{Intro}\label{intro}

So far in our Python explorations, we've been working with simple,
single values, like numbers and strings. But, as you know, data usually
comes in the form of larger structures. The structure most familiar to
you is a table, with rows and columns.

In this lesson, we're going to explore the building blocks for
organizing data in Python, building up through lists, dictionaries,
series, and finally tables, or, more formally,dataframes.

Let's dive in!

\section{Learning objectives}\label{learning-objectives-3}

\begin{itemize}
\tightlist
\item
  Create and work with Python lists and dictionaries
\item
  Understand and use Pandas Series
\item
  Explore Pandas DataFrames for organizing structured data
\end{itemize}

\section{Imports}\label{imports}

We need pandas for this lesson. You can import it like this:

\begin{Shaded}
\begin{Highlighting}[]
\ImportTok{import}\NormalTok{ pandas }\ImportTok{as}\NormalTok{ pd}
\end{Highlighting}
\end{Shaded}

If you get an error, you probably need to install it. You can do this by
running \texttt{!pip\ install\ pandas} in a cell.

\section{Python Lists}\label{python-lists}

Lists are like ordered containers that can hold different types of
information. For example, you might have a list of things to buy:

\begin{Shaded}
\begin{Highlighting}[]
\NormalTok{shopping }\OperatorTok{=}\NormalTok{ [}\StringTok{"apples"}\NormalTok{, }\StringTok{"bananas"}\NormalTok{, }\StringTok{"milk"}\NormalTok{, }\StringTok{"bread"}\NormalTok{] }
\NormalTok{shopping}
\end{Highlighting}
\end{Shaded}

\begin{verbatim}
['apples', 'bananas', 'milk', 'bread']
\end{verbatim}

In Python, we use something called ``zero-based indexing'' to access
items in a list. This means we start counting positions from 0, not 1.

Let's see some examples:

\begin{Shaded}
\begin{Highlighting}[]
\BuiltInTok{print}\NormalTok{(shopping[}\DecValTok{0}\NormalTok{])  }\CommentTok{\# First item (remember, we start at 0!)}
\BuiltInTok{print}\NormalTok{(shopping[}\DecValTok{1}\NormalTok{])  }\CommentTok{\# Second item}
\BuiltInTok{print}\NormalTok{(shopping[}\DecValTok{2}\NormalTok{])  }\CommentTok{\# Third item}
\end{Highlighting}
\end{Shaded}

\begin{verbatim}
apples
bananas
milk
\end{verbatim}

It might seem odd at first, but it's a common practice in many
programming languages. It has to do with how computers store
information, and the ease of writing algorithms.

We can change the contents of a list after we've created it, using the
same indexing system.

\begin{Shaded}
\begin{Highlighting}[]
\NormalTok{shopping[}\DecValTok{1}\NormalTok{] }\OperatorTok{=} \StringTok{"oranges"}  \CommentTok{\# Replace the second item (at index 1)}
\NormalTok{shopping}
\end{Highlighting}
\end{Shaded}

\begin{verbatim}
['apples', 'oranges', 'milk', 'bread']
\end{verbatim}

There are many methods accessible to lists. For example, we can add
elements to a list using the \texttt{append()} method.

\begin{Shaded}
\begin{Highlighting}[]
\NormalTok{shopping.append(}\StringTok{"eggs"}\NormalTok{)}
\NormalTok{shopping}
\end{Highlighting}
\end{Shaded}

\begin{verbatim}
['apples', 'oranges', 'milk', 'bread', 'eggs']
\end{verbatim}

In the initial stages of your Python data journey, you may not work with
lists too often, so we'll keep this intro brief.

\begin{tcolorbox}[enhanced jigsaw, colframe=quarto-callout-tip-color-frame, opacityback=0, titlerule=0mm, bottomrule=.15mm, breakable, leftrule=.75mm, colbacktitle=quarto-callout-tip-color!10!white, title=\textcolor{quarto-callout-tip-color}{\faLightbulb}\hspace{0.5em}{Practice}, rightrule=.15mm, coltitle=black, opacitybacktitle=0.6, colback=white, left=2mm, arc=.35mm, toptitle=1mm, bottomtitle=1mm, toprule=.15mm]

\subsection{Practice: Working with
Lists}\label{practice-working-with-lists}

\begin{enumerate}
\def\labelenumi{\arabic{enumi}.}
\tightlist
\item
  Create a list called \texttt{temperatures} with these values: 1,2,3,4
\item
  Print the first element of the list
\item
  Change the last element to 6
\end{enumerate}

\begin{Shaded}
\begin{Highlighting}[]
\CommentTok{\# Your code here}
\end{Highlighting}
\end{Shaded}

\end{tcolorbox}

\section{Python Dictionaries}\label{python-dictionaries}

Dictionaries are like labeled storage boxes for your data. Each piece of
data (value) has a unique label (key). Below, we have a dictionary of
grades for some students.

\begin{Shaded}
\begin{Highlighting}[]
\NormalTok{grades }\OperatorTok{=}\NormalTok{ \{}\StringTok{"Alice"}\NormalTok{: }\DecValTok{90}\NormalTok{, }\StringTok{"Bob"}\NormalTok{: }\DecValTok{85}\NormalTok{, }\StringTok{"Charlie"}\NormalTok{: }\DecValTok{92}\NormalTok{\}}
\NormalTok{grades}
\end{Highlighting}
\end{Shaded}

\begin{verbatim}
{'Alice': 90, 'Bob': 85, 'Charlie': 92}
\end{verbatim}

As you can see, dictionaries are defined using curly braces
\texttt{\{\}}, with keys and values separated by colons \texttt{:}, and
the key-value pairs are separated by commas.

We use the key to get the associated value.

\begin{Shaded}
\begin{Highlighting}[]
\NormalTok{grades[}\StringTok{"Bob"}\NormalTok{]}
\end{Highlighting}
\end{Shaded}

\begin{verbatim}
85
\end{verbatim}

\subsection{Adding/Modifying Entries}\label{addingmodifying-entries}

We can easily add new information or change existing data in a
dictionary.

\begin{Shaded}
\begin{Highlighting}[]
\NormalTok{grades[}\StringTok{"David"}\NormalTok{] }\OperatorTok{=} \DecValTok{88}  \CommentTok{\# Add a new student}
\NormalTok{grades}
\end{Highlighting}
\end{Shaded}

\begin{verbatim}
{'Alice': 90, 'Bob': 85, 'Charlie': 92, 'David': 88}
\end{verbatim}

\begin{Shaded}
\begin{Highlighting}[]
\NormalTok{grades[}\StringTok{"Alice"}\NormalTok{] }\OperatorTok{=} \DecValTok{95}  \CommentTok{\# Update Alice\textquotesingle{}s grade}
\NormalTok{grades}
\end{Highlighting}
\end{Shaded}

\begin{verbatim}
{'Alice': 95, 'Bob': 85, 'Charlie': 92, 'David': 88}
\end{verbatim}

\begin{tcolorbox}[enhanced jigsaw, colframe=quarto-callout-tip-color-frame, opacityback=0, titlerule=0mm, bottomrule=.15mm, breakable, leftrule=.75mm, colbacktitle=quarto-callout-tip-color!10!white, title=\textcolor{quarto-callout-tip-color}{\faLightbulb}\hspace{0.5em}{Practice}, rightrule=.15mm, coltitle=black, opacitybacktitle=0.6, colback=white, left=2mm, arc=.35mm, toptitle=1mm, bottomtitle=1mm, toprule=.15mm]

\subsection{Practice: Working with
Dictionaries}\label{practice-working-with-dictionaries}

\begin{enumerate}
\def\labelenumi{\arabic{enumi}.}
\tightlist
\item
  Create a dictionary called \texttt{prices} with these pairs:
  ``apple'': 0.50, ``banana'': 0.25, ``orange'': 0.75
\item
  Print the price of an orange by using the key
\item
  Add a new fruit ``grape'' with a price of 1.5
\item
  Change the price of ``banana'' to 0.30
\end{enumerate}

\begin{Shaded}
\begin{Highlighting}[]
\CommentTok{\# Your code here}
\end{Highlighting}
\end{Shaded}

\end{tcolorbox}

\section{Pandas Series}\label{pandas-series}

Pandas provides a data structure called a Series that is similar to a
list, but with additional features that are particularly useful for data
analysis.

Let's create a simple Series:

\begin{Shaded}
\begin{Highlighting}[]
\NormalTok{temps }\OperatorTok{=}\NormalTok{ pd.Series([}\DecValTok{1}\NormalTok{, }\DecValTok{2}\NormalTok{, }\DecValTok{3}\NormalTok{, }\DecValTok{4}\NormalTok{, }\DecValTok{5}\NormalTok{])}
\NormalTok{temps}
\end{Highlighting}
\end{Shaded}

\begin{verbatim}
0    1
1    2
2    3
3    4
4    5
dtype: int64
\end{verbatim}

We can use built-in Series methods to calculate summary statistics.

\begin{Shaded}
\begin{Highlighting}[]
\NormalTok{temps.mean()}
\NormalTok{temps.median()}
\NormalTok{temps.std()}
\end{Highlighting}
\end{Shaded}

\begin{verbatim}
np.float64(1.5811388300841898)
\end{verbatim}

An important feature of Series is that they can have a custom index for
intuitive access.

\begin{Shaded}
\begin{Highlighting}[]
\NormalTok{temps\_labeled }\OperatorTok{=}\NormalTok{ pd.Series([}\DecValTok{1}\NormalTok{, }\DecValTok{2}\NormalTok{, }\DecValTok{3}\NormalTok{, }\DecValTok{4}\NormalTok{], index}\OperatorTok{=}\NormalTok{[}\StringTok{\textquotesingle{}Mon\textquotesingle{}}\NormalTok{, }\StringTok{\textquotesingle{}Tue\textquotesingle{}}\NormalTok{, }\StringTok{\textquotesingle{}Wed\textquotesingle{}}\NormalTok{, }\StringTok{\textquotesingle{}Thu\textquotesingle{}}\NormalTok{])}
\NormalTok{temps\_labeled}
\NormalTok{temps\_labeled[}\StringTok{\textquotesingle{}Wed\textquotesingle{}}\NormalTok{]}
\end{Highlighting}
\end{Shaded}

\begin{verbatim}
np.int64(3)
\end{verbatim}

This makes them similar to dictionaries.

\begin{tcolorbox}[enhanced jigsaw, colframe=quarto-callout-tip-color-frame, opacityback=0, titlerule=0mm, bottomrule=.15mm, breakable, leftrule=.75mm, colbacktitle=quarto-callout-tip-color!10!white, title=\textcolor{quarto-callout-tip-color}{\faLightbulb}\hspace{0.5em}{Practice}, rightrule=.15mm, coltitle=black, opacitybacktitle=0.6, colback=white, left=2mm, arc=.35mm, toptitle=1mm, bottomtitle=1mm, toprule=.15mm]

\subsection{Practice: Working with
Series}\label{practice-working-with-series}

\begin{enumerate}
\def\labelenumi{\arabic{enumi}.}
\tightlist
\item
  Create a Series called \texttt{rainfall} with these values: 5, 2, 7,
  4, 1
\item
  Get the mean and median rainfall
\end{enumerate}

\begin{Shaded}
\begin{Highlighting}[]
\CommentTok{\# Your code here}
\end{Highlighting}
\end{Shaded}

\end{tcolorbox}

\section{Pandas DataFrames}\label{pandas-dataframes}

Next up, let's consider Pandas DataFrames, which are like Series but in
two dimensions - think spreadsheets or database tables.

This is the most important data structure for data analysis.

A DataFrame is like a spreadsheet in Python. It has rows and columns,
making it perfect for organizing structured data.

Most of the time, you will be importing external data frames, but you
should know how to data frames from scratch within Python as well.

Let's create three lists first:

\begin{Shaded}
\begin{Highlighting}[]
\CommentTok{\# Create three lists}
\NormalTok{names }\OperatorTok{=}\NormalTok{ [}\StringTok{"Alice"}\NormalTok{, }\StringTok{"Bob"}\NormalTok{, }\StringTok{"Charlie"}\NormalTok{]}
\NormalTok{ages }\OperatorTok{=}\NormalTok{ [}\DecValTok{25}\NormalTok{, }\DecValTok{30}\NormalTok{, }\DecValTok{28}\NormalTok{]}
\NormalTok{cities }\OperatorTok{=}\NormalTok{ [}\StringTok{"Lagos"}\NormalTok{, }\StringTok{"London"}\NormalTok{, }\StringTok{"Lima"}\NormalTok{]}
\end{Highlighting}
\end{Shaded}

Then we combined them into a dictionary, and finally into a dataframe.

\begin{Shaded}
\begin{Highlighting}[]
\NormalTok{data }\OperatorTok{=}\NormalTok{ \{}\StringTok{\textquotesingle{}name\textquotesingle{}}\NormalTok{: names,}
        \StringTok{\textquotesingle{}age\textquotesingle{}}\NormalTok{: ages,}
        \StringTok{\textquotesingle{}city\textquotesingle{}}\NormalTok{: cities\}}

\NormalTok{people\_df }\OperatorTok{=}\NormalTok{ pd.DataFrame(data)}
\NormalTok{people\_df}
\end{Highlighting}
\end{Shaded}

\begin{longtable}[]{@{}llll@{}}
\toprule\noalign{}
& name & age & city \\
\midrule\noalign{}
\endhead
\bottomrule\noalign{}
\endlastfoot
0 & Alice & 25 & Lagos \\
1 & Bob & 30 & London \\
2 & Charlie & 28 & Lima \\
\end{longtable}

Note that we could have created the dataframe without the intermediate
series:

\begin{Shaded}
\begin{Highlighting}[]
\NormalTok{people\_df }\OperatorTok{=}\NormalTok{ pd.DataFrame(}
\NormalTok{    \{}
        \StringTok{"name"}\NormalTok{: [}\StringTok{"Alice"}\NormalTok{, }\StringTok{"Bob"}\NormalTok{, }\StringTok{"Charlie"}\NormalTok{],}
        \StringTok{"age"}\NormalTok{: [}\DecValTok{25}\NormalTok{, }\DecValTok{30}\NormalTok{, }\DecValTok{28}\NormalTok{],}
        \StringTok{"city"}\NormalTok{: [}\StringTok{"Lagos"}\NormalTok{, }\StringTok{"London"}\NormalTok{, }\StringTok{"Lima"}\NormalTok{],}
\NormalTok{    \}}
\NormalTok{)}
\NormalTok{people\_df}
\end{Highlighting}
\end{Shaded}

\begin{longtable}[]{@{}llll@{}}
\toprule\noalign{}
& name & age & city \\
\midrule\noalign{}
\endhead
\bottomrule\noalign{}
\endlastfoot
0 & Alice & 25 & Lagos \\
1 & Bob & 30 & London \\
2 & Charlie & 28 & Lima \\
\end{longtable}

We can select specific columns or rows from our DataFrame.

\begin{Shaded}
\begin{Highlighting}[]
\NormalTok{people\_df[}\StringTok{"city"}\NormalTok{]  }\CommentTok{\# Selecting a column. Note that this returns a Series.}
\NormalTok{people\_df.loc[}\DecValTok{0}\NormalTok{]  }\CommentTok{\# Selecting a row by its label. This also returns a Series.}
\end{Highlighting}
\end{Shaded}

\begin{verbatim}
name    Alice
age        25
city    Lagos
Name: 0, dtype: object
\end{verbatim}

We can call methods on the dataframe.

\begin{Shaded}
\begin{Highlighting}[]
\NormalTok{people\_df.describe() }\CommentTok{\# This is a summary of the numerical columns}
\NormalTok{people\_df.info() }\CommentTok{\# This is a summary of the data types}
\end{Highlighting}
\end{Shaded}

\begin{verbatim}
<class 'pandas.core.frame.DataFrame'>
RangeIndex: 3 entries, 0 to 2
Data columns (total 3 columns):
 #   Column  Non-Null Count  Dtype 
---  ------  --------------  ----- 
 0   name    3 non-null      object
 1   age     3 non-null      int64 
 2   city    3 non-null      object
dtypes: int64(1), object(2)
memory usage: 204.0+ bytes
\end{verbatim}

And we can call methods on the Series objects that result from selecting
columns.

For example, we can get summary statistics on the ``city'' column.

\begin{Shaded}
\begin{Highlighting}[]
\NormalTok{people\_df[}\StringTok{"city"}\NormalTok{].describe()  }\CommentTok{\# This is a summary of the "city" column}
\NormalTok{people\_df[}\StringTok{"age"}\NormalTok{].mean()  }\CommentTok{\# This is the mean of the "age" column}
\end{Highlighting}
\end{Shaded}

\begin{verbatim}
np.float64(27.666666666666668)
\end{verbatim}

In a future series of lessons, we'll dive deeper into slicing and
manipulating DataFrames. Our goal in this lesson is just to get you
familiar with the basic syntax and concepts.

\begin{tcolorbox}[enhanced jigsaw, colframe=quarto-callout-tip-color-frame, opacityback=0, titlerule=0mm, bottomrule=.15mm, breakable, leftrule=.75mm, colbacktitle=quarto-callout-tip-color!10!white, title=\textcolor{quarto-callout-tip-color}{\faLightbulb}\hspace{0.5em}{Practice}, rightrule=.15mm, coltitle=black, opacitybacktitle=0.6, colback=white, left=2mm, arc=.35mm, toptitle=1mm, bottomtitle=1mm, toprule=.15mm]

\subsection{Practice: Working with
DataFrames}\label{practice-working-with-dataframes}

\begin{enumerate}
\def\labelenumi{\arabic{enumi}.}
\tightlist
\item
  Create a DataFrame called \texttt{students} with this information:

  \begin{itemize}
  \tightlist
  \item
    Columns: ``Name'', ``Age'', ``Grade''
  \item
    Data:

    \begin{itemize}
    \tightlist
    \item
      {[}``Alice'', 15, ``A''{]}
    \item
      {[}``Bob'', 16, ``B''{]}
    \item
      {[}``Charlie'', 15, ``A''{]}
    \end{itemize}
  \end{itemize}
\item
  Display the entire DataFrame
\item
  Show only the ``Grade'' column
\item
  Display the row for Bob
\item
  Calculate and show the average age of the students
\end{enumerate}

\begin{Shaded}
\begin{Highlighting}[]
\CommentTok{\# Your code here}
\end{Highlighting}
\end{Shaded}

\end{tcolorbox}

\section{Wrap-up}\label{wrap-up-2}

We've explored the main data structures for Python data analysis. From
basic lists and dictionaries to Pandas Series and DataFrames, these
tools are essential for organizing and analyzing data. They will be the
foundation for more advanced data work in future lessons.

\chapter{Intro to Loops in Python}\label{intro-to-loops-in-python}

\section{Introduction}\label{introduction-4}

At the heart of programming is the concept of repeating a task multiple
times. A \texttt{for} loop is one fundamental way to do that. Loops
enable efficient repetition, saving time and effort.

Mastering this concept is essential for writing intelligent Python code.

Let's dive in and enhance your coding skills!

\section{Learning Objectives}\label{learning-objectives-4}

By the end of this lesson, you will be able to:

\begin{itemize}
\tightlist
\item
  Use basic \texttt{for} loops in Python
\item
  Use index variables to iterate through lists in a loop
\item
  Format output using f-strings within loops
\item
  Apply loops to generate multiple plots for data visualization
\end{itemize}

\section{Packages}\label{packages}

In this lesson, we will use the following Python libraries:

\begin{Shaded}
\begin{Highlighting}[]
\ImportTok{import}\NormalTok{ pandas }\ImportTok{as}\NormalTok{ pd}
\ImportTok{import}\NormalTok{ plotly.express }\ImportTok{as}\NormalTok{ px}
\ImportTok{from}\NormalTok{ vega\_datasets }\ImportTok{import}\NormalTok{ data}
\end{Highlighting}
\end{Shaded}

\section{\texorpdfstring{Intro to \texttt{for}
Loops}{Intro to for Loops}}\label{intro-to-for-loops}

Let's start with a simple example. Suppose we have a list of children's
ages in years, and we want to convert these to months:

\begin{Shaded}
\begin{Highlighting}[]
\NormalTok{ages }\OperatorTok{=}\NormalTok{ [}\DecValTok{7}\NormalTok{, }\DecValTok{8}\NormalTok{, }\DecValTok{9}\NormalTok{]  }\CommentTok{\# List of ages in years}
\end{Highlighting}
\end{Shaded}

We could try to directly multiply the list by 12:

\begin{Shaded}
\begin{Highlighting}[]
\NormalTok{ages }\OperatorTok{*} \DecValTok{12}
\end{Highlighting}
\end{Shaded}

\begin{verbatim}
[7,
 8,
 9,
 7,
 8,
 9,
 7,
 8,
 9,
 7,
 8,
 9,
 7,
 8,
 9,
 7,
 8,
 9,
 7,
 8,
 9,
 7,
 8,
 9,
 7,
 8,
 9,
 7,
 8,
 9,
 7,
 8,
 9,
 7,
 8,
 9]
\end{verbatim}

But this does not do what we want. It repeats the list 12 times.

Rather, we need to loop through each element in the list and multiply it
by 12:

\begin{Shaded}
\begin{Highlighting}[]
\ControlFlowTok{for}\NormalTok{ age }\KeywordTok{in}\NormalTok{ ages:}
    \BuiltInTok{print}\NormalTok{(age }\OperatorTok{*} \DecValTok{12}\NormalTok{)}
\end{Highlighting}
\end{Shaded}

\begin{verbatim}
84
96
108
\end{verbatim}

\texttt{for} and \texttt{in} are required keywords in the loop. The
colon and the indentation on the second line are also required.

In this loop, \texttt{age} is a temporary variable that takes the value
of each element in \texttt{ages} during each iteration. First,
\texttt{age} is 7, then 8, then 9.

You can choose any name for this variable:

\begin{Shaded}
\begin{Highlighting}[]
\ControlFlowTok{for}\NormalTok{ random\_name }\KeywordTok{in}\NormalTok{ ages:}
    \BuiltInTok{print}\NormalTok{(random\_name }\OperatorTok{*} \DecValTok{12}\NormalTok{)}
\end{Highlighting}
\end{Shaded}

\begin{verbatim}
84
96
108
\end{verbatim}

Note that we need the print statement since the loop does not
automatically print the result:

\begin{Shaded}
\begin{Highlighting}[]
\ControlFlowTok{for}\NormalTok{ age }\KeywordTok{in}\NormalTok{ ages:}
\NormalTok{    age }\OperatorTok{*} \DecValTok{12}
\end{Highlighting}
\end{Shaded}

\begin{tcolorbox}[enhanced jigsaw, colframe=quarto-callout-tip-color-frame, opacityback=0, titlerule=0mm, bottomrule=.15mm, breakable, leftrule=.75mm, colbacktitle=quarto-callout-tip-color!10!white, title=\textcolor{quarto-callout-tip-color}{\faLightbulb}\hspace{0.5em}{Practice}, rightrule=.15mm, coltitle=black, opacitybacktitle=0.6, colback=white, left=2mm, arc=.35mm, toptitle=1mm, bottomtitle=1mm, toprule=.15mm]

\subsection{Hours to Minutes Basic
Loop}\label{hours-to-minutes-basic-loop}

Try converting hours to minutes using a \texttt{for} loop. Start with
this list of hours:

\begin{Shaded}
\begin{Highlighting}[]
\NormalTok{hours }\OperatorTok{=}\NormalTok{ [}\DecValTok{3}\NormalTok{, }\DecValTok{4}\NormalTok{, }\DecValTok{5}\NormalTok{]  }\CommentTok{\# List of hours}
\CommentTok{\# Your code here}
\end{Highlighting}
\end{Shaded}

\end{tcolorbox}

\section{Printing with f-strings}\label{printing-with-f-strings}

We might want to print both the result and the original age. We could do
this by concatenating strings with the \texttt{+} operator. But we need
to convert the age to a string with \texttt{str()}.

\begin{Shaded}
\begin{Highlighting}[]
\ControlFlowTok{for}\NormalTok{ age }\KeywordTok{in}\NormalTok{ ages:}
    \BuiltInTok{print}\NormalTok{(}\BuiltInTok{str}\NormalTok{(age) }\OperatorTok{+} \StringTok{" years is "} \OperatorTok{+} \BuiltInTok{str}\NormalTok{(age }\OperatorTok{*} \DecValTok{12}\NormalTok{) }\OperatorTok{+} \StringTok{" months"}\NormalTok{ )}
\end{Highlighting}
\end{Shaded}

\begin{verbatim}
7 years is 84 months
8 years is 96 months
9 years is 108 months
\end{verbatim}

Alternatively, we can use something called an f-string. This is a string
that allows us to embed variables directly.

\begin{Shaded}
\begin{Highlighting}[]
\ControlFlowTok{for}\NormalTok{ age }\KeywordTok{in}\NormalTok{ ages:}
    \BuiltInTok{print}\NormalTok{(}\SpecialStringTok{f"}\SpecialCharTok{\{}\NormalTok{age}\SpecialCharTok{\}}\SpecialStringTok{ years is }\SpecialCharTok{\{}\NormalTok{age }\OperatorTok{*} \DecValTok{12}\SpecialCharTok{\}}\SpecialStringTok{ months"}\NormalTok{)}
\end{Highlighting}
\end{Shaded}

\begin{verbatim}
7 years is 84 months
8 years is 96 months
9 years is 108 months
\end{verbatim}

Within the f-string, we use curly braces \texttt{\{\}} to embed the
variables.

\begin{tcolorbox}[enhanced jigsaw, colframe=quarto-callout-tip-color-frame, opacityback=0, titlerule=0mm, bottomrule=.15mm, breakable, leftrule=.75mm, colbacktitle=quarto-callout-tip-color!10!white, title=\textcolor{quarto-callout-tip-color}{\faLightbulb}\hspace{0.5em}{Practice}, rightrule=.15mm, coltitle=black, opacitybacktitle=0.6, colback=white, left=2mm, arc=.35mm, toptitle=1mm, bottomtitle=1mm, toprule=.15mm]

\subsection{Practice: F-String}\label{practice-f-string}

Again convert the list of hours below to minutes. Use f-strings to print
both the original hours and the converted minutes.

\begin{Shaded}
\begin{Highlighting}[]
\NormalTok{hours }\OperatorTok{=}\NormalTok{ [}\DecValTok{3}\NormalTok{, }\DecValTok{4}\NormalTok{, }\DecValTok{5}\NormalTok{]  }\CommentTok{\# List of hours}
\CommentTok{\# Your code here}
\CommentTok{\# Example output "3 hours is 180 minutes"}
\end{Highlighting}
\end{Shaded}

\end{tcolorbox}

\section{\texorpdfstring{Are \texttt{for} Loops Useful in
Python?}{Are for Loops Useful in Python?}}\label{are-for-loops-useful-in-python}

While \texttt{for} loops are useful, in many cases there are more
efficient ways to perform operations over collections of data.

For example, our initial age conversion could be achieved using pandas
Series:

\begin{Shaded}
\begin{Highlighting}[]
\ImportTok{import}\NormalTok{ pandas }\ImportTok{as}\NormalTok{ pd}

\NormalTok{ages }\OperatorTok{=}\NormalTok{ pd.Series([}\DecValTok{7}\NormalTok{, }\DecValTok{8}\NormalTok{, }\DecValTok{9}\NormalTok{])}
\NormalTok{months }\OperatorTok{=}\NormalTok{ ages }\OperatorTok{*} \DecValTok{12}
\BuiltInTok{print}\NormalTok{(months)}
\end{Highlighting}
\end{Shaded}

\begin{verbatim}
0     84
1     96
2    108
dtype: int64
\end{verbatim}

But while libraries like pandas offer powerful ways to work with data,
for loops are essential for tasks that can't be easily vectorized or
when you need fine-grained control over the iteration process.

\section{Looping with an Index and
Value}\label{looping-with-an-index-and-value}

Sometimes, we want to access both the position (index) and the value of
items in a list. The \texttt{enumerate()} function helps us do this
easily.

Let's look at our \texttt{ages} list again:

\begin{Shaded}
\begin{Highlighting}[]
\NormalTok{ages }\OperatorTok{=}\NormalTok{ [}\DecValTok{7}\NormalTok{, }\DecValTok{8}\NormalTok{, }\DecValTok{9}\NormalTok{]  }\CommentTok{\# List of ages in years}
\end{Highlighting}
\end{Shaded}

First, let's see what \texttt{enumerate()} actually does:

\begin{Shaded}
\begin{Highlighting}[]
\ControlFlowTok{for}\NormalTok{ item }\KeywordTok{in} \BuiltInTok{enumerate}\NormalTok{(ages):}
    \BuiltInTok{print}\NormalTok{(item)}
\end{Highlighting}
\end{Shaded}

\begin{verbatim}
(0, 7)
(1, 8)
(2, 9)
\end{verbatim}

As you can see, \texttt{enumerate()} gives us pairs of (index, value).

We can unpack these pairs directly in the \texttt{for} loop:

\begin{Shaded}
\begin{Highlighting}[]
\ControlFlowTok{for}\NormalTok{ i, age }\KeywordTok{in} \BuiltInTok{enumerate}\NormalTok{(ages):}
    \BuiltInTok{print}\NormalTok{(}\SpecialStringTok{f"The person at index }\SpecialCharTok{\{}\NormalTok{i}\SpecialCharTok{\}}\SpecialStringTok{ is aged }\SpecialCharTok{\{}\NormalTok{age}\SpecialCharTok{\}}\SpecialStringTok{"}\NormalTok{)}
\end{Highlighting}
\end{Shaded}

\begin{verbatim}
The person at index 0 is aged 7
The person at index 1 is aged 8
The person at index 2 is aged 9
\end{verbatim}

Here, \texttt{i} is the index, and \texttt{age} is the value at that
index.

Now, let's create a more detailed output using both the index and value:

\begin{Shaded}
\begin{Highlighting}[]
\ControlFlowTok{for}\NormalTok{ i, age }\KeywordTok{in} \BuiltInTok{enumerate}\NormalTok{(ages):}
    \BuiltInTok{print}\NormalTok{(}\SpecialStringTok{f"The person at index }\SpecialCharTok{\{}\NormalTok{i}\SpecialCharTok{\}}\SpecialStringTok{ is aged }\SpecialCharTok{\{}\NormalTok{age}\SpecialCharTok{\}}\SpecialStringTok{ years which is }\SpecialCharTok{\{}\NormalTok{age }\OperatorTok{*} \DecValTok{12}\SpecialCharTok{\}}\SpecialStringTok{ months"}\NormalTok{)}
\end{Highlighting}
\end{Shaded}

\begin{verbatim}
The person at index 0 is aged 7 years which is 84 months
The person at index 1 is aged 8 years which is 96 months
The person at index 2 is aged 9 years which is 108 months
\end{verbatim}

This is particularly useful when you need both the position and the
value in your loop.

\begin{tcolorbox}[enhanced jigsaw, colframe=quarto-callout-tip-color-frame, opacityback=0, titlerule=0mm, bottomrule=.15mm, breakable, leftrule=.75mm, colbacktitle=quarto-callout-tip-color!10!white, title=\textcolor{quarto-callout-tip-color}{\faLightbulb}\hspace{0.5em}{Practice}, rightrule=.15mm, coltitle=black, opacitybacktitle=0.6, colback=white, left=2mm, arc=.35mm, toptitle=1mm, bottomtitle=1mm, toprule=.15mm]

\subsection{Practice: Enumerate with
F-strings}\label{practice-enumerate-with-f-strings}

Use \texttt{enumerate()} and f-strings to print a sentence for each hour
in the list:

\begin{Shaded}
\begin{Highlighting}[]
\NormalTok{hours }\OperatorTok{=}\NormalTok{ [}\DecValTok{3}\NormalTok{, }\DecValTok{4}\NormalTok{, }\DecValTok{5}\NormalTok{]  }\CommentTok{\# List of hours}

\CommentTok{\# Your code here}
\CommentTok{\# Example output: "Hour 3 at index 0 is equal to 180 minutes"}
\end{Highlighting}
\end{Shaded}

\end{tcolorbox}

\chapter{Real Loops Application: Generating Multiple
Plots}\label{real-loops-application-generating-multiple-plots}

Now that you have a solid understanding of \texttt{for} loops, let's
apply our knowledge to a more realistic looping task: generating
multiple plots.

We'll use the \texttt{gapminder} dataset from Vega datasets to
demonstrate this. Our aim is to create line plots for a few selected
countries in the gapminder dataset.

First, let's load the data:

\begin{Shaded}
\begin{Highlighting}[]
\CommentTok{\# Load gapminder dataset}
\NormalTok{gapminder }\OperatorTok{=}\NormalTok{ data.gapminder()}
\NormalTok{gapminder.head()}
\end{Highlighting}
\end{Shaded}

\begin{longtable}[]{@{}lllllll@{}}
\toprule\noalign{}
& year & country & cluster & pop & life\_expect & fertility \\
\midrule\noalign{}
\endhead
\bottomrule\noalign{}
\endlastfoot
0 & 1955 & Afghanistan & 0 & 8891209 & 30.332 & 7.7 \\
1 & 1960 & Afghanistan & 0 & 9829450 & 31.997 & 7.7 \\
2 & 1965 & Afghanistan & 0 & 10997885 & 34.020 & 7.7 \\
3 & 1970 & Afghanistan & 0 & 12430623 & 36.088 & 7.7 \\
4 & 1975 & Afghanistan & 0 & 14132019 & 38.438 & 7.7 \\
\end{longtable}

Now let's create a line chart for a single country. We'll use the
\texttt{query} method to filter the data for the country ``Canada''. We
have not learned this method yet; it is a pandas function that allows us
to filter the data based on a condition. We will learn more about it
later in the course.

\begin{Shaded}
\begin{Highlighting}[]
\CommentTok{\# Filter data for China}
\NormalTok{china\_data }\OperatorTok{=}\NormalTok{ gapminder.query(}\StringTok{"country == \textquotesingle{}China\textquotesingle{}"}\NormalTok{)}

\CommentTok{\# Create line chart}
\NormalTok{fig }\OperatorTok{=}\NormalTok{ px.line(china\_data, x}\OperatorTok{=}\StringTok{"year"}\NormalTok{, y}\OperatorTok{=}\StringTok{"life\_expect"}\NormalTok{, title}\OperatorTok{=}\StringTok{"Life Expectancy in China"}\NormalTok{)}
\NormalTok{fig.show()}
\end{Highlighting}
\end{Shaded}

\begin{verbatim}
Unable to display output for mime type(s): text/html
\end{verbatim}

\begin{verbatim}
Unable to display output for mime type(s): text/html
\end{verbatim}

Now, let's create a loop to create line plots for a few selected
countries.

\begin{Shaded}
\begin{Highlighting}[]
\NormalTok{countries }\OperatorTok{=}\NormalTok{ [}\StringTok{"India"}\NormalTok{, }\StringTok{"China"}\NormalTok{, }\StringTok{"United States"}\NormalTok{, }\StringTok{"Indonesia"}\NormalTok{, }\StringTok{"Pakistan"}\NormalTok{]}

\ControlFlowTok{for}\NormalTok{ country\_name }\KeywordTok{in}\NormalTok{ countries:}
\NormalTok{    country\_data }\OperatorTok{=}\NormalTok{ gapminder.query(}\StringTok{"country == @country\_name"}\NormalTok{)}
\NormalTok{    fig }\OperatorTok{=}\NormalTok{ px.line(}
\NormalTok{        country\_data,}
\NormalTok{        x}\OperatorTok{=}\StringTok{"year"}\NormalTok{,}
\NormalTok{        y}\OperatorTok{=}\StringTok{"life\_expect"}\NormalTok{,}
\NormalTok{        title}\OperatorTok{=}\SpecialStringTok{f"Life Expectancy in }\SpecialCharTok{\{}\NormalTok{country\_name}\SpecialCharTok{\}}\SpecialStringTok{"}\NormalTok{,}
\NormalTok{    )}
\NormalTok{    fig.show()}
\end{Highlighting}
\end{Shaded}

\begin{verbatim}
Unable to display output for mime type(s): text/html
\end{verbatim}

\begin{verbatim}
Unable to display output for mime type(s): text/html
\end{verbatim}

\begin{verbatim}
Unable to display output for mime type(s): text/html
\end{verbatim}

\begin{verbatim}
Unable to display output for mime type(s): text/html
\end{verbatim}

\begin{verbatim}
Unable to display output for mime type(s): text/html
\end{verbatim}

This loop creates a separate line plot for each country in our list,
showing how life expectancy has changed over time.

\begin{tcolorbox}[enhanced jigsaw, colframe=quarto-callout-tip-color-frame, opacityback=0, titlerule=0mm, bottomrule=.15mm, breakable, leftrule=.75mm, colbacktitle=quarto-callout-tip-color!10!white, title=\textcolor{quarto-callout-tip-color}{\faLightbulb}\hspace{0.5em}{Practice}, rightrule=.15mm, coltitle=black, opacitybacktitle=0.6, colback=white, left=2mm, arc=.35mm, toptitle=1mm, bottomtitle=1mm, toprule=.15mm]

\subsection{Practice: Population Over
Time}\label{practice-population-over-time}

Using the \texttt{gapminder} dataset, create a bar chart (with
\texttt{px.bar}) of the population for the countries
\texttt{United\ States}, \texttt{Canada}, \texttt{Mexico} and
\texttt{Jamaica} over time.

\begin{Shaded}
\begin{Highlighting}[]
\CommentTok{\# Your code here}
\end{Highlighting}
\end{Shaded}

\end{tcolorbox}

\begin{tcolorbox}[enhanced jigsaw, colframe=quarto-callout-tip-color-frame, opacityback=0, titlerule=0mm, bottomrule=.15mm, breakable, leftrule=.75mm, colbacktitle=quarto-callout-tip-color!10!white, title=\textcolor{quarto-callout-tip-color}{\faLightbulb}\hspace{0.5em}{Practice}, rightrule=.15mm, coltitle=black, opacitybacktitle=0.6, colback=white, left=2mm, arc=.35mm, toptitle=1mm, bottomtitle=1mm, toprule=.15mm]

\subsection{Practice: Tips Histogram by
Day}\label{practice-tips-histogram-by-day}

Using the \texttt{tips} dataset, create a histogram of the total bill
for each day of the week.

\begin{Shaded}
\begin{Highlighting}[]
\CommentTok{\# Load tips dataset}
\NormalTok{tips }\OperatorTok{=}\NormalTok{ px.data.tips()}
\NormalTok{tips.head()}
\end{Highlighting}
\end{Shaded}

\begin{longtable}[]{@{}llllllll@{}}
\toprule\noalign{}
& total\_bill & tip & sex & smoker & day & time & size \\
\midrule\noalign{}
\endhead
\bottomrule\noalign{}
\endlastfoot
0 & 16.99 & 1.01 & Female & No & Sun & Dinner & 2 \\
1 & 10.34 & 1.66 & Male & No & Sun & Dinner & 3 \\
2 & 21.01 & 3.50 & Male & No & Sun & Dinner & 3 \\
3 & 23.68 & 3.31 & Male & No & Sun & Dinner & 2 \\
4 & 24.59 & 3.61 & Female & No & Sun & Dinner & 4 \\
\end{longtable}

\begin{Shaded}
\begin{Highlighting}[]
\CommentTok{\# List of days}
\NormalTok{days }\OperatorTok{=}\NormalTok{ [}\StringTok{"Thur"}\NormalTok{, }\StringTok{"Fri"}\NormalTok{, }\StringTok{"Sat"}\NormalTok{, }\StringTok{"Sun"}\NormalTok{]}

\CommentTok{\# Your loop here}
\end{Highlighting}
\end{Shaded}

\end{tcolorbox}

\chapter{Wrap Up!}\label{wrap-up-3}

We've covered the very basics of \texttt{for} loops in Python, from
simple syntax to practical data analysis applications. Loops are
essential for efficient coding, allowing you to automate repetitive
tasks. As we progress in the course, we will encounter many other
applications of this key programming construct.

\part{Data Visualization}

\chapter{Data Visualization Types}\label{data-visualization-types}

\begin{Shaded}
\begin{Highlighting}[]
\ImportTok{import}\NormalTok{ plotly.express }\ImportTok{as}\NormalTok{ px}
\ImportTok{import}\NormalTok{ pandas }\ImportTok{as}\NormalTok{ pd}
\ImportTok{import}\NormalTok{ numpy }\ImportTok{as}\NormalTok{ np}
\ImportTok{import}\NormalTok{ plotly.io }\ImportTok{as}\NormalTok{ pio}

\CommentTok{\# Create a custom template}
\NormalTok{custom\_template }\OperatorTok{=}\NormalTok{ pio.templates[}\StringTok{"plotly\_white"}\NormalTok{]}
\NormalTok{custom\_template.layout.update(}
\NormalTok{    font}\OperatorTok{=}\BuiltInTok{dict}\NormalTok{(size}\OperatorTok{=}\DecValTok{40}\NormalTok{),  }\CommentTok{\# Increase the default font size}
\NormalTok{    title}\OperatorTok{=}\BuiltInTok{dict}\NormalTok{(font}\OperatorTok{=}\BuiltInTok{dict}\NormalTok{(size}\OperatorTok{=}\DecValTok{45}\NormalTok{)),  }\CommentTok{\# Increase title font size}
\NormalTok{    xaxis}\OperatorTok{=}\BuiltInTok{dict}\NormalTok{(title}\OperatorTok{=}\BuiltInTok{dict}\NormalTok{(font}\OperatorTok{=}\BuiltInTok{dict}\NormalTok{(size}\OperatorTok{=}\DecValTok{40}\NormalTok{)), tickfont}\OperatorTok{=}\BuiltInTok{dict}\NormalTok{(size}\OperatorTok{=}\DecValTok{30}\NormalTok{)),  }\CommentTok{\# Increase x{-}axis title font size}
\NormalTok{    yaxis}\OperatorTok{=}\BuiltInTok{dict}\NormalTok{(title}\OperatorTok{=}\BuiltInTok{dict}\NormalTok{(font}\OperatorTok{=}\BuiltInTok{dict}\NormalTok{(size}\OperatorTok{=}\DecValTok{40}\NormalTok{)), tickfont}\OperatorTok{=}\BuiltInTok{dict}\NormalTok{(size}\OperatorTok{=}\DecValTok{30}\NormalTok{)),  }\CommentTok{\# Increase y{-}axis title font size}
\NormalTok{    colorway}\OperatorTok{=}\NormalTok{[}\StringTok{"\#2f828a"}\NormalTok{]  }\CommentTok{\# Set default color}
\NormalTok{)}

\CommentTok{\# Set the custom template as default}
\NormalTok{pio.templates.default }\OperatorTok{=}\NormalTok{ custom\_template}

\NormalTok{tips }\OperatorTok{=}\NormalTok{ px.data.tips()}
\end{Highlighting}
\end{Shaded}

\chapter{Univariate Graphs}\label{univariate-graphs}

\section{Numeric}\label{numeric}

\begin{Shaded}
\begin{Highlighting}[]
\NormalTok{px.histogram(tips, x}\OperatorTok{=}\StringTok{\textquotesingle{}tip\textquotesingle{}}\NormalTok{)}
\end{Highlighting}
\end{Shaded}

\begin{verbatim}
Unable to display output for mime type(s): text/html
\end{verbatim}

\begin{verbatim}
Unable to display output for mime type(s): text/html
\end{verbatim}

\begin{Shaded}
\begin{Highlighting}[]
\NormalTok{px.box(tips, x}\OperatorTok{=}\StringTok{\textquotesingle{}tip\textquotesingle{}}\NormalTok{)}
\end{Highlighting}
\end{Shaded}

\begin{verbatim}
Unable to display output for mime type(s): text/html
\end{verbatim}

\begin{Shaded}
\begin{Highlighting}[]
\NormalTok{px.violin(tips, x}\OperatorTok{=}\StringTok{\textquotesingle{}tip\textquotesingle{}}\NormalTok{, box}\OperatorTok{=}\VariableTok{True}\NormalTok{, points}\OperatorTok{=}\StringTok{"all"}\NormalTok{)}

\NormalTok{?px.violin}
\end{Highlighting}
\end{Shaded}

\chapter{Categorical}\label{categorical}

\begin{Shaded}
\begin{Highlighting}[]
\NormalTok{sex\_categ }\OperatorTok{=}\NormalTok{ px.histogram(tips, x}\OperatorTok{=}\StringTok{\textquotesingle{}sex\textquotesingle{}}\NormalTok{, color}\OperatorTok{=}\StringTok{\textquotesingle{}sex\textquotesingle{}}\NormalTok{, color\_discrete\_sequence}\OperatorTok{=}\NormalTok{ [}\StringTok{\textquotesingle{}\#deb221\textquotesingle{}}\NormalTok{, }\StringTok{\textquotesingle{}\#2f828a\textquotesingle{}}\NormalTok{])}
\CommentTok{\# remove the legend}
\NormalTok{sex\_categ.update\_layout(showlegend}\OperatorTok{=}\VariableTok{False}\NormalTok{)}
\end{Highlighting}
\end{Shaded}

\begin{verbatim}
Unable to display output for mime type(s): text/html
\end{verbatim}

\begin{Shaded}
\begin{Highlighting}[]
\CommentTok{\# pie}
\NormalTok{pie\_categ }\OperatorTok{=}\NormalTok{ px.pie(tips, values}\OperatorTok{=}\StringTok{\textquotesingle{}tip\textquotesingle{}}\NormalTok{, names}\OperatorTok{=}\StringTok{\textquotesingle{}sex\textquotesingle{}}\NormalTok{, color}\OperatorTok{=}\StringTok{\textquotesingle{}sex\textquotesingle{}}\NormalTok{, color\_discrete\_sequence}\OperatorTok{=}\NormalTok{[}\StringTok{\textquotesingle{}\#deb221\textquotesingle{}}\NormalTok{, }\StringTok{\textquotesingle{}\#2f828a\textquotesingle{}}\NormalTok{])}
\NormalTok{pie\_categ.update\_layout(showlegend}\OperatorTok{=}\VariableTok{False}\NormalTok{)}
\NormalTok{pie\_categ.update\_traces(textposition}\OperatorTok{=}\StringTok{\textquotesingle{}none\textquotesingle{}}\NormalTok{)}
\end{Highlighting}
\end{Shaded}

\begin{verbatim}
Unable to display output for mime type(s): text/html
\end{verbatim}

\chapter{Bivariate Graphs}\label{bivariate-graphs}

\section{Numeric vs Numeric}\label{numeric-vs-numeric}

\begin{Shaded}
\begin{Highlighting}[]
\NormalTok{px.scatter(tips, x}\OperatorTok{=}\StringTok{\textquotesingle{}total\_bill\textquotesingle{}}\NormalTok{, y}\OperatorTok{=}\StringTok{\textquotesingle{}tip\textquotesingle{}}\NormalTok{)}
\end{Highlighting}
\end{Shaded}

\begin{verbatim}
Unable to display output for mime type(s): text/html
\end{verbatim}

\chapter{Numeric vs Categorical}\label{numeric-vs-categorical}

\begin{Shaded}
\begin{Highlighting}[]
\CommentTok{\# grouped histogram}
\NormalTok{px.histogram(tips, x}\OperatorTok{=}\StringTok{\textquotesingle{}tip\textquotesingle{}}\NormalTok{, color}\OperatorTok{=}\StringTok{\textquotesingle{}sex\textquotesingle{}}\NormalTok{, barmode}\OperatorTok{=}\StringTok{\textquotesingle{}overlay\textquotesingle{}}\NormalTok{, color\_discrete\_sequence}\OperatorTok{=}\NormalTok{ [}\StringTok{\textquotesingle{}\#deb221\textquotesingle{}}\NormalTok{, }\StringTok{\textquotesingle{}\#2f828a\textquotesingle{}}\NormalTok{])}
\end{Highlighting}
\end{Shaded}

\begin{verbatim}
Unable to display output for mime type(s): text/html
\end{verbatim}

\begin{Shaded}
\begin{Highlighting}[]
\CommentTok{\# grouped violin plot}
\NormalTok{grouped\_violin }\OperatorTok{=}\NormalTok{ px.violin(tips, y}\OperatorTok{=}\StringTok{\textquotesingle{}sex\textquotesingle{}}\NormalTok{, x}\OperatorTok{=}\StringTok{\textquotesingle{}tip\textquotesingle{}}\NormalTok{, color}\OperatorTok{=}\StringTok{\textquotesingle{}sex\textquotesingle{}}\NormalTok{, box}\OperatorTok{=}\VariableTok{True}\NormalTok{, points}\OperatorTok{=}\StringTok{"all"}\NormalTok{, color\_discrete\_sequence}\OperatorTok{=}\NormalTok{ [}\StringTok{\textquotesingle{}\#deb221\textquotesingle{}}\NormalTok{, }\StringTok{\textquotesingle{}\#2f828a\textquotesingle{}}\NormalTok{])}
\NormalTok{grouped\_violin.update\_layout(showlegend}\OperatorTok{=}\VariableTok{False}\NormalTok{)}
\end{Highlighting}
\end{Shaded}

\begin{verbatim}
Unable to display output for mime type(s): text/html
\end{verbatim}

\begin{Shaded}
\begin{Highlighting}[]
\CommentTok{\# summary plot (e.g. mean + std tip by sex, bar plot)}
\CommentTok{\# first calculate the mean and std in a single data frame with assign }
\NormalTok{summary\_df }\OperatorTok{=}\NormalTok{ tips.groupby(}\StringTok{\textquotesingle{}sex\textquotesingle{}}\NormalTok{).agg(\{}\StringTok{\textquotesingle{}tip\textquotesingle{}}\NormalTok{: [}\StringTok{\textquotesingle{}mean\textquotesingle{}}\NormalTok{, }\StringTok{\textquotesingle{}std\textquotesingle{}}\NormalTok{]\}).reset\_index()}
\NormalTok{summary\_df.columns }\OperatorTok{=}\NormalTok{ [}\StringTok{\textquotesingle{}sex\textquotesingle{}}\NormalTok{, }\StringTok{\textquotesingle{}mean\_tip\textquotesingle{}}\NormalTok{, }\StringTok{\textquotesingle{}std\_tip\textquotesingle{}}\NormalTok{]}

\NormalTok{sex\_bar }\OperatorTok{=}\NormalTok{   px.bar(summary\_df, y}\OperatorTok{=}\StringTok{\textquotesingle{}sex\textquotesingle{}}\NormalTok{, x}\OperatorTok{=}\StringTok{\textquotesingle{}mean\_tip\textquotesingle{}}\NormalTok{, error\_x}\OperatorTok{=}\StringTok{\textquotesingle{}std\_tip\textquotesingle{}}\NormalTok{, color}\OperatorTok{=}\StringTok{\textquotesingle{}sex\textquotesingle{}}\NormalTok{, color\_discrete\_sequence}\OperatorTok{=}\NormalTok{[}\StringTok{\textquotesingle{}\#deb221\textquotesingle{}}\NormalTok{, }\StringTok{\textquotesingle{}\#2f828a\textquotesingle{}}\NormalTok{])}
\NormalTok{sex\_bar.update\_layout(showlegend}\OperatorTok{=}\VariableTok{False}\NormalTok{)}
\end{Highlighting}
\end{Shaded}

\begin{verbatim}
Unable to display output for mime type(s): text/html
\end{verbatim}

\chapter{Categorical vs Categorical}\label{categorical-vs-categorical}

\begin{Shaded}
\begin{Highlighting}[]
\CommentTok{\# grouped bar plot}
\NormalTok{categ\_categ\_grouped\_bar }\OperatorTok{=}\NormalTok{ px.histogram(tips, x}\OperatorTok{=}\StringTok{\textquotesingle{}day\textquotesingle{}}\NormalTok{, color}\OperatorTok{=}\StringTok{\textquotesingle{}sex\textquotesingle{}}\NormalTok{, barmode}\OperatorTok{=}\StringTok{\textquotesingle{}group\textquotesingle{}}\NormalTok{, color\_discrete\_sequence}\OperatorTok{=}\NormalTok{ [}\StringTok{\textquotesingle{}\#deb221\textquotesingle{}}\NormalTok{, }\StringTok{\textquotesingle{}\#2f828a\textquotesingle{}}\NormalTok{])}
\NormalTok{categ\_categ\_grouped\_bar.update\_layout(showlegend}\OperatorTok{=}\VariableTok{False}\NormalTok{)}
\end{Highlighting}
\end{Shaded}

\begin{verbatim}
Unable to display output for mime type(s): text/html
\end{verbatim}

\begin{Shaded}
\begin{Highlighting}[]
\CommentTok{\# stacked}
\NormalTok{categ\_categ\_stacked\_bar }\OperatorTok{=}\NormalTok{ px.histogram(tips, x}\OperatorTok{=}\StringTok{\textquotesingle{}day\textquotesingle{}}\NormalTok{, color}\OperatorTok{=}\StringTok{\textquotesingle{}sex\textquotesingle{}}\NormalTok{, color\_discrete\_sequence}\OperatorTok{=}\NormalTok{ [}\StringTok{\textquotesingle{}\#deb221\textquotesingle{}}\NormalTok{, }\StringTok{\textquotesingle{}\#2f828a\textquotesingle{}}\NormalTok{])}
\NormalTok{categ\_categ\_stacked\_bar.update\_layout(showlegend}\OperatorTok{=}\VariableTok{False}\NormalTok{)}
\end{Highlighting}
\end{Shaded}

\begin{verbatim}
Unable to display output for mime type(s): text/html
\end{verbatim}

\begin{Shaded}
\begin{Highlighting}[]
\CommentTok{\# percent stacked}
\NormalTok{percent\_stacked\_df }\OperatorTok{=}\NormalTok{ (}
\NormalTok{    tips.groupby([}\StringTok{"sex"}\NormalTok{, }\StringTok{"day"}\NormalTok{])}
\NormalTok{    .size()}
\NormalTok{    .reset\_index(name}\OperatorTok{=}\StringTok{\textquotesingle{}count\textquotesingle{}}\NormalTok{)}
\NormalTok{    .assign(percent}\OperatorTok{=}\KeywordTok{lambda}\NormalTok{ x: x.groupby(}\StringTok{\textquotesingle{}day\textquotesingle{}}\NormalTok{)[}\StringTok{\textquotesingle{}count\textquotesingle{}}\NormalTok{].transform(}\KeywordTok{lambda}\NormalTok{ y: y }\OperatorTok{/}\NormalTok{ y.}\BuiltInTok{sum}\NormalTok{() }\OperatorTok{*} \DecValTok{100}\NormalTok{))}
\NormalTok{)}

\NormalTok{categ\_categ\_percent\_stacked\_bar }\OperatorTok{=}\NormalTok{ px.bar(}
\NormalTok{    percent\_stacked\_df,}
\NormalTok{    x}\OperatorTok{=}\StringTok{"day"}\NormalTok{,}
\NormalTok{    y}\OperatorTok{=}\StringTok{"percent"}\NormalTok{,}
\NormalTok{    color}\OperatorTok{=}\StringTok{"sex"}\NormalTok{,}
\NormalTok{    barmode}\OperatorTok{=}\StringTok{"relative"}\NormalTok{,}
\NormalTok{    color\_discrete\_sequence}\OperatorTok{=}\NormalTok{[}\StringTok{"\#deb221"}\NormalTok{, }\StringTok{"\#2f828a"}\NormalTok{],}
\NormalTok{)}
\NormalTok{categ\_categ\_percent\_stacked\_bar.update\_layout(showlegend}\OperatorTok{=}\VariableTok{False}\NormalTok{)}
\end{Highlighting}
\end{Shaded}

\begin{verbatim}
Unable to display output for mime type(s): text/html
\end{verbatim}

\begin{Shaded}
\begin{Highlighting}[]
\NormalTok{px.histogram(tips, x}\OperatorTok{=}\StringTok{\textquotesingle{}day\textquotesingle{}}\NormalTok{, color}\OperatorTok{=}\StringTok{\textquotesingle{}sex\textquotesingle{}}\NormalTok{, barmode}\OperatorTok{=}\StringTok{\textquotesingle{}stack\textquotesingle{}}\NormalTok{, barnorm}\OperatorTok{=}\StringTok{\textquotesingle{}percent\textquotesingle{}}\NormalTok{, color\_discrete\_sequence}\OperatorTok{=}\NormalTok{ [}\StringTok{\textquotesingle{}\#deb221\textquotesingle{}}\NormalTok{, }\StringTok{\textquotesingle{}\#2f828a\textquotesingle{}}\NormalTok{])}
\end{Highlighting}
\end{Shaded}

\begin{verbatim}
Unable to display output for mime type(s): text/html
\end{verbatim}

\chapter{Practice}\label{practice-1}

\begin{Shaded}
\begin{Highlighting}[]
\NormalTok{gap\_dat }\OperatorTok{=}\NormalTok{ px.data.gapminder()}

\NormalTok{gap\_2007 }\OperatorTok{=}\NormalTok{ (gap\_dat}
\NormalTok{    .query(}\StringTok{\textquotesingle{}year == 2007\textquotesingle{}}\NormalTok{)}
\NormalTok{    .drop(columns}\OperatorTok{=}\NormalTok{[}\StringTok{\textquotesingle{}year\textquotesingle{}}\NormalTok{, }\StringTok{\textquotesingle{}iso\_alpha\textquotesingle{}}\NormalTok{, }\StringTok{\textquotesingle{}iso\_num\textquotesingle{}}\NormalTok{])}
\NormalTok{    .assign(income\_group}\OperatorTok{=}\KeywordTok{lambda}\NormalTok{ df: np.where(df.gdpPercap }\OperatorTok{\textgreater{}} \DecValTok{15000}\NormalTok{, }\StringTok{\textquotesingle{}High Income\textquotesingle{}}\NormalTok{, }\StringTok{\textquotesingle{}Low \& Middle Income\textquotesingle{}}\NormalTok{))}
\NormalTok{)}
\end{Highlighting}
\end{Shaded}

\begin{enumerate}
\def\labelenumi{\arabic{enumi}.}
\tightlist
\item
  How does country GDP per capita vary across continents?
\end{enumerate}

\begin{Shaded}
\begin{Highlighting}[]
\NormalTok{gdp\_per\_cap\_violin }\OperatorTok{=}\NormalTok{ px.violin(}
\NormalTok{    gap\_2007,}
\NormalTok{    x}\OperatorTok{=}\StringTok{"gdpPercap"}\NormalTok{,}
\NormalTok{    y}\OperatorTok{=}\StringTok{"continent"}\NormalTok{,}
\NormalTok{    color}\OperatorTok{=}\StringTok{"continent"}\NormalTok{,}
\NormalTok{    box}\OperatorTok{=}\VariableTok{True}\NormalTok{,}
\NormalTok{    points}\OperatorTok{=}\StringTok{"all"}\NormalTok{,}
\NormalTok{    color\_discrete\_sequence}\OperatorTok{=}\NormalTok{px.colors.qualitative.G10,}
\NormalTok{)}

\NormalTok{gdp\_per\_cap\_violin.update\_layout(showlegend}\OperatorTok{=}\VariableTok{False}\NormalTok{)}
\end{Highlighting}
\end{Shaded}

\begin{verbatim}
Unable to display output for mime type(s): text/html
\end{verbatim}

\begin{enumerate}
\def\labelenumi{\arabic{enumi}.}
\setcounter{enumi}{1}
\tightlist
\item
  Is there a relationship between GDP per capita \& life expectancy?
\end{enumerate}

\begin{Shaded}
\begin{Highlighting}[]
\NormalTok{px.scatter(gap\_2007, x}\OperatorTok{=}\StringTok{\textquotesingle{}gdpPercap\textquotesingle{}}\NormalTok{, y}\OperatorTok{=}\StringTok{\textquotesingle{}lifeExp\textquotesingle{}}\NormalTok{)}
\end{Highlighting}
\end{Shaded}

\begin{verbatim}
Unable to display output for mime type(s): text/html
\end{verbatim}

\begin{enumerate}
\def\labelenumi{\arabic{enumi}.}
\setcounter{enumi}{2}
\tightlist
\item
  How does life expectancy vary between the income groups?
\end{enumerate}

\begin{Shaded}
\begin{Highlighting}[]
\NormalTok{px.strip(gap\_2007, x}\OperatorTok{=}\StringTok{"income\_group"}\NormalTok{, y}\OperatorTok{=}\StringTok{"lifeExp"}\NormalTok{)}
\NormalTok{px.violin(gap\_2007, x}\OperatorTok{=}\StringTok{"income\_group"}\NormalTok{, y}\OperatorTok{=}\StringTok{"lifeExp"}\NormalTok{, box}\OperatorTok{=}\VariableTok{True}\NormalTok{, points}\OperatorTok{=}\StringTok{"all"}\NormalTok{)}
\end{Highlighting}
\end{Shaded}

\begin{verbatim}
Unable to display output for mime type(s): text/html
\end{verbatim}

\begin{enumerate}
\def\labelenumi{\arabic{enumi}.}
\setcounter{enumi}{3}
\tightlist
\item
  What is the relationship between continent \& income group?
\end{enumerate}

\begin{Shaded}
\begin{Highlighting}[]
\NormalTok{income\_group\_continent\_bar }\OperatorTok{=}\NormalTok{ px.histogram(gap\_2007, x}\OperatorTok{=}\StringTok{\textquotesingle{}continent\textquotesingle{}}\NormalTok{, color}\OperatorTok{=}\StringTok{\textquotesingle{}income\_group\textquotesingle{}}\NormalTok{, barmode}\OperatorTok{=}\StringTok{\textquotesingle{}stack\textquotesingle{}}\NormalTok{, color\_discrete\_sequence}\OperatorTok{=}\NormalTok{ [}\StringTok{\textquotesingle{}\#deb221\textquotesingle{}}\NormalTok{, }\StringTok{\textquotesingle{}\#2f828a\textquotesingle{}}\NormalTok{])}
\NormalTok{income\_group\_continent\_bar.update\_layout(showlegend}\OperatorTok{=}\VariableTok{False}\NormalTok{)}
\end{Highlighting}
\end{Shaded}

\begin{verbatim}
Unable to display output for mime type(s): text/html
\end{verbatim}

\chapter{Time series}\label{time-series}

Nigeria population over time

\begin{Shaded}
\begin{Highlighting}[]
\CommentTok{\# as a bar chart}
\NormalTok{nigeria\_pop }\OperatorTok{=}\NormalTok{ gap\_dat.query(}\StringTok{\textquotesingle{}country == "Nigeria"\textquotesingle{}}\NormalTok{)}


\CommentTok{\# Bar chart}
\NormalTok{px.bar(nigeria\_pop, x}\OperatorTok{=}\StringTok{\textquotesingle{}year\textquotesingle{}}\NormalTok{, y}\OperatorTok{=}\StringTok{\textquotesingle{}pop\textquotesingle{}}\NormalTok{)}

\CommentTok{\# Line chart}
\NormalTok{px.line(nigeria\_pop, x}\OperatorTok{=}\StringTok{\textquotesingle{}year\textquotesingle{}}\NormalTok{, y}\OperatorTok{=}\StringTok{\textquotesingle{}pop\textquotesingle{}}\NormalTok{)}

\CommentTok{\# Line chart with points}
\NormalTok{px.line(nigeria\_pop, x}\OperatorTok{=}\StringTok{\textquotesingle{}year\textquotesingle{}}\NormalTok{, y}\OperatorTok{=}\StringTok{\textquotesingle{}pop\textquotesingle{}}\NormalTok{, markers}\OperatorTok{=}\VariableTok{True}\NormalTok{)}
\end{Highlighting}
\end{Shaded}

\begin{verbatim}
Unable to display output for mime type(s): text/html
\end{verbatim}

\chapter{Univariate Graphs with Plotly
Express}\label{univariate-graphs-with-plotly-express}

\section{Intro}\label{intro-1}

In this lesson, you'll learn how to create univariate graphs using
Plotly Express. Univariate graphs are essential for understanding the
distribution of a single variable, whether it's categorical or
quantitative.

Let's get started!

\section{Learning objectives}\label{learning-objectives-5}

\begin{itemize}
\tightlist
\item
  Create bar charts, pie charts, and treemaps for categorical data using
  Plotly Express
\item
  Generate histograms for quantitative data using Plotly Express
\item
  Customize graph appearance and labels
\end{itemize}

\section{Imports}\label{imports-1}

This lesson requires plotly.express, pandas, and vega\_datasets. Install
them if you haven't already.

\begin{Shaded}
\begin{Highlighting}[]
\ImportTok{import}\NormalTok{ plotly.express }\ImportTok{as}\NormalTok{ px}
\ImportTok{import}\NormalTok{ pandas }\ImportTok{as}\NormalTok{ pd}
\ImportTok{from}\NormalTok{ vega\_datasets }\ImportTok{import}\NormalTok{ data}
\end{Highlighting}
\end{Shaded}

\section{Quantitative Data}\label{quantitative-data}

\subsection{Histogram}\label{histogram}

Histograms are used to visualize the distribution of continuous
variables.

Let's make a histogram of the tip amounts in the tips dataset.

\begin{Shaded}
\begin{Highlighting}[]
\NormalTok{tips }\OperatorTok{=}\NormalTok{ px.data.tips()}
\NormalTok{tips.head() }\CommentTok{\# view the first 5 rows}
\end{Highlighting}
\end{Shaded}

\begin{longtable}[]{@{}llllllll@{}}
\toprule\noalign{}
& total\_bill & tip & sex & smoker & day & time & size \\
\midrule\noalign{}
\endhead
\bottomrule\noalign{}
\endlastfoot
0 & 16.99 & 1.01 & Female & No & Sun & Dinner & 2 \\
1 & 10.34 & 1.66 & Male & No & Sun & Dinner & 3 \\
2 & 21.01 & 3.50 & Male & No & Sun & Dinner & 3 \\
3 & 23.68 & 3.31 & Male & No & Sun & Dinner & 2 \\
4 & 24.59 & 3.61 & Female & No & Sun & Dinner & 4 \\
\end{longtable}

\begin{Shaded}
\begin{Highlighting}[]
\NormalTok{px.histogram(tips, x}\OperatorTok{=}\StringTok{\textquotesingle{}tip\textquotesingle{}}\NormalTok{)}
\end{Highlighting}
\end{Shaded}

\begin{verbatim}
Unable to display output for mime type(s): text/html
\end{verbatim}

\begin{verbatim}
Unable to display output for mime type(s): text/html
\end{verbatim}

We can see that the highest bar, corresponding to tips between 1.75 and
2.24, has a frequency of 55. This means that there were 55 tips between
1.75 and 2.24.

\begin{tcolorbox}[enhanced jigsaw, colframe=quarto-callout-note-color-frame, opacityback=0, titlerule=0mm, bottomrule=.15mm, breakable, leftrule=.75mm, colbacktitle=quarto-callout-note-color!10!white, title=\textcolor{quarto-callout-note-color}{\faInfo}\hspace{0.5em}{Side-note}, rightrule=.15mm, coltitle=black, opacitybacktitle=0.6, colback=white, left=2mm, arc=.35mm, toptitle=1mm, bottomtitle=1mm, toprule=.15mm]

Notice that plotly charts are interactive. You can hover over the bars
to see the exact number of tips in each bin.

Try playing with the buttons at the top right. The button to download
the chart as a png is especially useful.

\end{tcolorbox}

\begin{tcolorbox}[enhanced jigsaw, colframe=quarto-callout-tip-color-frame, opacityback=0, titlerule=0mm, bottomrule=.15mm, breakable, leftrule=.75mm, colbacktitle=quarto-callout-tip-color!10!white, title=\textcolor{quarto-callout-tip-color}{\faLightbulb}\hspace{0.5em}{Practice}, rightrule=.15mm, coltitle=black, opacitybacktitle=0.6, colback=white, left=2mm, arc=.35mm, toptitle=1mm, bottomtitle=1mm, toprule=.15mm]

\subsection{Practice: Speed Distribution
Histogram}\label{practice-speed-distribution-histogram}

Following the example of the histogram of tips, create a histogram of
the speed distribution (Speed\_IAS\_in\_knots) using the birdstrikes
dataset.

\begin{Shaded}
\begin{Highlighting}[]
\NormalTok{birdstrikes }\OperatorTok{=}\NormalTok{ data.birdstrikes()}
\NormalTok{birdstrikes.head()}
\CommentTok{\# Your code here}
\end{Highlighting}
\end{Shaded}

\begin{longtable}[]{@{}lllllllllllllll@{}}
\toprule\noalign{}
& Airport\_\_Name & Aircraft\_\_Make\_Model &
Effect\_\_Amount\_of\_damage & Flight\_Date &
Aircraft\_\_Airline\_Operator & Origin\_State &
When\_\_Phase\_of\_flight & Wildlife\_\_Size & Wildlife\_\_Species &
When\_\_Time\_of\_day & Cost\_\_Other & Cost\_\_Repair &
Cost\_\_Total\_\$ & Speed\_IAS\_in\_knots \\
\midrule\noalign{}
\endhead
\bottomrule\noalign{}
\endlastfoot
0 & BARKSDALE AIR FORCE BASE ARPT & T-38A & None & 1/8/90 0:00 &
MILITARY & Louisiana & Climb & Large & Turkey vulture & Day & 0 & 0 & 0
& 300.0 \\
1 & BARKSDALE AIR FORCE BASE ARPT & KC-10A & None & 1/9/90 0:00 &
MILITARY & Louisiana & Approach & Medium & Unknown bird or bat & Night &
0 & 0 & 0 & 200.0 \\
2 & BARKSDALE AIR FORCE BASE ARPT & B-52 & None & 1/11/90 0:00 &
MILITARY & Louisiana & Take-off run & Medium & Unknown bird or bat & Day
& 0 & 0 & 0 & 130.0 \\
3 & NEW ORLEANS INTL & B-737-300 & Substantial & 1/11/90 0:00 &
SOUTHWEST AIRLINES & Louisiana & Take-off run & Small & Rock pigeon &
Day & 0 & 0 & 0 & 140.0 \\
4 & BARKSDALE AIR FORCE BASE ARPT & KC-10A & None & 1/12/90 0:00 &
MILITARY & Louisiana & Climb & Medium & Unknown bird or bat & Day & 0 &
0 & 0 & 160.0 \\
\end{longtable}

\end{tcolorbox}

We can view the help documentation for the function by typing
\texttt{px.histogram?} in a cell and running it.

\begin{Shaded}
\begin{Highlighting}[]
\NormalTok{px.histogram?}
\end{Highlighting}
\end{Shaded}

From the help documentation, we can see that the \texttt{px.histogram}
function has many arguments that we can use to customize the graph.

Let's make the histogram a bit nicer by adding a title, customizing the
x axis label, and changing the color.

\begin{Shaded}
\begin{Highlighting}[]
\NormalTok{px.histogram(}
\NormalTok{    tips,}
\NormalTok{    x}\OperatorTok{=}\StringTok{"tip"}\NormalTok{,}
\NormalTok{    labels}\OperatorTok{=}\NormalTok{\{}\StringTok{"tip"}\NormalTok{: }\StringTok{"Tip Amount ($)"}\NormalTok{\},}
\NormalTok{    title}\OperatorTok{=}\StringTok{"Distribution of Tips"}\NormalTok{, }
\NormalTok{    color\_discrete\_sequence}\OperatorTok{=}\NormalTok{[}\StringTok{"lightseagreen"}\NormalTok{]}
\NormalTok{)}
\end{Highlighting}
\end{Shaded}

\begin{verbatim}
Unable to display output for mime type(s): text/html
\end{verbatim}

Color names are based on standard CSS color naming from Mozilla. You can
see the full list
\href{https://developer.mozilla.org/en-US/docs/Web/CSS/named-color}{here}.

Alternatively, you can use hex color codes, like \texttt{\#1f77b4}. You
can get these easily by using a color picker. Search for ``color
picker'' on Google.

\begin{Shaded}
\begin{Highlighting}[]
\NormalTok{px.histogram(}
\NormalTok{    tips,}
\NormalTok{    x}\OperatorTok{=}\StringTok{"tip"}\NormalTok{,}
\NormalTok{    labels}\OperatorTok{=}\NormalTok{\{}\StringTok{"tip"}\NormalTok{: }\StringTok{"Tip Amount ($)"}\NormalTok{\},}
\NormalTok{    title}\OperatorTok{=}\StringTok{"Distribution of Tips"}\NormalTok{, }
\NormalTok{    color\_discrete\_sequence}\OperatorTok{=}\NormalTok{[}\StringTok{"\#6a5acd"}\NormalTok{]}
\NormalTok{)}
\end{Highlighting}
\end{Shaded}

\begin{verbatim}
Unable to display output for mime type(s): text/html
\end{verbatim}

\begin{tcolorbox}[enhanced jigsaw, colframe=quarto-callout-tip-color-frame, opacityback=0, titlerule=0mm, bottomrule=.15mm, breakable, leftrule=.75mm, colbacktitle=quarto-callout-tip-color!10!white, title=\textcolor{quarto-callout-tip-color}{\faLightbulb}\hspace{0.5em}{Practice}, rightrule=.15mm, coltitle=black, opacitybacktitle=0.6, colback=white, left=2mm, arc=.35mm, toptitle=1mm, bottomtitle=1mm, toprule=.15mm]

\subsection{Practice: Bird Strikes Histogram
Custom}\label{practice-bird-strikes-histogram-custom}

Update your birdstrikes histogram to use a hex code color, add a title,
and change the x-axis label to ``Speed (Nautical Miles Per Hour)''.

\begin{Shaded}
\begin{Highlighting}[]
\CommentTok{\# Your code here}
\end{Highlighting}
\end{Shaded}

\end{tcolorbox}

\subsection{Counts on bars}\label{counts-on-bars}

We can add counts to the bars with the \texttt{text\_auto} argument.

\begin{Shaded}
\begin{Highlighting}[]
\NormalTok{px.histogram(tips, x}\OperatorTok{=}\StringTok{\textquotesingle{}tip\textquotesingle{}}\NormalTok{, text\_auto}\OperatorTok{=} \VariableTok{True}\NormalTok{)}
\end{Highlighting}
\end{Shaded}

\begin{verbatim}
Unable to display output for mime type(s): text/html
\end{verbatim}

\subsubsection{Bins and bandwidths}\label{bins-and-bandwidths}

We can adjust the number of bins or bin width to better represent the
data using the \texttt{nbins} argument. Let's make a histogram with just
10 bins:

\begin{Shaded}
\begin{Highlighting}[]
\NormalTok{px.histogram(tips, x}\OperatorTok{=}\StringTok{\textquotesingle{}tip\textquotesingle{}}\NormalTok{, nbins}\OperatorTok{=}\DecValTok{10}\NormalTok{)}
\end{Highlighting}
\end{Shaded}

\begin{verbatim}
Unable to display output for mime type(s): text/html
\end{verbatim}

Now we have broader tip amount groups.

\begin{tcolorbox}[enhanced jigsaw, colframe=quarto-callout-tip-color-frame, opacityback=0, titlerule=0mm, bottomrule=.15mm, breakable, leftrule=.75mm, colbacktitle=quarto-callout-tip-color!10!white, title=\textcolor{quarto-callout-tip-color}{\faLightbulb}\hspace{0.5em}{Practice}, rightrule=.15mm, coltitle=black, opacitybacktitle=0.6, colback=white, left=2mm, arc=.35mm, toptitle=1mm, bottomtitle=1mm, toprule=.15mm]

\subsection{Practice: Speed Distribution Histogram
Custom}\label{practice-speed-distribution-histogram-custom}

Create a histogram of the speed distribution (Speed\_IAS\_in\_knots)
with 15 bins using the birdstrikes dataset. Add counts to the bars, use
a color of your choice, and add an appropriate title.

\begin{Shaded}
\begin{Highlighting}[]
\CommentTok{\# Your code here}
\end{Highlighting}
\end{Shaded}

\end{tcolorbox}

\section{Categorical Data}\label{categorical-data}

\subsection{Bar chart}\label{bar-chart}

Bar charts can be used to display the frequency of a single categorical
variable.

Plotly has a \texttt{px.bar} function that we will see later. But for
\textbf{single categorical variables}, the function plotly wants you to
use is actually \texttt{px.histogram}. (Statisticians everywhere are
crying; histograms are supposed to be used for just quantitative data!)

Let's create a basic bar chart showing the distribution of sex in the
tips dataset:

\begin{Shaded}
\begin{Highlighting}[]
\NormalTok{px.histogram(tips, x}\OperatorTok{=}\StringTok{\textquotesingle{}sex\textquotesingle{}}\NormalTok{)   }
\end{Highlighting}
\end{Shaded}

\begin{verbatim}
Unable to display output for mime type(s): text/html
\end{verbatim}

Let's add counts to the bars.

\begin{Shaded}
\begin{Highlighting}[]
\NormalTok{px.histogram(tips, x}\OperatorTok{=}\StringTok{\textquotesingle{}sex\textquotesingle{}}\NormalTok{, text\_auto}\OperatorTok{=} \VariableTok{True}\NormalTok{)}
\end{Highlighting}
\end{Shaded}

\begin{verbatim}
Unable to display output for mime type(s): text/html
\end{verbatim}

We can enhance the chart by adding a color axis, and customizing the
labels and title.

\begin{Shaded}
\begin{Highlighting}[]
\NormalTok{px.histogram(tips, x}\OperatorTok{=}\StringTok{\textquotesingle{}sex\textquotesingle{}}\NormalTok{, text\_auto}\OperatorTok{=}\VariableTok{True}\NormalTok{, color}\OperatorTok{=}\StringTok{\textquotesingle{}sex\textquotesingle{}}\NormalTok{, }
\NormalTok{             labels}\OperatorTok{=}\NormalTok{\{}\StringTok{\textquotesingle{}sex\textquotesingle{}}\NormalTok{: }\StringTok{\textquotesingle{}Gender\textquotesingle{}}\NormalTok{\},}
\NormalTok{             title}\OperatorTok{=}\StringTok{\textquotesingle{}Distribution of Customers by Gender\textquotesingle{}}\NormalTok{)}
\end{Highlighting}
\end{Shaded}

\begin{verbatim}
Unable to display output for mime type(s): text/html
\end{verbatim}

Arguably, in this plot, we do not need the \texttt{color} axis, since
the \texttt{sex} variable is already represented by the x axis. But
public audiences like colors, so it may still be worth including.

However, we should remove the legend. Let's also use custom colors.

For this, we can first create a figure object, then use the
\texttt{.layout.update} method from that object to update the legend.

\begin{Shaded}
\begin{Highlighting}[]
\NormalTok{tips\_by\_sex }\OperatorTok{=}\NormalTok{ px.histogram(}
\NormalTok{    tips,}
\NormalTok{    x}\OperatorTok{=}\StringTok{"sex"}\NormalTok{,}
\NormalTok{    text\_auto}\OperatorTok{=}\VariableTok{True}\NormalTok{,}
\NormalTok{    color}\OperatorTok{=}\StringTok{"sex"}\NormalTok{,}
\NormalTok{    labels}\OperatorTok{=}\NormalTok{\{}\StringTok{"sex"}\NormalTok{: }\StringTok{"Gender"}\NormalTok{\},}
\NormalTok{    title}\OperatorTok{=}\StringTok{"Distribution of Customers by Gender"}\NormalTok{,}
\NormalTok{    color\_discrete\_sequence}\OperatorTok{=}\NormalTok{[}\StringTok{"\#1f77b4"}\NormalTok{, }\StringTok{"\#ff7f0e"}\NormalTok{],}
\NormalTok{)}

\NormalTok{tips\_by\_sex.update\_layout(showlegend}\OperatorTok{=}\VariableTok{False}\NormalTok{)}
\end{Highlighting}
\end{Shaded}

\begin{verbatim}
Unable to display output for mime type(s): text/html
\end{verbatim}

\begin{tcolorbox}[enhanced jigsaw, colframe=quarto-callout-tip-color-frame, opacityback=0, titlerule=0mm, bottomrule=.15mm, breakable, leftrule=.75mm, colbacktitle=quarto-callout-tip-color!10!white, title=\textcolor{quarto-callout-tip-color}{\faLightbulb}\hspace{0.5em}{Practice}, rightrule=.15mm, coltitle=black, opacitybacktitle=0.6, colback=white, left=2mm, arc=.35mm, toptitle=1mm, bottomtitle=1mm, toprule=.15mm]

\subsection{Practice: Bird Strikes by Phase of
Flight}\label{practice-bird-strikes-by-phase-of-flight}

Create a bar chart showing the frequency of bird strikes by the phase of
flight, \texttt{When\_\_Phase\_of\_flight}. Add appropriate labels and a
title. Use colors of your choice, and remove the legend.

\begin{Shaded}
\begin{Highlighting}[]
\CommentTok{\# Your code here}
\end{Highlighting}
\end{Shaded}

\end{tcolorbox}

\subsubsection{Sorting categories}\label{sorting-categories}

It is sometimes useful to dictate a specific order for the categories in
a bar chart.

Consider this bar chart of the election winners by district in the 2013
Montreal mayoral election.

\begin{Shaded}
\begin{Highlighting}[]
\NormalTok{election }\OperatorTok{=}\NormalTok{ px.data.election()}
\NormalTok{election.head()}
\end{Highlighting}
\end{Shaded}

\begin{longtable}[]{@{}lllllllll@{}}
\toprule\noalign{}
& district & Coderre & Bergeron & Joly & total & winner & result &
district\_id \\
\midrule\noalign{}
\endhead
\bottomrule\noalign{}
\endlastfoot
0 & 101-Bois-de-Liesse & 2481 & 1829 & 3024 & 7334 & Joly & plurality &
101 \\
1 & 102-Cap-Saint-Jacques & 2525 & 1163 & 2675 & 6363 & Joly & plurality
& 102 \\
2 & 11-Sault-au-Récollet & 3348 & 2770 & 2532 & 8650 & Coderre &
plurality & 11 \\
3 & 111-Mile-End & 1734 & 4782 & 2514 & 9030 & Bergeron & majority &
111 \\
4 & 112-DeLorimier & 1770 & 5933 & 3044 & 10747 & Bergeron & majority &
112 \\
\end{longtable}

\begin{Shaded}
\begin{Highlighting}[]
\NormalTok{px.histogram(election, x}\OperatorTok{=}\StringTok{\textquotesingle{}winner\textquotesingle{}}\NormalTok{)}
\end{Highlighting}
\end{Shaded}

\begin{verbatim}
Unable to display output for mime type(s): text/html
\end{verbatim}

Let's define a custom order for the categories. ``Bergeron'' will be
first, then ``Joly'' then ``Coderre''.

\begin{Shaded}
\begin{Highlighting}[]
\NormalTok{custom\_order }\OperatorTok{=}\NormalTok{ [}\StringTok{"Bergeron"}\NormalTok{, }\StringTok{"Joly"}\NormalTok{, }\StringTok{"Coderre"}\NormalTok{]}
\NormalTok{election\_chart }\OperatorTok{=}\NormalTok{ px.histogram(election, x}\OperatorTok{=}\StringTok{\textquotesingle{}winner\textquotesingle{}}\NormalTok{, category\_orders}\OperatorTok{=}\NormalTok{\{}\StringTok{\textquotesingle{}winner\textquotesingle{}}\NormalTok{: custom\_order\})}
\NormalTok{election\_chart}
\end{Highlighting}
\end{Shaded}

\begin{verbatim}
Unable to display output for mime type(s): text/html
\end{verbatim}

We can also sort the categories by frequency.

We can sort the categories by frequency using the \texttt{categoryorder}
attribute of the x axis.

\begin{Shaded}
\begin{Highlighting}[]
\NormalTok{election\_chart }\OperatorTok{=}\NormalTok{ px.histogram(election, x}\OperatorTok{=}\StringTok{"winner"}\NormalTok{)}
\NormalTok{election\_chart.update\_xaxes(categoryorder}\OperatorTok{=}\StringTok{"total descending"}\NormalTok{)}
\end{Highlighting}
\end{Shaded}

\begin{verbatim}
Unable to display output for mime type(s): text/html
\end{verbatim}

Or in ascending order:

\begin{Shaded}
\begin{Highlighting}[]
\NormalTok{election\_chart }\OperatorTok{=}\NormalTok{ px.histogram(election, x}\OperatorTok{=}\StringTok{"winner"}\NormalTok{)}
\NormalTok{election\_chart.update\_xaxes(categoryorder}\OperatorTok{=}\StringTok{"total ascending"}\NormalTok{)}
\end{Highlighting}
\end{Shaded}

\begin{verbatim}
Unable to display output for mime type(s): text/html
\end{verbatim}

\begin{tcolorbox}[enhanced jigsaw, colframe=quarto-callout-tip-color-frame, opacityback=0, titlerule=0mm, bottomrule=.15mm, breakable, leftrule=.75mm, colbacktitle=quarto-callout-tip-color!10!white, title=\textcolor{quarto-callout-tip-color}{\faLightbulb}\hspace{0.5em}{Practice}, rightrule=.15mm, coltitle=black, opacitybacktitle=0.6, colback=white, left=2mm, arc=.35mm, toptitle=1mm, bottomtitle=1mm, toprule=.15mm]

\subsection{Practice: Sorted Origin State Bar
Chart}\label{practice-sorted-origin-state-bar-chart}

Create a sorted bar chart showing the distribution of bird strikes by
origin state. Sort the bars in ascending order of frequency.

\begin{Shaded}
\begin{Highlighting}[]
\CommentTok{\# Your code here}
\end{Highlighting}
\end{Shaded}

\end{tcolorbox}

\subsection{Horizontal bar chart}\label{horizontal-bar-chart}

When you have many categories, horizontal bar charts are often easier to
read than vertical bar charts. To make a horizontal bar chart, simply
use the \texttt{y} axis instead of the \texttt{x} axis.

\begin{Shaded}
\begin{Highlighting}[]
\NormalTok{px.histogram(tips, y}\OperatorTok{=}\StringTok{\textquotesingle{}day\textquotesingle{}}\NormalTok{)}
\end{Highlighting}
\end{Shaded}

\begin{verbatim}
Unable to display output for mime type(s): text/html
\end{verbatim}

\begin{tcolorbox}[enhanced jigsaw, colframe=quarto-callout-tip-color-frame, opacityback=0, titlerule=0mm, bottomrule=.15mm, breakable, leftrule=.75mm, colbacktitle=quarto-callout-tip-color!10!white, title=\textcolor{quarto-callout-tip-color}{\faLightbulb}\hspace{0.5em}{Practice}, rightrule=.15mm, coltitle=black, opacitybacktitle=0.6, colback=white, left=2mm, arc=.35mm, toptitle=1mm, bottomtitle=1mm, toprule=.15mm]

\subsection{Practice: Horizontal Bar Chart of Origin
State}\label{practice-horizontal-bar-chart-of-origin-state}

Create a horizontal bar chart showing the distribution of bird strikes
by origin state.

\begin{Shaded}
\begin{Highlighting}[]
\CommentTok{\# Your code here}
\end{Highlighting}
\end{Shaded}

\end{tcolorbox}

\subsection{Pie chart}\label{pie-chart}

Pie charts are also useful for showing the proportion of categorical
variables. They are best used when you have a \emph{small} number of
categories. For larger numbers of categories, pie charts are hard to
read.

Let's make a pie chart of the distribution of tips by day of the week.

\begin{Shaded}
\begin{Highlighting}[]
\NormalTok{px.pie(tips, names}\OperatorTok{=}\StringTok{"day"}\NormalTok{)}
\end{Highlighting}
\end{Shaded}

\begin{verbatim}
Unable to display output for mime type(s): text/html
\end{verbatim}

We can add labels to the pie chart to make it easier to read.

\begin{Shaded}
\begin{Highlighting}[]
\NormalTok{tips\_by\_day }\OperatorTok{=}\NormalTok{ px.pie(tips, names}\OperatorTok{=}\StringTok{"day"}\NormalTok{)}
\NormalTok{tips\_by\_day\_with\_labels }\OperatorTok{=}\NormalTok{ tips\_by\_day.update\_traces(textposition}\OperatorTok{=}\StringTok{"inside"}\NormalTok{, textinfo}\OperatorTok{=}\StringTok{"percent+label"}\NormalTok{)}
\NormalTok{tips\_by\_day\_with\_labels}
\end{Highlighting}
\end{Shaded}

\begin{verbatim}
Unable to display output for mime type(s): text/html
\end{verbatim}

The legend is no longer needed, so we can remove it.

\begin{Shaded}
\begin{Highlighting}[]
\NormalTok{tips\_by\_day\_with\_labels.update\_layout(showlegend}\OperatorTok{=}\VariableTok{False}\NormalTok{)}
\end{Highlighting}
\end{Shaded}

\begin{verbatim}
Unable to display output for mime type(s): text/html
\end{verbatim}

\begin{tcolorbox}[enhanced jigsaw, colframe=quarto-callout-note-color-frame, opacityback=0, titlerule=0mm, bottomrule=.15mm, breakable, leftrule=.75mm, colbacktitle=quarto-callout-note-color!10!white, title=\textcolor{quarto-callout-note-color}{\faInfo}\hspace{0.5em}{Pro}, rightrule=.15mm, coltitle=black, opacitybacktitle=0.6, colback=white, left=2mm, arc=.35mm, toptitle=1mm, bottomtitle=1mm, toprule=.15mm]

If you forget how to make simple changes like this, don't hesitate to
consult the plotly documentation, Google or ChatGPT.

\end{tcolorbox}

\begin{tcolorbox}[enhanced jigsaw, colframe=quarto-callout-tip-color-frame, opacityback=0, titlerule=0mm, bottomrule=.15mm, breakable, leftrule=.75mm, colbacktitle=quarto-callout-tip-color!10!white, title=\textcolor{quarto-callout-tip-color}{\faLightbulb}\hspace{0.5em}{Practice}, rightrule=.15mm, coltitle=black, opacitybacktitle=0.6, colback=white, left=2mm, arc=.35mm, toptitle=1mm, bottomtitle=1mm, toprule=.15mm]

\subsection{Practice: Wildlife Size Pie
Chart}\label{practice-wildlife-size-pie-chart}

Create a pie chart showing the distribution of bird strikes by wildlife
size. Include percentages and labels inside the pie slices.

\begin{Shaded}
\begin{Highlighting}[]
\CommentTok{\# Your code here}
\end{Highlighting}
\end{Shaded}

\end{tcolorbox}

\section{Summary}\label{summary}

In this lesson, you learned how to create univariate graphs using Plotly
Express. You should now feel confident in your ability to create bar
charts, pie charts, and histograms. You should also feel comfortable
customizing the appearance of your graphs.

See you in the next lesson.

\chapter{Bivariate \& Multivariate Graphs with Plotly
Express}\label{bivariate-multivariate-graphs-with-plotly-express}

\section{Introduction}\label{introduction-5}

In this lesson, you'll learn how to create bivariate and multivariate
graphs using Plotly Express. These types of graphs are essential for
exploring relationships between two or more variables, whether they are
quantitative or categorical. Understanding these relationships can
provide deeper insights into your data.

Let's dive in!

\section{Learning Objectives}\label{learning-objectives-6}

By the end of this lesson, you will be able to:

\begin{itemize}
\tightlist
\item
  Create scatter plots for quantitative vs.~quantitative data
\item
  Generate grouped histograms and violin plots for quantitative
  vs.~categorical data
\item
  Create grouped, stacked, and percent-stacked bar charts for
  categorical vs.~categorical data
\item
  Visualize time series data using bar charts and line charts
\item
  Create bubble charts to display relationships between three or more
  variables
\item
  Use faceting to compare distributions across subsets of data
\end{itemize}

\section{Imports}\label{imports-2}

This lesson requires \texttt{plotly.express}, \texttt{pandas},
\texttt{numpy}, and \texttt{vega\_datasets}. Install them if you haven't
already.

\begin{Shaded}
\begin{Highlighting}[]
\ImportTok{import}\NormalTok{ plotly.express }\ImportTok{as}\NormalTok{ px}
\ImportTok{import}\NormalTok{ pandas }\ImportTok{as}\NormalTok{ pd}
\ImportTok{import}\NormalTok{ numpy }\ImportTok{as}\NormalTok{ np}
\ImportTok{from}\NormalTok{ vega\_datasets }\ImportTok{import}\NormalTok{ data}
\end{Highlighting}
\end{Shaded}

\section{Numeric vs.~Numeric Data}\label{numeric-vs.-numeric-data}

When both variables are quantitative, scatter plots are an excellent way
to visualize their relationship.

\subsection{Scatter Plot}\label{scatter-plot}

Let's create a scatter plot to examine the relationship between
\texttt{total\_bill} and \texttt{tip} in the tips dataset. The tips
dataset is included in Plotly Express and contains information about
restaurant bills and tips that were collected by a waiter in a US
restaurant.

First, we'll load the dataset and view the first five rows:

\begin{Shaded}
\begin{Highlighting}[]
\NormalTok{tips }\OperatorTok{=}\NormalTok{ px.data.tips()}
\NormalTok{tips}
\end{Highlighting}
\end{Shaded}

\begin{longtable}[]{@{}llllllll@{}}
\toprule\noalign{}
& total\_bill & tip & sex & smoker & day & time & size \\
\midrule\noalign{}
\endhead
\bottomrule\noalign{}
\endlastfoot
0 & 16.99 & 1.01 & Female & No & Sun & Dinner & 2 \\
1 & 10.34 & 1.66 & Male & No & Sun & Dinner & 3 \\
2 & 21.01 & 3.50 & Male & No & Sun & Dinner & 3 \\
3 & 23.68 & 3.31 & Male & No & Sun & Dinner & 2 \\
4 & 24.59 & 3.61 & Female & No & Sun & Dinner & 4 \\
... & ... & ... & ... & ... & ... & ... & ... \\
239 & 29.03 & 5.92 & Male & No & Sat & Dinner & 3 \\
240 & 27.18 & 2.00 & Female & Yes & Sat & Dinner & 2 \\
241 & 22.67 & 2.00 & Male & Yes & Sat & Dinner & 2 \\
242 & 17.82 & 1.75 & Male & No & Sat & Dinner & 2 \\
243 & 18.78 & 3.00 & Female & No & Thur & Dinner & 2 \\
\end{longtable}

Next, we'll create a basic scatter plot. We do this with the
\texttt{px.scatter} function.

\begin{Shaded}
\begin{Highlighting}[]
\NormalTok{px.scatter(tips, x}\OperatorTok{=}\StringTok{\textquotesingle{}total\_bill\textquotesingle{}}\NormalTok{, y}\OperatorTok{=}\StringTok{\textquotesingle{}tip\textquotesingle{}}\NormalTok{)}
\end{Highlighting}
\end{Shaded}

\begin{verbatim}
Unable to display output for mime type(s): text/html
\end{verbatim}

\begin{verbatim}
Unable to display output for mime type(s): text/html
\end{verbatim}

From the scatter plot, we can observe that as the total bill increases,
the tip amount tends to increase as well.

Let's enhance the scatter plot by adding labels and a title.

\begin{Shaded}
\begin{Highlighting}[]
\NormalTok{px.scatter(}
\NormalTok{    tips,}
\NormalTok{    x}\OperatorTok{=}\StringTok{"total\_bill"}\NormalTok{,}
\NormalTok{    y}\OperatorTok{=}\StringTok{"tip"}\NormalTok{,}
\NormalTok{    labels}\OperatorTok{=}\NormalTok{\{}\StringTok{"total\_bill"}\NormalTok{: }\StringTok{"Total Bill ($)"}\NormalTok{, }\StringTok{"tip"}\NormalTok{: }\StringTok{"Tip ($)"}\NormalTok{\},}
\NormalTok{    title}\OperatorTok{=}\StringTok{"Relationship Between Total Bill and Tip Amount"}\NormalTok{,}
\NormalTok{)}
\end{Highlighting}
\end{Shaded}

\begin{verbatim}
Unable to display output for mime type(s): text/html
\end{verbatim}

Recall that you can see additional information about the function by
typing \texttt{px.scatter?} in a cell and executing the cell.

\begin{Shaded}
\begin{Highlighting}[]
\NormalTok{px.scatter?}
\end{Highlighting}
\end{Shaded}

\begin{tcolorbox}[enhanced jigsaw, colframe=quarto-callout-tip-color-frame, opacityback=0, titlerule=0mm, bottomrule=.15mm, breakable, leftrule=.75mm, colbacktitle=quarto-callout-tip-color!10!white, title=\textcolor{quarto-callout-tip-color}{\faLightbulb}\hspace{0.5em}{Practice}, rightrule=.15mm, coltitle=black, opacitybacktitle=0.6, colback=white, left=2mm, arc=.35mm, toptitle=1mm, bottomtitle=1mm, toprule=.15mm]

\subsection{Practice: Life Expectancy vs.~GDP Per
Capita}\label{practice-life-expectancy-vs.-gdp-per-capita}

Using the Gapminder dataset (the 2007 subset, \texttt{g\_2007}, defined
below), create a scatter plot showing the relationship between
\texttt{gdpPercap} (GDP per capita) and \texttt{lifeExp} (life
expectancy).

According to the plot, what is the relationship between GDP per capita
and life expectancy?

\begin{Shaded}
\begin{Highlighting}[]
\NormalTok{gapminder }\OperatorTok{=}\NormalTok{ px.data.gapminder()}
\NormalTok{g\_2007 }\OperatorTok{=}\NormalTok{ gapminder.query(}\StringTok{\textquotesingle{}year == 2007\textquotesingle{}}\NormalTok{)}
\NormalTok{g\_2007.head()}
\CommentTok{\# Your code here}
\end{Highlighting}
\end{Shaded}

\begin{longtable}[]{@{}lllllllll@{}}
\toprule\noalign{}
& country & continent & year & lifeExp & pop & gdpPercap & iso\_alpha &
iso\_num \\
\midrule\noalign{}
\endhead
\bottomrule\noalign{}
\endlastfoot
11 & Afghanistan & Asia & 2007 & 43.828 & 31889923 & 974.580338 & AFG &
4 \\
23 & Albania & Europe & 2007 & 76.423 & 3600523 & 5937.029526 & ALB &
8 \\
35 & Algeria & Africa & 2007 & 72.301 & 33333216 & 6223.367465 & DZA &
12 \\
47 & Angola & Africa & 2007 & 42.731 & 12420476 & 4797.231267 & AGO &
24 \\
59 & Argentina & Americas & 2007 & 75.320 & 40301927 & 12779.379640 &
ARG & 32 \\
\end{longtable}

\end{tcolorbox}

\section{Numeric vs.~Categorical
Data}\label{numeric-vs.-categorical-data}

When one variable is quantitative and the other is categorical, we can
use grouped histograms, violin plots, or box plots to visualize the
distribution of the quantitative variable across different categories.

\subsection{Grouped Histograms}\label{grouped-histograms}

First, here's how you can create a regular histogram of all tips:

\begin{Shaded}
\begin{Highlighting}[]
\NormalTok{px.histogram(tips, x}\OperatorTok{=}\StringTok{\textquotesingle{}tip\textquotesingle{}}\NormalTok{)}
\end{Highlighting}
\end{Shaded}

\begin{verbatim}
Unable to display output for mime type(s): text/html
\end{verbatim}

To create a grouped histogram, use the \texttt{color} parameter to
specify the categorical variable. Here, we'll color the histogram by
\texttt{sex}:

\begin{Shaded}
\begin{Highlighting}[]
\NormalTok{px.histogram(tips, x}\OperatorTok{=}\StringTok{\textquotesingle{}tip\textquotesingle{}}\NormalTok{, color}\OperatorTok{=}\StringTok{\textquotesingle{}sex\textquotesingle{}}\NormalTok{)}
\end{Highlighting}
\end{Shaded}

\begin{verbatim}
Unable to display output for mime type(s): text/html
\end{verbatim}

By default, the histograms for each category are stacked. To change this
behavior, you can use the \texttt{barmode} parameter. For example,
\texttt{barmode=\textquotesingle{}overlay\textquotesingle{}} will create
an overlaid histogram:

\begin{Shaded}
\begin{Highlighting}[]
\NormalTok{px.histogram(tips, x}\OperatorTok{=}\StringTok{"tip"}\NormalTok{, color}\OperatorTok{=}\StringTok{"sex"}\NormalTok{, barmode}\OperatorTok{=}\StringTok{"overlay"}\NormalTok{)}
\end{Highlighting}
\end{Shaded}

\begin{verbatim}
Unable to display output for mime type(s): text/html
\end{verbatim}

This creates two semi-transparent histograms overlaid on top of each
other, allowing for direct comparison of the distributions.

\begin{tcolorbox}[enhanced jigsaw, colframe=quarto-callout-tip-color-frame, opacityback=0, titlerule=0mm, bottomrule=.15mm, breakable, leftrule=.75mm, colbacktitle=quarto-callout-tip-color!10!white, title=\textcolor{quarto-callout-tip-color}{\faLightbulb}\hspace{0.5em}{Practice}, rightrule=.15mm, coltitle=black, opacitybacktitle=0.6, colback=white, left=2mm, arc=.35mm, toptitle=1mm, bottomtitle=1mm, toprule=.15mm]

\subsection{Practice: Age Distribution by
Gender}\label{practice-age-distribution-by-gender}

Using the \texttt{la\_riots} dataset from \texttt{vega\_datasets},
create a grouped histogram of \texttt{age} by \texttt{gender}. Compare
the age distributions between different genders.

According to the plot, was the oldest victim male or female?

\begin{Shaded}
\begin{Highlighting}[]
\NormalTok{la\_riots }\OperatorTok{=}\NormalTok{ data.la\_riots()}
\NormalTok{la\_riots.head()}
\CommentTok{\# Your code here}
\end{Highlighting}
\end{Shaded}

\begin{longtable}[]{@{}llllllllllll@{}}
\toprule\noalign{}
& first\_name & last\_name & age & gender & race & death\_date & address
& neighborhood & type & longitude & latitude \\
\midrule\noalign{}
\endhead
\bottomrule\noalign{}
\endlastfoot
0 & Cesar A. & Aguilar & 18.0 & Male & Latino & 1992-04-30 & 2009 W. 6th
St. & Westlake & Officer-involved shooting & -118.273976 & 34.059281 \\
1 & George & Alvarez & 42.0 & Male & Latino & 1992-05-01 & Main \&
College streets & Chinatown & Not riot-related & -118.234098 &
34.062690 \\
2 & Wilson & Alvarez & 40.0 & Male & Latino & 1992-05-23 & 3100
Rosecrans Ave. & Hawthorne & Homicide & -118.326816 & 33.901662 \\
3 & Brian E. & Andrew & 30.0 & Male & Black & 1992-04-30 & Rosecrans \&
Chester avenues & Compton & Officer-involved shooting & -118.215390 &
33.903457 \\
4 & Vivian & Austin & 87.0 & Female & Black & 1992-05-03 & 1600 W. 60th
St. & Harvard Park & Death & -118.304741 & 33.985667 \\
\end{longtable}

\end{tcolorbox}

\subsection{Violin \& Box Plots}\label{violin-box-plots}

Violin plots are useful for comparing the distribution of a quantitative
variable across different categories. They show the probability density
of the data at different values and can include a box plot to summarize
key statistics.

First, let's create a violin plot of all tips:

\begin{Shaded}
\begin{Highlighting}[]
\NormalTok{px.violin(tips, y}\OperatorTok{=}\StringTok{"tip"}\NormalTok{)}
\end{Highlighting}
\end{Shaded}

\begin{verbatim}
Unable to display output for mime type(s): text/html
\end{verbatim}

We can add a box plot to the violin plot by setting the \texttt{box}
parameter to \texttt{True}:

\begin{Shaded}
\begin{Highlighting}[]
\NormalTok{px.violin(tips, y}\OperatorTok{=}\StringTok{"tip"}\NormalTok{, box}\OperatorTok{=}\VariableTok{True}\NormalTok{)}
\end{Highlighting}
\end{Shaded}

\begin{verbatim}
Unable to display output for mime type(s): text/html
\end{verbatim}

For just the box plot, we can use \texttt{px.box}:

\begin{Shaded}
\begin{Highlighting}[]
\NormalTok{px.box(tips, y}\OperatorTok{=}\StringTok{"tip"}\NormalTok{)}
\end{Highlighting}
\end{Shaded}

\begin{verbatim}
Unable to display output for mime type(s): text/html
\end{verbatim}

To add jitter points to the violin or box plots, we can use the
\texttt{points\ =\ \textquotesingle{}all\textquotesingle{}} parameter.

\begin{Shaded}
\begin{Highlighting}[]
\NormalTok{px.violin(tips, y}\OperatorTok{=}\StringTok{"tip"}\NormalTok{, points}\OperatorTok{=}\StringTok{"all"}\NormalTok{)}
\end{Highlighting}
\end{Shaded}

\begin{verbatim}
Unable to display output for mime type(s): text/html
\end{verbatim}

Now, to create a violin plot of tips by gender, use the \texttt{x}
parameter to specify the categorical variable:

\begin{Shaded}
\begin{Highlighting}[]
\NormalTok{px.violin(tips, y}\OperatorTok{=}\StringTok{"tip"}\NormalTok{, x}\OperatorTok{=}\StringTok{"sex"}\NormalTok{, box}\OperatorTok{=}\VariableTok{True}\NormalTok{)}
\end{Highlighting}
\end{Shaded}

\begin{verbatim}
Unable to display output for mime type(s): text/html
\end{verbatim}

We can also add a color axis to differentiate the violins:

\begin{Shaded}
\begin{Highlighting}[]
\NormalTok{px.violin(tips, y}\OperatorTok{=}\StringTok{"tip"}\NormalTok{, x}\OperatorTok{=}\StringTok{"sex"}\NormalTok{, color}\OperatorTok{=}\StringTok{"sex"}\NormalTok{, box}\OperatorTok{=}\VariableTok{True}\NormalTok{)}
\end{Highlighting}
\end{Shaded}

\begin{verbatim}
Unable to display output for mime type(s): text/html
\end{verbatim}

\begin{tcolorbox}[enhanced jigsaw, colframe=quarto-callout-tip-color-frame, opacityback=0, titlerule=0mm, bottomrule=.15mm, breakable, leftrule=.75mm, colbacktitle=quarto-callout-tip-color!10!white, title=\textcolor{quarto-callout-tip-color}{\faLightbulb}\hspace{0.5em}{Practice}, rightrule=.15mm, coltitle=black, opacitybacktitle=0.6, colback=white, left=2mm, arc=.35mm, toptitle=1mm, bottomtitle=1mm, toprule=.15mm]

\subsection{Practice: Life Expectancy by
Continent}\label{practice-life-expectancy-by-continent}

Using the \texttt{g\_2007} dataset, create a violin plot showing the
distribution of \texttt{lifeExp} by \texttt{continent}.

According to the plot, which continent has the highest median country
life expectancy?

\begin{Shaded}
\begin{Highlighting}[]
\NormalTok{g\_2007 }\OperatorTok{=}\NormalTok{ gapminder.query(}\StringTok{"year == 2007"}\NormalTok{)}
\NormalTok{g\_2007.head()}
\CommentTok{\# Your code here}
\end{Highlighting}
\end{Shaded}

\begin{longtable}[]{@{}lllllllll@{}}
\toprule\noalign{}
& country & continent & year & lifeExp & pop & gdpPercap & iso\_alpha &
iso\_num \\
\midrule\noalign{}
\endhead
\bottomrule\noalign{}
\endlastfoot
11 & Afghanistan & Asia & 2007 & 43.828 & 31889923 & 974.580338 & AFG &
4 \\
23 & Albania & Europe & 2007 & 76.423 & 3600523 & 5937.029526 & ALB &
8 \\
35 & Algeria & Africa & 2007 & 72.301 & 33333216 & 6223.367465 & DZA &
12 \\
47 & Angola & Africa & 2007 & 42.731 & 12420476 & 4797.231267 & AGO &
24 \\
59 & Argentina & Americas & 2007 & 75.320 & 40301927 & 12779.379640 &
ARG & 32 \\
\end{longtable}

\end{tcolorbox}

\subsection{Summary Bar Charts (Mean and Standard
Deviation)}\label{summary-bar-charts-mean-and-standard-deviation}

Sometimes it's useful to display the mean and standard deviation of a
quantitative variable across different categories. This can be
visualized using a bar chart with error bars.

First, let's calculate the mean and standard deviation of tips for each
gender. You have not yet learned how to do this, but you will in a later
lesson.

\begin{Shaded}
\begin{Highlighting}[]
\CommentTok{\# Calculate the mean and standard deviation}
\NormalTok{summary\_df }\OperatorTok{=}\NormalTok{ (}
\NormalTok{    tips.groupby(}\StringTok{"sex"}\NormalTok{)}
\NormalTok{    .agg(mean\_tip}\OperatorTok{=}\NormalTok{(}\StringTok{"tip"}\NormalTok{, }\StringTok{"mean"}\NormalTok{), std\_tip}\OperatorTok{=}\NormalTok{(}\StringTok{"tip"}\NormalTok{, }\StringTok{"std"}\NormalTok{))}
\NormalTok{    .reset\_index()}
\NormalTok{)}
\NormalTok{summary\_df}
\end{Highlighting}
\end{Shaded}

\begin{longtable}[]{@{}llll@{}}
\toprule\noalign{}
& sex & mean\_tip & std\_tip \\
\midrule\noalign{}
\endhead
\bottomrule\noalign{}
\endlastfoot
0 & Female & 2.833448 & 1.159495 \\
1 & Male & 3.089618 & 1.489102 \\
\end{longtable}

Next, we'll create a bar chart using \texttt{px.bar} and add error bars
using the \texttt{error\_y} parameter:

\begin{Shaded}
\begin{Highlighting}[]
\CommentTok{\# Create the bar chart}
\NormalTok{px.bar(summary\_df, x}\OperatorTok{=}\StringTok{"sex"}\NormalTok{, y}\OperatorTok{=}\StringTok{"mean\_tip"}\NormalTok{, error\_y}\OperatorTok{=}\StringTok{"std\_tip"}\NormalTok{)}
\end{Highlighting}
\end{Shaded}

\begin{verbatim}
Unable to display output for mime type(s): text/html
\end{verbatim}

This bar chart displays the average tip amount for each gender, with
error bars representing the standard deviation.

\begin{tcolorbox}[enhanced jigsaw, colframe=quarto-callout-tip-color-frame, opacityback=0, titlerule=0mm, bottomrule=.15mm, breakable, leftrule=.75mm, colbacktitle=quarto-callout-tip-color!10!white, title=\textcolor{quarto-callout-tip-color}{\faLightbulb}\hspace{0.5em}{Practice}, rightrule=.15mm, coltitle=black, opacitybacktitle=0.6, colback=white, left=2mm, arc=.35mm, toptitle=1mm, bottomtitle=1mm, toprule=.15mm]

\subsection{Practice: Average Total Bill by
Day}\label{practice-average-total-bill-by-day}

Using the \texttt{tips} dataset, create a bar chart of mean
\texttt{total\_bill} by \texttt{day} with standard deviation error bars.
You should copy and paste the code from the example above and modify it
to create this plot.

According to the plot, which day has the highest average total bill?

\begin{Shaded}
\begin{Highlighting}[]
\NormalTok{tips.head()  }\CommentTok{\# View the tips dataset}
\CommentTok{\# Your code here}
\end{Highlighting}
\end{Shaded}

\begin{longtable}[]{@{}llllllll@{}}
\toprule\noalign{}
& total\_bill & tip & sex & smoker & day & time & size \\
\midrule\noalign{}
\endhead
\bottomrule\noalign{}
\endlastfoot
0 & 16.99 & 1.01 & Female & No & Sun & Dinner & 2 \\
1 & 10.34 & 1.66 & Male & No & Sun & Dinner & 3 \\
2 & 21.01 & 3.50 & Male & No & Sun & Dinner & 3 \\
3 & 23.68 & 3.31 & Male & No & Sun & Dinner & 2 \\
4 & 24.59 & 3.61 & Female & No & Sun & Dinner & 4 \\
\end{longtable}

\end{tcolorbox}

\begin{tcolorbox}[enhanced jigsaw, colframe=quarto-callout-note-color-frame, opacityback=0, titlerule=0mm, bottomrule=.15mm, breakable, leftrule=.75mm, colbacktitle=quarto-callout-note-color!10!white, title=\textcolor{quarto-callout-note-color}{\faInfo}\hspace{0.5em}{Side Note: Difference between \texttt{px.bar} and \texttt{px.histogram}}, rightrule=.15mm, coltitle=black, opacitybacktitle=0.6, colback=white, left=2mm, arc=.35mm, toptitle=1mm, bottomtitle=1mm, toprule=.15mm]

Notice that this is the first time we are using the \texttt{px.bar}
function. For past plots, we have used \texttt{px.histogram} to make bar
charts.

The bar chart function generally expects that the numeric variable being
plotted is already in it's own column, while the histogram function does
the grouping for you.

For example, in the cell below, we use \texttt{px.histogram} to make a
bar chart of the \texttt{sex} column. The resulting plot compares the
number of male and female customers in the dataset.

\begin{Shaded}
\begin{Highlighting}[]
\NormalTok{px.histogram(tips, x}\OperatorTok{=}\StringTok{\textquotesingle{}sex\textquotesingle{}}\NormalTok{)}
\end{Highlighting}
\end{Shaded}

\begin{verbatim}
Unable to display output for mime type(s): text/html
\end{verbatim}

To make the same plot using \texttt{px.bar}, we first need to group by
the \texttt{sex} column and count the number of rows for each sex.

\begin{Shaded}
\begin{Highlighting}[]
\NormalTok{sex\_counts }\OperatorTok{=}\NormalTok{ tips[}\StringTok{\textquotesingle{}sex\textquotesingle{}}\NormalTok{].value\_counts().reset\_index()}
\NormalTok{sex\_counts}
\end{Highlighting}
\end{Shaded}

\begin{longtable}[]{@{}lll@{}}
\toprule\noalign{}
& sex & count \\
\midrule\noalign{}
\endhead
\bottomrule\noalign{}
\endlastfoot
0 & Male & 157 \\
1 & Female & 87 \\
\end{longtable}

We can then plot the \texttt{day} column using \texttt{px.bar}:

\begin{Shaded}
\begin{Highlighting}[]
\NormalTok{px.bar(sex\_counts, x}\OperatorTok{=}\StringTok{"sex"}\NormalTok{, y}\OperatorTok{=}\StringTok{"count"}\NormalTok{)}
\end{Highlighting}
\end{Shaded}

\begin{verbatim}
Unable to display output for mime type(s): text/html
\end{verbatim}

This produces a bar chart with one bar for each sex.

\end{tcolorbox}

\section{Categorical vs.~Categorical
Data}\label{categorical-vs.-categorical-data}

When both variables are categorical, bar charts with a color axis are
effective for visualizing the frequency distribution across categories.
We will focus on three types of bar charts: stacked bar charts,
percent-stacked bar charts, and grouped/clustered bar charts.

\subsection{Stacked Bar Charts}\label{stacked-bar-charts}

Stacked bar charts show the total counts and the breakdown within each
category. To make a stacked bar chart, use the \texttt{color} parameter
to specify the categorical variable:

\begin{Shaded}
\begin{Highlighting}[]
\NormalTok{px.histogram(}
\NormalTok{    tips,}
\NormalTok{    x}\OperatorTok{=}\StringTok{\textquotesingle{}day\textquotesingle{}}\NormalTok{,}
\NormalTok{    color}\OperatorTok{=}\StringTok{\textquotesingle{}sex\textquotesingle{}}
\NormalTok{)}
\end{Highlighting}
\end{Shaded}

\begin{verbatim}
Unable to display output for mime type(s): text/html
\end{verbatim}

Let's add numbers to the bars to show the exact counts, and also improve
the color palette with custom colors.

\begin{Shaded}
\begin{Highlighting}[]
\NormalTok{px.histogram(}
\NormalTok{    tips,}
\NormalTok{    x}\OperatorTok{=}\StringTok{"day"}\NormalTok{,}
\NormalTok{    color}\OperatorTok{=}\StringTok{"sex"}\NormalTok{,}
\NormalTok{    text\_auto}\OperatorTok{=}\VariableTok{True}\NormalTok{,}
\NormalTok{    color\_discrete\_sequence}\OperatorTok{=}\NormalTok{[}\StringTok{"\#deb221"}\NormalTok{, }\StringTok{"\#2f828a"}\NormalTok{],}
\NormalTok{)}
\end{Highlighting}
\end{Shaded}

\begin{verbatim}
Unable to display output for mime type(s): text/html
\end{verbatim}

This stacked bar chart shows the total number of customers each day,
broken down by gender.

\begin{tcolorbox}[enhanced jigsaw, colframe=quarto-callout-tip-color-frame, opacityback=0, titlerule=0mm, bottomrule=.15mm, breakable, leftrule=.75mm, colbacktitle=quarto-callout-tip-color!10!white, title=\textcolor{quarto-callout-tip-color}{\faLightbulb}\hspace{0.5em}{Practice}, rightrule=.15mm, coltitle=black, opacitybacktitle=0.6, colback=white, left=2mm, arc=.35mm, toptitle=1mm, bottomtitle=1mm, toprule=.15mm]

\subsection{Practice: High and Low Income Countries by
Continent}\label{practice-high-and-low-income-countries-by-continent}

Using the \texttt{g\_2007\_income} dataset, create a stacked bar chart
showing the count of high and low income countries in each continent.

\begin{Shaded}
\begin{Highlighting}[]
\NormalTok{gap\_dat }\OperatorTok{=}\NormalTok{ px.data.gapminder()}

\NormalTok{g\_2007\_income }\OperatorTok{=}\NormalTok{ (}
\NormalTok{    gap\_dat.query(}\StringTok{"year == 2007"}\NormalTok{)}
\NormalTok{    .drop(columns}\OperatorTok{=}\NormalTok{[}\StringTok{"year"}\NormalTok{, }\StringTok{"iso\_alpha"}\NormalTok{, }\StringTok{"iso\_num"}\NormalTok{])}
\NormalTok{    .assign(}
\NormalTok{        income\_group}\OperatorTok{=}\KeywordTok{lambda}\NormalTok{ df: np.where(}
\NormalTok{            df.gdpPercap }\OperatorTok{\textgreater{}} \DecValTok{15000}\NormalTok{, }\StringTok{"High Income"}\NormalTok{, }\StringTok{"Low \& Middle Income"}
\NormalTok{        )}
\NormalTok{    )}
\NormalTok{)}

\NormalTok{g\_2007\_income.head()}
\CommentTok{\# Your code here}
\end{Highlighting}
\end{Shaded}

\begin{longtable}[]{@{}lllllll@{}}
\toprule\noalign{}
& country & continent & lifeExp & pop & gdpPercap & income\_group \\
\midrule\noalign{}
\endhead
\bottomrule\noalign{}
\endlastfoot
11 & Afghanistan & Asia & 43.828 & 31889923 & 974.580338 & Low \& Middle
Income \\
23 & Albania & Europe & 76.423 & 3600523 & 5937.029526 & Low \& Middle
Income \\
35 & Algeria & Africa & 72.301 & 33333216 & 6223.367465 & Low \& Middle
Income \\
47 & Angola & Africa & 42.731 & 12420476 & 4797.231267 & Low \& Middle
Income \\
59 & Argentina & Americas & 75.320 & 40301927 & 12779.379640 & Low \&
Middle Income \\
\end{longtable}

\end{tcolorbox}

\subsection{Percent-Stacked Bar
Charts}\label{percent-stacked-bar-charts}

To show proportions instead of counts, we can create percent-stacked bar
charts by setting the \texttt{barnorm} parameter to
\texttt{\textquotesingle{}percent\textquotesingle{}}:

\begin{Shaded}
\begin{Highlighting}[]
\CommentTok{\# Create the percent{-}stacked bar chart}
\NormalTok{px.histogram(tips, x}\OperatorTok{=}\StringTok{"day"}\NormalTok{, color}\OperatorTok{=}\StringTok{"sex"}\NormalTok{, barnorm}\OperatorTok{=}\StringTok{"percent"}\NormalTok{)}
\end{Highlighting}
\end{Shaded}

\begin{verbatim}
Unable to display output for mime type(s): text/html
\end{verbatim}

This chart normalizes the bar heights to represent percentages, showing
the proportion of each gender for each day.

We can also add text labels to the bars to show the exact percentages:

\begin{Shaded}
\begin{Highlighting}[]
\NormalTok{px.histogram(tips, x}\OperatorTok{=}\StringTok{"day"}\NormalTok{, color}\OperatorTok{=}\StringTok{"sex"}\NormalTok{, barnorm}\OperatorTok{=}\StringTok{"percent"}\NormalTok{, text\_auto}\OperatorTok{=}\StringTok{".1f"}\NormalTok{)}
\end{Highlighting}
\end{Shaded}

\begin{verbatim}
Unable to display output for mime type(s): text/html
\end{verbatim}

The symbol \texttt{.1f} in the \texttt{text\_auto} parameter formats the
text labels to one decimal place.

\begin{tcolorbox}[enhanced jigsaw, colframe=quarto-callout-tip-color-frame, opacityback=0, titlerule=0mm, bottomrule=.15mm, breakable, leftrule=.75mm, colbacktitle=quarto-callout-tip-color!10!white, title=\textcolor{quarto-callout-tip-color}{\faLightbulb}\hspace{0.5em}{Practice}, rightrule=.15mm, coltitle=black, opacitybacktitle=0.6, colback=white, left=2mm, arc=.35mm, toptitle=1mm, bottomtitle=1mm, toprule=.15mm]

\subsection{Practice: Proportion of High and Low Income Countries by
Continent}\label{practice-proportion-of-high-and-low-income-countries-by-continent}

Again using the \texttt{g\_2007\_income} dataset, create a
percent-stacked bar chart showing the proportion of high and low income
countries in each continent. Add text labels to the bars to show the
exact percentages.

According the plot, which continent has the highest proportion of high
income countries? Are there any limitations to this plot?

\begin{Shaded}
\begin{Highlighting}[]
\CommentTok{\# Your code here}
\end{Highlighting}
\end{Shaded}

\end{tcolorbox}

\subsection{Clustered Bar Charts}\label{clustered-bar-charts}

For clustered bar charts, set the \texttt{barmode} parameter to
\texttt{\textquotesingle{}group\textquotesingle{}} to place the bars for
each category side by side:

\begin{Shaded}
\begin{Highlighting}[]
\NormalTok{px.histogram(tips, x}\OperatorTok{=}\StringTok{"day"}\NormalTok{, color}\OperatorTok{=}\StringTok{"sex"}\NormalTok{, barmode}\OperatorTok{=}\StringTok{"group"}\NormalTok{)}
\end{Highlighting}
\end{Shaded}

\begin{verbatim}
Unable to display output for mime type(s): text/html
\end{verbatim}

This layout makes it easier to compare values across categories
directly.

\section{Time Series Data}\label{time-series-data}

Time series data represents observations collected at different points
in time. It's crucial for analyzing trends, patterns, and changes over
time. Let's explore some basic time series visualizations using
Nigeria's population data from the Gapminder dataset.

First, let's prepare our data:

\begin{Shaded}
\begin{Highlighting}[]
\CommentTok{\# Load the Gapminder dataset}
\NormalTok{gapminder }\OperatorTok{=}\NormalTok{ px.data.gapminder()}

\CommentTok{\# Subset the data for Nigeria}
\NormalTok{nigeria\_pop }\OperatorTok{=}\NormalTok{ gapminder.query(}\StringTok{\textquotesingle{}country == "Nigeria"\textquotesingle{}}\NormalTok{)[[}\StringTok{\textquotesingle{}year\textquotesingle{}}\NormalTok{, }\StringTok{\textquotesingle{}pop\textquotesingle{}}\NormalTok{]]}
\NormalTok{nigeria\_pop}
\end{Highlighting}
\end{Shaded}

\begin{longtable}[]{@{}lll@{}}
\toprule\noalign{}
& year & pop \\
\midrule\noalign{}
\endhead
\bottomrule\noalign{}
\endlastfoot
1128 & 1952 & 33119096 \\
1129 & 1957 & 37173340 \\
1130 & 1962 & 41871351 \\
1131 & 1967 & 47287752 \\
1132 & 1972 & 53740085 \\
1133 & 1977 & 62209173 \\
1134 & 1982 & 73039376 \\
1135 & 1987 & 81551520 \\
1136 & 1992 & 93364244 \\
1137 & 1997 & 106207839 \\
1138 & 2002 & 119901274 \\
1139 & 2007 & 135031164 \\
\end{longtable}

\subsection{Bar Chart}\label{bar-chart-1}

A bar chart can be used to plot time series data.

\begin{Shaded}
\begin{Highlighting}[]
\CommentTok{\# Bar chart}
\NormalTok{px.bar(nigeria\_pop, x}\OperatorTok{=}\StringTok{"year"}\NormalTok{, y}\OperatorTok{=}\StringTok{"pop"}\NormalTok{)}
\end{Highlighting}
\end{Shaded}

\begin{verbatim}
Unable to display output for mime type(s): text/html
\end{verbatim}

This bar chart gives us a clear view of how Nigeria's population has
changed over the years, with each bar representing the population at a
specific year.

\subsection{Line Chart}\label{line-chart}

A line chart is excellent for showing continuous changes over time:

\begin{Shaded}
\begin{Highlighting}[]
\CommentTok{\# Line chart}
\NormalTok{px.line(nigeria\_pop, x}\OperatorTok{=}\StringTok{"year"}\NormalTok{, y}\OperatorTok{=}\StringTok{"pop"}\NormalTok{)}
\end{Highlighting}
\end{Shaded}

\begin{verbatim}
Unable to display output for mime type(s): text/html
\end{verbatim}

The line chart connects the population values, making it easier to see
the overall trend of population growth.

Adding markers to a line chart can highlight specific data points:

\begin{Shaded}
\begin{Highlighting}[]
\CommentTok{\# Line chart with points}
\NormalTok{px.line(nigeria\_pop, x}\OperatorTok{=}\StringTok{\textquotesingle{}year\textquotesingle{}}\NormalTok{, y}\OperatorTok{=}\StringTok{\textquotesingle{}pop\textquotesingle{}}\NormalTok{, markers}\OperatorTok{=}\VariableTok{True}\NormalTok{)}
\end{Highlighting}
\end{Shaded}

\begin{verbatim}
Unable to display output for mime type(s): text/html
\end{verbatim}

We can also compare the population growth of multiple countries by
adding a \texttt{color} parameter:

\begin{Shaded}
\begin{Highlighting}[]
\NormalTok{nigeria\_ghana }\OperatorTok{=}\NormalTok{ gapminder.query(}\StringTok{\textquotesingle{}country in ["Nigeria", "Ghana"]\textquotesingle{}}\NormalTok{)}
\NormalTok{px.line(nigeria\_ghana, x}\OperatorTok{=}\StringTok{"year"}\NormalTok{, y}\OperatorTok{=}\StringTok{"pop"}\NormalTok{, color}\OperatorTok{=}\StringTok{"country"}\NormalTok{, markers}\OperatorTok{=}\VariableTok{True}\NormalTok{)}
\end{Highlighting}
\end{Shaded}

\begin{verbatim}
Unable to display output for mime type(s): text/html
\end{verbatim}

This chart allows us to compare the population trends of Nigeria and
Ghana over time.

\begin{tcolorbox}[enhanced jigsaw, colframe=quarto-callout-tip-color-frame, opacityback=0, titlerule=0mm, bottomrule=.15mm, breakable, leftrule=.75mm, colbacktitle=quarto-callout-tip-color!10!white, title=\textcolor{quarto-callout-tip-color}{\faLightbulb}\hspace{0.5em}{Practice}, rightrule=.15mm, coltitle=black, opacitybacktitle=0.6, colback=white, left=2mm, arc=.35mm, toptitle=1mm, bottomtitle=1mm, toprule=.15mm]

\subsection{Practice: GDP per Capita Time
Series}\label{practice-gdp-per-capita-time-series}

Using the Gapminder dataset, create a time series visualization for the
GDP per capita of Iraq.

\begin{Shaded}
\begin{Highlighting}[]
\CommentTok{\# Your code here}
\end{Highlighting}
\end{Shaded}

What happened to Iraq in the 1980s that might explain the graph shown?

\end{tcolorbox}

\section{Plots with three or more
variables}\label{plots-with-three-or-more-variables}

Although bivariate visualizations are the most common types of
visualizations, plots with three or more variables are also sometimes
useful. Let's explore a few examples.

\subsection{Bubble Charts}\label{bubble-charts}

Bubble charts show the relationship between three variables by mapping
the size of the points to a third variable. Below, we plot the
relationship between \texttt{gdpPercap} and \texttt{lifeExp} with the
size of the points representing the population of the country.

\begin{Shaded}
\begin{Highlighting}[]
\NormalTok{px.scatter(g\_2007, x}\OperatorTok{=}\StringTok{"gdpPercap"}\NormalTok{, y}\OperatorTok{=}\StringTok{"lifeExp"}\NormalTok{, size}\OperatorTok{=}\StringTok{"pop"}\NormalTok{)}
\end{Highlighting}
\end{Shaded}

\begin{verbatim}
Unable to display output for mime type(s): text/html
\end{verbatim}

We can easily spot the largest countries by population, such as China,
India, and the United States. We can also add a color axis to
differentiate between continents:

\begin{Shaded}
\begin{Highlighting}[]
\NormalTok{px.scatter(g\_2007, x}\OperatorTok{=}\StringTok{"gdpPercap"}\NormalTok{, y}\OperatorTok{=}\StringTok{"lifeExp"}\NormalTok{, size}\OperatorTok{=}\StringTok{"pop"}\NormalTok{, color}\OperatorTok{=}\StringTok{"continent"}\NormalTok{)}
\end{Highlighting}
\end{Shaded}

\begin{verbatim}
Unable to display output for mime type(s): text/html
\end{verbatim}

Now we have four different variables being plotted:

\begin{itemize}
\tightlist
\item
  \texttt{gdpPercap} on the x-axis
\item
  \texttt{lifeExp} on the y-axis
\item
  \texttt{pop} as the size of the points
\item
  \texttt{continent} as the color of the points
\end{itemize}

\begin{tcolorbox}[enhanced jigsaw, colframe=quarto-callout-tip-color-frame, opacityback=0, titlerule=0mm, bottomrule=.15mm, breakable, leftrule=.75mm, colbacktitle=quarto-callout-tip-color!10!white, title=\textcolor{quarto-callout-tip-color}{\faLightbulb}\hspace{0.5em}{Practice}, rightrule=.15mm, coltitle=black, opacitybacktitle=0.6, colback=white, left=2mm, arc=.35mm, toptitle=1mm, bottomtitle=1mm, toprule=.15mm]

\subsection{Practice: Tips Bubble
Chart}\label{practice-tips-bubble-chart}

Using the \texttt{tips} dataset, create a bubble chart showing the
relationship between \texttt{total\_bill} and \texttt{tip} with the size
of the points representing the \texttt{size} of the party, and the color
representing the \texttt{day} of the week.

Use the plot to answer the question:

\begin{itemize}
\tightlist
\item
  The highest two tip amounts were on which days and what was the table
  size?
\end{itemize}

\begin{Shaded}
\begin{Highlighting}[]
\NormalTok{tips.head()}
\CommentTok{\# Your code here}
\end{Highlighting}
\end{Shaded}

\begin{longtable}[]{@{}llllllll@{}}
\toprule\noalign{}
& total\_bill & tip & sex & smoker & day & time & size \\
\midrule\noalign{}
\endhead
\bottomrule\noalign{}
\endlastfoot
0 & 16.99 & 1.01 & Female & No & Sun & Dinner & 2 \\
1 & 10.34 & 1.66 & Male & No & Sun & Dinner & 3 \\
2 & 21.01 & 3.50 & Male & No & Sun & Dinner & 3 \\
3 & 23.68 & 3.31 & Male & No & Sun & Dinner & 2 \\
4 & 24.59 & 3.61 & Female & No & Sun & Dinner & 4 \\
\end{longtable}

\end{tcolorbox}

\subsection{Facet Plots}\label{facet-plots}

Faceting splits a single plot into multiple plots, with each plot
showing a different subset of the data. This is useful for comparing
distributions across subsets.

For example, we can facet the bubble chart by continent:

\begin{Shaded}
\begin{Highlighting}[]
\NormalTok{px.scatter(}
\NormalTok{    g\_2007,}
\NormalTok{    x}\OperatorTok{=}\StringTok{"gdpPercap"}\NormalTok{,}
\NormalTok{    y}\OperatorTok{=}\StringTok{"lifeExp"}\NormalTok{,}
\NormalTok{    size}\OperatorTok{=}\StringTok{"pop"}\NormalTok{,}
\NormalTok{    color}\OperatorTok{=}\StringTok{"continent"}\NormalTok{,}
\NormalTok{    facet\_col}\OperatorTok{=}\StringTok{"continent"}\NormalTok{,}
\NormalTok{)}
\end{Highlighting}
\end{Shaded}

\begin{verbatim}
Unable to display output for mime type(s): text/html
\end{verbatim}

We can change the arrangement of the facets by changing the
\texttt{facet\_col\_wrap} parameter. For example,
\texttt{facet\_col\_wrap=2} will wrap the facets into two columns:

\begin{Shaded}
\begin{Highlighting}[]
\NormalTok{px.scatter(}
\NormalTok{    g\_2007,}
\NormalTok{    x}\OperatorTok{=}\StringTok{"gdpPercap"}\NormalTok{,}
\NormalTok{    y}\OperatorTok{=}\StringTok{"lifeExp"}\NormalTok{,}
\NormalTok{    size}\OperatorTok{=}\StringTok{"pop"}\NormalTok{,}
\NormalTok{    color}\OperatorTok{=}\StringTok{"continent"}\NormalTok{,}
\NormalTok{    facet\_col}\OperatorTok{=}\StringTok{"continent"}\NormalTok{,}
\NormalTok{    facet\_col\_wrap}\OperatorTok{=}\DecValTok{2}\NormalTok{,}
\NormalTok{)}
\end{Highlighting}
\end{Shaded}

\begin{verbatim}
Unable to display output for mime type(s): text/html
\end{verbatim}

Similarly, we can facet the violin plots of tips by day of the week:

\begin{Shaded}
\begin{Highlighting}[]
\NormalTok{px.violin(}
\NormalTok{    tips,}
\NormalTok{    x}\OperatorTok{=}\StringTok{"sex"}\NormalTok{,}
\NormalTok{    y}\OperatorTok{=}\StringTok{"tip"}\NormalTok{,}
\NormalTok{    color}\OperatorTok{=}\StringTok{"sex"}\NormalTok{,}
\NormalTok{    facet\_col}\OperatorTok{=}\StringTok{"day"}\NormalTok{,}
\NormalTok{    facet\_col\_wrap}\OperatorTok{=}\DecValTok{2}\NormalTok{,}
\NormalTok{)}
\end{Highlighting}
\end{Shaded}

\begin{verbatim}
Unable to display output for mime type(s): text/html
\end{verbatim}

Faceting allows us to compare distributions across different days,
providing more granular insights.

\begin{tcolorbox}[enhanced jigsaw, colframe=quarto-callout-tip-color-frame, opacityback=0, titlerule=0mm, bottomrule=.15mm, breakable, leftrule=.75mm, colbacktitle=quarto-callout-tip-color!10!white, title=\textcolor{quarto-callout-tip-color}{\faLightbulb}\hspace{0.5em}{Practice}, rightrule=.15mm, coltitle=black, opacitybacktitle=0.6, colback=white, left=2mm, arc=.35mm, toptitle=1mm, bottomtitle=1mm, toprule=.15mm]

\subsection{Practice: Tips Facet Plot}\label{practice-tips-facet-plot}

Using the \texttt{tips} dataset, create a percent-stacked bar chart of
the \texttt{time} column, colored by the \texttt{sex} column, and
facetted by the \texttt{day} column.

Which day-time has the highest proportion of male customers (e.g.~Friday
Lunch, Saturday Dinner, etc.)?

\begin{Shaded}
\begin{Highlighting}[]
\NormalTok{tips.head()}
\CommentTok{\# Your code here}
\end{Highlighting}
\end{Shaded}

\begin{longtable}[]{@{}llllllll@{}}
\toprule\noalign{}
& total\_bill & tip & sex & smoker & day & time & size \\
\midrule\noalign{}
\endhead
\bottomrule\noalign{}
\endlastfoot
0 & 16.99 & 1.01 & Female & No & Sun & Dinner & 2 \\
1 & 10.34 & 1.66 & Male & No & Sun & Dinner & 3 \\
2 & 21.01 & 3.50 & Male & No & Sun & Dinner & 3 \\
3 & 23.68 & 3.31 & Male & No & Sun & Dinner & 2 \\
4 & 24.59 & 3.61 & Female & No & Sun & Dinner & 4 \\
\end{longtable}

\end{tcolorbox}

\section{Summary}\label{summary-1}

In this lesson, you learned how to create bivariate and multivariate
graphs using Plotly Express. Understanding these visualization
techniques will help you explore and communicate relationships in your
data more effectively.

See you in the next lesson!

\part{Data Manipulation}

\chapter{Subsetting columns}\label{subsetting-columns}

\section{Introduction}\label{introduction-6}

Today we will begin our exploration of pandas for data manipulation!

Our first focus will be on selecting and renaming columns. Often your
dataset comes with many columns that you do not need, and you would like
to narrow it down to just a few. Pandas makes this easy. Let's see how.

\section{Learning objectives}\label{learning-objectives-7}

\begin{itemize}
\tightlist
\item
  You can keep or drop columns from a DataFrame using pandas methods
  like \texttt{loc{[}{]}}, \texttt{filter()}, and \texttt{drop()}.
\item
  You can select columns based on regex patterns with \texttt{filter()}.
\item
  You can use \texttt{rename()} to change column names.
\item
  You can use regex to clean column names.
\end{itemize}

\section{About pandas}\label{about-pandas}

Pandas is a popular library for data manipulation and analysis. It is
designed to make it easy to work with tabular data in Python.

Install pandas with the following command in your terminal if it is not
already installed:

\begin{Shaded}
\begin{Highlighting}[]
\NormalTok{pip install pandas }
\end{Highlighting}
\end{Shaded}

Then import pandas with the following command in your script:

\begin{Shaded}
\begin{Highlighting}[]
\ImportTok{import}\NormalTok{ pandas }\ImportTok{as}\NormalTok{ pd}
\end{Highlighting}
\end{Shaded}

\section{The Yaounde COVID-19
dataset}\label{the-yaounde-covid-19-dataset}

In this lesson, we analyse results from a COVID-19 survey conducted in
Yaounde, Cameroon in late 2020. The survey estimated how many people had
been infected with COVID-19 in the region, by testing for antibodies.

You can find out more about this dataset here:
https://www.nature.com/articles/s41467-021-25946-0

Let's load and examine the dataset:

\begin{Shaded}
\begin{Highlighting}[]
\NormalTok{yao }\OperatorTok{=}\NormalTok{ pd.read\_csv(}\StringTok{"data/yaounde\_data.csv"}\NormalTok{)}
\NormalTok{yao}
\end{Highlighting}
\end{Shaded}

\begin{longtable}[]{@{}llllllllllllllllllllll@{}}
\toprule\noalign{}
& id & date\_surveyed & age & age\_category & age\_category\_3 & sex &
highest\_education & occupation & weight\_kg & height\_cm & ... &
is\_drug\_antibio & is\_drug\_hydrocortisone &
is\_drug\_other\_anti\_inflam & is\_drug\_antiviral & is\_drug\_chloro &
is\_drug\_tradn & is\_drug\_oxygen & is\_drug\_other &
is\_drug\_no\_resp & is\_drug\_none \\
\midrule\noalign{}
\endhead
\bottomrule\noalign{}
\endlastfoot
0 & BRIQUETERIE\_000\_0001 & 2020-10-22 & 45 & 45 - 64 & Adult & Female
& Secondary & Informal worker & 95 & 169 & ... & 0.0 & 0.0 & 0.0 & 0.0 &
0.0 & 0.0 & 0.0 & 0.0 & 0.0 & 0.0 \\
1 & BRIQUETERIE\_000\_0002 & 2020-10-24 & 55 & 45 - 64 & Adult & Male &
University & Salaried worker & 96 & 185 & ... & NaN & NaN & NaN & NaN &
NaN & NaN & NaN & NaN & NaN & NaN \\
... & ... & ... & ... & ... & ... & ... & ... & ... & ... & ... & ... &
... & ... & ... & ... & ... & ... & ... & ... & ... & ... \\
969 & TSINGAOLIGA\_026\_0002 & 2020-11-11 & 31 & 30 - 44 & Adult &
Female & Secondary & Unemployed & 66 & 169 & ... & NaN & NaN & NaN & NaN
& NaN & NaN & NaN & NaN & NaN & NaN \\
970 & TSINGAOLIGA\_026\_0003 & 2020-11-11 & 17 & 15 - 29 & Child &
Female & Secondary & Unemployed & 67 & 162 & ... & NaN & NaN & NaN & NaN
& NaN & NaN & NaN & NaN & NaN & NaN \\
\end{longtable}

\section{\texorpdfstring{Selecting columns with square brackets
\texttt{{[}{]}}}{Selecting columns with square brackets {[}{]}}}\label{selecting-columns-with-square-brackets}

In pandas, the most common way to select a column is simply to use
square brackets \texttt{{[}{]}} and the column name. For example, to
select the \texttt{age} and \texttt{sex} columns, we type:

\begin{Shaded}
\begin{Highlighting}[]
\NormalTok{yao[[}\StringTok{"age"}\NormalTok{, }\StringTok{"sex"}\NormalTok{]]}
\end{Highlighting}
\end{Shaded}

\begin{longtable}[]{@{}lll@{}}
\toprule\noalign{}
& age & sex \\
\midrule\noalign{}
\endhead
\bottomrule\noalign{}
\endlastfoot
0 & 45 & Female \\
1 & 55 & Male \\
... & ... & ... \\
969 & 31 & Female \\
970 & 17 & Female \\
\end{longtable}

Note the double square brackets \texttt{{[}{[}{]}{]}}. Without it, you
will get an error:

\begin{Shaded}
\begin{Highlighting}[]
\NormalTok{yao[}\StringTok{"age"}\NormalTok{, }\StringTok{"sex"}\NormalTok{]}
\end{Highlighting}
\end{Shaded}

\begin{verbatim}
KeyError: ('age', 'sex')
\end{verbatim}

If you want to select a single column, you \emph{may omit the double
square brackets}, but your output will no longer be a DataFrame. Compare
the following:

\begin{Shaded}
\begin{Highlighting}[]
\NormalTok{yao[}\StringTok{"age"}\NormalTok{] }\CommentTok{\# does not return a DataFrame}
\end{Highlighting}
\end{Shaded}

\begin{verbatim}
0      45
1      55
       ..
969    31
970    17
Name: age, Length: 971, dtype: int64
\end{verbatim}

\begin{Shaded}
\begin{Highlighting}[]
\NormalTok{yao[[}\StringTok{"age"}\NormalTok{]]  }\CommentTok{\# returns a DataFrame}
\end{Highlighting}
\end{Shaded}

\begin{longtable}[]{@{}ll@{}}
\toprule\noalign{}
& age \\
\midrule\noalign{}
\endhead
\bottomrule\noalign{}
\endlastfoot
0 & 45 \\
1 & 55 \\
... & ... \\
969 & 31 \\
970 & 17 \\
\end{longtable}

\begin{tcolorbox}[enhanced jigsaw, colframe=quarto-callout-note-color-frame, opacityback=0, titlerule=0mm, bottomrule=.15mm, breakable, leftrule=.75mm, colbacktitle=quarto-callout-note-color!10!white, title=\textcolor{quarto-callout-note-color}{\faInfo}\hspace{0.5em}{Key Point}, rightrule=.15mm, coltitle=black, opacitybacktitle=0.6, colback=white, left=2mm, arc=.35mm, toptitle=1mm, bottomtitle=1mm, toprule=.15mm]

\section{Storing data subsets}\label{storing-data-subsets}

Note that these selections are not modifying the DataFrame itself. If we
want a modified version, we create a new DataFrame to store the subset.
For example, below we create a subset with only three columns:

\begin{Shaded}
\begin{Highlighting}[]
\NormalTok{yao\_subset }\OperatorTok{=}\NormalTok{ yao[[}\StringTok{"age"}\NormalTok{, }\StringTok{"sex"}\NormalTok{, }\StringTok{"igg\_result"}\NormalTok{]]}
\NormalTok{yao\_subset}
\end{Highlighting}
\end{Shaded}

\begin{longtable}[]{@{}llll@{}}
\toprule\noalign{}
& age & sex & igg\_result \\
\midrule\noalign{}
\endhead
\bottomrule\noalign{}
\endlastfoot
0 & 45 & Female & Negative \\
1 & 55 & Male & Positive \\
... & ... & ... & ... \\
969 & 31 & Female & Negative \\
970 & 17 & Female & Negative \\
\end{longtable}

And if we want to overwrite a DataFrame, we can assign the subset back
to the original DataFrame. Let's overwrite the \texttt{yao\_subset}
DataFrame to have only the \texttt{age} column:

\begin{Shaded}
\begin{Highlighting}[]
\NormalTok{yao\_subset }\OperatorTok{=}\NormalTok{ yao\_subset[[}\StringTok{"age"}\NormalTok{]]}
\NormalTok{yao\_subset}
\end{Highlighting}
\end{Shaded}

\begin{longtable}[]{@{}ll@{}}
\toprule\noalign{}
& age \\
\midrule\noalign{}
\endhead
\bottomrule\noalign{}
\endlastfoot
0 & 45 \\
1 & 55 \\
... & ... \\
969 & 31 \\
970 & 17 \\
\end{longtable}

The \texttt{yao\_subset} DataFrame has gone from having 3 columns to
having 1 column.

\end{tcolorbox}

\begin{tcolorbox}[enhanced jigsaw, colframe=quarto-callout-tip-color-frame, opacityback=0, titlerule=0mm, bottomrule=.15mm, breakable, leftrule=.75mm, colbacktitle=quarto-callout-tip-color!10!white, title=\textcolor{quarto-callout-tip-color}{\faLightbulb}\hspace{0.5em}{Practice}, rightrule=.15mm, coltitle=black, opacitybacktitle=0.6, colback=white, left=2mm, arc=.35mm, toptitle=1mm, bottomtitle=1mm, toprule=.15mm]

\subsection{\texorpdfstring{Practice Q: Select Columns with
\texttt{{[}{]}}}{Practice Q: Select Columns with {[}{]}}}\label{practice-q-select-columns-with}

\begin{itemize}
\tightlist
\item
  Use the \texttt{{[}{]}} operator to select the ``weight\_kg'' and
  ``height\_cm'' variables in the \texttt{yao} DataFrame. Assign the
  result to a new DataFrame called \texttt{yao\_weight\_height}. Then
  print this new DataFrame.
\end{itemize}

\begin{Shaded}
\begin{Highlighting}[]
\CommentTok{\# Your code here}
\end{Highlighting}
\end{Shaded}

\end{tcolorbox}

\begin{tcolorbox}[enhanced jigsaw, colframe=quarto-callout-note-color-frame, opacityback=0, titlerule=0mm, bottomrule=.15mm, breakable, leftrule=.75mm, colbacktitle=quarto-callout-note-color!10!white, title=\textcolor{quarto-callout-note-color}{\faInfo}\hspace{0.5em}{Pro tip}, rightrule=.15mm, coltitle=black, opacitybacktitle=0.6, colback=white, left=2mm, arc=.35mm, toptitle=1mm, bottomtitle=1mm, toprule=.15mm]

There are many ways to select columns in pandas. In your free time, you
may choose to explore the \texttt{.loc{[}{]}} and \texttt{.take()}
methods, which provide additional functionality.

\end{tcolorbox}

\section{\texorpdfstring{Excluding columns with
\texttt{drop()}}{Excluding columns with drop()}}\label{excluding-columns-with-drop}

Sometimes it is more useful to drop columns you do not need than to
explicitly select the ones that you do need.

To drop columns, we can use the \texttt{drop()} method with the
\texttt{columns} argument. To drop the age column, we type:

\begin{Shaded}
\begin{Highlighting}[]
\NormalTok{yao.drop(columns}\OperatorTok{=}\NormalTok{[}\StringTok{"age"}\NormalTok{])}
\end{Highlighting}
\end{Shaded}

\begin{longtable}[]{@{}llllllllllllllllllllll@{}}
\toprule\noalign{}
& id & date\_surveyed & age\_category & age\_category\_3 & sex &
highest\_education & occupation & weight\_kg & height\_cm & is\_smoker &
... & is\_drug\_antibio & is\_drug\_hydrocortisone &
is\_drug\_other\_anti\_inflam & is\_drug\_antiviral & is\_drug\_chloro &
is\_drug\_tradn & is\_drug\_oxygen & is\_drug\_other &
is\_drug\_no\_resp & is\_drug\_none \\
\midrule\noalign{}
\endhead
\bottomrule\noalign{}
\endlastfoot
0 & BRIQUETERIE\_000\_0001 & 2020-10-22 & 45 - 64 & Adult & Female &
Secondary & Informal worker & 95 & 169 & Non-smoker & ... & 0.0 & 0.0 &
0.0 & 0.0 & 0.0 & 0.0 & 0.0 & 0.0 & 0.0 & 0.0 \\
1 & BRIQUETERIE\_000\_0002 & 2020-10-24 & 45 - 64 & Adult & Male &
University & Salaried worker & 96 & 185 & Ex-smoker & ... & NaN & NaN &
NaN & NaN & NaN & NaN & NaN & NaN & NaN & NaN \\
... & ... & ... & ... & ... & ... & ... & ... & ... & ... & ... & ... &
... & ... & ... & ... & ... & ... & ... & ... & ... & ... \\
969 & TSINGAOLIGA\_026\_0002 & 2020-11-11 & 30 - 44 & Adult & Female &
Secondary & Unemployed & 66 & 169 & Non-smoker & ... & NaN & NaN & NaN &
NaN & NaN & NaN & NaN & NaN & NaN & NaN \\
970 & TSINGAOLIGA\_026\_0003 & 2020-11-11 & 15 - 29 & Child & Female &
Secondary & Unemployed & 67 & 162 & Non-smoker & ... & NaN & NaN & NaN &
NaN & NaN & NaN & NaN & NaN & NaN & NaN \\
\end{longtable}

To drop several columns:

\begin{Shaded}
\begin{Highlighting}[]
\NormalTok{yao.drop(columns}\OperatorTok{=}\NormalTok{[}\StringTok{"age"}\NormalTok{, }\StringTok{"sex"}\NormalTok{])}
\end{Highlighting}
\end{Shaded}

\begin{longtable}[]{@{}llllllllllllllllllllll@{}}
\toprule\noalign{}
& id & date\_surveyed & age\_category & age\_category\_3 &
highest\_education & occupation & weight\_kg & height\_cm & is\_smoker &
is\_pregnant & ... & is\_drug\_antibio & is\_drug\_hydrocortisone &
is\_drug\_other\_anti\_inflam & is\_drug\_antiviral & is\_drug\_chloro &
is\_drug\_tradn & is\_drug\_oxygen & is\_drug\_other &
is\_drug\_no\_resp & is\_drug\_none \\
\midrule\noalign{}
\endhead
\bottomrule\noalign{}
\endlastfoot
0 & BRIQUETERIE\_000\_0001 & 2020-10-22 & 45 - 64 & Adult & Secondary &
Informal worker & 95 & 169 & Non-smoker & No & ... & 0.0 & 0.0 & 0.0 &
0.0 & 0.0 & 0.0 & 0.0 & 0.0 & 0.0 & 0.0 \\
1 & BRIQUETERIE\_000\_0002 & 2020-10-24 & 45 - 64 & Adult & University &
Salaried worker & 96 & 185 & Ex-smoker & NaN & ... & NaN & NaN & NaN &
NaN & NaN & NaN & NaN & NaN & NaN & NaN \\
... & ... & ... & ... & ... & ... & ... & ... & ... & ... & ... & ... &
... & ... & ... & ... & ... & ... & ... & ... & ... & ... \\
969 & TSINGAOLIGA\_026\_0002 & 2020-11-11 & 30 - 44 & Adult & Secondary
& Unemployed & 66 & 169 & Non-smoker & No & ... & NaN & NaN & NaN & NaN
& NaN & NaN & NaN & NaN & NaN & NaN \\
970 & TSINGAOLIGA\_026\_0003 & 2020-11-11 & 15 - 29 & Child & Secondary
& Unemployed & 67 & 162 & Non-smoker & No response & ... & NaN & NaN &
NaN & NaN & NaN & NaN & NaN & NaN & NaN & NaN \\
\end{longtable}

Again, note that this is not modifying the DataFrame itself. If we want
a modified version, we create a new DataFrame to store the subset. For
example, below we create a subset age and sex dropped:

\begin{Shaded}
\begin{Highlighting}[]
\NormalTok{yao\_subset }\OperatorTok{=}\NormalTok{ yao.drop(columns}\OperatorTok{=}\NormalTok{[}\StringTok{"age"}\NormalTok{, }\StringTok{"sex"}\NormalTok{])}
\NormalTok{yao\_subset}
\end{Highlighting}
\end{Shaded}

\begin{longtable}[]{@{}llllllllllllllllllllll@{}}
\toprule\noalign{}
& id & date\_surveyed & age\_category & age\_category\_3 &
highest\_education & occupation & weight\_kg & height\_cm & is\_smoker &
is\_pregnant & ... & is\_drug\_antibio & is\_drug\_hydrocortisone &
is\_drug\_other\_anti\_inflam & is\_drug\_antiviral & is\_drug\_chloro &
is\_drug\_tradn & is\_drug\_oxygen & is\_drug\_other &
is\_drug\_no\_resp & is\_drug\_none \\
\midrule\noalign{}
\endhead
\bottomrule\noalign{}
\endlastfoot
0 & BRIQUETERIE\_000\_0001 & 2020-10-22 & 45 - 64 & Adult & Secondary &
Informal worker & 95 & 169 & Non-smoker & No & ... & 0.0 & 0.0 & 0.0 &
0.0 & 0.0 & 0.0 & 0.0 & 0.0 & 0.0 & 0.0 \\
1 & BRIQUETERIE\_000\_0002 & 2020-10-24 & 45 - 64 & Adult & University &
Salaried worker & 96 & 185 & Ex-smoker & NaN & ... & NaN & NaN & NaN &
NaN & NaN & NaN & NaN & NaN & NaN & NaN \\
... & ... & ... & ... & ... & ... & ... & ... & ... & ... & ... & ... &
... & ... & ... & ... & ... & ... & ... & ... & ... & ... \\
969 & TSINGAOLIGA\_026\_0002 & 2020-11-11 & 30 - 44 & Adult & Secondary
& Unemployed & 66 & 169 & Non-smoker & No & ... & NaN & NaN & NaN & NaN
& NaN & NaN & NaN & NaN & NaN & NaN \\
970 & TSINGAOLIGA\_026\_0003 & 2020-11-11 & 15 - 29 & Child & Secondary
& Unemployed & 67 & 162 & Non-smoker & No response & ... & NaN & NaN &
NaN & NaN & NaN & NaN & NaN & NaN & NaN & NaN \\
\end{longtable}

\begin{tcolorbox}[enhanced jigsaw, colframe=quarto-callout-tip-color-frame, opacityback=0, titlerule=0mm, bottomrule=.15mm, breakable, leftrule=.75mm, colbacktitle=quarto-callout-tip-color!10!white, title=\textcolor{quarto-callout-tip-color}{\faLightbulb}\hspace{0.5em}{Practice}, rightrule=.15mm, coltitle=black, opacitybacktitle=0.6, colback=white, left=2mm, arc=.35mm, toptitle=1mm, bottomtitle=1mm, toprule=.15mm]

\subsection{\texorpdfstring{Practice Q: Drop Columns with
\texttt{drop()}}{Practice Q: Drop Columns with drop()}}\label{practice-q-drop-columns-with-drop}

\begin{itemize}
\tightlist
\item
  From the \texttt{yao} DataFrame, \textbf{remove} the columns
  \texttt{highest\_education} and \texttt{consultation}. Assign the
  result to a new DataFrame called
  \texttt{yao\_no\_education\_consultation}. Print this new DataFrame.
\end{itemize}

\begin{Shaded}
\begin{Highlighting}[]
\CommentTok{\# Your code here}
\end{Highlighting}
\end{Shaded}

\end{tcolorbox}

\section{\texorpdfstring{Using \texttt{filter()} to select columns by
regex}{Using filter() to select columns by regex}}\label{using-filter-to-select-columns-by-regex}

The \texttt{filter()} method and its \texttt{regex} argument offer a
powerful way to select columns based on patterns in their names. As an
example, to select columns containing the string ``ig'', we can write:

\begin{Shaded}
\begin{Highlighting}[]
\NormalTok{yao.}\BuiltInTok{filter}\NormalTok{(regex}\OperatorTok{=}\StringTok{"ig"}\NormalTok{)}
\end{Highlighting}
\end{Shaded}

\begin{longtable}[]{@{}llllllll@{}}
\toprule\noalign{}
& highest\_education & weight\_kg & height\_cm & neighborhood &
igg\_result & igm\_result & symp\_fatigue \\
\midrule\noalign{}
\endhead
\bottomrule\noalign{}
\endlastfoot
0 & Secondary & 95 & 169 & Briqueterie & Negative & Negative & No \\
1 & University & 96 & 185 & Briqueterie & Positive & Negative & No \\
... & ... & ... & ... & ... & ... & ... & ... \\
969 & Secondary & 66 & 169 & Tsinga Oliga & Negative & Negative & No \\
970 & Secondary & 67 & 162 & Tsinga Oliga & Negative & Negative & No \\
\end{longtable}

The argument \texttt{regex} specifies the pattern to match. Regex stands
for regular expression and refers to a sequence of characters that
define a search pattern.

To select columns \textbf{starting with} the string ``ig'', we write:

\begin{Shaded}
\begin{Highlighting}[]
\NormalTok{yao.}\BuiltInTok{filter}\NormalTok{(regex}\OperatorTok{=}\StringTok{"\^{}ig"}\NormalTok{)}
\end{Highlighting}
\end{Shaded}

\begin{longtable}[]{@{}lll@{}}
\toprule\noalign{}
& igg\_result & igm\_result \\
\midrule\noalign{}
\endhead
\bottomrule\noalign{}
\endlastfoot
0 & Negative & Negative \\
1 & Positive & Negative \\
... & ... & ... \\
969 & Negative & Negative \\
970 & Negative & Negative \\
\end{longtable}

The symbol \texttt{\^{}} is a regex character that matches the beginning
of the string.

To select columns \textbf{ending with} the string ``result'', we can
write:

\begin{Shaded}
\begin{Highlighting}[]
\NormalTok{yao.}\BuiltInTok{filter}\NormalTok{(regex}\OperatorTok{=}\StringTok{"result$"}\NormalTok{)}
\end{Highlighting}
\end{Shaded}

\begin{longtable}[]{@{}lll@{}}
\toprule\noalign{}
& igg\_result & igm\_result \\
\midrule\noalign{}
\endhead
\bottomrule\noalign{}
\endlastfoot
0 & Negative & Negative \\
1 & Positive & Negative \\
... & ... & ... \\
969 & Negative & Negative \\
970 & Negative & Negative \\
\end{longtable}

The character \texttt{\$} is regex that matches the end of the string.

\begin{tcolorbox}[enhanced jigsaw, colframe=quarto-callout-tip-color-frame, opacityback=0, titlerule=0mm, bottomrule=.15mm, breakable, leftrule=.75mm, colbacktitle=quarto-callout-tip-color!10!white, title=\textcolor{quarto-callout-tip-color}{\faLightbulb}\hspace{0.5em}{Practice}, rightrule=.15mm, coltitle=black, opacitybacktitle=0.6, colback=white, left=2mm, arc=.35mm, toptitle=1mm, bottomtitle=1mm, toprule=.15mm]

\subsection{Practice Q: Select Columns with
Regex}\label{practice-q-select-columns-with-regex}

\begin{itemize}
\tightlist
\item
  Select all columns in the \texttt{yao} DataFrame that start with
  ``is''. Assign the result to a new DataFrame called
  \texttt{yao\_is\_columns}. Then print this new DataFrame.
\end{itemize}

\begin{Shaded}
\begin{Highlighting}[]
\CommentTok{\# Your code here}
\end{Highlighting}
\end{Shaded}

\end{tcolorbox}

\section{\texorpdfstring{Change column names with
\texttt{rename()}}{Change column names with rename()}}\label{change-column-names-with-rename}

We can use the \texttt{rename()} method to change column names:

\begin{Shaded}
\begin{Highlighting}[]
\NormalTok{yao.rename(columns}\OperatorTok{=}\NormalTok{\{}\StringTok{"age"}\NormalTok{: }\StringTok{"patient\_age"}\NormalTok{, }\StringTok{"sex"}\NormalTok{: }\StringTok{"patient\_sex"}\NormalTok{\})}
\end{Highlighting}
\end{Shaded}

\begin{longtable}[]{@{}llllllllllllllllllllll@{}}
\toprule\noalign{}
& id & date\_surveyed & patient\_age & age\_category & age\_category\_3
& patient\_sex & highest\_education & occupation & weight\_kg &
height\_cm & ... & is\_drug\_antibio & is\_drug\_hydrocortisone &
is\_drug\_other\_anti\_inflam & is\_drug\_antiviral & is\_drug\_chloro &
is\_drug\_tradn & is\_drug\_oxygen & is\_drug\_other &
is\_drug\_no\_resp & is\_drug\_none \\
\midrule\noalign{}
\endhead
\bottomrule\noalign{}
\endlastfoot
0 & BRIQUETERIE\_000\_0001 & 2020-10-22 & 45 & 45 - 64 & Adult & Female
& Secondary & Informal worker & 95 & 169 & ... & 0.0 & 0.0 & 0.0 & 0.0 &
0.0 & 0.0 & 0.0 & 0.0 & 0.0 & 0.0 \\
1 & BRIQUETERIE\_000\_0002 & 2020-10-24 & 55 & 45 - 64 & Adult & Male &
University & Salaried worker & 96 & 185 & ... & NaN & NaN & NaN & NaN &
NaN & NaN & NaN & NaN & NaN & NaN \\
... & ... & ... & ... & ... & ... & ... & ... & ... & ... & ... & ... &
... & ... & ... & ... & ... & ... & ... & ... & ... & ... \\
969 & TSINGAOLIGA\_026\_0002 & 2020-11-11 & 31 & 30 - 44 & Adult &
Female & Secondary & Unemployed & 66 & 169 & ... & NaN & NaN & NaN & NaN
& NaN & NaN & NaN & NaN & NaN & NaN \\
970 & TSINGAOLIGA\_026\_0003 & 2020-11-11 & 17 & 15 - 29 & Child &
Female & Secondary & Unemployed & 67 & 162 & ... & NaN & NaN & NaN & NaN
& NaN & NaN & NaN & NaN & NaN & NaN \\
\end{longtable}

\begin{tcolorbox}[enhanced jigsaw, colframe=quarto-callout-tip-color-frame, opacityback=0, titlerule=0mm, bottomrule=.15mm, breakable, leftrule=.75mm, colbacktitle=quarto-callout-tip-color!10!white, title=\textcolor{quarto-callout-tip-color}{\faLightbulb}\hspace{0.5em}{Practice}, rightrule=.15mm, coltitle=black, opacitybacktitle=0.6, colback=white, left=2mm, arc=.35mm, toptitle=1mm, bottomtitle=1mm, toprule=.15mm]

\subsection{\texorpdfstring{Practice Q: Rename Columns with
\texttt{rename()}}{Practice Q: Rename Columns with rename()}}\label{practice-q-rename-columns-with-rename}

\begin{itemize}
\tightlist
\item
  Rename the \texttt{age\_category} column in the \texttt{yao} DataFrame
  to \texttt{age\_cat}. Assign the result to a new DataFrame called
  \texttt{yao\_age\_cat}. Then print this new DataFrame.
\end{itemize}

\begin{Shaded}
\begin{Highlighting}[]
\CommentTok{\# Your code here}
\end{Highlighting}
\end{Shaded}

\end{tcolorbox}

\section{Cleaning messy column names}\label{cleaning-messy-column-names}

For automatic cleaning of column names, you can use regular expressions
with the \texttt{str.replace()} method in pandas. This will allow you to
replace everything that is not a letter or a number with an underscore
(`\_').

Here's how you can do it on a test DataFrame with messy column names.
Messy column names are names with spaces, special characters, or other
non-alphanumeric characters.

\begin{Shaded}
\begin{Highlighting}[]
\NormalTok{test\_df }\OperatorTok{=}\NormalTok{ pd.DataFrame(\{}\StringTok{"good\_name"}\NormalTok{: [}\DecValTok{1}\NormalTok{], }\StringTok{"bad name"}\NormalTok{: [}\DecValTok{2}\NormalTok{], }\StringTok{"bad*@name*2"}\NormalTok{: [}\DecValTok{3}\NormalTok{]\})}
\NormalTok{test\_df}
\end{Highlighting}
\end{Shaded}

\begin{longtable}[]{@{}llll@{}}
\toprule\noalign{}
& good\_name & bad name & bad*@name*2 \\
\midrule\noalign{}
\endhead
\bottomrule\noalign{}
\endlastfoot
0 & 1 & 2 & 3 \\
\end{longtable}

Such column names are not ideal because, for example, we cannot select
them with the dot operator the way we can for clean names:

\begin{Shaded}
\begin{Highlighting}[]
\NormalTok{test\_df.good\_name  }\CommentTok{\# this works}
\end{Highlighting}
\end{Shaded}

\begin{verbatim}
0    1
Name: good_name, dtype: int64
\end{verbatim}

But this does not work:

\begin{Shaded}
\begin{Highlighting}[]
\NormalTok{test\_df.bad name}
\end{Highlighting}
\end{Shaded}

\begin{verbatim}
      test_df.bad name
                 ^
SyntaxError: invalid syntax
\end{verbatim}

We can automatically clean such names using the \texttt{str.replace()}
method along with regular expressions.

\begin{Shaded}
\begin{Highlighting}[]
\NormalTok{clean\_names }\OperatorTok{=}\NormalTok{ test\_df.columns.}\BuiltInTok{str}\NormalTok{.replace(}\VerbatimStringTok{r\textquotesingle{}[\^{}a{-}zA{-}Z0{-}9]\textquotesingle{}}\NormalTok{, }\StringTok{\textquotesingle{}\_\textquotesingle{}}\NormalTok{, regex}\OperatorTok{=}\VariableTok{True}\NormalTok{)}
\end{Highlighting}
\end{Shaded}

The regular expression
\texttt{r\textquotesingle{}{[}\^{}a-zA-Z0-9{]}\textquotesingle{}}
matches any character that is not a letter (either uppercase or
lowercase) or a digit. The \texttt{str.replace()} method replaces these
characters with an underscore (`\_') to make the column names more
legible and usable in dot notation.

Now we can replace the column names in the DataFrame with the cleaned
names:

\begin{Shaded}
\begin{Highlighting}[]
\NormalTok{test\_df.columns }\OperatorTok{=}\NormalTok{ clean\_names}
\NormalTok{test\_df}
\end{Highlighting}
\end{Shaded}

\begin{longtable}[]{@{}llll@{}}
\toprule\noalign{}
& good\_name & bad\_name & bad\_\_name\_2 \\
\midrule\noalign{}
\endhead
\bottomrule\noalign{}
\endlastfoot
0 & 1 & 2 & 3 \\
\end{longtable}

\begin{tcolorbox}[enhanced jigsaw, colframe=quarto-callout-tip-color-frame, opacityback=0, titlerule=0mm, bottomrule=.15mm, breakable, leftrule=.75mm, colbacktitle=quarto-callout-tip-color!10!white, title=\textcolor{quarto-callout-tip-color}{\faLightbulb}\hspace{0.5em}{Practice}, rightrule=.15mm, coltitle=black, opacitybacktitle=0.6, colback=white, left=2mm, arc=.35mm, toptitle=1mm, bottomtitle=1mm, toprule=.15mm]

\subsection{Practice Q: Clean Column Names with
Regex}\label{practice-q-clean-column-names-with-regex}

\begin{itemize}
\tightlist
\item
  Consider the data frame defined below with messy column names. Use the
  \texttt{str.replace()} method to clean the column names.
\end{itemize}

\begin{Shaded}
\begin{Highlighting}[]
\NormalTok{cleaning\_practice }\OperatorTok{=}\NormalTok{ pd.DataFrame(}
\NormalTok{    \{}\StringTok{"Aloha"}\NormalTok{: }\BuiltInTok{range}\NormalTok{(}\DecValTok{3}\NormalTok{), }\StringTok{"Bell Chart"}\NormalTok{: }\BuiltInTok{range}\NormalTok{(}\DecValTok{3}\NormalTok{), }\StringTok{"Animals@\#$\%\^{}"}\NormalTok{: }\BuiltInTok{range}\NormalTok{(}\DecValTok{3}\NormalTok{)\}}
\NormalTok{)}
\NormalTok{cleaning\_practice}
\end{Highlighting}
\end{Shaded}

\begin{longtable}[]{@{}llll@{}}
\toprule\noalign{}
& Aloha & Bell Chart & Animals@\#\$\%\^{} \\
\midrule\noalign{}
\endhead
\bottomrule\noalign{}
\endlastfoot
0 & 0 & 0 & 0 \\
1 & 1 & 1 & 1 \\
2 & 2 & 2 & 2 \\
\end{longtable}

\end{tcolorbox}

\begin{center}\rule{0.5\linewidth}{0.5pt}\end{center}

\section{Wrap up}\label{wrap-up-4}

Hopefully this lesson has shown you how intuitive and useful pandas is
for data manipulation!

This is the first of a series of basic data wrangling techniques: see
you in the next lesson to learn more.

\chapter{Subsetting rows}\label{subsetting-rows}

\section{Intro}\label{intro-2}

In our previous lesson we saw how to select variables (columns). In this
lesson we will see how to keep or drop data entries.

Dropping abnormal data entries or keeping subsets of your data points is
another essential aspect of data wrangling.

Let's get started!

\section{Learning objectives}\label{learning-objectives-8}

\begin{enumerate}
\def\labelenumi{\arabic{enumi}.}
\tightlist
\item
  You can use the \texttt{query()} method to keep or drop rows from a
  DataFrame.
\item
  You can specify conditions using relational operators like greater
  than (\texttt{\textgreater{}}), less than (\texttt{\textless{}}),
  equal to (\texttt{==}), and not equal to (\texttt{!=}).
\item
  You can combine conditions with \texttt{\&} and \texttt{\textbar{}}.
\item
  You can negate conditions with \texttt{\textasciitilde{}}.
\item
  You can use the \texttt{isna()} and \texttt{notna()} methods.
\end{enumerate}

\section{The Yaounde COVID-19
dataset}\label{the-yaounde-covid-19-dataset-1}

In this lesson, we will again use the data from the COVID-19 serological
survey conducted in Yaounde, Cameroon.

You can find out more about this dataset here:
https://www.nature.com/articles/s41467-021-25946-0

Let's load the data into a pandas DataFrame.

\begin{Shaded}
\begin{Highlighting}[]
\ImportTok{import}\NormalTok{ pandas }\ImportTok{as}\NormalTok{ pd}

\NormalTok{yaounde }\OperatorTok{=}\NormalTok{ pd.read\_csv(}\StringTok{"data/yaounde\_data.csv"}\NormalTok{)}
\CommentTok{\# a smaller subset of variables}
\NormalTok{yao }\OperatorTok{=}\NormalTok{ yaounde[}
\NormalTok{    [}
        \StringTok{"age"}\NormalTok{,}
        \StringTok{"sex"}\NormalTok{,}
        \StringTok{"weight\_kg"}\NormalTok{,}
        \StringTok{"highest\_education"}\NormalTok{,}
        \StringTok{"neighborhood"}\NormalTok{,}
        \StringTok{"occupation"}\NormalTok{,}
        \StringTok{"symptoms"}\NormalTok{,}
        \StringTok{"is\_smoker"}\NormalTok{,}
        \StringTok{"is\_pregnant"}\NormalTok{,}
        \StringTok{"igg\_result"}\NormalTok{,}
        \StringTok{"igm\_result"}\NormalTok{,}
\NormalTok{    ]}
\NormalTok{]}
\NormalTok{yao.head()}
\end{Highlighting}
\end{Shaded}

\begin{longtable}[]{@{}llllllllllll@{}}
\toprule\noalign{}
& age & sex & weight\_kg & highest\_education & neighborhood &
occupation & symptoms & is\_smoker & is\_pregnant & igg\_result &
igm\_result \\
\midrule\noalign{}
\endhead
\bottomrule\noalign{}
\endlastfoot
0 & 45 & Female & 95 & Secondary & Briqueterie & Informal worker &
Muscle pain & Non-smoker & No & Negative & Negative \\
1 & 55 & Male & 96 & University & Briqueterie & Salaried worker & No
symptoms & Ex-smoker & NaN & Positive & Negative \\
2 & 23 & Male & 74 & University & Briqueterie & Student & No symptoms &
Smoker & NaN & Negative & Negative \\
3 & 20 & Female & 70 & Secondary & Briqueterie & Student &
Rhinitis-\/-Sneezing-\/-Anosmia or ageusia & Non-smoker & No & Positive
& Negative \\
4 & 55 & Female & 67 & Primary & Briqueterie & Trader-\/-Farmer & No
symptoms & Non-smoker & No & Positive & Negative \\
\end{longtable}

\section{\texorpdfstring{Introducing
\texttt{query()}}{Introducing query()}}\label{introducing-query}

We can use the \texttt{query()} method to keep rows that satisfy a set
of conditions. Let's take a look at a simple example. If we want to keep
just the male records, we run:

\begin{Shaded}
\begin{Highlighting}[]
\NormalTok{yao.query(}\StringTok{\textquotesingle{}sex == "Male"\textquotesingle{}}\NormalTok{)}
\end{Highlighting}
\end{Shaded}

\begin{longtable}[]{@{}llllllllllll@{}}
\toprule\noalign{}
& age & sex & weight\_kg & highest\_education & neighborhood &
occupation & symptoms & is\_smoker & is\_pregnant & igg\_result &
igm\_result \\
\midrule\noalign{}
\endhead
\bottomrule\noalign{}
\endlastfoot
1 & 55 & Male & 96 & University & Briqueterie & Salaried worker & No
symptoms & Ex-smoker & NaN & Positive & Negative \\
2 & 23 & Male & 74 & University & Briqueterie & Student & No symptoms &
Smoker & NaN & Negative & Negative \\
... & ... & ... & ... & ... & ... & ... & ... & ... & ... & ... & ... \\
966 & 32 & Male & 54 & Secondary & Tsinga Oliga & Informal worker &
Rhinitis-\/-Sneezing-\/-Diarrhoea & Smoker & NaN & Negative &
Negative \\
968 & 35 & Male & 77 & University & Tsinga Oliga & Informal worker &
Headache & Smoker & NaN & Positive & Negative \\
\end{longtable}

As you can see, the \texttt{query()} syntax is quite simple. It may be a
bit surprising to have to put code in quotes, but it is quite readable.

Note the use of double equals (\texttt{==}) instead of single equals
(\texttt{=}) there. The \texttt{==} sign tests for equality, while the
single equals sign assigns a value. This is a common source of errors
when you are a beginner, so watch out for it.

We can chain \texttt{query()} with \texttt{shape{[}0{]}} to count the
number of male respondents.

\begin{Shaded}
\begin{Highlighting}[]
\NormalTok{yao.query(}\StringTok{\textquotesingle{}sex == "Male"\textquotesingle{}}\NormalTok{).shape[}\DecValTok{0}\NormalTok{]}
\end{Highlighting}
\end{Shaded}

\begin{verbatim}
422
\end{verbatim}

\begin{tcolorbox}[enhanced jigsaw, colframe=quarto-callout-note-color-frame, opacityback=0, titlerule=0mm, bottomrule=.15mm, breakable, leftrule=.75mm, colbacktitle=quarto-callout-note-color!10!white, title=\textcolor{quarto-callout-note-color}{\faInfo}\hspace{0.5em}{Reminder}, rightrule=.15mm, coltitle=black, opacitybacktitle=0.6, colback=white, left=2mm, arc=.35mm, toptitle=1mm, bottomtitle=1mm, toprule=.15mm]

Recall that the \texttt{shape} property returns the number of rows and
columns in a DataFrame. The first element, \texttt{shape{[}0{]}}, is the
number of rows

\end{tcolorbox}

\begin{tcolorbox}[enhanced jigsaw, colframe=quarto-callout-note-color-frame, opacityback=0, titlerule=0mm, bottomrule=.15mm, breakable, leftrule=.75mm, colbacktitle=quarto-callout-note-color!10!white, title=\textcolor{quarto-callout-note-color}{\faInfo}\hspace{0.5em}{Key Point}, rightrule=.15mm, coltitle=black, opacitybacktitle=0.6, colback=white, left=2mm, arc=.35mm, toptitle=1mm, bottomtitle=1mm, toprule=.15mm]

Note that these subsets are not modifying the DataFrame itself. If we
want a modified version, we create a new DataFrame to store the subset.
For example, below we create a subset of male respondents:

\begin{Shaded}
\begin{Highlighting}[]
\NormalTok{yao\_male }\OperatorTok{=}\NormalTok{ yao.query(}\StringTok{\textquotesingle{}sex == "Male"\textquotesingle{}}\NormalTok{)}
\NormalTok{yao\_male}
\end{Highlighting}
\end{Shaded}

\begin{longtable}[]{@{}llllllllllll@{}}
\toprule\noalign{}
& age & sex & weight\_kg & highest\_education & neighborhood &
occupation & symptoms & is\_smoker & is\_pregnant & igg\_result &
igm\_result \\
\midrule\noalign{}
\endhead
\bottomrule\noalign{}
\endlastfoot
1 & 55 & Male & 96 & University & Briqueterie & Salaried worker & No
symptoms & Ex-smoker & NaN & Positive & Negative \\
2 & 23 & Male & 74 & University & Briqueterie & Student & No symptoms &
Smoker & NaN & Negative & Negative \\
... & ... & ... & ... & ... & ... & ... & ... & ... & ... & ... & ... \\
966 & 32 & Male & 54 & Secondary & Tsinga Oliga & Informal worker &
Rhinitis-\/-Sneezing-\/-Diarrhoea & Smoker & NaN & Negative &
Negative \\
968 & 35 & Male & 77 & University & Tsinga Oliga & Informal worker &
Headache & Smoker & NaN & Positive & Negative \\
\end{longtable}

But for ease of explanation, in the examples below, we are simply
printing the result, without storing it in a variable.

\end{tcolorbox}

\begin{tcolorbox}[enhanced jigsaw, colframe=quarto-callout-tip-color-frame, opacityback=0, titlerule=0mm, bottomrule=.15mm, breakable, leftrule=.75mm, colbacktitle=quarto-callout-tip-color!10!white, title=\textcolor{quarto-callout-tip-color}{\faLightbulb}\hspace{0.5em}{Practice}, rightrule=.15mm, coltitle=black, opacitybacktitle=0.6, colback=white, left=2mm, arc=.35mm, toptitle=1mm, bottomtitle=1mm, toprule=.15mm]

\subsection{Practice Q: Subset for Pregnant
Respondents}\label{practice-q-subset-for-pregnant-respondents}

Subset the \texttt{yao} data frame to respondents who were pregnant
during the survey. Assign the result to a new DataFrame called
\texttt{yao\_pregnant}. Then print this new DataFrame. There should be
24 rows.

\begin{Shaded}
\begin{Highlighting}[]
\CommentTok{\# Your code here}
\end{Highlighting}
\end{Shaded}

\end{tcolorbox}

\section{Relational operators}\label{relational-operators}

The \texttt{==} operator introduced above is an example of a
``relational'' operator, as it tests the relation between two values.
Here is a list of some more of these operators. You will use these often
when you are querying rows in your data.

\begin{longtable}[]{@{}ll@{}}
\toprule\noalign{}
\endhead
\bottomrule\noalign{}
\endlastfoot
\textbf{Operator} & \textbf{is True if} \\
A \textless{} B & A is \textbf{less than} B \\
A \textless= B & A is \textbf{less than or equal} to B \\
A \textgreater{} B & A is \textbf{greater than} B \\
A \textgreater= B & A is \textbf{greater than or equal to} B \\
A == B & A is \textbf{equal} to B \\
A != B & A is \textbf{not equal} to B \\
A.isin(B) & A \textbf{is an element of} B \\
\end{longtable}

Let's see how to use these with \texttt{query()}:

\begin{Shaded}
\begin{Highlighting}[]
\NormalTok{yao.query(}\StringTok{\textquotesingle{}sex == "Female"\textquotesingle{}}\NormalTok{)  }\CommentTok{\# keep rows where \textasciigrave{}sex\textasciigrave{} is female}
\NormalTok{yao.query(}\StringTok{\textquotesingle{}sex != "Male"\textquotesingle{}}\NormalTok{)  }\CommentTok{\# keep rows where \textasciigrave{}sex\textasciigrave{} is not "Male"}
\NormalTok{yao.query(}\StringTok{"age \textless{} 6"}\NormalTok{)  }\CommentTok{\# keep respondents under 6}
\NormalTok{yao.query(}\StringTok{"age \textgreater{}= 70"}\NormalTok{)  }\CommentTok{\# keep respondents aged at least 70}

\CommentTok{\# keep respondents whose highest education is "Primary" or "Secondary"}
\NormalTok{yao.query(}\StringTok{\textquotesingle{}highest\_education.isin(["Primary", "Secondary"])\textquotesingle{}}\NormalTok{)}
\end{Highlighting}
\end{Shaded}

\begin{longtable}[]{@{}llllllllllll@{}}
\toprule\noalign{}
& age & sex & weight\_kg & highest\_education & neighborhood &
occupation & symptoms & is\_smoker & is\_pregnant & igg\_result &
igm\_result \\
\midrule\noalign{}
\endhead
\bottomrule\noalign{}
\endlastfoot
0 & 45 & Female & 95 & Secondary & Briqueterie & Informal worker &
Muscle pain & Non-smoker & No & Negative & Negative \\
3 & 20 & Female & 70 & Secondary & Briqueterie & Student &
Rhinitis-\/-Sneezing-\/-Anosmia or ageusia & Non-smoker & No & Positive
& Negative \\
... & ... & ... & ... & ... & ... & ... & ... & ... & ... & ... & ... \\
969 & 31 & Female & 66 & Secondary & Tsinga Oliga & Unemployed & No
symptoms & Non-smoker & No & Negative & Negative \\
970 & 17 & Female & 67 & Secondary & Tsinga Oliga & Unemployed & No
symptoms & Non-smoker & No response & Negative & Negative \\
\end{longtable}

\begin{tcolorbox}[enhanced jigsaw, colframe=quarto-callout-tip-color-frame, opacityback=0, titlerule=0mm, bottomrule=.15mm, breakable, leftrule=.75mm, colbacktitle=quarto-callout-tip-color!10!white, title=\textcolor{quarto-callout-tip-color}{\faLightbulb}\hspace{0.5em}{Practice}, rightrule=.15mm, coltitle=black, opacitybacktitle=0.6, colback=white, left=2mm, arc=.35mm, toptitle=1mm, bottomtitle=1mm, toprule=.15mm]

\subsection{Practice Q: Subset for
Children}\label{practice-q-subset-for-children}

\begin{itemize}
\tightlist
\item
  From \texttt{yao}, keep only respondents who were children (under 18).
  Assign the result to a new DataFrame called \texttt{yao\_children}.
  There should be 291 rows.
\end{itemize}

\begin{Shaded}
\begin{Highlighting}[]
\CommentTok{\# Your code here}
\end{Highlighting}
\end{Shaded}

\end{tcolorbox}

\begin{tcolorbox}[enhanced jigsaw, colframe=quarto-callout-tip-color-frame, opacityback=0, titlerule=0mm, bottomrule=.15mm, breakable, leftrule=.75mm, colbacktitle=quarto-callout-tip-color!10!white, title=\textcolor{quarto-callout-tip-color}{\faLightbulb}\hspace{0.5em}{Practice}, rightrule=.15mm, coltitle=black, opacitybacktitle=0.6, colback=white, left=2mm, arc=.35mm, toptitle=1mm, bottomtitle=1mm, toprule=.15mm]

\subsection{Practice Q: Subset for Tsinga and
Messa}\label{practice-q-subset-for-tsinga-and-messa}

\begin{itemize}
\tightlist
\item
  With \texttt{isin()}, keep only respondents who live in the ``Tsinga''
  or ``Messa'' neighborhoods. Assign the result to a new DataFrame
  called \texttt{yao\_tsinga\_messa}. There should be 129 rows
\end{itemize}

\begin{Shaded}
\begin{Highlighting}[]
\CommentTok{\# Your code here}
\end{Highlighting}
\end{Shaded}

\end{tcolorbox}

\section{\texorpdfstring{Accessing external variables in
\texttt{query()}}{Accessing external variables in query()}}\label{accessing-external-variables-in-query}

The \texttt{query()} method allows you to access variables outside the
DataFrame using the \texttt{@} symbol. This is useful when you want to
use dynamic values in your query conditions.

For example, say you have the variable \texttt{min\_age} that you want
to use in your query. You can do this as follows:

\begin{Shaded}
\begin{Highlighting}[]
\NormalTok{min\_age }\OperatorTok{=} \DecValTok{25}

\CommentTok{\# Query using external variables}
\NormalTok{yao.query(}\StringTok{\textquotesingle{}age \textgreater{}= @min\_age\textquotesingle{}}\NormalTok{)}
\end{Highlighting}
\end{Shaded}

\begin{longtable}[]{@{}llllllllllll@{}}
\toprule\noalign{}
& age & sex & weight\_kg & highest\_education & neighborhood &
occupation & symptoms & is\_smoker & is\_pregnant & igg\_result &
igm\_result \\
\midrule\noalign{}
\endhead
\bottomrule\noalign{}
\endlastfoot
0 & 45 & Female & 95 & Secondary & Briqueterie & Informal worker &
Muscle pain & Non-smoker & No & Negative & Negative \\
1 & 55 & Male & 96 & University & Briqueterie & Salaried worker & No
symptoms & Ex-smoker & NaN & Positive & Negative \\
... & ... & ... & ... & ... & ... & ... & ... & ... & ... & ... & ... \\
968 & 35 & Male & 77 & University & Tsinga Oliga & Informal worker &
Headache & Smoker & NaN & Positive & Negative \\
969 & 31 & Female & 66 & Secondary & Tsinga Oliga & Unemployed & No
symptoms & Non-smoker & No & Negative & Negative \\
\end{longtable}

This feature is helpful when you need to filter data based on values
that may change or are determined at runtime.

\begin{tcolorbox}[enhanced jigsaw, colframe=quarto-callout-tip-color-frame, opacityback=0, titlerule=0mm, bottomrule=.15mm, breakable, leftrule=.75mm, colbacktitle=quarto-callout-tip-color!10!white, title=\textcolor{quarto-callout-tip-color}{\faLightbulb}\hspace{0.5em}{Practice}, rightrule=.15mm, coltitle=black, opacitybacktitle=0.6, colback=white, left=2mm, arc=.35mm, toptitle=1mm, bottomtitle=1mm, toprule=.15mm]

\subsection{Practice Q: Subset for Young
Respondents}\label{practice-q-subset-for-young-respondents}

\begin{itemize}
\tightlist
\item
  From \texttt{yao}, keep respondents who are less than or equal to the
  variable \texttt{max\_age}, defined below. Assign the result to a new
  DataFrame called \texttt{yao\_young}. There should be 590 rows.
\end{itemize}

\begin{Shaded}
\begin{Highlighting}[]
\NormalTok{max\_age }\OperatorTok{=} \DecValTok{30}
\CommentTok{\# Your code here}
\end{Highlighting}
\end{Shaded}

\end{tcolorbox}

\section{\texorpdfstring{Combining conditions with \texttt{\&} and
\texttt{\textbar{}}}{Combining conditions with \& and \textbar{}}}\label{combining-conditions-with-and}

We can pass multiple conditions to \texttt{query()} using \texttt{\&}
(the ``ampersand'' symbol) for AND and \texttt{\textbar{}} (the
``vertical bar'' or ``pipe'' symbol) for OR.

For example, to keep respondents who are either younger than 18 OR older
than 65, we can write:

\begin{Shaded}
\begin{Highlighting}[]
\NormalTok{yao.query(}\StringTok{"age \textless{} 18 | age \textgreater{} 65"}\NormalTok{)}
\end{Highlighting}
\end{Shaded}

\begin{longtable}[]{@{}llllllllllll@{}}
\toprule\noalign{}
& age & sex & weight\_kg & highest\_education & neighborhood &
occupation & symptoms & is\_smoker & is\_pregnant & igg\_result &
igm\_result \\
\midrule\noalign{}
\endhead
\bottomrule\noalign{}
\endlastfoot
5 & 17 & Female & 65 & Secondary & Briqueterie & Student &
Fever-\/-Cough-\/-Rhinitis-\/-Nausea or vomiting-\/-Di... & Non-smoker &
No & Negative & Negative \\
6 & 13 & Female & 65 & Secondary & Briqueterie & Student & Sneezing &
Non-smoker & No & Positive & Negative \\
... & ... & ... & ... & ... & ... & ... & ... & ... & ... & ... & ... \\
962 & 15 & Male & 44 & Secondary & Tsinga Oliga & Student &
Fever-\/-Cough-\/-Rhinitis & Non-smoker & NaN & Positive & Negative \\
970 & 17 & Female & 67 & Secondary & Tsinga Oliga & Unemployed & No
symptoms & Non-smoker & No response & Negative & Negative \\
\end{longtable}

To keep respondents who are pregnant AND are ex-smokers, we write:

\begin{Shaded}
\begin{Highlighting}[]
\NormalTok{yao.query(}\StringTok{\textquotesingle{}is\_pregnant == "Yes" \& is\_smoker == "Ex{-}smoker"\textquotesingle{}}\NormalTok{)}
\end{Highlighting}
\end{Shaded}

\begin{longtable}[]{@{}llllllllllll@{}}
\toprule\noalign{}
& age & sex & weight\_kg & highest\_education & neighborhood &
occupation & symptoms & is\_smoker & is\_pregnant & igg\_result &
igm\_result \\
\midrule\noalign{}
\endhead
\bottomrule\noalign{}
\endlastfoot
273 & 25 & Female & 90 & Secondary & Carriere & Home-maker &
Cough-\/-Rhinitis-\/-Sneezing & Ex-smoker & Yes & Positive & Negative \\
\end{longtable}

There is only one person who is both pregnant and ex-smoker.

\begin{tcolorbox}[enhanced jigsaw, colframe=quarto-callout-tip-color-frame, opacityback=0, titlerule=0mm, bottomrule=.15mm, breakable, leftrule=.75mm, colbacktitle=quarto-callout-tip-color!10!white, title=\textcolor{quarto-callout-tip-color}{\faLightbulb}\hspace{0.5em}{Practice}, rightrule=.15mm, coltitle=black, opacitybacktitle=0.6, colback=white, left=2mm, arc=.35mm, toptitle=1mm, bottomtitle=1mm, toprule=.15mm]

\subsection{Practice Q: Subset for IgG Positive
Men}\label{practice-q-subset-for-igg-positive-men}

Subset \texttt{yao} to only keep men who tested IgG positive. Assign the
result to a new DataFrame called \texttt{yao\_igg\_positive\_men}. There
should be 148 rows after your filter. Think carefully about whether to
use \texttt{\&} or \texttt{\textbar{}}.

\begin{Shaded}
\begin{Highlighting}[]
\CommentTok{\# Your code here}
\end{Highlighting}
\end{Shaded}

Subset \texttt{yao} to keep both children (under 18) and anyone whose
highest education is primary school. Assign the result to a new
DataFrame called \texttt{yao\_children\_primary}. There should be 444
rows after your filter. Think carefully about \texttt{\&} vs
\texttt{\textbar{}}. It may be counterintuitive!

\begin{Shaded}
\begin{Highlighting}[]
\CommentTok{\# Your code here}
\end{Highlighting}
\end{Shaded}

\end{tcolorbox}

Note that we can also chain queries together instead of using
\texttt{\&} or \texttt{\textbar{}}. For example, to keep respondents who
are above 18, who smoke and who live in Tsinga, we could write:

\begin{Shaded}
\begin{Highlighting}[]
\NormalTok{yao.query(}\StringTok{\textquotesingle{}age \textgreater{} 18 \& is\_smoker == "Ex{-}smoker" \& neighborhood == "Tsinga"\textquotesingle{}}\NormalTok{)}
\end{Highlighting}
\end{Shaded}

\begin{longtable}[]{@{}llllllllllll@{}}
\toprule\noalign{}
& age & sex & weight\_kg & highest\_education & neighborhood &
occupation & symptoms & is\_smoker & is\_pregnant & igg\_result &
igm\_result \\
\midrule\noalign{}
\endhead
\bottomrule\noalign{}
\endlastfoot
827 & 24 & Male & 85 & Secondary & Tsinga & Student-\/-Informal worker &
No symptoms & Ex-smoker & NaN & Positive & Positive \\
835 & 22 & Male & 70 & University & Tsinga & Student &
Fever-\/-Headache-\/-Fatigue & Ex-smoker & NaN & Positive & Negative \\
... & ... & ... & ... & ... & ... & ... & ... & ... & ... & ... & ... \\
899 & 29 & Female & 64 & Doctorate & Tsinga & Informal worker &
Fever-\/-Headache-\/-Anosmia or ageusia & Ex-smoker & No & Positive &
Negative \\
903 & 28 & Male & 69 & Secondary & Tsinga & Trader &
Sneezing-\/-Headache & Ex-smoker & NaN & Negative & Negative \\
\end{longtable}

But we could also write:

\begin{Shaded}
\begin{Highlighting}[]
\NormalTok{(yao}
\NormalTok{.query(}\StringTok{\textquotesingle{}age \textgreater{} 18\textquotesingle{}}\NormalTok{)}
\NormalTok{.query(}\StringTok{\textquotesingle{}is\_smoker == "Ex{-}smoker"\textquotesingle{}}\NormalTok{)}
\NormalTok{.query(}\StringTok{\textquotesingle{}neighborhood == "Tsinga"\textquotesingle{}}\NormalTok{))}
\end{Highlighting}
\end{Shaded}

\begin{longtable}[]{@{}llllllllllll@{}}
\toprule\noalign{}
& age & sex & weight\_kg & highest\_education & neighborhood &
occupation & symptoms & is\_smoker & is\_pregnant & igg\_result &
igm\_result \\
\midrule\noalign{}
\endhead
\bottomrule\noalign{}
\endlastfoot
827 & 24 & Male & 85 & Secondary & Tsinga & Student-\/-Informal worker &
No symptoms & Ex-smoker & NaN & Positive & Positive \\
835 & 22 & Male & 70 & University & Tsinga & Student &
Fever-\/-Headache-\/-Fatigue & Ex-smoker & NaN & Positive & Negative \\
... & ... & ... & ... & ... & ... & ... & ... & ... & ... & ... & ... \\
899 & 29 & Female & 64 & Doctorate & Tsinga & Informal worker &
Fever-\/-Headache-\/-Anosmia or ageusia & Ex-smoker & No & Positive &
Negative \\
903 & 28 & Male & 69 & Secondary & Tsinga & Trader &
Sneezing-\/-Headache & Ex-smoker & NaN & Negative & Negative \\
\end{longtable}

When you have a very long sequence of conditions, this kind of chaining
can be more readable than using \texttt{\&} multiple times. The example
above is not \emph{that} long, so it may not be a good candidate for
splitting the query up, but now you know how to do it in case you need
to.

\section{\texorpdfstring{Negating conditions with the
\texttt{\textasciitilde{}}
operator}{Negating conditions with the \textasciitilde{} operator}}\label{negating-conditions-with-the-operator}

To negate conditions in \texttt{query()}, we use the
\texttt{\textasciitilde{}} operator (pronounced ``tilde'').

Let's use this to drop respondents who are students:

\begin{Shaded}
\begin{Highlighting}[]
\NormalTok{yao.query(}\StringTok{\textquotesingle{}\textasciitilde{} (occupation == "Student")\textquotesingle{}}\NormalTok{)}
\end{Highlighting}
\end{Shaded}

\begin{longtable}[]{@{}llllllllllll@{}}
\toprule\noalign{}
& age & sex & weight\_kg & highest\_education & neighborhood &
occupation & symptoms & is\_smoker & is\_pregnant & igg\_result &
igm\_result \\
\midrule\noalign{}
\endhead
\bottomrule\noalign{}
\endlastfoot
0 & 45 & Female & 95 & Secondary & Briqueterie & Informal worker &
Muscle pain & Non-smoker & No & Negative & Negative \\
1 & 55 & Male & 96 & University & Briqueterie & Salaried worker & No
symptoms & Ex-smoker & NaN & Positive & Negative \\
... & ... & ... & ... & ... & ... & ... & ... & ... & ... & ... & ... \\
969 & 31 & Female & 66 & Secondary & Tsinga Oliga & Unemployed & No
symptoms & Non-smoker & No & Negative & Negative \\
970 & 17 & Female & 67 & Secondary & Tsinga Oliga & Unemployed & No
symptoms & Non-smoker & No response & Negative & Negative \\
\end{longtable}

Notice that we have to enclose the condition in parentheses.

Of course, in this case, we could more simply use \texttt{!=} to drop
students, so this is not a great use case for \texttt{not}.

\begin{Shaded}
\begin{Highlighting}[]
\NormalTok{yao.query(}\StringTok{\textquotesingle{}occupation != "Student"\textquotesingle{}}\NormalTok{)}
\end{Highlighting}
\end{Shaded}

\begin{longtable}[]{@{}llllllllllll@{}}
\toprule\noalign{}
& age & sex & weight\_kg & highest\_education & neighborhood &
occupation & symptoms & is\_smoker & is\_pregnant & igg\_result &
igm\_result \\
\midrule\noalign{}
\endhead
\bottomrule\noalign{}
\endlastfoot
0 & 45 & Female & 95 & Secondary & Briqueterie & Informal worker &
Muscle pain & Non-smoker & No & Negative & Negative \\
1 & 55 & Male & 96 & University & Briqueterie & Salaried worker & No
symptoms & Ex-smoker & NaN & Positive & Negative \\
... & ... & ... & ... & ... & ... & ... & ... & ... & ... & ... & ... \\
969 & 31 & Female & 66 & Secondary & Tsinga Oliga & Unemployed & No
symptoms & Non-smoker & No & Negative & Negative \\
970 & 17 & Female & 67 & Secondary & Tsinga Oliga & Unemployed & No
symptoms & Non-smoker & No response & Negative & Negative \\
\end{longtable}

The \texttt{\textasciitilde{}} operator is more useful when we have more
complex conditions that we want to negate.

Imagine we want to give out a drug, but respondents below 18 (children)
or who weigh less than 30kg (too light) are not eligible. The code below
selects the children and these light respondents:

\begin{Shaded}
\begin{Highlighting}[]
\NormalTok{yao.query(}\StringTok{"age \textless{} 18 | weight\_kg \textless{} 30"}\NormalTok{)}
\end{Highlighting}
\end{Shaded}

\begin{longtable}[]{@{}llllllllllll@{}}
\toprule\noalign{}
& age & sex & weight\_kg & highest\_education & neighborhood &
occupation & symptoms & is\_smoker & is\_pregnant & igg\_result &
igm\_result \\
\midrule\noalign{}
\endhead
\bottomrule\noalign{}
\endlastfoot
5 & 17 & Female & 65 & Secondary & Briqueterie & Student &
Fever-\/-Cough-\/-Rhinitis-\/-Nausea or vomiting-\/-Di... & Non-smoker &
No & Negative & Negative \\
6 & 13 & Female & 65 & Secondary & Briqueterie & Student & Sneezing &
Non-smoker & No & Positive & Negative \\
... & ... & ... & ... & ... & ... & ... & ... & ... & ... & ... & ... \\
962 & 15 & Male & 44 & Secondary & Tsinga Oliga & Student &
Fever-\/-Cough-\/-Rhinitis & Non-smoker & NaN & Positive & Negative \\
970 & 17 & Female & 67 & Secondary & Tsinga Oliga & Unemployed & No
symptoms & Non-smoker & No response & Negative & Negative \\
\end{longtable}

Now to drop these individuals, we can negate the condition with
\texttt{\textasciitilde{}}:

\begin{Shaded}
\begin{Highlighting}[]
\NormalTok{yao.query(}\StringTok{"\textasciitilde{} (age \textless{} 18 | weight\_kg \textless{} 30)"}\NormalTok{)}
\end{Highlighting}
\end{Shaded}

\begin{longtable}[]{@{}llllllllllll@{}}
\toprule\noalign{}
& age & sex & weight\_kg & highest\_education & neighborhood &
occupation & symptoms & is\_smoker & is\_pregnant & igg\_result &
igm\_result \\
\midrule\noalign{}
\endhead
\bottomrule\noalign{}
\endlastfoot
0 & 45 & Female & 95 & Secondary & Briqueterie & Informal worker &
Muscle pain & Non-smoker & No & Negative & Negative \\
1 & 55 & Male & 96 & University & Briqueterie & Salaried worker & No
symptoms & Ex-smoker & NaN & Positive & Negative \\
... & ... & ... & ... & ... & ... & ... & ... & ... & ... & ... & ... \\
968 & 35 & Male & 77 & University & Tsinga Oliga & Informal worker &
Headache & Smoker & NaN & Positive & Negative \\
969 & 31 & Female & 66 & Secondary & Tsinga Oliga & Unemployed & No
symptoms & Non-smoker & No & Negative & Negative \\
\end{longtable}

Note how we enclosed the conditions in parentheses. This allows the not
operator to `act' on the whole condition, rather than just the first
part of it.

\begin{tcolorbox}[enhanced jigsaw, colframe=quarto-callout-tip-color-frame, opacityback=0, titlerule=0mm, bottomrule=.15mm, breakable, leftrule=.75mm, colbacktitle=quarto-callout-tip-color!10!white, title=\textcolor{quarto-callout-tip-color}{\faLightbulb}\hspace{0.5em}{Practice}, rightrule=.15mm, coltitle=black, opacitybacktitle=0.6, colback=white, left=2mm, arc=.35mm, toptitle=1mm, bottomtitle=1mm, toprule=.15mm]

\subsection{Practice Q: Drop Smokers and drop those over
50}\label{practice-q-drop-smokers-and-drop-those-over-50}

From \texttt{yao}, drop respondents that are either above 50 or who are
smokers. Assign the result to a new DataFrame called
\texttt{yao\_dropped}. Your output should have 810 rows.

\begin{Shaded}
\begin{Highlighting}[]
\CommentTok{\# Your code here}
\end{Highlighting}
\end{Shaded}

\end{tcolorbox}

\section{\texorpdfstring{\texttt{NaN}
values}{NaN values}}\label{nan-values}

The relational operators introduced so far do not work with null values
like \texttt{NaN}.

For example, the \texttt{is\_pregnant} column contains missing values
for some respondents. To keep the rows with missing
\texttt{is\_pregnant} values, we could try writing:

\begin{Shaded}
\begin{Highlighting}[]
\NormalTok{yao.query(}\StringTok{"is\_pregnant == NaN"}\NormalTok{)  }\CommentTok{\# does not work}
\end{Highlighting}
\end{Shaded}

But this will not work. This is because \texttt{NaN} is a non-existent
value. So the system cannot evaluate whether it is ``equal to'' or ``not
equal to'' anything.

Instead, we can use the \texttt{isna()} method to select rows with
missing values:

\begin{Shaded}
\begin{Highlighting}[]
\NormalTok{yao.query(}\StringTok{"is\_pregnant.isna()"}\NormalTok{)}
\end{Highlighting}
\end{Shaded}

\begin{longtable}[]{@{}llllllllllll@{}}
\toprule\noalign{}
& age & sex & weight\_kg & highest\_education & neighborhood &
occupation & symptoms & is\_smoker & is\_pregnant & igg\_result &
igm\_result \\
\midrule\noalign{}
\endhead
\bottomrule\noalign{}
\endlastfoot
1 & 55 & Male & 96 & University & Briqueterie & Salaried worker & No
symptoms & Ex-smoker & NaN & Positive & Negative \\
2 & 23 & Male & 74 & University & Briqueterie & Student & No symptoms &
Smoker & NaN & Negative & Negative \\
... & ... & ... & ... & ... & ... & ... & ... & ... & ... & ... & ... \\
966 & 32 & Male & 54 & Secondary & Tsinga Oliga & Informal worker &
Rhinitis-\/-Sneezing-\/-Diarrhoea & Smoker & NaN & Negative &
Negative \\
968 & 35 & Male & 77 & University & Tsinga Oliga & Informal worker &
Headache & Smoker & NaN & Positive & Negative \\
\end{longtable}

Or we can select rows that are not missing with \texttt{notna()}:

\begin{Shaded}
\begin{Highlighting}[]
\NormalTok{yao.query(}\StringTok{"is\_pregnant.notna()"}\NormalTok{)}
\end{Highlighting}
\end{Shaded}

\begin{longtable}[]{@{}llllllllllll@{}}
\toprule\noalign{}
& age & sex & weight\_kg & highest\_education & neighborhood &
occupation & symptoms & is\_smoker & is\_pregnant & igg\_result &
igm\_result \\
\midrule\noalign{}
\endhead
\bottomrule\noalign{}
\endlastfoot
0 & 45 & Female & 95 & Secondary & Briqueterie & Informal worker &
Muscle pain & Non-smoker & No & Negative & Negative \\
3 & 20 & Female & 70 & Secondary & Briqueterie & Student &
Rhinitis-\/-Sneezing-\/-Anosmia or ageusia & Non-smoker & No & Positive
& Negative \\
... & ... & ... & ... & ... & ... & ... & ... & ... & ... & ... & ... \\
969 & 31 & Female & 66 & Secondary & Tsinga Oliga & Unemployed & No
symptoms & Non-smoker & No & Negative & Negative \\
970 & 17 & Female & 67 & Secondary & Tsinga Oliga & Unemployed & No
symptoms & Non-smoker & No response & Negative & Negative \\
\end{longtable}

\begin{tcolorbox}[enhanced jigsaw, colframe=quarto-callout-tip-color-frame, opacityback=0, titlerule=0mm, bottomrule=.15mm, breakable, leftrule=.75mm, colbacktitle=quarto-callout-tip-color!10!white, title=\textcolor{quarto-callout-tip-color}{\faLightbulb}\hspace{0.5em}{Practice}, rightrule=.15mm, coltitle=black, opacitybacktitle=0.6, colback=white, left=2mm, arc=.35mm, toptitle=1mm, bottomtitle=1mm, toprule=.15mm]

\subsection{Practice Q: Keep Missing Smoking
Status}\label{practice-q-keep-missing-smoking-status}

From the \texttt{yao} dataset, keep all the respondents who had missing
records for the report of their smoking status.

\begin{Shaded}
\begin{Highlighting}[]
\CommentTok{\# Your code here}
\end{Highlighting}
\end{Shaded}

\end{tcolorbox}

\section{Querying Based on String
Patterns}\label{querying-based-on-string-patterns}

Sometimes, we need to filter our data based on whether a string column
contains a certain substring. This is particularly useful when dealing
with multi-answer type variables, where responses may contain multiple
values separated by delimiters. Let's explore this using the
\texttt{occupation} column in our dataset.

First, let's take a look at the unique values in the \texttt{occupation}
column:

\begin{Shaded}
\begin{Highlighting}[]
\NormalTok{occup\_df }\OperatorTok{=}\NormalTok{ yao[[}\StringTok{"occupation"}\NormalTok{, }\StringTok{"sex"}\NormalTok{]]}
\NormalTok{occup\_df.value\_counts().to\_dict()}
\end{Highlighting}
\end{Shaded}

\begin{verbatim}
{('Student', 'Female'): 203,
 ('Student', 'Male'): 180,
 ('Informal worker', 'Male'): 112,
 ('Informal worker', 'Female'): 77,
 ('Trader', 'Female'): 69,
 ('Home-maker', 'Female'): 65,
 ('Unemployed', 'Female'): 50,
 ('Trader', 'Male'): 42,
 ('Salaried worker', 'Female'): 34,
 ('Salaried worker', 'Male'): 20,
 ('Unemployed', 'Male'): 18,
 ('Retired', 'Male'): 14,
 ('Retired', 'Female'): 13,
 ('Student--Informal worker', 'Male'): 11,
 ('No response', 'Female'): 8,
 ('Other', 'Male'): 7,
 ('Other', 'Female'): 6,
 ('Trader--Farmer', 'Female'): 4,
 ('Student--Trader', 'Male'): 4,
 ('Informal worker--Trader', 'Male'): 3,
 ('Home-maker--Trader', 'Female'): 3,
 ('Home-maker--Informal worker', 'Female'): 3,
 ('Farmer', 'Female'): 3,
 ('Informal worker--Other', 'Male'): 2,
 ('Farmer', 'Male'): 2,
 ('Home-maker--Farmer', 'Female'): 2,
 ('Student--Informal worker', 'Female'): 2,
 ('Retired--Informal worker', 'Female'): 2,
 ('Farmer--Other', 'Male'): 1,
 ('Trader--Unemployed', 'Female'): 1,
 ('Informal worker--Unemployed', 'Male'): 1,
 ('No response', 'Male'): 1,
 ('Retired--Other', 'Male'): 1,
 ('Student--Other', 'Female'): 1,
 ('Retired--Trader', 'Female'): 1,
 ('Home-maker--Informal worker--Farmer', 'Female'): 1,
 ('Retired--Informal worker', 'Male'): 1,
 ('Informal worker--Trader--Farmer--Other', 'Male'): 1,
 ('Informal worker--Trader', 'Female'): 1,
 ('Student--Informal worker--Other', 'Male'): 1}
\end{verbatim}

As we can see, some respondents have multiple occupations, separated by
``--''. To query based on string containment, we can use the
\texttt{str.contains()} method within our \texttt{query()}.

\subsection{Basic String Containment}\label{basic-string-containment}

To find all respondents who are students (either solely or in
combination with other occupations), we can use:

\begin{Shaded}
\begin{Highlighting}[]
\NormalTok{occup\_df.query(}\StringTok{"occupation.str.contains(\textquotesingle{}Student\textquotesingle{})"}\NormalTok{)}
\end{Highlighting}
\end{Shaded}

\begin{longtable}[]{@{}lll@{}}
\toprule\noalign{}
& occupation & sex \\
\midrule\noalign{}
\endhead
\bottomrule\noalign{}
\endlastfoot
2 & Student & Male \\
3 & Student & Female \\
... & ... & ... \\
963 & Student & Female \\
964 & Student-\/-Informal worker & Male \\
\end{longtable}

This query will return all rows where the \texttt{occupation} column
contains the word ``Student'', regardless of whether it's the only
occupation or part of a multiple-occupation entry.

\subsection{Combining with Other
Conditions}\label{combining-with-other-conditions}

You can combine string containment queries with other conditions. For
example, to find female students:

\begin{Shaded}
\begin{Highlighting}[]
\NormalTok{occup\_df.query(}\StringTok{"occupation.str.contains(\textquotesingle{}Student\textquotesingle{}) \& sex == \textquotesingle{}Female\textquotesingle{}"}\NormalTok{)}
\end{Highlighting}
\end{Shaded}

\begin{longtable}[]{@{}lll@{}}
\toprule\noalign{}
& occupation & sex \\
\midrule\noalign{}
\endhead
\bottomrule\noalign{}
\endlastfoot
3 & Student & Female \\
5 & Student & Female \\
... & ... & ... \\
949 & Student & Female \\
963 & Student & Female \\
\end{longtable}

\subsection{Negating String
Containment}\label{negating-string-containment}

To find respondents who are not students (i.e., their occupation does
not contain ``Student''), you can use the \texttt{\textasciitilde{}}
operator:

\begin{Shaded}
\begin{Highlighting}[]
\NormalTok{occup\_df.query(}\StringTok{"\textasciitilde{}occupation.str.contains(\textquotesingle{}Student\textquotesingle{})"}\NormalTok{)}
\end{Highlighting}
\end{Shaded}

\begin{longtable}[]{@{}lll@{}}
\toprule\noalign{}
& occupation & sex \\
\midrule\noalign{}
\endhead
\bottomrule\noalign{}
\endlastfoot
0 & Informal worker & Female \\
1 & Salaried worker & Male \\
... & ... & ... \\
969 & Unemployed & Female \\
970 & Unemployed & Female \\
\end{longtable}

\subsection{Search for non-alphabetic
characters}\label{search-for-non-alphabetic-characters}

If you want to find respondents who have multiple occupations, you can
search for the delimiter:

\begin{Shaded}
\begin{Highlighting}[]
\NormalTok{occup\_df.query(}\StringTok{"occupation.str.contains(\textquotesingle{}{-}{-}\textquotesingle{})"}\NormalTok{)}
\end{Highlighting}
\end{Shaded}

\begin{longtable}[]{@{}lll@{}}
\toprule\noalign{}
& occupation & sex \\
\midrule\noalign{}
\endhead
\bottomrule\noalign{}
\endlastfoot
4 & Trader-\/-Farmer & Female \\
45 & Retired-\/-Informal worker & Male \\
... & ... & ... \\
964 & Student-\/-Informal worker & Male \\
967 & Informal worker-\/-Trader & Female \\
\end{longtable}

This will return all rows where the occupation contains ``--'',
indicating multiple occupations.

\begin{tcolorbox}[enhanced jigsaw, colframe=quarto-callout-tip-color-frame, opacityback=0, titlerule=0mm, bottomrule=.15mm, breakable, leftrule=.75mm, colbacktitle=quarto-callout-tip-color!10!white, title=\textcolor{quarto-callout-tip-color}{\faLightbulb}\hspace{0.5em}{Practice}, rightrule=.15mm, coltitle=black, opacitybacktitle=0.6, colback=white, left=2mm, arc=.35mm, toptitle=1mm, bottomtitle=1mm, toprule=.15mm]

\subsection{Practice Q: Symptoms}\label{practice-q-symptoms}

The symptoms column contains a list of symptoms that respondents
reported.

Query \texttt{yao} to find respondents who reported ``Cough'' or
``Fever'' as symptoms. Your answer should have 219 rows.

\begin{Shaded}
\begin{Highlighting}[]
\CommentTok{\# Your code here}
\end{Highlighting}
\end{Shaded}

\end{tcolorbox}

\section{\texorpdfstring{Subsetting with square brackets,
\texttt{{[}{]}}}{Subsetting with square brackets, {[}{]}}}\label{subsetting-with-square-brackets}

As a final note, you can also use logical subsetting with
\texttt{{[}{]}} (square brackets) to filter rows.

We don't recommend this for filtering DataFrames, as it's difficult to
use properly in method chains, but we will use it at other points in our
Python journey, so it's good to know about.

Let's see an example using a pandas Series.

\begin{Shaded}
\begin{Highlighting}[]
\NormalTok{s }\OperatorTok{=}\NormalTok{ pd.Series([}\DecValTok{1}\NormalTok{, }\DecValTok{2}\NormalTok{, }\DecValTok{3}\NormalTok{, }\DecValTok{4}\NormalTok{, }\DecValTok{5}\NormalTok{])}
\NormalTok{s}
\end{Highlighting}
\end{Shaded}

\begin{verbatim}
0    1
1    2
2    3
3    4
4    5
dtype: int64
\end{verbatim}

We can use logical subsetting to filter this series:

\begin{Shaded}
\begin{Highlighting}[]
\NormalTok{s[s }\OperatorTok{\textgreater{}} \DecValTok{3}\NormalTok{]  }\CommentTok{\# subset s to just the values greater than 3}
\end{Highlighting}
\end{Shaded}

\begin{verbatim}
3    4
4    5
dtype: int64
\end{verbatim}

We obtain a new Series with the values that are greater than 3. Series
cannot be filtered with \texttt{query()}, so they are a good use case
for \texttt{{[}{]}}.

We can do the same with DataFrames. Let's filter our \texttt{yao}
DataFrame to keep only females:

\begin{Shaded}
\begin{Highlighting}[]
\NormalTok{yao[yao.sex }\OperatorTok{==} \StringTok{\textquotesingle{}Female\textquotesingle{}}\NormalTok{]}
\end{Highlighting}
\end{Shaded}

\begin{longtable}[]{@{}llllllllllll@{}}
\toprule\noalign{}
& age & sex & weight\_kg & highest\_education & neighborhood &
occupation & symptoms & is\_smoker & is\_pregnant & igg\_result &
igm\_result \\
\midrule\noalign{}
\endhead
\bottomrule\noalign{}
\endlastfoot
0 & 45 & Female & 95 & Secondary & Briqueterie & Informal worker &
Muscle pain & Non-smoker & No & Negative & Negative \\
3 & 20 & Female & 70 & Secondary & Briqueterie & Student &
Rhinitis-\/-Sneezing-\/-Anosmia or ageusia & Non-smoker & No & Positive
& Negative \\
... & ... & ... & ... & ... & ... & ... & ... & ... & ... & ... & ... \\
969 & 31 & Female & 66 & Secondary & Tsinga Oliga & Unemployed & No
symptoms & Non-smoker & No & Negative & Negative \\
970 & 17 & Female & 67 & Secondary & Tsinga Oliga & Unemployed & No
symptoms & Non-smoker & No response & Negative & Negative \\
\end{longtable}

This returns a DataFrame with only female respondents.

We can combine conditions using \texttt{\&} for AND and
\texttt{\textbar{}} for OR. For example, to keep females who are over
30:

\begin{Shaded}
\begin{Highlighting}[]
\NormalTok{yao[(yao.sex }\OperatorTok{==} \StringTok{\textquotesingle{}Female\textquotesingle{}}\NormalTok{) }\OperatorTok{\&}\NormalTok{ (yao.age }\OperatorTok{\textgreater{}} \DecValTok{30}\NormalTok{)]}
\end{Highlighting}
\end{Shaded}

\begin{longtable}[]{@{}llllllllllll@{}}
\toprule\noalign{}
& age & sex & weight\_kg & highest\_education & neighborhood &
occupation & symptoms & is\_smoker & is\_pregnant & igg\_result &
igm\_result \\
\midrule\noalign{}
\endhead
\bottomrule\noalign{}
\endlastfoot
0 & 45 & Female & 95 & Secondary & Briqueterie & Informal worker &
Muscle pain & Non-smoker & No & Negative & Negative \\
4 & 55 & Female & 67 & Primary & Briqueterie & Trader-\/-Farmer & No
symptoms & Non-smoker & No & Positive & Negative \\
... & ... & ... & ... & ... & ... & ... & ... & ... & ... & ... & ... \\
960 & 48 & Female & 46 & Primary & Tsinga Oliga & Trader-\/-Farmer & No
symptoms & Non-smoker & No & Negative & Negative \\
969 & 31 & Female & 66 & Secondary & Tsinga Oliga & Unemployed & No
symptoms & Non-smoker & No & Negative & Negative \\
\end{longtable}

Note the use of parentheses around each condition.

\begin{tcolorbox}[enhanced jigsaw, colframe=quarto-callout-tip-color-frame, opacityback=0, titlerule=0mm, bottomrule=.15mm, breakable, leftrule=.75mm, colbacktitle=quarto-callout-tip-color!10!white, title=\textcolor{quarto-callout-tip-color}{\faLightbulb}\hspace{0.5em}{Practice}, rightrule=.15mm, coltitle=black, opacitybacktitle=0.6, colback=white, left=2mm, arc=.35mm, toptitle=1mm, bottomtitle=1mm, toprule=.15mm]

\subsection{Practice Q: Heights subset}\label{practice-q-heights-subset}

Create a Series called \texttt{heights} with the following values:
{[}160, 175, 182, 168, 190, 173{]}. Then, use logical subsetting to
create a new Series called \texttt{tall\_people} that only includes
heights greater than 180 cm.

\begin{Shaded}
\begin{Highlighting}[]
\CommentTok{\# Your code here}
\end{Highlighting}
\end{Shaded}

\end{tcolorbox}

\section{Wrap up}\label{wrap-up-5}

Great job! You've learned how to select specific columns and filter rows
based on various conditions.

These skills allow you to focus on relevant data and create targeted
subsets for analysis.

Next, we'll explore how to modify and transform your data, further
expanding your data wrangling toolkit. See you in the next lesson!

\chapter{\texorpdfstring{Creating and modifying variables with
\texttt{assign()}}{Creating and modifying variables with assign()}}\label{creating-and-modifying-variables-with-assign}

\section{Intro}\label{intro-3}

Today you will learn how to modify existing variables or create new
ones, using the \texttt{assign()} method from pandas. This is an
essential step in most data analysis projects.

Let's get started!

\section{Learning objective}\label{learning-objective}

\begin{itemize}
\tightlist
\item
  You can use the \texttt{assign()} method from pandas to create new
  variables or modify existing variables.
\end{itemize}

\section{Imports}\label{imports-3}

This lesson will require the pandas package. You can import it with the
following code:

\begin{Shaded}
\begin{Highlighting}[]
\ImportTok{import}\NormalTok{ pandas }\ImportTok{as}\NormalTok{ pd}
\end{Highlighting}
\end{Shaded}

\section{Datasets}\label{datasets}

In this lesson, we will use a dataset of counties of the United States
with demographic and economic data.

\begin{Shaded}
\begin{Highlighting}[]
\NormalTok{counties }\OperatorTok{=}\NormalTok{ pd.read\_csv(}\StringTok{"data/us\_counties\_data.csv"}\NormalTok{)}
\NormalTok{counties}
\end{Highlighting}
\end{Shaded}

\begin{longtable}[]{@{}lllllllllll@{}}
\toprule\noalign{}
& state & county & pop\_20 & area\_sq\_miles & hh\_inc\_21 & econ\_type
& unemp\_20 & foreign\_born\_num & pop\_change\_2010\_2020 &
pct\_emp\_change\_2010\_2021 \\
\midrule\noalign{}
\endhead
\bottomrule\noalign{}
\endlastfoot
0 & AL & Autauga, AL & 58877.0 & 594.456107 & 66444.0 & Nonspecialized &
5.4 & 1241.0 & 7.758700 & 9.0 \\
1 & AL & Baldwin, AL & 233140.0 & 1589.836014 & 65658.0 & Recreation &
6.2 & 7938.0 & 27.159356 & 28.2 \\
... & ... & ... & ... & ... & ... & ... & ... & ... & ... & ... \\
3224 & PR & Yabucoa, PR & 30364.0 & 55.214614 & NaN & NaN & NaN & NaN &
-19.807069 & 0.1 \\
3225 & PR & Yauco, PR & 34062.0 & 67.711484 & NaN & NaN & NaN & NaN &
-18.721309 & -5.3 \\
\end{longtable}

The variables in the dataset are:

\begin{itemize}
\tightlist
\item
  \texttt{state}, US state
\item
  \texttt{county}, US county
\item
  \texttt{pop\_20}, population estimate for 2020
\item
  \texttt{area\_sq\_miles}, area in square miles
\item
  \texttt{hh\_inc\_21}, median household income for 2021
\item
  \texttt{econ\_type}, economic type of the county
\item
  \texttt{pop\_change\_2010\_2020}, population change between 2010 and
  2020
\item
  \texttt{unemp\_20}, unemployment rate for 2020
\item
  \texttt{pct\_emp\_change\_2010\_2021}, percentage change in employment
  between 2010 and 2021
\end{itemize}

The variables are collected from a range of sources, including the US
Census Bureau, the Bureau of Labor Statistics, and the American
Community Survey.

Let's create a small subset of the variables, with just area and
population, for illustration.

\begin{Shaded}
\begin{Highlighting}[]
\CommentTok{\#\# small subset for illustration}
\NormalTok{area\_pop }\OperatorTok{=}\NormalTok{ counties[[}\StringTok{"county"}\NormalTok{, }\StringTok{"area\_sq\_miles"}\NormalTok{, }\StringTok{"pop\_20"}\NormalTok{]]}
\NormalTok{area\_pop}
\end{Highlighting}
\end{Shaded}

\begin{longtable}[]{@{}llll@{}}
\toprule\noalign{}
& county & area\_sq\_miles & pop\_20 \\
\midrule\noalign{}
\endhead
\bottomrule\noalign{}
\endlastfoot
0 & Autauga, AL & 594.456107 & 58877.0 \\
1 & Baldwin, AL & 1589.836014 & 233140.0 \\
... & ... & ... & ... \\
3224 & Yabucoa, PR & 55.214614 & 30364.0 \\
3225 & Yauco, PR & 67.711484 & 34062.0 \\
\end{longtable}

\section{\texorpdfstring{Introducing \texttt{assign()} and lambda
functions}{Introducing assign() and lambda functions}}\label{introducing-assign-and-lambda-functions}

We use the \texttt{assign()} method from pandas to create new variables
or modify existing variables.

Let's see a quick example.

The \texttt{area\_pop} dataset shows the area of each county in square
miles. We want to \textbf{create a new variable} with this converted to
square kilometers so we must multiply the \texttt{area\_sq\_miles}
variable by 2.59. With \texttt{assign()}, we can write:

\begin{Shaded}
\begin{Highlighting}[]
\NormalTok{area\_pop.assign(area\_sq\_km}\OperatorTok{=}\KeywordTok{lambda}\NormalTok{ x: x.area\_sq\_miles }\OperatorTok{*} \FloatTok{2.59}\NormalTok{)}
\end{Highlighting}
\end{Shaded}

\begin{longtable}[]{@{}lllll@{}}
\toprule\noalign{}
& county & area\_sq\_miles & pop\_20 & area\_sq\_km \\
\midrule\noalign{}
\endhead
\bottomrule\noalign{}
\endlastfoot
0 & Autauga, AL & 594.456107 & 58877.0 & 1539.641317 \\
1 & Baldwin, AL & 1589.836014 & 233140.0 & 4117.675277 \\
... & ... & ... & ... & ... \\
3224 & Yabucoa, PR & 55.214614 & 30364.0 & 143.005851 \\
3225 & Yauco, PR & 67.711484 & 34062.0 & 175.372743 \\
\end{longtable}

Nice! The syntax may appear confusing, so let's break it down.

We can read \texttt{lambda\ x:\ x.area\_sq\_miles\ *\ 2.59} as, ``define
a function that takes a DataFrame \texttt{x}, then multiplies the
\texttt{area\_sq\_miles} variable in that DataFrame by 2.59.''

A \texttt{lambda} is a term to refer to a small function defined without
a name. This part of the code is essential.

But the symbol called \texttt{x} above can be anything you want; it's
simply a placeholder for the data frame. For example, we could call it
\texttt{dat}:

\begin{Shaded}
\begin{Highlighting}[]
\NormalTok{area\_pop.assign(area\_sq\_km}\OperatorTok{=}\KeywordTok{lambda}\NormalTok{ dat: dat.area\_sq\_miles }\OperatorTok{*} \FloatTok{2.59}\NormalTok{)}
\end{Highlighting}
\end{Shaded}

\begin{longtable}[]{@{}lllll@{}}
\toprule\noalign{}
& county & area\_sq\_miles & pop\_20 & area\_sq\_km \\
\midrule\noalign{}
\endhead
\bottomrule\noalign{}
\endlastfoot
0 & Autauga, AL & 594.456107 & 58877.0 & 1539.641317 \\
1 & Baldwin, AL & 1589.836014 & 233140.0 & 4117.675277 \\
... & ... & ... & ... & ... \\
3224 & Yabucoa, PR & 55.214614 & 30364.0 & 143.005851 \\
3225 & Yauco, PR & 67.711484 & 34062.0 & 175.372743 \\
\end{longtable}

Let's see another example. We'll add a variable showing the area in
hectares by multiplying \texttt{area\_sq\_miles} by 259. We'll also
store the new DataFrame in a variable called
\texttt{area\_pop\_converted}. (Remember that if we don't store the new
DataFrame in a variable, as in our example above, it will not be saved.)

\begin{Shaded}
\begin{Highlighting}[]
\NormalTok{area\_pop\_converted }\OperatorTok{=}\NormalTok{ area\_pop.assign(}
\NormalTok{    area\_sq\_km}\OperatorTok{=}\KeywordTok{lambda}\NormalTok{ x: x.area\_sq\_miles }\OperatorTok{*} \FloatTok{2.59}\NormalTok{,}
\NormalTok{    area\_hectares}\OperatorTok{=}\KeywordTok{lambda}\NormalTok{ x: x.area\_sq\_miles }\OperatorTok{*} \DecValTok{259}\NormalTok{,}
\NormalTok{)}
\NormalTok{area\_pop\_converted  }\CommentTok{\# view the new DataFrame}
\end{Highlighting}
\end{Shaded}

\begin{longtable}[]{@{}llllll@{}}
\toprule\noalign{}
& county & area\_sq\_miles & pop\_20 & area\_sq\_km & area\_hectares \\
\midrule\noalign{}
\endhead
\bottomrule\noalign{}
\endlastfoot
0 & Autauga, AL & 594.456107 & 58877.0 & 1539.641317 & 153964.131747 \\
1 & Baldwin, AL & 1589.836014 & 233140.0 & 4117.675277 &
411767.527703 \\
... & ... & ... & ... & ... & ... \\
3224 & Yabucoa, PR & 55.214614 & 30364.0 & 143.005851 & 14300.585058 \\
3225 & Yauco, PR & 67.711484 & 34062.0 & 175.372743 & 17537.274254 \\
\end{longtable}

\begin{tcolorbox}[enhanced jigsaw, colframe=quarto-callout-tip-color-frame, opacityback=0, titlerule=0mm, bottomrule=.15mm, breakable, leftrule=.75mm, colbacktitle=quarto-callout-tip-color!10!white, title=\textcolor{quarto-callout-tip-color}{\faLightbulb}\hspace{0.5em}{Practice}, rightrule=.15mm, coltitle=black, opacitybacktitle=0.6, colback=white, left=2mm, arc=.35mm, toptitle=1mm, bottomtitle=1mm, toprule=.15mm]

\subsection{Practice Q: Area in acres}\label{practice-q-area-in-acres}

\begin{itemize}
\tightlist
\item
  Using the \texttt{area\_pop} dataset, create a new column called
  \texttt{area\_acres} by multiplying the \texttt{area\_sq\_miles}
  variable by 640. Store the new DataFrame in an object called
  \texttt{conversion\_question} and print it.
\end{itemize}

\begin{Shaded}
\begin{Highlighting}[]
\CommentTok{\# your code here}
\end{Highlighting}
\end{Shaded}

\end{tcolorbox}

\section{Overwriting variables}\label{overwriting-variables}

We can also use \texttt{assign()} to overwrite existing variables. For
example, to round the \texttt{area\_sq\_km} variable to one decimal
place, we can write:

\begin{Shaded}
\begin{Highlighting}[]
\NormalTok{area\_pop\_converted.assign(area\_sq\_km}\OperatorTok{=}\KeywordTok{lambda}\NormalTok{ x: }\BuiltInTok{round}\NormalTok{(x.area\_sq\_miles }\OperatorTok{*} \FloatTok{2.59}\NormalTok{, }\DecValTok{1}\NormalTok{))}
\end{Highlighting}
\end{Shaded}

\begin{longtable}[]{@{}llllll@{}}
\toprule\noalign{}
& county & area\_sq\_miles & pop\_20 & area\_sq\_km & area\_hectares \\
\midrule\noalign{}
\endhead
\bottomrule\noalign{}
\endlastfoot
0 & Autauga, AL & 594.456107 & 58877.0 & 1539.6 & 153964.131747 \\
1 & Baldwin, AL & 1589.836014 & 233140.0 & 4117.7 & 411767.527703 \\
... & ... & ... & ... & ... & ... \\
3224 & Yabucoa, PR & 55.214614 & 30364.0 & 143.0 & 14300.585058 \\
3225 & Yauco, PR & 67.711484 & 34062.0 & 175.4 & 17537.274254 \\
\end{longtable}

\begin{tcolorbox}[enhanced jigsaw, colframe=quarto-callout-tip-color-frame, opacityback=0, titlerule=0mm, bottomrule=.15mm, breakable, leftrule=.75mm, colbacktitle=quarto-callout-tip-color!10!white, title=\textcolor{quarto-callout-tip-color}{\faLightbulb}\hspace{0.5em}{Practice}, rightrule=.15mm, coltitle=black, opacitybacktitle=0.6, colback=white, left=2mm, arc=.35mm, toptitle=1mm, bottomtitle=1mm, toprule=.15mm]

\subsection{Practice Q: Area in acres
rounded}\label{practice-q-area-in-acres-rounded}

\begin{itemize}
\tightlist
\item
  Using the \texttt{conversion\_question} dataset you created above,
  round the \texttt{area\_acres} variable to one decimal place. Store
  the new DataFrame in an object called
  \texttt{conversion\_question\_rounded} and print it.
\end{itemize}

\begin{Shaded}
\begin{Highlighting}[]
\CommentTok{\# your code here}
\end{Highlighting}
\end{Shaded}

\end{tcolorbox}

\section{More complex assignments}\label{more-complex-assignments}

To get further practice with the \texttt{assign()} method, let's look at
creating new columns that combine multiple existing variables.

For example, to calculate the population density per square kilometer,
we can write:

\begin{Shaded}
\begin{Highlighting}[]
\NormalTok{area\_pop\_converted.assign(pop\_per\_sq\_km}\OperatorTok{=}\KeywordTok{lambda}\NormalTok{ x: x.pop\_20 }\OperatorTok{/}\NormalTok{ x.area\_sq\_km)}
\end{Highlighting}
\end{Shaded}

\begin{longtable}[]{@{}lllllll@{}}
\toprule\noalign{}
& county & area\_sq\_miles & pop\_20 & area\_sq\_km & area\_hectares &
pop\_per\_sq\_km \\
\midrule\noalign{}
\endhead
\bottomrule\noalign{}
\endlastfoot
0 & Autauga, AL & 594.456107 & 58877.0 & 1539.641317 & 153964.131747 &
38.240725 \\
1 & Baldwin, AL & 1589.836014 & 233140.0 & 4117.675277 & 411767.527703 &
56.619326 \\
... & ... & ... & ... & ... & ... & ... \\
3224 & Yabucoa, PR & 55.214614 & 30364.0 & 143.005851 & 14300.585058 &
212.326977 \\
3225 & Yauco, PR & 67.711484 & 34062.0 & 175.372743 & 17537.274254 &
194.226306 \\
\end{longtable}

We can also round the \texttt{pop\_per\_sq\_km} variable to one decimal
place within the \texttt{assign()} method, either on the same line:

\begin{Shaded}
\begin{Highlighting}[]
\NormalTok{area\_pop\_converted.assign(}
\NormalTok{    pop\_per\_sq\_km}\OperatorTok{=}\KeywordTok{lambda}\NormalTok{ x: }\BuiltInTok{round}\NormalTok{(x.pop\_20 }\OperatorTok{/}\NormalTok{ x.area\_sq\_km, }\DecValTok{1}\NormalTok{)}
\NormalTok{)}
\end{Highlighting}
\end{Shaded}

\begin{longtable}[]{@{}lllllll@{}}
\toprule\noalign{}
& county & area\_sq\_miles & pop\_20 & area\_sq\_km & area\_hectares &
pop\_per\_sq\_km \\
\midrule\noalign{}
\endhead
\bottomrule\noalign{}
\endlastfoot
0 & Autauga, AL & 594.456107 & 58877.0 & 1539.641317 & 153964.131747 &
38.2 \\
1 & Baldwin, AL & 1589.836014 & 233140.0 & 4117.675277 & 411767.527703 &
56.6 \\
... & ... & ... & ... & ... & ... & ... \\
3224 & Yabucoa, PR & 55.214614 & 30364.0 & 143.005851 & 14300.585058 &
212.3 \\
3225 & Yauco, PR & 67.711484 & 34062.0 & 175.372743 & 17537.274254 &
194.2 \\
\end{longtable}

Or on a new line like so:

\begin{Shaded}
\begin{Highlighting}[]
\NormalTok{area\_pop\_converted.assign(}
\NormalTok{    pop\_per\_sq\_km}\OperatorTok{=}\KeywordTok{lambda}\NormalTok{ x: x.pop\_20 }\OperatorTok{/}\NormalTok{ x.area\_sq\_km,}
\NormalTok{    pop\_per\_sq\_km\_rounded}\OperatorTok{=}\KeywordTok{lambda}\NormalTok{ x: }\BuiltInTok{round}\NormalTok{(x.pop\_per\_sq\_km, }\DecValTok{1}\NormalTok{),}
\NormalTok{)}
\end{Highlighting}
\end{Shaded}

\begin{longtable}[]{@{}llllllll@{}}
\toprule\noalign{}
& county & area\_sq\_miles & pop\_20 & area\_sq\_km & area\_hectares &
pop\_per\_sq\_km & pop\_per\_sq\_km\_rounded \\
\midrule\noalign{}
\endhead
\bottomrule\noalign{}
\endlastfoot
0 & Autauga, AL & 594.456107 & 58877.0 & 1539.641317 & 153964.131747 &
38.240725 & 38.2 \\
1 & Baldwin, AL & 1589.836014 & 233140.0 & 4117.675277 & 411767.527703 &
56.619326 & 56.6 \\
... & ... & ... & ... & ... & ... & ... & ... \\
3224 & Yabucoa, PR & 55.214614 & 30364.0 & 143.005851 & 14300.585058 &
212.326977 & 212.3 \\
3225 & Yauco, PR & 67.711484 & 34062.0 & 175.372743 & 17537.274254 &
194.226306 & 194.2 \\
\end{longtable}

As a final step, let's practice method chaining by arranging the dataset
by the \texttt{pop\_per\_sq\_km} variable in descending order, and store
this in a new variable called \texttt{area\_pop\_converted\_sorted}.

\begin{Shaded}
\begin{Highlighting}[]
\NormalTok{area\_pop\_converted\_sorted }\OperatorTok{=}\NormalTok{ area\_pop\_converted.assign(}
\NormalTok{    pop\_per\_sq\_km}\OperatorTok{=}\KeywordTok{lambda}\NormalTok{ x: x.pop\_20 }\OperatorTok{/}\NormalTok{ x.area\_sq\_km,}
\NormalTok{    pop\_per\_sq\_km\_rounded}\OperatorTok{=}\KeywordTok{lambda}\NormalTok{ x: }\BuiltInTok{round}\NormalTok{(x.pop\_per\_sq\_km, }\DecValTok{1}\NormalTok{),}
\NormalTok{).sort\_values(}\StringTok{"pop\_per\_sq\_km"}\NormalTok{, ascending}\OperatorTok{=}\VariableTok{False}\NormalTok{)}

\NormalTok{area\_pop\_converted\_sorted}
\end{Highlighting}
\end{Shaded}

\begin{longtable}[]{@{}llllllll@{}}
\toprule\noalign{}
& county & area\_sq\_miles & pop\_20 & area\_sq\_km & area\_hectares &
pop\_per\_sq\_km & pop\_per\_sq\_km\_rounded \\
\midrule\noalign{}
\endhead
\bottomrule\noalign{}
\endlastfoot
1863 & New York, NY & 22.656266 & 1687834.0 & 58.679729 & 5867.972888 &
28763.493499 & 28763.5 \\
1856 & Kings, NY & 69.376570 & 2727393.0 & 179.685318 & 17968.531752 &
15178.719317 & 15178.7 \\
... & ... & ... & ... & ... & ... & ... & ... \\
98 & Wrangell-Petersburg, AK & NaN & NaN & NaN & NaN & NaN & NaN \\
2921 & Bedford, VA & NaN & NaN & NaN & NaN & NaN & NaN \\
\end{longtable}

We see that New York county has the highest population density in the
country.

\begin{tcolorbox}[enhanced jigsaw, colframe=quarto-callout-tip-color-frame, opacityback=0, titlerule=0mm, bottomrule=.15mm, breakable, leftrule=.75mm, colbacktitle=quarto-callout-tip-color!10!white, title=\textcolor{quarto-callout-tip-color}{\faLightbulb}\hspace{0.5em}{Practice}, rightrule=.15mm, coltitle=black, opacitybacktitle=0.6, colback=white, left=2mm, arc=.35mm, toptitle=1mm, bottomtitle=1mm, toprule=.15mm]

\subsection{Practice Q: Rounding all
variables}\label{practice-q-rounding-all-variables}

\begin{itemize}
\tightlist
\item
  Consider the sample dataset created below. Use \texttt{assign()} to
  round all variables to 1 decimal place. Your final dataframe should be
  called \texttt{sample\_df\_rounded} and should still have three
  columns, \texttt{a}, \texttt{b}, and \texttt{c}.
\end{itemize}

\begin{Shaded}
\begin{Highlighting}[]
\NormalTok{sample\_df }\OperatorTok{=}\NormalTok{ pd.DataFrame(}
\NormalTok{    \{}
        \StringTok{"a"}\NormalTok{: [}\FloatTok{1.111}\NormalTok{, }\FloatTok{2.222}\NormalTok{, }\FloatTok{3.333}\NormalTok{],}
        \StringTok{"b"}\NormalTok{: [}\FloatTok{4.444}\NormalTok{, }\FloatTok{5.555}\NormalTok{, }\FloatTok{6.666}\NormalTok{],}
        \StringTok{"c"}\NormalTok{: [}\FloatTok{7.777}\NormalTok{, }\FloatTok{8.888}\NormalTok{, }\FloatTok{9.999}\NormalTok{],}
\NormalTok{    \}}
\NormalTok{)}
\NormalTok{sample\_df}
\end{Highlighting}
\end{Shaded}

\begin{longtable}[]{@{}llll@{}}
\toprule\noalign{}
& a & b & c \\
\midrule\noalign{}
\endhead
\bottomrule\noalign{}
\endlastfoot
0 & 1.111 & 4.444 & 7.777 \\
1 & 2.222 & 5.555 & 8.888 \\
2 & 3.333 & 6.666 & 9.999 \\
\end{longtable}

\begin{Shaded}
\begin{Highlighting}[]
\CommentTok{\# your code here}
\end{Highlighting}
\end{Shaded}

\end{tcolorbox}

\begin{tcolorbox}[enhanced jigsaw, colframe=quarto-callout-note-color-frame, opacityback=0, titlerule=0mm, bottomrule=.15mm, breakable, leftrule=.75mm, colbacktitle=quarto-callout-note-color!10!white, title=\textcolor{quarto-callout-note-color}{\faInfo}\hspace{0.5em}{Pro-tip}, rightrule=.15mm, coltitle=black, opacitybacktitle=0.6, colback=white, left=2mm, arc=.35mm, toptitle=1mm, bottomtitle=1mm, toprule=.15mm]

\subsection{\texorpdfstring{Why use lambda functions in
\texttt{assign()}?}{Why use lambda functions in assign()?}}\label{why-use-lambda-functions-in-assign}

You may be wondering whether the lambda function is really necessary
within assign. After all, we could run code like this:

\begin{Shaded}
\begin{Highlighting}[]
\NormalTok{area\_pop.assign(}
\NormalTok{    area\_sq\_km}\OperatorTok{=}\NormalTok{area\_pop.area\_sq\_miles }\OperatorTok{*} \FloatTok{2.59}\NormalTok{,}
\NormalTok{    area\_hectares}\OperatorTok{=}\NormalTok{area\_pop.area\_sq\_miles }\OperatorTok{*} \DecValTok{259}\NormalTok{,}
\NormalTok{)}
\end{Highlighting}
\end{Shaded}

\begin{longtable}[]{@{}llllll@{}}
\toprule\noalign{}
& county & area\_sq\_miles & pop\_20 & area\_sq\_km & area\_hectares \\
\midrule\noalign{}
\endhead
\bottomrule\noalign{}
\endlastfoot
0 & Autauga, AL & 594.456107 & 58877.0 & 1539.641317 & 153964.131747 \\
1 & Baldwin, AL & 1589.836014 & 233140.0 & 4117.675277 &
411767.527703 \\
... & ... & ... & ... & ... & ... \\
3224 & Yabucoa, PR & 55.214614 & 30364.0 & 143.005851 & 14300.585058 \\
3225 & Yauco, PR & 67.711484 & 34062.0 & 175.372743 & 17537.274254 \\
\end{longtable}

Here, we are referring back to the \texttt{area\_pop} DataFrame within
the \texttt{assign()} method.

The problem with this is we cannot access variables created in the same
\texttt{assign()} call or created in the same method chain.

For example, below, if you try to use the \texttt{area\_sq\_km} variable
to calculate the population density, you will get an error:

\begin{Shaded}
\begin{Highlighting}[]
\NormalTok{area\_pop.assign(}
\NormalTok{    area\_sq\_km}\OperatorTok{=}\NormalTok{area\_pop.area\_sq\_miles }\OperatorTok{*} \FloatTok{2.59}\NormalTok{,}
\NormalTok{    area\_hectares}\OperatorTok{=}\NormalTok{area\_pop.area\_sq\_miles }\OperatorTok{*} \DecValTok{259}\NormalTok{,}
\NormalTok{    pop\_per\_sq\_km}\OperatorTok{=}\NormalTok{area\_pop.pop\_20 }\OperatorTok{/}\NormalTok{ area\_pop.area\_sq\_km,}
\NormalTok{)}
\end{Highlighting}
\end{Shaded}

\begin{verbatim}
AttributeError: 'DataFrame' object has no attribute 'area_sq_km'
\end{verbatim}

Python cannot find the \texttt{area\_sq\_km} variable because it is
created in the same \texttt{assign()} call. So the \texttt{area\_pop}
DataFrame is does not yet have that variable!

Lambda functions in \texttt{assign()} allow you to create new columns
based on intermediate results within the same call. So the code below
works:

\begin{Shaded}
\begin{Highlighting}[]
\NormalTok{area\_pop.assign(}
\NormalTok{    area\_sq\_km}\OperatorTok{=}\KeywordTok{lambda}\NormalTok{ x: x.area\_sq\_miles }\OperatorTok{*} \FloatTok{2.59}\NormalTok{,}
\NormalTok{    area\_hectares}\OperatorTok{=}\KeywordTok{lambda}\NormalTok{ x: x.area\_sq\_miles }\OperatorTok{*} \DecValTok{259}\NormalTok{,}
    \CommentTok{\# area\_sq\_km is created in the previous line, but is already available here}
\NormalTok{    pop\_per\_sq\_km}\OperatorTok{=}\KeywordTok{lambda}\NormalTok{ x: x.pop\_20 }\OperatorTok{/}\NormalTok{ x.area\_sq\_km,}
\NormalTok{)}
\end{Highlighting}
\end{Shaded}

\begin{longtable}[]{@{}lllllll@{}}
\toprule\noalign{}
& county & area\_sq\_miles & pop\_20 & area\_sq\_km & area\_hectares &
pop\_per\_sq\_km \\
\midrule\noalign{}
\endhead
\bottomrule\noalign{}
\endlastfoot
0 & Autauga, AL & 594.456107 & 58877.0 & 1539.641317 & 153964.131747 &
38.240725 \\
1 & Baldwin, AL & 1589.836014 & 233140.0 & 4117.675277 & 411767.527703 &
56.619326 \\
... & ... & ... & ... & ... & ... & ... \\
3224 & Yabucoa, PR & 55.214614 & 30364.0 & 143.005851 & 14300.585058 &
212.326977 \\
3225 & Yauco, PR & 67.711484 & 34062.0 & 175.372743 & 17537.274254 &
194.226306 \\
\end{longtable}

\end{tcolorbox}

\section{Creating Boolean variables}\label{creating-boolean-variables}

You can use \texttt{assign()} to create a Boolean variable to categorize
part of your dataset.

Consider the \texttt{pop\_change\_2010\_2020} variable in the
\texttt{counties} dataset, which shows the percentage change in
population between 2010 and 2020. This is shown below in our
\texttt{changes\_df} subset.

\begin{Shaded}
\begin{Highlighting}[]
\NormalTok{changes\_df }\OperatorTok{=}\NormalTok{ counties[}
\NormalTok{    [}\StringTok{"county"}\NormalTok{, }\StringTok{"pop\_change\_2010\_2020"}\NormalTok{, }\StringTok{"pct\_emp\_change\_2010\_2021"}\NormalTok{]}
\NormalTok{]  }\CommentTok{\# Make dataset subset}
\NormalTok{changes\_df}
\end{Highlighting}
\end{Shaded}

\begin{longtable}[]{@{}llll@{}}
\toprule\noalign{}
& county & pop\_change\_2010\_2020 & pct\_emp\_change\_2010\_2021 \\
\midrule\noalign{}
\endhead
\bottomrule\noalign{}
\endlastfoot
0 & Autauga, AL & 7.758700 & 9.0 \\
1 & Baldwin, AL & 27.159356 & 28.2 \\
... & ... & ... & ... \\
3224 & Yabucoa, PR & -19.807069 & 0.1 \\
3225 & Yauco, PR & -18.721309 & -5.3 \\
\end{longtable}

Below we create a Boolean variable, \texttt{pop\_increased}, that is
\texttt{True} if the growth rate is greater than 0 and \texttt{False} if
it is not.

\begin{Shaded}
\begin{Highlighting}[]
\NormalTok{changes\_df.assign(pop\_increased}\OperatorTok{=}\KeywordTok{lambda}\NormalTok{ x: x.pop\_change\_2010\_2020 }\OperatorTok{\textgreater{}} \DecValTok{0}\NormalTok{)}
\end{Highlighting}
\end{Shaded}

\begin{longtable}[]{@{}lllll@{}}
\toprule\noalign{}
& county & pop\_change\_2010\_2020 & pct\_emp\_change\_2010\_2021 &
pop\_increased \\
\midrule\noalign{}
\endhead
\bottomrule\noalign{}
\endlastfoot
0 & Autauga, AL & 7.758700 & 9.0 & True \\
1 & Baldwin, AL & 27.159356 & 28.2 & True \\
... & ... & ... & ... & ... \\
3224 & Yabucoa, PR & -19.807069 & 0.1 & False \\
3225 & Yauco, PR & -18.721309 & -5.3 & False \\
\end{longtable}

The code
\texttt{x{[}\textquotesingle{}pop\_change\_2010\_2020\textquotesingle{}{]}\ \textgreater{}\ 0}
evaluates whether each growth rate is greater than 0. Growth rates that
match that condition (growth rates greater than 0) are \texttt{True} and
those that fail the condition are \texttt{False}.

Let's do the same for the employment change variable and store the
results in our dataset.

\begin{Shaded}
\begin{Highlighting}[]
\NormalTok{changes\_df }\OperatorTok{=}\NormalTok{ changes\_df.assign(}
\NormalTok{    pop\_increased}\OperatorTok{=}\KeywordTok{lambda}\NormalTok{ x: x.pop\_change\_2010\_2020 }\OperatorTok{\textgreater{}} \DecValTok{0}\NormalTok{,}
\NormalTok{    emp\_increased}\OperatorTok{=}\KeywordTok{lambda}\NormalTok{ x: x.pct\_emp\_change\_2010\_2021 }\OperatorTok{\textgreater{}} \DecValTok{0}\NormalTok{,}
\NormalTok{)}
\NormalTok{changes\_df}
\end{Highlighting}
\end{Shaded}

\begin{longtable}[]{@{}llllll@{}}
\toprule\noalign{}
& county & pop\_change\_2010\_2020 & pct\_emp\_change\_2010\_2021 &
pop\_increased & emp\_increased \\
\midrule\noalign{}
\endhead
\bottomrule\noalign{}
\endlastfoot
0 & Autauga, AL & 7.758700 & 9.0 & True & True \\
1 & Baldwin, AL & 27.159356 & 28.2 & True & True \\
... & ... & ... & ... & ... & ... \\
3224 & Yabucoa, PR & -19.807069 & 0.1 & False & True \\
3225 & Yauco, PR & -18.721309 & -5.3 & False & False \\
\end{longtable}

We can now query the dataset to, for example, see which counties had a
population increase but an employment decrease. From a policy
perspective, this would be a concern, since employment is not able to
keep up with population growth.

\begin{Shaded}
\begin{Highlighting}[]
\NormalTok{changes\_df.query(}\StringTok{"pop\_increased == True \& emp\_increased == False"}\NormalTok{)}
\CommentTok{\# Or more succintly:}
\NormalTok{changes\_df.query(}\StringTok{"pop\_increased \& not emp\_increased"}\NormalTok{)}
\end{Highlighting}
\end{Shaded}

\begin{longtable}[]{@{}llllll@{}}
\toprule\noalign{}
& county & pop\_change\_2010\_2020 & pct\_emp\_change\_2010\_2021 &
pop\_increased & emp\_increased \\
\midrule\noalign{}
\endhead
\bottomrule\noalign{}
\endlastfoot
71 & Bethel, AK & 9.716099 & -0.7 & True & False \\
75 & Dillingham, AK & 0.206313 & -16.1 & True & False \\
... & ... & ... & ... & ... & ... \\
3127 & Campbell, WY & 1.935708 & -14.8 & True & False \\
3137 & Natrona, WY & 5.970842 & -0.2 & True & False \\
\end{longtable}

There are 242 such concerning counties.

\begin{tcolorbox}[enhanced jigsaw, colframe=quarto-callout-tip-color-frame, opacityback=0, titlerule=0mm, bottomrule=.15mm, breakable, leftrule=.75mm, colbacktitle=quarto-callout-tip-color!10!white, title=\textcolor{quarto-callout-tip-color}{\faLightbulb}\hspace{0.5em}{Practice}, rightrule=.15mm, coltitle=black, opacitybacktitle=0.6, colback=white, left=2mm, arc=.35mm, toptitle=1mm, bottomtitle=1mm, toprule=.15mm]

\subsection{Practice Q: Foreign-born
residents}\label{practice-q-foreign-born-residents}

\begin{itemize}
\tightlist
\item
  Use the \texttt{foreign\_born\_num} variable and the population
  estimate to calculate the percentage of foreign-born residents in each
  county. Then create a Boolean variable called
  \texttt{foreign\_born\_pct\_gt\_30} that is \texttt{True} if the
  percentage of foreign-born residents is greater than 30\% and
  \texttt{False} if it is not. Store the new DataFrame in a variable
  called \texttt{foreign\_born\_df\_question}.
\end{itemize}

\begin{Shaded}
\begin{Highlighting}[]
\CommentTok{\# your code here}
\end{Highlighting}
\end{Shaded}

\begin{itemize}
\tightlist
\item
  Use \texttt{.query()} on the \texttt{foreign\_born\_df\_question}
  DataFrame to return only the counties where the
  \texttt{foreign\_born\_pct\_gt\_30} variable is \texttt{True}. You
  should get 24 rows.
\end{itemize}

\begin{Shaded}
\begin{Highlighting}[]
\CommentTok{\# Your code here}
\end{Highlighting}
\end{Shaded}

\end{tcolorbox}

\begin{tcolorbox}[enhanced jigsaw, colframe=quarto-callout-tip-color-frame, opacityback=0, titlerule=0mm, bottomrule=.15mm, breakable, leftrule=.75mm, colbacktitle=quarto-callout-tip-color!10!white, title=\textcolor{quarto-callout-tip-color}{\faLightbulb}\hspace{0.5em}{Pro-tip}, rightrule=.15mm, coltitle=black, opacitybacktitle=0.6, colback=white, left=2mm, arc=.35mm, toptitle=1mm, bottomtitle=1mm, toprule=.15mm]

\subsection{\texorpdfstring{Accessing variables with \texttt{{[}{]}}
inside a lambda
function}{Accessing variables with {[}{]} inside a lambda function}}\label{accessing-variables-with-inside-a-lambda-function}

We can also use the square brackets \texttt{{[}{]}} to access the
variables. This is useful in two cases:

\begin{enumerate}
\def\labelenumi{\arabic{enumi}.}
\tightlist
\item
  When the variable name has special characters.
\item
  When the variable name is a reserved word or method name in Python.
\end{enumerate}

If the population variable were called \texttt{pop\ 20}, with a space,
we would not be able to access it with dot notation.

\begin{Shaded}
\begin{Highlighting}[]
\NormalTok{demo\_df }\OperatorTok{=}\NormalTok{ area\_pop.rename(columns}\OperatorTok{=}\NormalTok{\{}\StringTok{"pop\_20"}\NormalTok{: }\StringTok{"pop 20"}\NormalTok{\})}
\NormalTok{demo\_df}
\end{Highlighting}
\end{Shaded}

\begin{longtable}[]{@{}llll@{}}
\toprule\noalign{}
& county & area\_sq\_miles & pop 20 \\
\midrule\noalign{}
\endhead
\bottomrule\noalign{}
\endlastfoot
0 & Autauga, AL & 594.456107 & 58877.0 \\
1 & Baldwin, AL & 1589.836014 & 233140.0 \\
... & ... & ... & ... \\
3224 & Yabucoa, PR & 55.214614 & 30364.0 \\
3225 & Yauco, PR & 67.711484 & 34062.0 \\
\end{longtable}

\begin{Shaded}
\begin{Highlighting}[]
\NormalTok{demo\_df.assign(pop\_per\_sq\_km}\OperatorTok{=}\KeywordTok{lambda}\NormalTok{ x: x.pop }\DecValTok{20} \OperatorTok{/}\NormalTok{ x.area\_sq\_miles) }\CommentTok{\# gives error}
\end{Highlighting}
\end{Shaded}

\begin{verbatim}
area_pop.assign(pop_per_sq_km=lambda x: x.pop 20 / x.area_sq_miles)
                                            ^
SyntaxError: invalid syntax. Perhaps you forgot a comma?
\end{verbatim}

So we would need to use square brackets to access the variable.

\begin{Shaded}
\begin{Highlighting}[]
\NormalTok{demo\_df.assign(pop\_per\_sq\_km}\OperatorTok{=}\KeywordTok{lambda}\NormalTok{ x: x[}\StringTok{"pop 20"}\NormalTok{] }\OperatorTok{/}\NormalTok{ x.area\_sq\_miles)}
\end{Highlighting}
\end{Shaded}

\begin{longtable}[]{@{}lllll@{}}
\toprule\noalign{}
& county & area\_sq\_miles & pop 20 & pop\_per\_sq\_km \\
\midrule\noalign{}
\endhead
\bottomrule\noalign{}
\endlastfoot
0 & Autauga, AL & 594.456107 & 58877.0 & 99.043477 \\
1 & Baldwin, AL & 1589.836014 & 233140.0 & 146.644055 \\
... & ... & ... & ... & ... \\
3224 & Yabucoa, PR & 55.214614 & 30364.0 & 549.926871 \\
3225 & Yauco, PR & 67.711484 & 34062.0 & 503.046133 \\
\end{longtable}

In reality, we should probably just rename such a variable though!

\begin{center}\rule{0.5\linewidth}{0.5pt}\end{center}

The square brackets are also needed if our variable name is a reserved
word or method name in Python.

For example, if our population variable was called \texttt{pop}, we
would get an error when accessing it with dot notation in the
\texttt{assign()} method, because \texttt{pop} is a method name.

\begin{Shaded}
\begin{Highlighting}[]
\NormalTok{demo\_df }\OperatorTok{=}\NormalTok{ area\_pop.rename(columns}\OperatorTok{=}\NormalTok{\{}\StringTok{"pop\_20"}\NormalTok{: }\StringTok{"pop"}\NormalTok{\})}
\NormalTok{demo\_df}
\end{Highlighting}
\end{Shaded}

\begin{longtable}[]{@{}llll@{}}
\toprule\noalign{}
& county & area\_sq\_miles & pop \\
\midrule\noalign{}
\endhead
\bottomrule\noalign{}
\endlastfoot
0 & Autauga, AL & 594.456107 & 58877.0 \\
1 & Baldwin, AL & 1589.836014 & 233140.0 \\
... & ... & ... & ... \\
3224 & Yabucoa, PR & 55.214614 & 30364.0 \\
3225 & Yauco, PR & 67.711484 & 34062.0 \\
\end{longtable}

\begin{Shaded}
\begin{Highlighting}[]
\NormalTok{demo\_df.assign(pop\_per\_sq\_mile}\OperatorTok{=}\KeywordTok{lambda}\NormalTok{ x: x.pop }\OperatorTok{/}\NormalTok{ x.area\_sq\_miles)}
\end{Highlighting}
\end{Shaded}

\begin{verbatim}
TypeError: unsupported operand type(s) for /: 'method' and 'float'
\end{verbatim}

In such a case, we can use the square brackets \texttt{{[}{]}} to access
the \texttt{pop} variable.

\begin{Shaded}
\begin{Highlighting}[]
\NormalTok{demo\_df.assign(pop\_per\_sq\_mile}\OperatorTok{=}\KeywordTok{lambda}\NormalTok{ x: x[}\StringTok{"pop"}\NormalTok{] }\OperatorTok{/}\NormalTok{ x.area\_sq\_miles)}
\end{Highlighting}
\end{Shaded}

\begin{longtable}[]{@{}lllll@{}}
\toprule\noalign{}
& county & area\_sq\_miles & pop & pop\_per\_sq\_mile \\
\midrule\noalign{}
\endhead
\bottomrule\noalign{}
\endlastfoot
0 & Autauga, AL & 594.456107 & 58877.0 & 99.043477 \\
1 & Baldwin, AL & 1589.836014 & 233140.0 & 146.644055 \\
... & ... & ... & ... & ... \\
3224 & Yabucoa, PR & 55.214614 & 30364.0 & 549.926871 \\
3225 & Yauco, PR & 67.711484 & 34062.0 & 503.046133 \\
\end{longtable}

\end{tcolorbox}

\section{Wrap up}\label{wrap-up-6}

As you can imagine, transforming data is an essential step in any data
analysis workflow. It is often required to clean data and to prepare it
for further statistical analysis or for making plots. And as you have
seen, it is quite simple to transform data with pandas'
\texttt{assign()} method.

Congrats on making it through.

But your data wrangling journey isn't over yet! In our next lessons, we
will learn how to create complex data summaries and how to create and
work with data frame groups. Intrigued? See you in the next lesson.

\chapter{Conditional assignment with case\_when() and
np.where()}\label{conditional-assignment-with-case_when-and-np.where}

\section{Introduction}\label{introduction-7}

In the previous lesson, you learned the basics of data transformation
using pandas' \texttt{assign()} method.

In that lesson, we looked at \emph{global} transformations; that is,
transformations that did the same thing to an entire variable. In this
lesson, we will look at how to \emph{conditionally} manipulate certain
rows based on whether or not they meet defined criteria.

For this, we will mostly use the powerful pandas\texttt{case\_when()}
method. Note that this was introduced recently, in pandas 2.2, so if you
are using an older version of pandas, you will need to upgrade.

Let's get started.

\section{Learning objectives}\label{learning-objectives-9}

\begin{enumerate}
\def\labelenumi{\arabic{enumi}.}
\item
  You can transform or create new variables based on conditions using
  \texttt{case\_when()}
\item
  You know how to use a default condition in \texttt{case\_when()} to
  match unmatched cases.
\item
  You can handle \texttt{NaN} values in \texttt{case\_when()}
  transformations.
\item
  You understand \texttt{case\_when()} conditions priority order.
\item
  You can use \texttt{np.where()} for binary conditional assignment.
\end{enumerate}

\section{Packages}\label{packages-1}

This lesson will require pandas and numpy:

\begin{Shaded}
\begin{Highlighting}[]
\ImportTok{import}\NormalTok{ pandas }\ImportTok{as}\NormalTok{ pd}
\ImportTok{import}\NormalTok{ numpy }\ImportTok{as}\NormalTok{ np}
\end{Highlighting}
\end{Shaded}

\section{Datasets}\label{datasets-1}

In this lesson, we will use a dataset of counties of the United States
with demographic and economic data.

Below we import it, then add some NAs to the \texttt{pop\_20} column, to
simulate missing data.

\begin{Shaded}
\begin{Highlighting}[]
\CommentTok{\# Import and view the dataset}
\NormalTok{counties }\OperatorTok{=}\NormalTok{ pd.read\_csv(}\StringTok{"data/us\_counties\_data.csv"}\NormalTok{)}

\CommentTok{\# Add NAs to the population column}
\NormalTok{counties.loc[counties.index }\OperatorTok{\%} \DecValTok{5} \OperatorTok{==} \DecValTok{0}\NormalTok{, }\StringTok{"pop\_20"}\NormalTok{] }\OperatorTok{=}\NormalTok{ np.nan}

\NormalTok{counties}
\end{Highlighting}
\end{Shaded}

\begin{longtable}[]{@{}lllllllllll@{}}
\toprule\noalign{}
& state & county & pop\_20 & area\_sq\_miles & hh\_inc\_21 & econ\_type
& unemp\_20 & foreign\_born\_num & pop\_change\_2010\_2020 &
pct\_emp\_change\_2010\_2021 \\
\midrule\noalign{}
\endhead
\bottomrule\noalign{}
\endlastfoot
0 & AL & Autauga, AL & NaN & 594.456107 & 66444.0 & Nonspecialized & 5.4
& 1241.0 & 7.758700 & 9.0 \\
1 & AL & Baldwin, AL & 233140.0 & 1589.836014 & 65658.0 & Recreation &
6.2 & 7938.0 & 27.159356 & 28.2 \\
2 & AL & Barbour, AL & 25180.0 & 885.008019 & 38649.0 & Manufacturing &
7.8 & 659.0 & -8.136359 & -13.9 \\
... & ... & ... & ... & ... & ... & ... & ... & ... & ... & ... \\
3223 & PR & Villalba, PR & 22044.0 & 35.636678 & NaN & NaN & NaN & NaN &
-15.264833 & -1.2 \\
3224 & PR & Yabucoa, PR & 30364.0 & 55.214614 & NaN & NaN & NaN & NaN &
-19.807069 & 0.1 \\
3225 & PR & Yauco, PR & NaN & 67.711484 & NaN & NaN & NaN & NaN &
-18.721309 & -5.3 \\
\end{longtable}

The variables in the dataset are:

\begin{itemize}
\tightlist
\item
  \texttt{state}, US state
\item
  \texttt{county}, US county
\item
  \texttt{pop\_20}, population estimate for 2020
\item
  \texttt{area\_sq\_miles}, area in square miles
\item
  \texttt{hh\_inc\_21}, median household income for 2021
\item
  \texttt{econ\_type}, economic type of the county
\item
  \texttt{pop\_change\_2010\_2020}, population change between 2010 and
  2020
\item
  \texttt{unemp\_20}, unemployment rate for 2020
\item
  \texttt{pct\_emp\_change\_2010\_2021}, percentage change in employment
  between 2010 and 2021
\end{itemize}

The variables are collected from a range of sources, including the US
Census Bureau, the Bureau of Labor Statistics, and the American
Community Survey.

Let's also make some subsets of the data to work with.

\begin{Shaded}
\begin{Highlighting}[]
\NormalTok{counties\_pop }\OperatorTok{=}\NormalTok{ counties[[}\StringTok{"pop\_20"}\NormalTok{]]}
\NormalTok{counties\_income }\OperatorTok{=}\NormalTok{ counties[[}\StringTok{"hh\_inc\_21"}\NormalTok{]]}
\end{Highlighting}
\end{Shaded}

\section{\texorpdfstring{Introduction to
\texttt{case\_when()}}{Introduction to case\_when()}}\label{introduction-to-case_when}

To get familiar with \texttt{case\_when()}, let's begin with a simple
conditional transformation on the \texttt{pop\_20} column of the
\texttt{counties} dataset.

We will make a new column, called ``pop\_class'', that has the value
``Smaller'' if the population is below 50,000, and ``Larger'' if the
population is 50,000 or more.

\begin{Shaded}
\begin{Highlighting}[]
\NormalTok{counties\_pop }\OperatorTok{=}\NormalTok{ counties\_pop.assign(}
\NormalTok{    pop\_class}\OperatorTok{=}\KeywordTok{lambda}\NormalTok{ x: x.pop\_20.case\_when(}
\NormalTok{        [}
\NormalTok{            (x.pop\_20 }\OperatorTok{\textless{}} \DecValTok{50000}\NormalTok{, }\StringTok{"Smaller"}\NormalTok{),}
\NormalTok{            (x.pop\_20 }\OperatorTok{\textgreater{}=} \DecValTok{50000}\NormalTok{, }\StringTok{"Larger"}\NormalTok{),}
\NormalTok{        ]}
\NormalTok{    )}
\NormalTok{)}
\NormalTok{counties\_pop}
\end{Highlighting}
\end{Shaded}

\begin{longtable}[]{@{}lll@{}}
\toprule\noalign{}
& pop\_20 & pop\_class \\
\midrule\noalign{}
\endhead
\bottomrule\noalign{}
\endlastfoot
0 & NaN & NaN \\
1 & 233140.0 & Larger \\
2 & 25180.0 & Smaller \\
... & ... & ... \\
3223 & 22044.0 & Smaller \\
3224 & 30364.0 & Smaller \\
3225 & NaN & NaN \\
\end{longtable}

The statement
\texttt{case\_when({[}(x.pop\_20\ \textless{}\ 50000,\ "Smaller"),\ (x.pop\_20\ \textgreater{}=\ 50000,\ "Larger"){]})}
can be read as: ``if \texttt{pop\_20} is below 50,000, input `Smaller',
else if \texttt{pop\_20} is greater than or equal to 50,000, input
`Larger'\,''.

The \texttt{case\_when()} syntax may seem a bit foreign. So let's break
it down:

\begin{itemize}
\tightlist
\item
  First, we use the \texttt{assign()} method to create a new column,
  \texttt{pop\_class}.
\item
  Then we use a lambda function to define the \texttt{pop\_class}
  column. Hopefullyou you recall how to write these.
\item
  Then, we use the \texttt{case\_when()} method to create the new
  column.
\item
  Inside the \texttt{case\_when()} method, we use a list of tuples.
\item
  Recall that lists are created with square brackets \texttt{{[}{]}}.
\item
  Tuples are created with parentheses \texttt{(itemA,\ itemB)}, with
  each element in the tuple separated by a comma.
\item
  So the list of tuples is
  \texttt{{[}(conditionA,\ valueA),\ (conditionB,\ valueB){]}}.
\end{itemize}

\begin{center}\rule{0.5\linewidth}{0.5pt}\end{center}

After creating a new variable with \texttt{case\_when()}, you shoul
inspect it to make sure the transformation worked as intended.

Let's use the \texttt{value\_counts()} method to ensure that the numbers
and proportions ``make sense'':

\begin{Shaded}
\begin{Highlighting}[]
\NormalTok{counties\_pop[}\StringTok{"pop\_class"}\NormalTok{].value\_counts() }\CommentTok{\# numbers}
\end{Highlighting}
\end{Shaded}

\begin{verbatim}
pop_class
Smaller    1773
Larger      801
Name: count, dtype: int64
\end{verbatim}

\begin{Shaded}
\begin{Highlighting}[]
\NormalTok{counties\_pop[}\StringTok{"pop\_class"}\NormalTok{].value\_counts(normalize}\OperatorTok{=}\VariableTok{True}\NormalTok{) }\CommentTok{\# proportions}
\end{Highlighting}
\end{Shaded}

\begin{verbatim}
pop_class
Smaller    0.688811
Larger     0.311189
Name: proportion, dtype: float64
\end{verbatim}

Do these seem reasonable based on the population size of US counties?
(Yes!)

Let's make it a bit more complex with a third condition for medium-sized
counties. Counties with populations from 30,000 to below 100,000 will be
labeled ``Medium'', and counties with populations of 100,000 and up will
be labeled ``Large''.

\begin{Shaded}
\begin{Highlighting}[]
\NormalTok{counties\_pop }\OperatorTok{=}\NormalTok{ counties\_pop.assign(}
\NormalTok{    pop\_class}\OperatorTok{=}\KeywordTok{lambda}\NormalTok{ x: x.pop\_20.case\_when(}
\NormalTok{        [}
\NormalTok{            (x.pop\_20 }\OperatorTok{\textless{}} \DecValTok{30000}\NormalTok{, }\StringTok{"Small"}\NormalTok{),}
\NormalTok{            ((x.pop\_20 }\OperatorTok{\textgreater{}=} \DecValTok{30000}\NormalTok{) }\OperatorTok{\&}\NormalTok{ (x.pop\_20 }\OperatorTok{\textless{}} \DecValTok{100000}\NormalTok{), }\StringTok{"Medium"}\NormalTok{),}
\NormalTok{            (x.pop\_20 }\OperatorTok{\textgreater{}=} \DecValTok{100000}\NormalTok{, }\StringTok{"Large"}\NormalTok{),}
\NormalTok{        ]}
\NormalTok{    )}
\NormalTok{)}
\NormalTok{counties\_pop}
\end{Highlighting}
\end{Shaded}

\begin{longtable}[]{@{}lll@{}}
\toprule\noalign{}
& pop\_20 & pop\_class \\
\midrule\noalign{}
\endhead
\bottomrule\noalign{}
\endlastfoot
0 & NaN & NaN \\
1 & 233140.0 & Large \\
2 & 25180.0 & Small \\
... & ... & ... \\
3223 & 22044.0 & Small \\
3224 & 30364.0 & Medium \\
3225 & NaN & NaN \\
\end{longtable}

Note that when we combine multiple conditions with \texttt{\&}, we must
use parentheses around each condition.

\begin{tcolorbox}[enhanced jigsaw, colframe=quarto-callout-tip-color-frame, opacityback=0, titlerule=0mm, bottomrule=.15mm, breakable, leftrule=.75mm, colbacktitle=quarto-callout-tip-color!10!white, title=\textcolor{quarto-callout-tip-color}{\faLightbulb}\hspace{0.5em}{Practice}, rightrule=.15mm, coltitle=black, opacitybacktitle=0.6, colback=white, left=2mm, arc=.35mm, toptitle=1mm, bottomtitle=1mm, toprule=.15mm]

\subsection{Practice Q: Recode income
class}\label{practice-q-recode-income-class}

The \texttt{counties\_income} dataset has a column called
\texttt{hh\_inc\_21} with the median household income in 2021. Make a
new column, called \texttt{income\_group}, with the following three
groups:

\begin{itemize}
\tightlist
\item
  ``Below 30k'' for counties with income under \$30,000
\item
  ``30k to 60k'' for counties with income between \$30,000 and \$60,000
  (exclusive of 60k)
\item
  ``60k and above'' for counties with income of \$60,000 and up
\end{itemize}

\begin{Shaded}
\begin{Highlighting}[]
\CommentTok{\# Complete the code with your answer:}
\NormalTok{Q\_income\_group }\OperatorTok{=}\NormalTok{ counties\_income.assign(}
\NormalTok{    income\_group}\OperatorTok{=}\KeywordTok{lambda}\NormalTok{ x: x.hh\_inc\_21.case\_when(}
\NormalTok{        [}
\NormalTok{            (x.hh\_inc\_21 }\OperatorTok{\textless{}} \DecValTok{30000}\NormalTok{, }\StringTok{"Below 30k"}\NormalTok{),}
\NormalTok{            \_\_\_\_\_\_\_\_\_\_\_\_\_\_\_\_\_\_\_\_\_\_\_\_\_\_\_\_\_\_\_\_\_}
\NormalTok{            \_\_\_\_\_\_\_\_\_\_\_\_\_\_\_\_\_\_\_\_\_\_\_\_\_\_\_\_\_\_\_\_\_}
\NormalTok{        ]}
\NormalTok{    )}
\NormalTok{)}
\NormalTok{Q\_income\_group[}\StringTok{"income\_group"}\NormalTok{].value\_counts(normalize}\OperatorTok{=}\VariableTok{True}\NormalTok{)}
\end{Highlighting}
\end{Shaded}

\begin{itemize}
\tightlist
\item
  Use the \texttt{value\_counts()} method to count the proportions of
  each income group. If you did it correctly, the proportion of counties
  with median household income below \$60,000 should be approximately
  60\%.
\end{itemize}

\end{tcolorbox}

\begin{tcolorbox}[enhanced jigsaw, colframe=quarto-callout-note-color-frame, opacityback=0, titlerule=0mm, bottomrule=.15mm, breakable, leftrule=.75mm, colbacktitle=quarto-callout-note-color!10!white, title=\textcolor{quarto-callout-note-color}{\faInfo}\hspace{0.5em}{Pro-tip}, rightrule=.15mm, coltitle=black, opacitybacktitle=0.6, colback=white, left=2mm, arc=.35mm, toptitle=1mm, bottomtitle=1mm, toprule=.15mm]

\section{Denoting intervals}\label{denoting-intervals}

When categorizing variables, it's important to clearly communicate
whether each category includes or excludes its boundaries. Here are two
common approaches:

\begin{enumerate}
\def\labelenumi{\arabic{enumi}.}
\tightlist
\item
  Mathematical notation:

  \begin{itemize}
  \tightlist
  \item
    Use square brackets \texttt{{[}{]}} for inclusive bounds and
    parentheses \texttt{()} for exclusive bounds.
  \item
    Example:

    \begin{itemize}
    \tightlist
    \item
      {[}0, 60,000) : This means under \$60,000
    \item
      {[}60,000, 100,000) : This means \$60,000 to \$99,999
    \item
      {[}100,000, ∞) : This means \$100,000 and up. The ∞ symbol stands
      for infinity.
    \end{itemize}
  \end{itemize}
\item
  Code-friendly notation:

  \begin{itemize}
  \tightlist
  \item
    Use \texttt{\textgreater{}=} and \texttt{\textless{}=} for inclusive
    bounds.
  \item
    Use \texttt{\textgreater{}} and \texttt{\textless{}} for exclusive
    bounds.
  \item
    Example:

    \begin{itemize}
    \tightlist
    \item
      x \textless{} \$60,000 : This means under \$60,000
    \item
      \$60,000 \textless{} x \textless{} \$100,000 : This means \$60,000
      to \$99,999.99\ldots{}
    \item
      x \textgreater= \$100,000 : This means \$100,000 and up
    \end{itemize}
  \end{itemize}
\item
  Rounded integer ranges (more reader-friendly):

  \begin{itemize}
  \tightlist
  \item
    Round values to integers and use descriptive language.
  \item
    Example:

    \begin{itemize}
    \tightlist
    \item
      \$0 to \$59,999
    \item
      \$60,000 to \$99,999
    \item
      \$100,000 and up
    \end{itemize}
  \end{itemize}
\end{enumerate}

Choose the method that best suits your audience and context.

\end{tcolorbox}

\section{A catch-all condition}\label{a-catch-all-condition}

In a \texttt{case\_when()} statement, you sometimes want to catch all
rows not matched with provided conditions. You can do this with a
catch-all condition. This should be something that always returns
\texttt{True}. Unfortunately, simply using \texttt{True} as the
condition will not work, since we need a \textbf{sequence} of True
values.

You could use \texttt{pd.Series(True,\ index=df.index)} to create a
sequence of \texttt{True} values, but perhaps an easier solution is to
test whether the values are equal to themselves, using, for example
\texttt{x.pop\_20\ ==\ x.pop\_20}:

\begin{Shaded}
\begin{Highlighting}[]
\NormalTok{counties\_pop }\OperatorTok{=}\NormalTok{ counties\_pop.assign(}
\NormalTok{    pop\_class}\OperatorTok{=}\KeywordTok{lambda}\NormalTok{ x: x.pop\_20.case\_when(}
\NormalTok{        [}
\NormalTok{            ((x.pop\_20 }\OperatorTok{\textgreater{}=} \DecValTok{30000}\NormalTok{) }\OperatorTok{\&}\NormalTok{ (x.pop\_20 }\OperatorTok{\textless{}} \DecValTok{100000}\NormalTok{), }\StringTok{"Medium"}\NormalTok{),}
\NormalTok{            (x.pop\_20 }\OperatorTok{==}\NormalTok{ x.pop\_20, }\StringTok{"Other"}\NormalTok{),}
\NormalTok{        ]}
\NormalTok{    )}
\NormalTok{)}
\NormalTok{counties\_pop}
\end{Highlighting}
\end{Shaded}

\begin{longtable}[]{@{}lll@{}}
\toprule\noalign{}
& pop\_20 & pop\_class \\
\midrule\noalign{}
\endhead
\bottomrule\noalign{}
\endlastfoot
0 & NaN & NaN \\
1 & 233140.0 & Other \\
2 & 25180.0 & Other \\
... & ... & ... \\
3223 & 22044.0 & Other \\
3224 & 30364.0 & Medium \\
3225 & NaN & NaN \\
\end{longtable}

This \texttt{x.pop\_20\ ==\ x.pop\_20} condition can be read as ``for
everything else\ldots{}''.

\begin{tcolorbox}[enhanced jigsaw, colframe=quarto-callout-caution-color-frame, opacityback=0, titlerule=0mm, bottomrule=.15mm, breakable, leftrule=.75mm, colbacktitle=quarto-callout-caution-color!10!white, title=\textcolor{quarto-callout-caution-color}{\faFire}\hspace{0.5em}{Watch Out}, rightrule=.15mm, coltitle=black, opacitybacktitle=0.6, colback=white, left=2mm, arc=.35mm, toptitle=1mm, bottomtitle=1mm, toprule=.15mm]

\subsection{Order of conditions}\label{order-of-conditions}

It is important to use \texttt{x.pop\_20\ ==\ x.pop\_20} as the
\emph{final} condition in \texttt{case\_when()}. If you use it as the
first condition, it will take precedence over all others, as seen here:

\begin{Shaded}
\begin{Highlighting}[]
\NormalTok{counties\_pop }\OperatorTok{=}\NormalTok{ counties\_pop.assign(}
\NormalTok{    pop\_class}\OperatorTok{=}\KeywordTok{lambda}\NormalTok{ x: x.pop\_20.case\_when(}
\NormalTok{        [}
\NormalTok{            (x.pop\_20 }\OperatorTok{==}\NormalTok{ x.pop\_20, }\StringTok{"Other"}\NormalTok{),}
\NormalTok{            ((x.pop\_20 }\OperatorTok{\textgreater{}=} \DecValTok{30000}\NormalTok{) }\OperatorTok{\&}\NormalTok{ (x.pop\_20 }\OperatorTok{\textless{}} \DecValTok{100000}\NormalTok{), }\StringTok{"Medium"}\NormalTok{),}
\NormalTok{        ]}
\NormalTok{    )}
\NormalTok{)}
\NormalTok{counties\_pop}
\end{Highlighting}
\end{Shaded}

\begin{longtable}[]{@{}lll@{}}
\toprule\noalign{}
& pop\_20 & pop\_class \\
\midrule\noalign{}
\endhead
\bottomrule\noalign{}
\endlastfoot
0 & NaN & NaN \\
1 & 233140.0 & Other \\
2 & 25180.0 & Other \\
... & ... & ... \\
3223 & 22044.0 & Other \\
3224 & 30364.0 & Other \\
3225 & NaN & NaN \\
\end{longtable}

As you can observe, all non-NaN counties are now coded with ``Other'',
because the \texttt{x.pop\_20\ ==\ x.pop\_20} condition was placed
first, and therefore took precedence.

Statements with case\_when are evaluated from top to bottom, so
conditions that should be evaluated first should be placed higher in the
list.

\end{tcolorbox}

\begin{tcolorbox}[enhanced jigsaw, colframe=quarto-callout-tip-color-frame, opacityback=0, titlerule=0mm, bottomrule=.15mm, breakable, leftrule=.75mm, colbacktitle=quarto-callout-tip-color!10!white, title=\textcolor{quarto-callout-tip-color}{\faLightbulb}\hspace{0.5em}{Practice}, rightrule=.15mm, coltitle=black, opacitybacktitle=0.6, colback=white, left=2mm, arc=.35mm, toptitle=1mm, bottomtitle=1mm, toprule=.15mm]

\subsection{Practice Q: Recode income class
2}\label{practice-q-recode-income-class-2}

Using the \texttt{counties\_income} data, and the \texttt{hh\_inc\_21}
column, create a new column called \texttt{income\_class} that has the
value ``Medium income'' for counties with median household income from
\$60,000 to below \$100,000 and ``Other'' for all other counties. (Use a
catch-all condition.) Your new dataframe should be called
\texttt{Q\_income\_class}.

\begin{Shaded}
\begin{Highlighting}[]
\CommentTok{\# Your code here}
\end{Highlighting}
\end{Shaded}

\end{tcolorbox}

\section{Counting counties by name
characteristics}\label{counting-counties-by-name-characteristics}

To get more practice with \texttt{case\_when()}, let's explore a fun
application of it by analyzing county names.

We'll look for counties that have the words ``River'' or ``Rio''
(spanish for river) in their names. This may be interesting to know, as
these often indicate counties with significant water features.

First, we'll use \texttt{case\_when()} to create a new column that flags
these counties:

\begin{Shaded}
\begin{Highlighting}[]
\NormalTok{counties\_names }\OperatorTok{=}\NormalTok{ counties[[}\StringTok{"county"}\NormalTok{]]}

\NormalTok{counties\_names }\OperatorTok{=}\NormalTok{ counties\_names.assign(}
\NormalTok{    water\_name}\OperatorTok{=}\KeywordTok{lambda}\NormalTok{ x: x.county.case\_when(}
\NormalTok{        [}
\NormalTok{            (x.county.}\BuiltInTok{str}\NormalTok{.contains(}\StringTok{"River|Rio"}\NormalTok{), }\StringTok{"River in name"}\NormalTok{),}
\NormalTok{            (x.county }\OperatorTok{==}\NormalTok{ x.county, }\StringTok{"Other"}\NormalTok{),}
\NormalTok{        ]}
\NormalTok{    )}
\NormalTok{)}
\NormalTok{counties\_names}
\end{Highlighting}
\end{Shaded}

\begin{longtable}[]{@{}lll@{}}
\toprule\noalign{}
& county & water\_name \\
\midrule\noalign{}
\endhead
\bottomrule\noalign{}
\endlastfoot
0 & Autauga, AL & Other \\
1 & Baldwin, AL & Other \\
2 & Barbour, AL & Other \\
... & ... & ... \\
3223 & Villalba, PR & Other \\
3224 & Yabucoa, PR & Other \\
3225 & Yauco, PR & Other \\
\end{longtable}

In this code, we're using a regular expression
\texttt{"River\textbar{}Rio"} to match either ``River'' or ``Rio'' in
the county name. The \texttt{\textbar{}} symbol in regex means ``or''.
We're also using our catch-all condition \texttt{x.county\ ==\ x.county}
to label all other counties as ``Other''.

Let's see how many counties we've identified:

\begin{Shaded}
\begin{Highlighting}[]
\NormalTok{counties\_names[}\StringTok{"water\_name"}\NormalTok{].value\_counts(normalize}\OperatorTok{=}\VariableTok{True}\NormalTok{)}
\end{Highlighting}
\end{Shaded}

\begin{verbatim}
water_name
Other            0.99597
River in name    0.00403
Name: proportion, dtype: float64
\end{verbatim}

Interesting! It looks like about X\% of counties have ``River'' or
``Rio'' in their name.

Now, let's use \texttt{.query()} to keep just those water-related
counties:

\begin{Shaded}
\begin{Highlighting}[]
\NormalTok{water\_counties }\OperatorTok{=}\NormalTok{ counties\_names.query(}\StringTok{"water\_name == \textquotesingle{}Water in name\textquotesingle{}"}\NormalTok{)}
\NormalTok{water\_counties}
\end{Highlighting}
\end{Shaded}

\begin{longtable}[]{@{}lll@{}}
\toprule\noalign{}
& county & water\_name \\
\midrule\noalign{}
\endhead
\bottomrule\noalign{}
\endlastfoot
\end{longtable}

Cool! Do you recognize any of these counties?

\begin{tcolorbox}[enhanced jigsaw, colframe=quarto-callout-tip-color-frame, opacityback=0, titlerule=0mm, bottomrule=.15mm, breakable, leftrule=.75mm, colbacktitle=quarto-callout-tip-color!10!white, title=\textcolor{quarto-callout-tip-color}{\faLightbulb}\hspace{0.5em}{Practice}, rightrule=.15mm, coltitle=black, opacitybacktitle=0.6, colback=white, left=2mm, arc=.35mm, toptitle=1mm, bottomtitle=1mm, toprule=.15mm]

\subsection{Practice Q: Find lake or bay
counties}\label{practice-q-find-lake-or-bay-counties}

In a similar way to what we did with ``River'' or ``Rio'', find counties
with ``Lake'' or ``Bay'' in their names. Create a new column called
\texttt{lake\_bay\_name} and then use \texttt{.query()} to create a
dataframe called \texttt{lake\_bay\_counties} with only these counties.

Start with the \texttt{counties\_names} dataframe.

\begin{Shaded}
\begin{Highlighting}[]
\CommentTok{\# Your code here}
\end{Highlighting}
\end{Shaded}

How does the proportion of ``Lake'' or ``Bay'' counties compare to
``River'' or ``Rio'' counties?

\end{tcolorbox}

This example shows how we can use \texttt{case\_when()} along with
string methods and regular expressions to create meaningful categories
based on text data. It's a powerful way to explore patterns in your
dataset!

\section{\texorpdfstring{Matching NaN's with
\texttt{isna()}}{Matching NaN's with isna()}}\label{matching-nans-with-isna}

As you may have noticed, the \texttt{pop\_20} column contains
\texttt{NaN} values, and these are not matched with any of the
conditions we provided.

We can match missing values manually with \texttt{isna()}. Below we
match \texttt{NaN} populations with \texttt{isna()} and set their
population size to ``Missing population'':

\begin{Shaded}
\begin{Highlighting}[]
\NormalTok{counties\_pop }\OperatorTok{=}\NormalTok{ counties\_pop.assign(}
\NormalTok{    pop\_class}\OperatorTok{=}\KeywordTok{lambda}\NormalTok{ x: x.pop\_20.case\_when(}
\NormalTok{        [}
\NormalTok{            (x.pop\_20 }\OperatorTok{\textless{}} \DecValTok{50000}\NormalTok{, }\StringTok{"Smaller"}\NormalTok{),}
\NormalTok{            (x.pop\_20 }\OperatorTok{\textgreater{}=} \DecValTok{50000}\NormalTok{, }\StringTok{"Larger"}\NormalTok{),}
\NormalTok{            (x.pop\_20.isna(), }\StringTok{"Missing"}\NormalTok{),}
\NormalTok{        ]}
\NormalTok{    )}
\NormalTok{)}
\NormalTok{counties\_pop}
\end{Highlighting}
\end{Shaded}

\begin{longtable}[]{@{}lll@{}}
\toprule\noalign{}
& pop\_20 & pop\_class \\
\midrule\noalign{}
\endhead
\bottomrule\noalign{}
\endlastfoot
0 & NaN & Missing \\
1 & 233140.0 & Larger \\
2 & 25180.0 & Smaller \\
... & ... & ... \\
3223 & 22044.0 & Smaller \\
3224 & 30364.0 & Smaller \\
3225 & NaN & Missing \\
\end{longtable}

\begin{tcolorbox}[enhanced jigsaw, colframe=quarto-callout-tip-color-frame, opacityback=0, titlerule=0mm, bottomrule=.15mm, breakable, leftrule=.75mm, colbacktitle=quarto-callout-tip-color!10!white, title=\textcolor{quarto-callout-tip-color}{\faLightbulb}\hspace{0.5em}{Practice}, rightrule=.15mm, coltitle=black, opacitybacktitle=0.6, colback=white, left=2mm, arc=.35mm, toptitle=1mm, bottomtitle=1mm, toprule=.15mm]

\subsection{Practice Q: Recode economic
type}\label{practice-q-recode-economic-type}

The \texttt{econ\_type} column of the \texttt{counties} dataset
describes the primary economic activity of the county:

\begin{Shaded}
\begin{Highlighting}[]
\NormalTok{practice\_econ }\OperatorTok{=}\NormalTok{ counties[[}\StringTok{"econ\_type"}\NormalTok{]]}
\NormalTok{practice\_econ.value\_counts(dropna}\OperatorTok{=}\VariableTok{False}\NormalTok{)}
\end{Highlighting}
\end{Shaded}

\begin{verbatim}
econ_type               
Nonspecialized              1237
Manufacturing                501
Farming                      444
Federal/State Government     407
Recreation                   333
Mining                       221
NaN                           83
Name: count, dtype: int64
\end{verbatim}

Implement the following recoding, storing the new data in a column
called \texttt{econ\_type\_recode}:

\begin{itemize}
\tightlist
\item
  Farming, Mining and Manufacturing to ``Industry''
\item
  Federal/State Government to ``Government''
\item
  Recreation and Nonspecialized to ``Other''
\item
  NaNs to ``Missing''
\end{itemize}

To get you started, here is how to recode ``Farming, Mining and
Manufacturing'' to ``Industry'':

\begin{Shaded}
\begin{Highlighting}[]
\CommentTok{\# Your code here}
\end{Highlighting}
\end{Shaded}

\end{tcolorbox}

\section{\texorpdfstring{Binary conditions:
\texttt{np.where()}}{Binary conditions: np.where()}}\label{binary-conditions-np.where}

There is another function similar to \texttt{case\_when()} for when we
want to apply a binary condition to a variable: \texttt{np.where()}. A
binary condition is either \texttt{True} or \texttt{False}.

Let's test it out with a simple Series:

\begin{Shaded}
\begin{Highlighting}[]
\NormalTok{x }\OperatorTok{=}\NormalTok{ pd.Series([}\DecValTok{1}\NormalTok{, }\DecValTok{2}\NormalTok{, }\DecValTok{3}\NormalTok{, }\DecValTok{4}\NormalTok{, }\DecValTok{5}\NormalTok{])}
\NormalTok{x}
\end{Highlighting}
\end{Shaded}

\begin{verbatim}
0    1
1    2
2    3
3    4
4    5
dtype: int64
\end{verbatim}

\begin{Shaded}
\begin{Highlighting}[]
\NormalTok{np.where(x }\OperatorTok{\textgreater{}} \DecValTok{3}\NormalTok{, }\StringTok{"Above 3"}\NormalTok{, }\StringTok{"3 or below"}\NormalTok{)}
\end{Highlighting}
\end{Shaded}

\begin{verbatim}
array(['3 or below', '3 or below', '3 or below', 'Above 3', 'Above 3'],
      dtype='<U10')
\end{verbatim}

As you can see, \texttt{np.where()} returns a new array with the values
``Above 3'' for all values greater than 3, and ``3 or below'' for all
values less than or equal to 3.

This is useful when you want to make a quick modification to a variable,
and don't want to type up a whole \texttt{case\_when()} statement.

Let's try it out on our dataset to classify counties as ``Larger'' if
their population is 50000 or more, and ``Smaller'' if their population
is less than 50000.

\begin{Shaded}
\begin{Highlighting}[]
\NormalTok{counties\_pop }\OperatorTok{=}\NormalTok{ counties\_pop.assign(}
\NormalTok{    pop\_class}\OperatorTok{=}\KeywordTok{lambda}\NormalTok{ x: np.where(x.pop\_20 }\OperatorTok{\textgreater{}=} \DecValTok{50000}\NormalTok{, }\StringTok{"Larger"}\NormalTok{, }\StringTok{"Smaller"}\NormalTok{)}
\NormalTok{)}
\NormalTok{counties\_pop}
\end{Highlighting}
\end{Shaded}

\begin{longtable}[]{@{}lll@{}}
\toprule\noalign{}
& pop\_20 & pop\_class \\
\midrule\noalign{}
\endhead
\bottomrule\noalign{}
\endlastfoot
0 & NaN & Smaller \\
1 & 233140.0 & Larger \\
2 & 25180.0 & Smaller \\
... & ... & ... \\
3223 & 22044.0 & Smaller \\
3224 & 30364.0 & Smaller \\
3225 & NaN & Smaller \\
\end{longtable}

Nice and easy!

Since it is only a binary condition, it's utility is limited, but it can
be useful for quick transformations.

\begin{tcolorbox}[enhanced jigsaw, colframe=quarto-callout-tip-color-frame, opacityback=0, titlerule=0mm, bottomrule=.15mm, breakable, leftrule=.75mm, colbacktitle=quarto-callout-tip-color!10!white, title=\textcolor{quarto-callout-tip-color}{\faLightbulb}\hspace{0.5em}{Practice}, rightrule=.15mm, coltitle=black, opacitybacktitle=0.6, colback=white, left=2mm, arc=.35mm, toptitle=1mm, bottomtitle=1mm, toprule=.15mm]

\subsection{Practice Q: Recode income class with
np.where()}\label{practice-q-recode-income-class-with-np.where}

With the \texttt{counties\_income} data, make a new column, called
\texttt{income\_class}, that has the value ``Medium income'' for
counties with median household income between \$60,000 and \$100,000
(exclusive of 100k) and ``Other'' for all other counties. Use the
\texttt{np.where()} function.

\begin{Shaded}
\begin{Highlighting}[]
\CommentTok{\# Complete the code with your answer:}
\NormalTok{Q\_income\_class }\OperatorTok{=}\NormalTok{ counties\_income.assign(}
\NormalTok{    income\_class}\OperatorTok{=}\KeywordTok{lambda}\NormalTok{ x: np.where(}
        \CommentTok{\# Your code here}
\NormalTok{    )}
\NormalTok{)}
\end{Highlighting}
\end{Shaded}

\end{tcolorbox}

\section{Wrap up}\label{wrap-up-7}

Changing or constructing your variables based on conditions on other
variables is one of the most repeated data wrangling tasks.

I hope now that you will feel comfortable using \texttt{case\_when()}
and \texttt{np.where()} within \texttt{assign()}.

Soon we will learn more complex pandas operations such as grouping
variables and summarizing them.

See you next time!

\chapter{Grouping and summarizing
data}\label{grouping-and-summarizing-data}

\section{Introduction}\label{introduction-8}

In this lesson, we'll explore two powerful pandas methods:
\texttt{agg()} and \texttt{groupby()}. These tools will enable you to
extract summary statistics and perform operations on grouped data
effortlessly.

Let's dive in and discover how to unlock deeper insights from your data!

\section{Learning objectives}\label{learning-objectives-10}

\begin{enumerate}
\def\labelenumi{\arabic{enumi}.}
\item
  You can use \texttt{pandas.DataFrame.agg()} to extract summary
  statistics from datasets.
\item
  You can use \texttt{pandas.DataFrame.groupby()} to group data by one
  or more variables before performing operations on them.
\item
  You can use \texttt{sum()} together with
  \texttt{groupby()}-\texttt{agg()} to count rows that meet a condition.
\item
  You can pass custom functions to \texttt{agg()} to compute summary
  statistics.
\end{enumerate}

\begin{center}\rule{0.5\linewidth}{0.5pt}\end{center}

\section{Libraries}\label{libraries}

Run the following lines to import the necessary libraries:

\begin{Shaded}
\begin{Highlighting}[]
\ImportTok{import}\NormalTok{ pandas }\ImportTok{as}\NormalTok{ pd}
\ImportTok{import}\NormalTok{ numpy }\ImportTok{as}\NormalTok{ np}
\end{Highlighting}
\end{Shaded}

\section{The Yaounde COVID-19
dataset}\label{the-yaounde-covid-19-dataset-2}

In this lesson, we will again use data from the COVID-19 serological
survey conducted in Yaounde, Cameroon.

\begin{Shaded}
\begin{Highlighting}[]
\NormalTok{yaounde }\OperatorTok{=}\NormalTok{ pd.read\_csv(}\StringTok{\textquotesingle{}data/yaounde\_data.csv\textquotesingle{}}\NormalTok{)}

\CommentTok{\#\# A smaller subset of variables}
\NormalTok{yao }\OperatorTok{=}\NormalTok{ yaounde[[}\StringTok{\textquotesingle{}age\textquotesingle{}}\NormalTok{, }\StringTok{\textquotesingle{}age\_category\_3\textquotesingle{}}\NormalTok{, }\StringTok{\textquotesingle{}sex\textquotesingle{}}\NormalTok{, }\StringTok{\textquotesingle{}weight\_kg\textquotesingle{}}\NormalTok{, }\StringTok{\textquotesingle{}height\_cm\textquotesingle{}}\NormalTok{,}
               \StringTok{\textquotesingle{}neighborhood\textquotesingle{}}\NormalTok{, }\StringTok{\textquotesingle{}is\_smoker\textquotesingle{}}\NormalTok{, }\StringTok{\textquotesingle{}is\_pregnant\textquotesingle{}}\NormalTok{, }\StringTok{\textquotesingle{}occupation\textquotesingle{}}\NormalTok{,}
               \StringTok{\textquotesingle{}treatment\_combinations\textquotesingle{}}\NormalTok{, }\StringTok{\textquotesingle{}symptoms\textquotesingle{}}\NormalTok{, }\StringTok{\textquotesingle{}n\_days\_miss\_work\textquotesingle{}}\NormalTok{, }\StringTok{\textquotesingle{}n\_bedridden\_days\textquotesingle{}}\NormalTok{,}
               \StringTok{\textquotesingle{}highest\_education\textquotesingle{}}\NormalTok{, }\StringTok{\textquotesingle{}igg\_result\textquotesingle{}}\NormalTok{]]}

\NormalTok{yao}
\end{Highlighting}
\end{Shaded}

\begin{longtable}[]{@{}llllllllllllllll@{}}
\toprule\noalign{}
& age & age\_category\_3 & sex & weight\_kg & height\_cm & neighborhood
& is\_smoker & is\_pregnant & occupation & treatment\_combinations &
symptoms & n\_days\_miss\_work & n\_bedridden\_days & highest\_education
& igg\_result \\
\midrule\noalign{}
\endhead
\bottomrule\noalign{}
\endlastfoot
0 & 45 & Adult & Female & 95 & 169 & Briqueterie & Non-smoker & No &
Informal worker & Paracetamol & Muscle pain & 0.0 & 0.0 & Secondary &
Negative \\
1 & 55 & Adult & Male & 96 & 185 & Briqueterie & Ex-smoker & NaN &
Salaried worker & NaN & No symptoms & NaN & NaN & University &
Positive \\
2 & 23 & Adult & Male & 74 & 180 & Briqueterie & Smoker & NaN & Student
& NaN & No symptoms & NaN & NaN & University & Negative \\
... & ... & ... & ... & ... & ... & ... & ... & ... & ... & ... & ... &
... & ... & ... & ... \\
968 & 35 & Adult & Male & 77 & 168 & Tsinga Oliga & Smoker & NaN &
Informal worker & Paracetamol & Headache & 0.0 & 0.0 & University &
Positive \\
969 & 31 & Adult & Female & 66 & 169 & Tsinga Oliga & Non-smoker & No &
Unemployed & NaN & No symptoms & NaN & NaN & Secondary & Negative \\
970 & 17 & Child & Female & 67 & 162 & Tsinga Oliga & Non-smoker & No
response & Unemployed & NaN & No symptoms & NaN & NaN & Secondary &
Negative \\
\end{longtable}

You can find out more about this dataset here:
https://www.nature.com/articles/s41467-021-25946-0

\section{What are summary
statistics?}\label{what-are-summary-statistics}

A summary statistic is a single value (such as a mean or median) that
describes a sequence of values (typically a column in your dataset).

Computing summary statistics is a very common operation in most data
analysis workflows, so it will be important to become fluent in
extracting them from your datasets. And for this task, there is no
better tool than the pandas method \texttt{agg()}! So let's see how to
use this powerful method.

\section{\texorpdfstring{Introducing
\texttt{pandas.DataFrame.agg()}}{Introducing pandas.DataFrame.agg()}}\label{introducing-pandas.dataframe.agg}

To get started, let's consider how to get simple summary statistics
\emph{without} using \texttt{agg()}, then we will consider why you
\emph{should} actually use \texttt{agg()}.

Imagine you were asked to find the mean age of respondents in the
\texttt{yao} data frame. You can do this by calling the \texttt{mean()}
method on the \texttt{age} column of the \texttt{yao} data frame:

\begin{Shaded}
\begin{Highlighting}[]
\NormalTok{yao[[}\StringTok{"age"}\NormalTok{]].mean()}
\end{Highlighting}
\end{Shaded}

\begin{verbatim}
age    29.017508
dtype: float64
\end{verbatim}

And that's it! You now have a simple summary statistic. Extremely easy,
right?

So why do we need \texttt{agg()} to get summary statistics if the
process is already so simple without it? We'll come back to the
\emph{why} question soon. First let's see \emph{how} to obtain summary
statistics with \texttt{agg()}.

Going back to the previous example, the correct syntax to get the mean
age with \texttt{agg()} would be:

\begin{Shaded}
\begin{Highlighting}[]
\NormalTok{yao.agg(mean\_age}\OperatorTok{=}\NormalTok{(}\StringTok{\textquotesingle{}age\textquotesingle{}}\NormalTok{, }\StringTok{\textquotesingle{}mean\textquotesingle{}}\NormalTok{))}
\end{Highlighting}
\end{Shaded}

\begin{longtable}[]{@{}ll@{}}
\toprule\noalign{}
& age \\
\midrule\noalign{}
\endhead
\bottomrule\noalign{}
\endlastfoot
mean\_age & 29.017508 \\
\end{longtable}

The anatomy of this syntax is:

\begin{Shaded}
\begin{Highlighting}[]
\NormalTok{dataframe.agg(new\_column\_name}\OperatorTok{=}\NormalTok{(}\StringTok{"COLUMN\_TO\_SUMMARIZE"}\NormalTok{, }\StringTok{"SUMMARY\_FUNCTION"}\NormalTok{))}
\end{Highlighting}
\end{Shaded}

\begin{center}\rule{0.5\linewidth}{0.5pt}\end{center}

You can also compute multiple summary statistics in a single
\texttt{agg()} statement. For example, if you wanted both the mean and
the median age, you could run:

\begin{Shaded}
\begin{Highlighting}[]
\NormalTok{yao.agg(mean\_age}\OperatorTok{=}\NormalTok{(}\StringTok{"age"}\NormalTok{, }\StringTok{"mean"}\NormalTok{), median\_age}\OperatorTok{=}\NormalTok{(}\StringTok{"age"}\NormalTok{, }\StringTok{"median"}\NormalTok{))}
\end{Highlighting}
\end{Shaded}

\begin{longtable}[]{@{}ll@{}}
\toprule\noalign{}
& age \\
\midrule\noalign{}
\endhead
\bottomrule\noalign{}
\endlastfoot
mean\_age & 29.017508 \\
median\_age & 26.000000 \\
\end{longtable}

Nice! Try your hand at the practice question below.

\begin{tcolorbox}[enhanced jigsaw, colframe=quarto-callout-tip-color-frame, opacityback=0, titlerule=0mm, bottomrule=.15mm, breakable, leftrule=.75mm, colbacktitle=quarto-callout-tip-color!10!white, title=\textcolor{quarto-callout-tip-color}{\faLightbulb}\hspace{0.5em}{Practice}, rightrule=.15mm, coltitle=black, opacitybacktitle=0.6, colback=white, left=2mm, arc=.35mm, toptitle=1mm, bottomtitle=1mm, toprule=.15mm]

\subsection{Practice Q: Mean and median
weight}\label{practice-q-mean-and-median-weight}

Use \texttt{agg()} and the relevant summary functions to obtain the mean
and median of respondent weights from the \texttt{weight\_kg} variable
of the \texttt{yao} data frame.

\begin{Shaded}
\begin{Highlighting}[]
\CommentTok{\# Your code here}
\end{Highlighting}
\end{Shaded}

\end{tcolorbox}

\begin{tcolorbox}[enhanced jigsaw, colframe=quarto-callout-tip-color-frame, opacityback=0, titlerule=0mm, bottomrule=.15mm, breakable, leftrule=.75mm, colbacktitle=quarto-callout-tip-color!10!white, title=\textcolor{quarto-callout-tip-color}{\faLightbulb}\hspace{0.5em}{Practice}, rightrule=.15mm, coltitle=black, opacitybacktitle=0.6, colback=white, left=2mm, arc=.35mm, toptitle=1mm, bottomtitle=1mm, toprule=.15mm]

\subsection{Practice Q: Min and max
height}\label{practice-q-min-and-max-height}

Use \texttt{agg()} and the relevant summary functions to obtain the
minimum and maximum respondent heights from the \texttt{height\_cm}
variable of the \texttt{yao} data frame. You may need to use a search
engine to find out the minimum and maximum functions in Python if you
don't remember them.

\begin{Shaded}
\begin{Highlighting}[]
\CommentTok{\# Your code here}
\end{Highlighting}
\end{Shaded}

\end{tcolorbox}

\section{\texorpdfstring{Grouped summaries with
\texttt{pandas.DataFrame.groupby()}}{Grouped summaries with pandas.DataFrame.groupby()}}\label{grouped-summaries-with-pandas.dataframe.groupby}

Now let's see how to use \texttt{groupby()} to obtain grouped summaries,
the primary reason for using \texttt{agg()} in the first place.

As its name suggests, \texttt{pandas.DataFrame.groupby()} lets you group
a data frame by the values in a variable (e.g.~male vs female sex). You
can then perform operations that are split according to these groups.

Let's try to group the \texttt{yao} data frame by sex and observe the
effect:

\begin{Shaded}
\begin{Highlighting}[]
\NormalTok{yao.groupby(}\StringTok{"sex"}\NormalTok{)}
\end{Highlighting}
\end{Shaded}

\begin{verbatim}
<pandas.core.groupby.generic.DataFrameGroupBy object at 0x106c62450>
\end{verbatim}

Hmm. Apparently nothing happened. We just get a \texttt{GroupBy} object.

But watch what happens when we chain the \texttt{groupby()} with the
\texttt{agg()} call we used in the previous section:

\begin{Shaded}
\begin{Highlighting}[]
\NormalTok{(}
\NormalTok{    yao.groupby(}\StringTok{"sex"}\NormalTok{)}
\NormalTok{    .agg(mean\_age}\OperatorTok{=}\NormalTok{(}\StringTok{"age"}\NormalTok{, }\StringTok{"mean"}\NormalTok{), median\_age}\OperatorTok{=}\NormalTok{(}\StringTok{"age"}\NormalTok{, }\StringTok{"median"}\NormalTok{))}
\NormalTok{)}
\end{Highlighting}
\end{Shaded}

\begin{longtable}[]{@{}lll@{}}
\toprule\noalign{}
& mean\_age & median\_age \\
sex & & \\
\midrule\noalign{}
\endhead
\bottomrule\noalign{}
\endlastfoot
Female & 29.495446 & 26.0 \\
Male & 28.395735 & 25.0 \\
\end{longtable}

Now we get a different statistic for each group! The mean age for female
respondents is about 29.5, while for male respondents it's about 28.4.

As was mentioned earlier, this kind of grouped summary is the primary
reason the \texttt{agg()} function is so useful.

You may notice that there are two header rows. This is because the
output has a hierarchical index (called a MultiIndex in pandas). While
this can be useful in some cases, it often makes further data
manipulation more difficult. We can reset the index to convert the group
labels back to a regular column with the \texttt{reset\_index()} method.

\begin{Shaded}
\begin{Highlighting}[]
\NormalTok{(}
\NormalTok{    yao.groupby(}\StringTok{"sex"}\NormalTok{)}
\NormalTok{    .agg(mean\_age}\OperatorTok{=}\NormalTok{(}\StringTok{"age"}\NormalTok{, }\StringTok{"mean"}\NormalTok{), median\_age}\OperatorTok{=}\NormalTok{(}\StringTok{"age"}\NormalTok{, }\StringTok{"median"}\NormalTok{))}
\NormalTok{    .reset\_index()}
\NormalTok{)}
\end{Highlighting}
\end{Shaded}

\begin{longtable}[]{@{}llll@{}}
\toprule\noalign{}
& sex & mean\_age & median\_age \\
\midrule\noalign{}
\endhead
\bottomrule\noalign{}
\endlastfoot
0 & Female & 29.495446 & 26.0 \\
1 & Male & 28.395735 & 25.0 \\
\end{longtable}

\begin{center}\rule{0.5\linewidth}{0.5pt}\end{center}

Let's see some more examples.

Suppose you were asked to obtain the maximum and minimum weights for
individuals in different neighborhoods, and also present the number of
individuals in each neighborhood. We can write:

\begin{Shaded}
\begin{Highlighting}[]
\NormalTok{(}
\NormalTok{    yao.groupby(}\StringTok{"neighborhood"}\NormalTok{)}
\NormalTok{    .agg(}
\NormalTok{        max\_weight}\OperatorTok{=}\NormalTok{(}\StringTok{"weight\_kg"}\NormalTok{, }\StringTok{"max"}\NormalTok{),}
\NormalTok{        min\_weight}\OperatorTok{=}\NormalTok{(}\StringTok{"weight\_kg"}\NormalTok{, }\StringTok{"min"}\NormalTok{),}
\NormalTok{        count}\OperatorTok{=}\NormalTok{(}\StringTok{"neighborhood"}\NormalTok{, }\StringTok{"size"}\NormalTok{),  }\CommentTok{\# the size function counts rows per group}
\NormalTok{    )}
\NormalTok{    .reset\_index()}
\NormalTok{) }
\end{Highlighting}
\end{Shaded}

\begin{longtable}[]{@{}lllll@{}}
\toprule\noalign{}
& neighborhood & max\_weight & min\_weight & count \\
\midrule\noalign{}
\endhead
\bottomrule\noalign{}
\endlastfoot
0 & Briqueterie & 128 & 20 & 106 \\
1 & Carriere & 129 & 14 & 236 \\
2 & Cité Verte & 118 & 16 & 72 \\
... & ... & ... & ... & ... \\
6 & Nkomkana & 161 & 15 & 75 \\
7 & Tsinga & 105 & 15 & 81 \\
8 & Tsinga Oliga & 100 & 17 & 67 \\
\end{longtable}

Great! With just a few code lines you are able to extract quite a lot of
information.

\begin{center}\rule{0.5\linewidth}{0.5pt}\end{center}

Let's see one more example for good measure. The variable
\texttt{n\_days\_miss\_work} tells us the number of days that
respondents missed work due to COVID-like symptoms. Individuals who
reported no COVID-like symptoms have an \texttt{NA} for this variable:

\begin{Shaded}
\begin{Highlighting}[]
\NormalTok{yao[[}\StringTok{\textquotesingle{}n\_days\_miss\_work\textquotesingle{}}\NormalTok{]]}
\end{Highlighting}
\end{Shaded}

\begin{longtable}[]{@{}ll@{}}
\toprule\noalign{}
& n\_days\_miss\_work \\
\midrule\noalign{}
\endhead
\bottomrule\noalign{}
\endlastfoot
0 & 0.0 \\
1 & NaN \\
2 & NaN \\
... & ... \\
968 & 0.0 \\
969 & NaN \\
970 & NaN \\
\end{longtable}

To count the total number of work days missed for each sex group, we can
write:

\begin{Shaded}
\begin{Highlighting}[]
\NormalTok{(}
\NormalTok{    yao.groupby(}\StringTok{"sex"}\NormalTok{)}
\NormalTok{    .agg(total\_days\_missed}\OperatorTok{=}\NormalTok{(}\StringTok{"n\_days\_miss\_work"}\NormalTok{, }\StringTok{"sum"}\NormalTok{))}
\NormalTok{    .reset\_index()}
\NormalTok{)}
\end{Highlighting}
\end{Shaded}

\begin{longtable}[]{@{}lll@{}}
\toprule\noalign{}
& sex & total\_days\_missed \\
\midrule\noalign{}
\endhead
\bottomrule\noalign{}
\endlastfoot
0 & Female & 256.0 \\
1 & Male & 272.0 \\
\end{longtable}

The output tells us that across all women in the sample, 256 work days
were missed due to COVID-like symptoms, and across all men, 272 days.

\begin{center}\rule{0.5\linewidth}{0.5pt}\end{center}

Hopefully now you see why \texttt{agg()} is so powerful. In combination
with \texttt{groupby()}, it lets you obtain highly informative grouped
summaries of your datasets with very few lines of code.

Producing such summaries is a very important part of most data analysis
workflows, so this skill is likely to come in handy soon!

\begin{tcolorbox}[enhanced jigsaw, colframe=quarto-callout-tip-color-frame, opacityback=0, titlerule=0mm, bottomrule=.15mm, breakable, leftrule=.75mm, colbacktitle=quarto-callout-tip-color!10!white, title=\textcolor{quarto-callout-tip-color}{\faLightbulb}\hspace{0.5em}{Practice}, rightrule=.15mm, coltitle=black, opacitybacktitle=0.6, colback=white, left=2mm, arc=.35mm, toptitle=1mm, bottomtitle=1mm, toprule=.15mm]

\subsection{Practice Q: Min and max height per
sex}\label{practice-q-min-and-max-height-per-sex}

Use \texttt{groupby()}, \texttt{agg()}, and the relevant summary
functions to obtain the minimum and maximum heights for each sex in the
\texttt{yao} data frame, as well as the number of individuals in each
sex group.

Your output should be a DataFrame that looks like this:

\begin{longtable}[]{@{}llll@{}}
\toprule\noalign{}
sex & min\_height\_cm & max\_height\_cm & count \\
\midrule\noalign{}
\endhead
\bottomrule\noalign{}
\endlastfoot
Female & & & \\
Male & & & \\
\end{longtable}

\begin{Shaded}
\begin{Highlighting}[]
\CommentTok{\# Your code here}
\end{Highlighting}
\end{Shaded}

\end{tcolorbox}

\begin{tcolorbox}[enhanced jigsaw, colframe=quarto-callout-note-color-frame, opacityback=0, titlerule=0mm, bottomrule=.15mm, breakable, leftrule=.75mm, colbacktitle=quarto-callout-note-color!10!white, title=\textcolor{quarto-callout-note-color}{\faInfo}\hspace{0.5em}{Pro-tip}, rightrule=.15mm, coltitle=black, opacitybacktitle=0.6, colback=white, left=2mm, arc=.35mm, toptitle=1mm, bottomtitle=1mm, toprule=.15mm]

\textbf{Groupby and agg are not always needed}

Sometimes you just want a quick summary of a variable, not per group.

Remember that there are quick ways to get such statistics. So you do not
always need to use \texttt{groupby()} and \texttt{agg()}.

\begin{Shaded}
\begin{Highlighting}[]
\NormalTok{yao[}\StringTok{\textquotesingle{}age\textquotesingle{}}\NormalTok{].describe() }\CommentTok{\# summary stats for a numeric variable}
\end{Highlighting}
\end{Shaded}

\begin{verbatim}
count    971.000000
mean      29.017508
std       17.340397
            ...    
50%       26.000000
75%       39.000000
max       79.000000
Name: age, Length: 8, dtype: float64
\end{verbatim}

\begin{Shaded}
\begin{Highlighting}[]
\NormalTok{yao[}\StringTok{\textquotesingle{}sex\textquotesingle{}}\NormalTok{].describe() }\CommentTok{\# summary stats for a categorical variable}
\end{Highlighting}
\end{Shaded}

\begin{verbatim}
count        971
unique         2
top       Female
freq         549
Name: sex, dtype: object
\end{verbatim}

\begin{Shaded}
\begin{Highlighting}[]
\NormalTok{yao[}\StringTok{\textquotesingle{}sex\textquotesingle{}}\NormalTok{].value\_counts() }\CommentTok{\# count of each category}
\end{Highlighting}
\end{Shaded}

\begin{verbatim}
sex
Female    549
Male      422
Name: count, dtype: int64
\end{verbatim}

\begin{Shaded}
\begin{Highlighting}[]
\NormalTok{yao.select\_dtypes(}\BuiltInTok{object}\NormalTok{).describe() }\CommentTok{\# summary stats for categorical variables}
\end{Highlighting}
\end{Shaded}

\begin{longtable}[]{@{}lllllllllll@{}}
\toprule\noalign{}
& age\_category\_3 & sex & neighborhood & is\_smoker & is\_pregnant &
occupation & treatment\_combinations & symptoms & highest\_education &
igg\_result \\
\midrule\noalign{}
\endhead
\bottomrule\noalign{}
\endlastfoot
count & 971 & 971 & 971 & 969 & 549 & 971 & 262 & 971 & 971 & 971 \\
unique & 3 & 2 & 9 & 3 & 3 & 28 & 31 & 122 & 7 & 2 \\
top & Adult & Female & Carriere & Non-smoker & No & Student &
Traditional meds. & No symptoms & Secondary & Negative \\
freq & 635 & 549 & 236 & 859 & 464 & 383 & 57 & 675 & 433 & 669 \\
\end{longtable}

\begin{Shaded}
\begin{Highlighting}[]
\NormalTok{yao.select\_dtypes(np.number).describe() }\CommentTok{\# summary stats for numeric variables}
\end{Highlighting}
\end{Shaded}

\begin{longtable}[]{@{}llllll@{}}
\toprule\noalign{}
& age & weight\_kg & height\_cm & n\_days\_miss\_work &
n\_bedridden\_days \\
\midrule\noalign{}
\endhead
\bottomrule\noalign{}
\endlastfoot
count & 971.000000 & 971.000000 & 971.000000 & 296.000000 &
295.000000 \\
mean & 29.017508 & 64.405767 & 158.944387 & 1.783784 & 1.145763 \\
std & 17.340397 & 23.053370 & 18.776356 & 4.583006 & 2.525201 \\
... & ... & ... & ... & ... & ... \\
50\% & 26.000000 & 66.000000 & 164.000000 & 0.000000 & 0.000000 \\
75\% & 39.000000 & 78.000000 & 170.000000 & 2.000000 & 2.000000 \\
max & 79.000000 & 162.000000 & 196.000000 & 45.000000 & 30.000000 \\
\end{longtable}

\end{tcolorbox}

\begin{tcolorbox}[enhanced jigsaw, colframe=quarto-callout-warning-color-frame, opacityback=0, titlerule=0mm, bottomrule=.15mm, breakable, leftrule=.75mm, colbacktitle=quarto-callout-warning-color!10!white, title=\textcolor{quarto-callout-warning-color}{\faExclamationTriangle}\hspace{0.5em}{Watch-out}, rightrule=.15mm, coltitle=black, opacitybacktitle=0.6, colback=white, left=2mm, arc=.35mm, toptitle=1mm, bottomtitle=1mm, toprule=.15mm]

\section{\texorpdfstring{NaN values in
\texttt{agg()}}{NaN values in agg()}}\label{nan-values-in-agg}

When using \texttt{agg()} to compute grouped summary statistics, pay
attention to whether your group of interest contains NaN values.

For example, to get mean weight by smoking status, we can write:

\begin{Shaded}
\begin{Highlighting}[]
\NormalTok{(}
\NormalTok{    yao.groupby(}\StringTok{"is\_smoker"}\NormalTok{)}
\NormalTok{    .agg(weight\_mean}\OperatorTok{=}\NormalTok{(}\StringTok{"weight\_kg"}\NormalTok{, }\StringTok{"mean"}\NormalTok{))}
\NormalTok{    .reset\_index()}
\NormalTok{)}
\end{Highlighting}
\end{Shaded}

\begin{longtable}[]{@{}lll@{}}
\toprule\noalign{}
& is\_smoker & weight\_mean \\
\midrule\noalign{}
\endhead
\bottomrule\noalign{}
\endlastfoot
0 & Ex-smoker & 76.366197 \\
1 & Non-smoker & 63.033760 \\
2 & Smoker & 72.410256 \\
\end{longtable}

But this actually excludes some rows with NaN smoking status from the
summary table.

We can include these individuals in the summary table by setting
\texttt{dropna=False} with the \texttt{groupby()} function.

\begin{Shaded}
\begin{Highlighting}[]
\NormalTok{(}
\NormalTok{    yao.groupby(}\StringTok{"is\_smoker"}\NormalTok{, dropna}\OperatorTok{=}\VariableTok{False}\NormalTok{)}
\NormalTok{    .agg(weight\_mean}\OperatorTok{=}\NormalTok{(}\StringTok{"weight\_kg"}\NormalTok{, }\StringTok{"mean"}\NormalTok{))}
\NormalTok{    .reset\_index()}
\NormalTok{)}
\end{Highlighting}
\end{Shaded}

\begin{longtable}[]{@{}lll@{}}
\toprule\noalign{}
& is\_smoker & weight\_mean \\
\midrule\noalign{}
\endhead
\bottomrule\noalign{}
\endlastfoot
0 & Ex-smoker & 76.366197 \\
1 & Non-smoker & 63.033760 \\
2 & Smoker & 72.410256 \\
3 & NaN & 73.000000 \\
\end{longtable}

Also recall that you can see how many individuals are in each smoking
status group by using the \texttt{size()} function. It is often useful
to include this information in your summary table, so that you know how
many individuals are behind each summary statistic.

\begin{Shaded}
\begin{Highlighting}[]
\NormalTok{(}
\NormalTok{    yao.groupby(}\StringTok{"is\_smoker"}\NormalTok{, dropna}\OperatorTok{=}\VariableTok{False}\NormalTok{)}
\NormalTok{    .agg(weight\_mean}\OperatorTok{=}\NormalTok{(}\StringTok{"weight\_kg"}\NormalTok{, }\StringTok{"mean"}\NormalTok{), }
\NormalTok{         count}\OperatorTok{=}\NormalTok{(}\StringTok{"is\_smoker"}\NormalTok{, }\StringTok{"size"}\NormalTok{))}
\NormalTok{    .reset\_index()}
\NormalTok{)}
\end{Highlighting}
\end{Shaded}

\begin{longtable}[]{@{}llll@{}}
\toprule\noalign{}
& is\_smoker & weight\_mean & count \\
\midrule\noalign{}
\endhead
\bottomrule\noalign{}
\endlastfoot
0 & Ex-smoker & 76.366197 & 71 \\
1 & Non-smoker & 63.033760 & 859 \\
2 & Smoker & 72.410256 & 39 \\
3 & NaN & 73.000000 & 2 \\
\end{longtable}

\end{tcolorbox}

\begin{tcolorbox}[enhanced jigsaw, colframe=quarto-callout-tip-color-frame, opacityback=0, titlerule=0mm, bottomrule=.15mm, breakable, leftrule=.75mm, colbacktitle=quarto-callout-tip-color!10!white, title=\textcolor{quarto-callout-tip-color}{\faLightbulb}\hspace{0.5em}{Practice}, rightrule=.15mm, coltitle=black, opacitybacktitle=0.6, colback=white, left=2mm, arc=.35mm, toptitle=1mm, bottomtitle=1mm, toprule=.15mm]

\subsection{Practice Q: Mean weight by pregnancy
status}\label{practice-q-mean-weight-by-pregnancy-status}

Use \texttt{groupby()}, \texttt{agg()}, and the \texttt{mean()} function
to obtain the mean weight (kg) by pregnancy status in the \texttt{yao}
data frame. Include individuals with NaN pregnancy status in the summary
table.

The output data frame should look something like this:

\begin{longtable}[]{@{}ll@{}}
\toprule\noalign{}
is\_pregnant & weight\_mean \\
\midrule\noalign{}
\endhead
\bottomrule\noalign{}
\endlastfoot
No & \\
No response & \\
Yes & \\
NaN & \\
\end{longtable}

\begin{Shaded}
\begin{Highlighting}[]
\CommentTok{\# your code here}
\end{Highlighting}
\end{Shaded}

\end{tcolorbox}

\section{Grouping by multiple variables (nested
grouping)}\label{grouping-by-multiple-variables-nested-grouping}

It is possible to group a data frame by more than one variable. This is
sometimes called ``nested'' grouping.

Suppose you want to know the mean age of men and women \emph{in each
neighbourhood}, you could put both \texttt{sex} and
\texttt{neighborhood} in the \texttt{groupby()} statement:

\begin{Shaded}
\begin{Highlighting}[]
\NormalTok{(}
\NormalTok{    yao}
\NormalTok{    .groupby([}\StringTok{\textquotesingle{}sex\textquotesingle{}}\NormalTok{, }\StringTok{\textquotesingle{}neighborhood\textquotesingle{}}\NormalTok{])}
\NormalTok{    .agg(mean\_age}\OperatorTok{=}\NormalTok{(}\StringTok{\textquotesingle{}age\textquotesingle{}}\NormalTok{, }\StringTok{\textquotesingle{}mean\textquotesingle{}}\NormalTok{))}
\NormalTok{    .reset\_index()}
\NormalTok{)}
\end{Highlighting}
\end{Shaded}

\begin{longtable}[]{@{}llll@{}}
\toprule\noalign{}
& sex & neighborhood & mean\_age \\
\midrule\noalign{}
\endhead
\bottomrule\noalign{}
\endlastfoot
0 & Female & Briqueterie & 31.622951 \\
1 & Female & Carriere & 28.164286 \\
2 & Female & Cité Verte & 31.750000 \\
... & ... & ... & ... \\
15 & Male & Nkomkana & 29.812500 \\
16 & Male & Tsinga & 28.820513 \\
17 & Male & Tsinga Oliga & 24.297297 \\
\end{longtable}

From this output data frame you can tell that, for example, women from
Briqueterie have a mean age of 31.6 years.

The order of the columns listed in \texttt{groupby()} is
interchangeable. So if you run
\texttt{groupby({[}\textquotesingle{}neighborhood\textquotesingle{},\ \textquotesingle{}sex\textquotesingle{}{]})}
instead of
\texttt{groupby({[}\textquotesingle{}sex\textquotesingle{},\ \textquotesingle{}neighborhood\textquotesingle{}{]})},
you'll get the same result, although it will be ordered differently:

\begin{Shaded}
\begin{Highlighting}[]
\NormalTok{(}
\NormalTok{    yao}
\NormalTok{    .groupby([}\StringTok{\textquotesingle{}neighborhood\textquotesingle{}}\NormalTok{, }\StringTok{\textquotesingle{}sex\textquotesingle{}}\NormalTok{])}
\NormalTok{    .agg(mean\_age}\OperatorTok{=}\NormalTok{(}\StringTok{\textquotesingle{}age\textquotesingle{}}\NormalTok{, }\StringTok{\textquotesingle{}mean\textquotesingle{}}\NormalTok{))}
\NormalTok{    .reset\_index()}
\NormalTok{)}
\end{Highlighting}
\end{Shaded}

\begin{longtable}[]{@{}llll@{}}
\toprule\noalign{}
& neighborhood & sex & mean\_age \\
\midrule\noalign{}
\endhead
\bottomrule\noalign{}
\endlastfoot
0 & Briqueterie & Female & 31.622951 \\
1 & Briqueterie & Male & 33.711111 \\
2 & Carriere & Female & 28.164286 \\
... & ... & ... & ... \\
15 & Tsinga & Male & 28.820513 \\
16 & Tsinga Oliga & Female & 24.266667 \\
17 & Tsinga Oliga & Male & 24.297297 \\
\end{longtable}

\begin{tcolorbox}[enhanced jigsaw, colframe=quarto-callout-tip-color-frame, opacityback=0, titlerule=0mm, bottomrule=.15mm, breakable, leftrule=.75mm, colbacktitle=quarto-callout-tip-color!10!white, title=\textcolor{quarto-callout-tip-color}{\faLightbulb}\hspace{0.5em}{Practice}, rightrule=.15mm, coltitle=black, opacitybacktitle=0.6, colback=white, left=2mm, arc=.35mm, toptitle=1mm, bottomtitle=1mm, toprule=.15mm]

\subsection{Practice Q: Total bedridden days per age-sex
group}\label{practice-q-total-bedridden-days-per-age-sex-group}

Use \texttt{groupby()}, \texttt{agg()}, and the \texttt{sum()} function
to calculate the total number of bedridden days (from the
\texttt{n\_bedridden\_days} variable) reported by respondents of each
age-sex group. For age group, use the \texttt{age\_category\_3}
variable.

Your output should be a data frame with three columns named as shown
below:

\begin{longtable}[]{@{}lll@{}}
\toprule\noalign{}
age\_category\_3 & sex & total\_bedridden\_days \\
\midrule\noalign{}
\endhead
\bottomrule\noalign{}
\endlastfoot
Adult & Female & \\
Adult & Male & \\
Child & Female & \\
Child & Male & \\
Senior & Female & \\
Senior & Male & \\
\end{longtable}

\begin{Shaded}
\begin{Highlighting}[]
\CommentTok{\# your code here}
\end{Highlighting}
\end{Shaded}

\end{tcolorbox}

\section{Custom summary statistics}\label{custom-summary-statistics}

Sometimes, you may want to apply custom summary statistics that aren't
available as built-in functions. In these cases, you can use lambda
functions or define your own functions to use with \texttt{agg()}.

For example, let's say we want to calculate the range (difference
between maximum and minimum) of weights in each neighborhood:

\begin{Shaded}
\begin{Highlighting}[]
\NormalTok{(}
\NormalTok{    yao}
\NormalTok{    .groupby(}\StringTok{\textquotesingle{}neighborhood\textquotesingle{}}\NormalTok{)}
\NormalTok{    .agg(weight\_range}\OperatorTok{=}\NormalTok{(}\StringTok{\textquotesingle{}weight\_kg\textquotesingle{}}\NormalTok{, }\KeywordTok{lambda}\NormalTok{ x: x.}\BuiltInTok{max}\NormalTok{() }\OperatorTok{{-}}\NormalTok{ x.}\BuiltInTok{min}\NormalTok{()))}
\NormalTok{    .reset\_index()}
\NormalTok{)}
\end{Highlighting}
\end{Shaded}

\begin{longtable}[]{@{}lll@{}}
\toprule\noalign{}
& neighborhood & weight\_range \\
\midrule\noalign{}
\endhead
\bottomrule\noalign{}
\endlastfoot
0 & Briqueterie & 108 \\
1 & Carriere & 115 \\
2 & Cité Verte & 102 \\
... & ... & ... \\
6 & Nkomkana & 146 \\
7 & Tsinga & 90 \\
8 & Tsinga Oliga & 83 \\
\end{longtable}

Notice that we still provide a tuple to the \texttt{agg()} function,
\texttt{(\textquotesingle{}weight\_kg\textquotesingle{},\ lambda\ x:\ x.max()\ -\ x.min())},
but the second element of the tuple is a lambda function.

This lambda function operates on the column provided in the tuple,
\texttt{weight\_kg}.

(Unlike the lambda functions we saw in the \texttt{assign()} method, we
do not need to access the column within the lambda with
\texttt{x.weight\_kg.max()\ -\ x.weight\_kg.min()}. We simply use
\texttt{x.max()\ -\ x.min()}.)

\begin{tcolorbox}[enhanced jigsaw, colframe=quarto-callout-tip-color-frame, opacityback=0, titlerule=0mm, bottomrule=.15mm, breakable, leftrule=.75mm, colbacktitle=quarto-callout-tip-color!10!white, title=\textcolor{quarto-callout-tip-color}{\faLightbulb}\hspace{0.5em}{Practice}, rightrule=.15mm, coltitle=black, opacitybacktitle=0.6, colback=white, left=2mm, arc=.35mm, toptitle=1mm, bottomtitle=1mm, toprule=.15mm]

\subsection{Practice Q: IQR of age by
neighborhood}\label{practice-q-iqr-of-age-by-neighborhood}

Find the interquartile range (IQR) of the age variable for each
neighborhood. The IQR is the difference between the 75th and 25th
percentiles. Your lambda will look like this:
\texttt{lambda\ x:\ x.quantile(0.75)\ -\ x.quantile(0.25)}

\begin{Shaded}
\begin{Highlighting}[]
\CommentTok{\# Your code here}
\end{Highlighting}
\end{Shaded}

\end{tcolorbox}

\section{Counting rows that meet a
condition}\label{counting-rows-that-meet-a-condition}

It is sometimes useful to count the rows that meet specific conditions
within a group. This can be done with the groupby and agg functions.

For example, to count the number of people under 18 in each
neighborhood, we can write:

\begin{Shaded}
\begin{Highlighting}[]
\NormalTok{(}
\NormalTok{    yao}
\NormalTok{    .groupby(}\StringTok{\textquotesingle{}neighborhood\textquotesingle{}}\NormalTok{)}
\NormalTok{    .agg(num\_children}\OperatorTok{=}\NormalTok{(}\StringTok{\textquotesingle{}age\textquotesingle{}}\NormalTok{, }\KeywordTok{lambda}\NormalTok{ x: (x }\OperatorTok{\textless{}} \DecValTok{18}\NormalTok{).}\BuiltInTok{sum}\NormalTok{()))}
\NormalTok{    .reset\_index()}
\NormalTok{)}
\end{Highlighting}
\end{Shaded}

\begin{longtable}[]{@{}lll@{}}
\toprule\noalign{}
& neighborhood & num\_children \\
\midrule\noalign{}
\endhead
\bottomrule\noalign{}
\endlastfoot
0 & Briqueterie & 28 \\
1 & Carriere & 58 \\
2 & Cité Verte & 19 \\
... & ... & ... \\
6 & Nkomkana & 22 \\
7 & Tsinga & 23 \\
8 & Tsinga Oliga & 25 \\
\end{longtable}

The lambda function \texttt{(x\ \textless{}\ 18).sum()} counts the
number of values in the \texttt{x} series that are less than 18. The
\texttt{x} series is the age variable for each neighborhood.

Let's broaden our scope and add the number of seniors (65+) to the
summary table.

\begin{Shaded}
\begin{Highlighting}[]
\NormalTok{(}
\NormalTok{    yao.groupby(}\StringTok{"neighborhood"}\NormalTok{)}
\NormalTok{    .agg(}
\NormalTok{        num\_children}\OperatorTok{=}\NormalTok{(}\StringTok{"age"}\NormalTok{, }\KeywordTok{lambda}\NormalTok{ x: (x }\OperatorTok{\textless{}} \DecValTok{18}\NormalTok{).}\BuiltInTok{sum}\NormalTok{()),}
\NormalTok{        num\_seniors}\OperatorTok{=}\NormalTok{(}\StringTok{"age"}\NormalTok{, }\KeywordTok{lambda}\NormalTok{ x: (x }\OperatorTok{\textgreater{}=} \DecValTok{65}\NormalTok{).}\BuiltInTok{sum}\NormalTok{())}
\NormalTok{    )}
\NormalTok{    .reset\_index()}
\NormalTok{)}
\end{Highlighting}
\end{Shaded}

\begin{longtable}[]{@{}llll@{}}
\toprule\noalign{}
& neighborhood & num\_children & num\_seniors \\
\midrule\noalign{}
\endhead
\bottomrule\noalign{}
\endlastfoot
0 & Briqueterie & 28 & 9 \\
1 & Carriere & 58 & 9 \\
2 & Cité Verte & 19 & 4 \\
... & ... & ... & ... \\
6 & Nkomkana & 22 & 6 \\
7 & Tsinga & 23 & 5 \\
8 & Tsinga Oliga & 25 & 0 \\
\end{longtable}

\section{Calculate percentage shares}\label{calculate-percentage-shares}

We can expand our previous example to calculate the percentage of
children and seniors in each neighborhood.

Calculating such percentage shares is a common task in data analysis.

\begin{Shaded}
\begin{Highlighting}[]
\NormalTok{(}
\NormalTok{    yao.groupby(}\StringTok{"neighborhood"}\NormalTok{)}
\NormalTok{    .agg(}
\NormalTok{        children\_percent}\OperatorTok{=}\NormalTok{(}\StringTok{"age"}\NormalTok{, }\KeywordTok{lambda}\NormalTok{ x: (x }\OperatorTok{\textless{}} \DecValTok{18}\NormalTok{).}\BuiltInTok{sum}\NormalTok{() }\OperatorTok{/} \BuiltInTok{len}\NormalTok{(x) }\OperatorTok{*} \DecValTok{100}\NormalTok{),}
\NormalTok{        seniors\_percent}\OperatorTok{=}\NormalTok{(}\StringTok{"age"}\NormalTok{, }\KeywordTok{lambda}\NormalTok{ x: (x }\OperatorTok{\textgreater{}=} \DecValTok{65}\NormalTok{).}\BuiltInTok{sum}\NormalTok{() }\OperatorTok{/} \BuiltInTok{len}\NormalTok{(x) }\OperatorTok{*} \DecValTok{100}\NormalTok{),}
\NormalTok{    )}
\NormalTok{    .reset\_index()}
\NormalTok{)}
\end{Highlighting}
\end{Shaded}

\begin{longtable}[]{@{}llll@{}}
\toprule\noalign{}
& neighborhood & children\_percent & seniors\_percent \\
\midrule\noalign{}
\endhead
\bottomrule\noalign{}
\endlastfoot
0 & Briqueterie & 26.415094 & 8.490566 \\
1 & Carriere & 24.576271 & 3.813559 \\
2 & Cité Verte & 26.388889 & 5.555556 \\
... & ... & ... & ... \\
6 & Nkomkana & 29.333333 & 8.000000 \\
7 & Tsinga & 28.395062 & 6.172840 \\
8 & Tsinga Oliga & 37.313433 & 0.000000 \\
\end{longtable}

Hopefully the code above is not too confusing. To calculate the
percentage of children, we divide the number of children
\texttt{(x\ \textless{}\ 18).sum()} by the total number of individuals
in the series, \texttt{len(x)}, and multiply by 100. Likewise for
seniors.

A more concise approach utilizes the fact that the mean of a boolean
series represents the proportion of True values:

\begin{Shaded}
\begin{Highlighting}[]
\NormalTok{(}
\NormalTok{    yao.groupby(}\StringTok{"neighborhood"}\NormalTok{)}
\NormalTok{    .agg(}
\NormalTok{        children\_percent}\OperatorTok{=}\NormalTok{(}\StringTok{"age"}\NormalTok{, }\KeywordTok{lambda}\NormalTok{ x: (x }\OperatorTok{\textless{}} \DecValTok{18}\NormalTok{).mean() }\OperatorTok{*} \DecValTok{100}\NormalTok{),}
\NormalTok{        seniors\_percent}\OperatorTok{=}\NormalTok{(}\StringTok{"age"}\NormalTok{, }\KeywordTok{lambda}\NormalTok{ x: (x }\OperatorTok{\textgreater{}=} \DecValTok{65}\NormalTok{).mean() }\OperatorTok{*} \DecValTok{100}\NormalTok{),}
\NormalTok{    )}
\NormalTok{    .reset\_index()}
\NormalTok{)}
\end{Highlighting}
\end{Shaded}

\begin{longtable}[]{@{}llll@{}}
\toprule\noalign{}
& neighborhood & children\_percent & seniors\_percent \\
\midrule\noalign{}
\endhead
\bottomrule\noalign{}
\endlastfoot
0 & Briqueterie & 26.415094 & 8.490566 \\
1 & Carriere & 24.576271 & 3.813559 \\
2 & Cité Verte & 26.388889 & 5.555556 \\
... & ... & ... & ... \\
6 & Nkomkana & 29.333333 & 8.000000 \\
7 & Tsinga & 28.395062 & 6.172840 \\
8 & Tsinga Oliga & 37.313433 & 0.000000 \\
\end{longtable}

While this method is more efficient, it may be less intuitive. If you
find it confusing, feel free to use the more explicit calculation shown
earlier.

\begin{center}\rule{0.5\linewidth}{0.5pt}\end{center}

As a final example, to count the number and percentage of people with
doctorate degrees in each neighborhood, we can write:

\begin{Shaded}
\begin{Highlighting}[]
\NormalTok{(}
\NormalTok{    yao.groupby(}\StringTok{"neighborhood"}\NormalTok{)}
\NormalTok{    .agg(}
\NormalTok{        count\_w\_doctorates}\OperatorTok{=}\NormalTok{(}\StringTok{"highest\_education"}\NormalTok{, }\KeywordTok{lambda}\NormalTok{ x: (x }\OperatorTok{==} \StringTok{"Doctorate"}\NormalTok{).}\BuiltInTok{sum}\NormalTok{()),}
\NormalTok{        pct\_in\_neighborhood\_w\_doctorates}\OperatorTok{=}\NormalTok{(}
            \StringTok{"highest\_education"}\NormalTok{,}
            \KeywordTok{lambda}\NormalTok{ x: (x }\OperatorTok{==} \StringTok{"Doctorate"}\NormalTok{).mean() }\OperatorTok{*} \DecValTok{100}\NormalTok{,}
\NormalTok{        ),}
\NormalTok{    )}
\NormalTok{    .reset\_index()}
\NormalTok{)}
\end{Highlighting}
\end{Shaded}

\begin{longtable}[]{@{}llll@{}}
\toprule\noalign{}
& neighborhood & count\_w\_doctorates &
pct\_in\_neighborhood\_w\_doctorates \\
\midrule\noalign{}
\endhead
\bottomrule\noalign{}
\endlastfoot
0 & Briqueterie & 2 & 1.886792 \\
1 & Carriere & 1 & 0.423729 \\
2 & Cité Verte & 1 & 1.388889 \\
... & ... & ... & ... \\
6 & Nkomkana & 4 & 5.333333 \\
7 & Tsinga & 3 & 3.703704 \\
8 & Tsinga Oliga & 3 & 4.477612 \\
\end{longtable}

\begin{tcolorbox}[enhanced jigsaw, colframe=quarto-callout-tip-color-frame, opacityback=0, titlerule=0mm, bottomrule=.15mm, breakable, leftrule=.75mm, colbacktitle=quarto-callout-tip-color!10!white, title=\textcolor{quarto-callout-tip-color}{\faLightbulb}\hspace{0.5em}{Practice}, rightrule=.15mm, coltitle=black, opacitybacktitle=0.6, colback=white, left=2mm, arc=.35mm, toptitle=1mm, bottomtitle=1mm, toprule=.15mm]

\subsection{Practice Q: Smoker rate by sex and age
category}\label{practice-q-smoker-rate-by-sex-and-age-category}

Calculate the percentage of people with are smokers (use the
\texttt{is\_smoker} variable) for each combination of sex and age
category (use \texttt{age\_category\_3}). The output should look like
this:

\begin{longtable}[]{@{}lll@{}}
\toprule\noalign{}
sex & age\_category\_3 & smoker\_rate \\
\midrule\noalign{}
\endhead
\bottomrule\noalign{}
\endlastfoot
Female & Adult & \\
Female & Child & \\
Female & Senior & \\
Male & Adult & \\
Male & Child & \\
Male & Senior & \\
\end{longtable}

Your summary table should show that male adults have the highest smoker
rate, followed by male seniors.

\begin{Shaded}
\begin{Highlighting}[]
\CommentTok{\# Your code here}
\end{Highlighting}
\end{Shaded}

\end{tcolorbox}

\section{Wrap up}\label{wrap-up-8}

In this lesson, you've learned how to obtain quick summary statistics
from your data using \texttt{agg()}, group your data using
\texttt{groupby()}, and combine \texttt{groupby()} with \texttt{agg()}
for powerful data summarization.

These skills are essential for both exploratory data analysis and
preparing data for presentation or plotting. The combination of
\texttt{groupby()} and \texttt{agg()} is one of the most common and
useful data manipulation techniques in pandas.

In our next lesson, we'll explore ways to combine \texttt{groupby()}
with other pandas methods.

See you there!

\chapter{Grouped Transformations and
Filtering}\label{grouped-transformations-and-filtering}

\section{Introduction to Grouped
Operations}\label{introduction-to-grouped-operations}

Data wrangling often involves applying the same operations separately to
different groups within the data. In this lesson, we'll learn how to use
\texttt{groupby()} with \texttt{transform()} and \texttt{assign()} to
conduct grouped transformations on a DataFrame.

\section{Learning Objectives}\label{learning-objectives-11}

\begin{enumerate}
\def\labelenumi{\arabic{enumi}.}
\tightlist
\item
  You can use \texttt{groupby()} with \texttt{transform()} and
  \texttt{assign()} to conduct grouped operations on a DataFrame.
\end{enumerate}

\section{Required Packages}\label{required-packages}

This lesson requires pandas and numpy:

\begin{Shaded}
\begin{Highlighting}[]
\ImportTok{import}\NormalTok{ pandas }\ImportTok{as}\NormalTok{ pd}
\ImportTok{import}\NormalTok{ numpy }\ImportTok{as}\NormalTok{ np}
\end{Highlighting}
\end{Shaded}

\section{Simple Toy Dataset}\label{simple-toy-dataset}

Let's start with a simple toy dataset to illustrate the concepts:

\begin{Shaded}
\begin{Highlighting}[]
\CommentTok{\# Create a simple toy dataset}
\NormalTok{toy\_data }\OperatorTok{=}\NormalTok{ pd.DataFrame(}
\NormalTok{    \{}
        \StringTok{"year"}\NormalTok{: [}\DecValTok{2022}\NormalTok{, }\DecValTok{2022}\NormalTok{, }\DecValTok{2023}\NormalTok{, }\DecValTok{2023}\NormalTok{],}
        \StringTok{"month"}\NormalTok{: [}\StringTok{"Jan"}\NormalTok{, }\StringTok{"Feb"}\NormalTok{, }\StringTok{"Jan"}\NormalTok{, }\StringTok{"Feb"}\NormalTok{],}
        \StringTok{"sales"}\NormalTok{: [}\DecValTok{2}\NormalTok{, }\DecValTok{4}\NormalTok{, }\DecValTok{6}\NormalTok{, }\DecValTok{8}\NormalTok{],}
\NormalTok{    \}}
\NormalTok{)}

\BuiltInTok{print}\NormalTok{(toy\_data)}
\end{Highlighting}
\end{Shaded}

\begin{verbatim}
   year month  sales
0  2022   Jan      2
1  2022   Feb      4
2  2023   Jan      6
3  2023   Feb      8
\end{verbatim}

\subsection{\texorpdfstring{Regular \texttt{assign()}
Call}{Regular assign() Call}}\label{regular-assign-call}

Let's add a column with the mean sales for the entire dataset, then
calculate the difference between each month's sales and the mean sales:

\begin{Shaded}
\begin{Highlighting}[]
\NormalTok{toy\_data.assign(}
\NormalTok{    mean\_sales}\OperatorTok{=}\KeywordTok{lambda}\NormalTok{ x: x.sales.mean(), }
\NormalTok{    sales\_diff}\OperatorTok{=}\KeywordTok{lambda}\NormalTok{ x: x.sales }\OperatorTok{{-}}\NormalTok{ x.mean\_sales}
\NormalTok{)}
\end{Highlighting}
\end{Shaded}

\begin{longtable}[]{@{}llllll@{}}
\toprule\noalign{}
& year & month & sales & mean\_sales & sales\_diff \\
\midrule\noalign{}
\endhead
\bottomrule\noalign{}
\endlastfoot
0 & 2022 & Jan & 2 & 5.0 & -3.0 \\
1 & 2022 & Feb & 4 & 5.0 & -1.0 \\
2 & 2023 & Jan & 6 & 5.0 & 1.0 \\
3 & 2023 & Feb & 8 & 5.0 & 3.0 \\
\end{longtable}

In this example, the \texttt{mean\_sales} column contains the mean sales
value for the entire dataset (5), repeated in each row. The
\texttt{sales\_diff} column shows the difference between each month's
sales and the overall mean sales.

\subsection{\texorpdfstring{Grouped \texttt{assign()}
Call}{Grouped assign() Call}}\label{grouped-assign-call}

But we might want to see how each month's sales compare to the mean
sales for that year.

\begin{Shaded}
\begin{Highlighting}[]
\NormalTok{toy\_data.assign(}
\NormalTok{    mean\_sales}\OperatorTok{=}\KeywordTok{lambda}\NormalTok{ x: x.groupby(}\StringTok{"year"}\NormalTok{).sales.mean(),}
\NormalTok{    sales\_diff}\OperatorTok{=}\KeywordTok{lambda}\NormalTok{ x: x.sales }\OperatorTok{{-}}\NormalTok{ x.mean\_sales}
\NormalTok{)}
\end{Highlighting}
\end{Shaded}

\begin{longtable}[]{@{}llllll@{}}
\toprule\noalign{}
& year & month & sales & mean\_sales & sales\_diff \\
\midrule\noalign{}
\endhead
\bottomrule\noalign{}
\endlastfoot
0 & 2022 & Jan & 2 & NaN & NaN \\
1 & 2022 & Feb & 4 & NaN & NaN \\
2 & 2023 & Jan & 6 & NaN & NaN \\
3 & 2023 & Feb & 8 & NaN & NaN \\
\end{longtable}

Now we have the mean sales for year 2022 as 3 and for year 2023 as 7.
and the sales\_diff column shows how far each month's sales are from the
mean sales for that year.

Let's break down the code, especially the first line:

\begin{itemize}
\tightlist
\item
  Within the assign method, we initiate the \texttt{mean\_sales} column
\item
  Then we pass the lambda function to the \texttt{mean\_sales} column
\item
  The lambda function takes the dataframe, denoted \texttt{x} as input,
  groups it by \texttt{year} (\texttt{x.groupby("year")}), then pulls
  out the \texttt{sales} column (\texttt{x.groupby("year").sales}) then
  to compute the mean for each year, we use the \texttt{transform()}
  method (\texttt{x.groupby("year").sales.transform("mean")}).
\end{itemize}

\section{Real Dataset: Austin
Housing}\label{real-dataset-austin-housing}

Now let's apply these concepts to a real dataset. We'll use the
\texttt{housing} dataset containing housing sales data for Austin,
Texas:

\begin{Shaded}
\begin{Highlighting}[]
\NormalTok{housing }\OperatorTok{=}\NormalTok{ pd.read\_csv(}\StringTok{"data/austinhousing.csv"}\NormalTok{, usecols}\OperatorTok{=}\NormalTok{[}\StringTok{"year"}\NormalTok{, }\StringTok{"month"}\NormalTok{, }\StringTok{"sales"}\NormalTok{])}

\CommentTok{\#\# Display first few rows of the dataset}
\NormalTok{housing}
\end{Highlighting}
\end{Shaded}

\begin{longtable}[]{@{}llll@{}}
\toprule\noalign{}
& year & month & sales \\
\midrule\noalign{}
\endhead
\bottomrule\noalign{}
\endlastfoot
0 & 2000 & 1 & 1025 \\
1 & 2000 & 2 & 1277 \\
2 & 2000 & 3 & 1603 \\
... & ... & ... & ... \\
184 & 2015 & 5 & 2999 \\
185 & 2015 & 6 & 3301 \\
186 & 2015 & 7 & 3466 \\
\end{longtable}

\subsection{\texorpdfstring{Regular \texttt{assign()}
Call}{Regular assign() Call}}\label{regular-assign-call-1}

Let's add a column with the mean sales for the entire dataset, then
calculate the difference between each month's sales and the mean sales:

\begin{Shaded}
\begin{Highlighting}[]
\NormalTok{housing.assign(}
\NormalTok{    mean\_sales}\OperatorTok{=}\KeywordTok{lambda}\NormalTok{ x: x.sales.mean(), }
\NormalTok{    sales\_diff}\OperatorTok{=}\KeywordTok{lambda}\NormalTok{ x: x.sales }\OperatorTok{{-}}\NormalTok{ x.mean\_sales}
\NormalTok{)}
\end{Highlighting}
\end{Shaded}

\begin{longtable}[]{@{}llllll@{}}
\toprule\noalign{}
& year & month & sales & mean\_sales & sales\_diff \\
\midrule\noalign{}
\endhead
\bottomrule\noalign{}
\endlastfoot
0 & 2000 & 1 & 1025 & 1996.68984 & -971.68984 \\
1 & 2000 & 2 & 1277 & 1996.68984 & -719.68984 \\
2 & 2000 & 3 & 1603 & 1996.68984 & -393.68984 \\
... & ... & ... & ... & ... & ... \\
184 & 2015 & 5 & 2999 & 1996.68984 & 1002.31016 \\
185 & 2015 & 6 & 3301 & 1996.68984 & 1304.31016 \\
186 & 2015 & 7 & 3466 & 1996.68984 & 1469.31016 \\
\end{longtable}

In this example, the \texttt{mean\_sales} column contains the mean sales
value for the entire dataset, repeated in each row. The
\texttt{sales\_diff} column shows the difference between each month's
sales and the overall mean sales.

\subsection{\texorpdfstring{Grouped \texttt{assign()}
Call}{Grouped assign() Call}}\label{grouped-assign-call-1}

Now, let's add a column with the mean sales for each year, and then
calculate the difference between each month's sales and the mean sales
for that year:

\begin{Shaded}
\begin{Highlighting}[]
\NormalTok{housing.assign(}
\NormalTok{    mean\_sales}\OperatorTok{=}\KeywordTok{lambda}\NormalTok{ x: x.groupby(}\StringTok{"year"}\NormalTok{).sales.transform(}\StringTok{"mean"}\NormalTok{),}
\NormalTok{    sales\_diff}\OperatorTok{=}\KeywordTok{lambda}\NormalTok{ x: x.sales }\OperatorTok{{-}}\NormalTok{ x.mean\_sales}
\NormalTok{)}
\end{Highlighting}
\end{Shaded}

\begin{longtable}[]{@{}llllll@{}}
\toprule\noalign{}
& year & month & sales & mean\_sales & sales\_diff \\
\midrule\noalign{}
\endhead
\bottomrule\noalign{}
\endlastfoot
0 & 2000 & 1 & 1025 & 1551.750000 & -526.750000 \\
1 & 2000 & 2 & 1277 & 1551.750000 & -274.750000 \\
2 & 2000 & 3 & 1603 & 1551.750000 & 51.250000 \\
... & ... & ... & ... & ... & ... \\
184 & 2015 & 5 & 2999 & 2696.857143 & 302.142857 \\
185 & 2015 & 6 & 3301 & 2696.857143 & 604.142857 \\
186 & 2015 & 7 & 3466 & 2696.857143 & 769.142857 \\
\end{longtable}

Here's what's happening: - Within the \texttt{assign()} method, we group
by year and use \texttt{transform("mean")} to compute the mean sales for
each year. - The \texttt{mean\_sales} column now contains the mean sales
for each year, repeated in each row of that year. - The
\texttt{sales\_diff} column shows how far each month's sales are from
the mean sales for that year, allowing us to see which months are high
and low performers within each year.

This grouped approach allows us to compare sales within each year,
accounting for overall trends or differences between years.

\begin{tcolorbox}[enhanced jigsaw, colframe=quarto-callout-tip-color-frame, opacityback=0, titlerule=0mm, bottomrule=.15mm, breakable, leftrule=.75mm, colbacktitle=quarto-callout-tip-color!10!white, title=\textcolor{quarto-callout-tip-color}{\faLightbulb}\hspace{0.5em}{Practice}, rightrule=.15mm, coltitle=black, opacitybacktitle=0.6, colback=white, left=2mm, arc=.35mm, toptitle=1mm, bottomtitle=1mm, toprule=.15mm]

\subsection{Practice Q: Compute Mean Temperature for Each
Day}\label{practice-q-compute-mean-temperature-for-each-day}

With the \texttt{gibraltar} dataframe, group by the date of measurement,
then in a new variable called \texttt{mean\_temp}, compute the mean
temperature for each day. Store the result in a new dataframe called
\texttt{mean\_temp\_question}.

\begin{Shaded}
\begin{Highlighting}[]
\CommentTok{\#\# Complete the code with your answer:}
\NormalTok{mean\_temp\_question }\OperatorTok{=}\NormalTok{ gibraltar.assign(}
\NormalTok{    mean\_temp}\OperatorTok{=}\NormalTok{\_\_\_\_\_\_\_\_\_\_\_\_\_\_\_\_\_\_\_\_\_\_\_\_\_\_\_\_\_\_\_\_\_\_\_}\OperatorTok{{-}}
\NormalTok{)}

\NormalTok{mean\_temp\_question}
\end{Highlighting}
\end{Shaded}

\end{tcolorbox}

\subsection{More Examples of Grouped
Transformations}\label{more-examples-of-grouped-transformations}

\subsubsection{Rank Sales Within Each
Year}\label{rank-sales-within-each-year}

Now let's look at another example. We can add a column that ranks the
sales volume for each year.

\begin{Shaded}
\begin{Highlighting}[]
\NormalTok{housing.assign(sales\_rank}\OperatorTok{=}\NormalTok{housing.groupby(}\StringTok{"year"}\NormalTok{).sales.rank(ascending}\OperatorTok{=}\VariableTok{False}\NormalTok{))}
\end{Highlighting}
\end{Shaded}

\begin{longtable}[]{@{}lllll@{}}
\toprule\noalign{}
& year & month & sales & sales\_rank \\
\midrule\noalign{}
\endhead
\bottomrule\noalign{}
\endlastfoot
0 & 2000 & 1 & 1025 & 12.0 \\
1 & 2000 & 2 & 1277 & 10.0 \\
2 & 2000 & 3 & 1603 & 5.0 \\
... & ... & ... & ... & ... \\
184 & 2015 & 5 & 2999 & 3.0 \\
185 & 2015 & 6 & 3301 & 2.0 \\
186 & 2015 & 7 & 3466 & 1.0 \\
\end{longtable}

In this example, we're creating a new column \texttt{sales\_rank} that
ranks the \texttt{sales} column in descending order within each year
group. This allows us to see which months had the highest sales for each
year.

These examples demonstrate how we can use grouped operations to perform
more nuanced analyses on our data, allowing us to uncover patterns and
insights within specific subgroups of our dataset.

\subsubsection{Cumulative Sales Column}\label{cumulative-sales-column}

Let's add a column that shows the cumulative sales for each year, adding
up the sales for each month within the year.

\begin{Shaded}
\begin{Highlighting}[]
\NormalTok{housing.assign(}
\NormalTok{    cumulative\_sales}\OperatorTok{=}\NormalTok{housing.groupby(}\StringTok{"year"}\NormalTok{).sales.transform(}\StringTok{"cumsum"}\NormalTok{)}
\NormalTok{)}
\end{Highlighting}
\end{Shaded}

\begin{longtable}[]{@{}lllll@{}}
\toprule\noalign{}
& year & month & sales & cumulative\_sales \\
\midrule\noalign{}
\endhead
\bottomrule\noalign{}
\endlastfoot
0 & 2000 & 1 & 1025 & 1025 \\
1 & 2000 & 2 & 1277 & 2302 \\
2 & 2000 & 3 & 1603 & 3905 \\
... & ... & ... & ... & ... \\
184 & 2015 & 5 & 2999 & 12111 \\
185 & 2015 & 6 & 3301 & 15412 \\
186 & 2015 & 7 & 3466 & 18878 \\
\end{longtable}

\subsubsection{Percent of Yearly Sales Done Each
Month}\label{percent-of-yearly-sales-done-each-month}

We can calculate the percent of sales done each month for that year.

\begin{Shaded}
\begin{Highlighting}[]
\NormalTok{housing.assign(}
\NormalTok{    percent\_sales}\OperatorTok{=}\KeywordTok{lambda}\NormalTok{ x: }\DecValTok{100} \OperatorTok{*}\NormalTok{ x.sales }\OperatorTok{/}\NormalTok{ x.groupby(}\StringTok{"year"}\NormalTok{).sales.transform(}\StringTok{"sum"}\NormalTok{)}
\NormalTok{)}
\end{Highlighting}
\end{Shaded}

\begin{longtable}[]{@{}lllll@{}}
\toprule\noalign{}
& year & month & sales & percent\_sales \\
\midrule\noalign{}
\endhead
\bottomrule\noalign{}
\endlastfoot
0 & 2000 & 1 & 1025 & 5.504538 \\
1 & 2000 & 2 & 1277 & 6.857849 \\
2 & 2000 & 3 & 1603 & 8.608560 \\
... & ... & ... & ... & ... \\
184 & 2015 & 5 & 2999 & 15.886217 \\
185 & 2015 & 6 & 3301 & 17.485962 \\
186 & 2015 & 7 & 3466 & 18.359996 \\
\end{longtable}

\begin{tcolorbox}[enhanced jigsaw, colframe=quarto-callout-tip-color-frame, opacityback=0, titlerule=0mm, bottomrule=.15mm, breakable, leftrule=.75mm, colbacktitle=quarto-callout-tip-color!10!white, title=\textcolor{quarto-callout-tip-color}{\faLightbulb}\hspace{0.5em}{Practice}, rightrule=.15mm, coltitle=black, opacitybacktitle=0.6, colback=white, left=2mm, arc=.35mm, toptitle=1mm, bottomtitle=1mm, toprule=.15mm]

\subsection{Practice Q: Cumulative Listings
Column}\label{practice-q-cumulative-listings-column}

With the \texttt{housing} dataframe, group by \texttt{year}, then create
a cumulative listings column for each year.

\begin{Shaded}
\begin{Highlighting}[]
\CommentTok{\# Your code here}
\end{Highlighting}
\end{Shaded}

\end{tcolorbox}

\section{\texorpdfstring{Grouped Filtering with
\texttt{query()}}{Grouped Filtering with query()}}\label{grouped-filtering-with-query}

To query per group, it is usually better to create a relevant column
with \texttt{groupby()} and \texttt{transform()}, then use
\texttt{query()} on the resulting dataframe to filter data.

\subsection{Filter for Month with Highest Sales Volume per
Year}\label{filter-for-month-with-highest-sales-volume-per-year}

For example, if we want to filter the data for the month with the
highest sales volume per year group (the month with the highest sales
volume for each year), we can use \texttt{groupby()} with
\texttt{transform()} to first create a column with the max sales volume
per year, then ungroup and filter.

\begin{Shaded}
\begin{Highlighting}[]
\NormalTok{(}
\NormalTok{housing}
\NormalTok{  .assign(max\_sales}\OperatorTok{=}\NormalTok{housing.groupby(}\StringTok{"year"}\NormalTok{).sales.transform(}\StringTok{"max"}\NormalTok{))}
\NormalTok{  .query(}\StringTok{"sales == max\_sales"}\NormalTok{)}
\NormalTok{)}
\end{Highlighting}
\end{Shaded}

\begin{longtable}[]{@{}lllll@{}}
\toprule\noalign{}
& year & month & sales & max\_sales \\
\midrule\noalign{}
\endhead
\bottomrule\noalign{}
\endlastfoot
4 & 2000 & 5 & 1980 & 1980 \\
18 & 2001 & 7 & 1871 & 1871 \\
28 & 2002 & 5 & 1931 & 1931 \\
... & ... & ... & ... & ... \\
162 & 2013 & 7 & 3376 & 3376 \\
173 & 2014 & 6 & 3195 & 3195 \\
186 & 2015 & 7 & 3466 & 3466 \\
\end{longtable}

We can drop the \texttt{max\_sales} column if we want to at the end.

\begin{Shaded}
\begin{Highlighting}[]
\NormalTok{(}
\NormalTok{housing}
\NormalTok{  .assign(max\_sales}\OperatorTok{=}\NormalTok{housing.groupby(}\StringTok{"year"}\NormalTok{).sales.transform(}\StringTok{"max"}\NormalTok{))}
\NormalTok{  .query(}\StringTok{"sales == max\_sales"}\NormalTok{)}
\NormalTok{  .drop(columns}\OperatorTok{=}\NormalTok{[}\StringTok{"max\_sales"}\NormalTok{])}
\NormalTok{)}
\end{Highlighting}
\end{Shaded}

\begin{longtable}[]{@{}llll@{}}
\toprule\noalign{}
& year & month & sales \\
\midrule\noalign{}
\endhead
\bottomrule\noalign{}
\endlastfoot
4 & 2000 & 5 & 1980 \\
18 & 2001 & 7 & 1871 \\
28 & 2002 & 5 & 1931 \\
... & ... & ... & ... \\
162 & 2013 & 7 & 3376 \\
173 & 2014 & 6 & 3195 \\
186 & 2015 & 7 & 3466 \\
\end{longtable}

\begin{tcolorbox}[enhanced jigsaw, colframe=quarto-callout-tip-color-frame, opacityback=0, titlerule=0mm, bottomrule=.15mm, breakable, leftrule=.75mm, colbacktitle=quarto-callout-tip-color!10!white, title=\textcolor{quarto-callout-tip-color}{\faLightbulb}\hspace{0.5em}{Practice}, rightrule=.15mm, coltitle=black, opacitybacktitle=0.6, colback=white, left=2mm, arc=.35mm, toptitle=1mm, bottomtitle=1mm, toprule=.15mm]

\subsection{Practice Q: Filter for Time of Day with Highest Wind
Speed}\label{practice-q-filter-for-time-of-day-with-highest-wind-speed}

With the \texttt{gibraltar} dataframe created above, group by the date
of measurement, calculate the time of day with the highest recorded wind
speed \texttt{wind\_speed} for each day, then filter the dataframe to
keep only the rows with the highest wind speed for each day.

Note that for some days, there may be a tie in the highest wind speed.

\begin{Shaded}
\begin{Highlighting}[]
\CommentTok{\# Your code here}
\end{Highlighting}
\end{Shaded}

\end{tcolorbox}

\subsection{Month with the Most ``Typical''
Sales}\label{month-with-the-most-typical-sales}

We can also use \texttt{groupby()} with \texttt{transform()} to find the
month with the most ``typical'' sales. We'll use the mean sales volume
for each year as the ``typical'' sales volume, then get the month with
the closest sales to that.

\begin{Shaded}
\begin{Highlighting}[]
\NormalTok{(}
\NormalTok{housing}
\NormalTok{  .assign(}
\NormalTok{    mean\_sales}\OperatorTok{=}\NormalTok{housing.groupby(}\StringTok{"year"}\NormalTok{).sales.transform(}\StringTok{"mean"}\NormalTok{),}
\NormalTok{    sales\_diff\_abs}\OperatorTok{=}\KeywordTok{lambda}\NormalTok{ x: (x.sales }\OperatorTok{{-}}\NormalTok{ x.mean\_sales).}\BuiltInTok{abs}\NormalTok{()}
\NormalTok{    )}
\NormalTok{  .sort\_values(}\StringTok{"sales\_diff\_abs"}\NormalTok{)}
\NormalTok{  .groupby(}\StringTok{"year"}\NormalTok{)}
\NormalTok{  .first()}
\NormalTok{)}
\end{Highlighting}
\end{Shaded}

\begin{longtable}[]{@{}lllll@{}}
\toprule\noalign{}
& month & sales & mean\_sales & sales\_diff\_abs \\
year & & & & \\
\midrule\noalign{}
\endhead
\bottomrule\noalign{}
\endlastfoot
2000 & 4 & 1556 & 1551.750000 & 4.250000 \\
2001 & 3 & 1553 & 1532.666667 & 20.333333 \\
2002 & 3 & 1550 & 1559.666667 & 9.666667 \\
... & ... & ... & ... & ... \\
2013 & 9 & 2544 & 2536.333333 & 7.666667 \\
2014 & 10 & 2588 & 2579.416667 & 8.583333 \\
2015 & 3 & 2677 & 2696.857143 & 19.857143 \\
\end{longtable}

\begin{tcolorbox}[enhanced jigsaw, colframe=quarto-callout-tip-color-frame, opacityback=0, titlerule=0mm, bottomrule=.15mm, breakable, leftrule=.75mm, colbacktitle=quarto-callout-tip-color!10!white, title=\textcolor{quarto-callout-tip-color}{\faLightbulb}\hspace{0.5em}{Practice}, rightrule=.15mm, coltitle=black, opacitybacktitle=0.6, colback=white, left=2mm, arc=.35mm, toptitle=1mm, bottomtitle=1mm, toprule=.15mm]

\subsection{Practice Q: Filter for Times with Above Mean
Temperature}\label{practice-q-filter-for-times-with-above-mean-temperature}

Group the \texttt{gibraltar} dataframe by \texttt{date}, then use
\texttt{apply()}, \texttt{assign()}, and \texttt{query()} as needed to
subset to only times when the temperature is above the mean temperature
for that day. You should have 698 rows in the output.

\begin{Shaded}
\begin{Highlighting}[]
\CommentTok{\# Your code here}
\end{Highlighting}
\end{Shaded}

\end{tcolorbox}

\section{Wrap Up}\label{wrap-up-9}

In this lesson, we explored powerful techniques for grouped operations
in pandas:

\begin{enumerate}
\def\labelenumi{\arabic{enumi}.}
\tightlist
\item
  Using \texttt{groupby()} with \texttt{transform()} and
  \texttt{assign()} for grouped transformations
\item
  Combining \texttt{groupby()}, \texttt{transform()}, and
  \texttt{query()} for grouped filtering
\end{enumerate}

These methods significantly enhance our ability to analyze and
manipulate data within specific groups.

Congratulations on making it through!

\chapter{Reshaping data}\label{reshaping-data}

\begin{Shaded}
\begin{Highlighting}[]
\ImportTok{import}\NormalTok{ pandas }\ImportTok{as}\NormalTok{ pd}

\NormalTok{pd.options.display.max\_rows }\OperatorTok{=} \DecValTok{10}
\end{Highlighting}
\end{Shaded}

\section{Intro}\label{intro-4}

Pivoting or reshaping is a data manipulation technique that involves
re-orienting the rows and columns of a dataset. This is often required
to make data easier to analyze or understand.

In this lesson, we will cover how to effectively pivot data using
\texttt{pandas} functions.

\section{Learning Objectives}\label{learning-objectives-12}

\begin{itemize}
\tightlist
\item
  Understand what wide data format is, and what long data format is.
\item
  Learn how to pivot long data to wide data using \texttt{melt()}.
\item
  Learn how to pivot wide data to long data using \texttt{pivot()}.
\end{itemize}

\section{What do wide and long mean?}\label{what-do-wide-and-long-mean}

The terms wide and long are best understood in the context of example
datasets. Let's take a look at some now.

Imagine that you have three products for which you collect sales data
over three months.

You can record the data in a wide format like this:

\begin{longtable}[]{@{}llll@{}}
\toprule\noalign{}
Product & Jan & Feb & Mar \\
\midrule\noalign{}
\endhead
\bottomrule\noalign{}
\endlastfoot
A & 100 & 120 & 110 \\
B & 90 & 95 & 100 \\
C & 80 & 85 & 90 \\
\end{longtable}

\begin{center}\rule{0.5\linewidth}{0.5pt}\end{center}

Or you could record the data in a long format as so:

\begin{longtable}[]{@{}lll@{}}
\toprule\noalign{}
Product & Month & Sales \\
\midrule\noalign{}
\endhead
\bottomrule\noalign{}
\endlastfoot
A & Jan & 100 \\
A & Feb & 120 \\
A & Mar & 110 \\
B & Jan & 90 \\
B & Feb & 95 \\
B & Mar & 100 \\
C & Jan & 80 \\
C & Feb & 85 \\
C & Mar & 90 \\
\end{longtable}

Take a minute to study the two datasets to make sure you understand the
relationship between them.

In the wide dataset, each observational unit (each product) occupies
only one row. And each measurement (sales in Jan, Feb, Mar) is in a
separate column.

In the long dataset, on the other hand, each observational unit (each
product) occupies multiple rows, with one row for each measurement.

\begin{center}\rule{0.5\linewidth}{0.5pt}\end{center}

Here is another example with mock data, in which the observational units
are countries:

Long format:

\begin{longtable}[]{@{}lll@{}}
\toprule\noalign{}
Country & Year & GDP \\
\midrule\noalign{}
\endhead
\bottomrule\noalign{}
\endlastfoot
USA & 2020 & 21433 \\
USA & 2021 & 22940 \\
China & 2020 & 14723 \\
China & 2021 & 17734 \\
\end{longtable}

Wide format:

\begin{longtable}[]{@{}lll@{}}
\toprule\noalign{}
Country & GDP\_2020 & GDP\_2021 \\
\midrule\noalign{}
\endhead
\bottomrule\noalign{}
\endlastfoot
USA & 21433 & 22940 \\
China & 14723 & 17734 \\
\end{longtable}

\begin{center}\rule{0.5\linewidth}{0.5pt}\end{center}

The examples above are both time-series datasets, because the
measurements are repeated across time. But the concepts of long and wide
are relevant to other kinds of data too.

Consider the example below, showing the number of employees in different
departments of three companies:

Wide format:

\begin{longtable}[]{@{}llll@{}}
\toprule\noalign{}
Company & HR & Sales & IT \\
\midrule\noalign{}
\endhead
\bottomrule\noalign{}
\endlastfoot
A & 10 & 20 & 15 \\
B & 8 & 25 & 20 \\
C & 12 & 18 & 22 \\
\end{longtable}

Long format:

\begin{longtable}[]{@{}lll@{}}
\toprule\noalign{}
Company & Department & Employees \\
\midrule\noalign{}
\endhead
\bottomrule\noalign{}
\endlastfoot
A & HR & 10 \\
A & Sales & 20 \\
A & IT & 15 \\
B & HR & 8 \\
B & Sales & 25 \\
B & IT & 20 \\
C & HR & 12 \\
C & Sales & 18 \\
C & IT & 22 \\
\end{longtable}

In the wide dataset, again, each observational unit (each company)
occupies only one row, with the repeated measurements for that unit
(number of employees in different departments) spread across multiple
columns.

In the long dataset, each observational unit is spread over multiple
lines.

\begin{tcolorbox}[enhanced jigsaw, colframe=quarto-callout-note-color-frame, opacityback=0, titlerule=0mm, bottomrule=.15mm, breakable, leftrule=.75mm, colbacktitle=quarto-callout-note-color!10!white, title=\textcolor{quarto-callout-note-color}{\faInfo}\hspace{0.5em}{Vocab}, rightrule=.15mm, coltitle=black, opacitybacktitle=0.6, colback=white, left=2mm, arc=.35mm, toptitle=1mm, bottomtitle=1mm, toprule=.15mm]

The ``observational units'', sometimes called ``statistical units'' of a
dataset are the primary entities or items described by the columns in
that dataset.

In the first example, the observational/statistical units were products;
in the second example, countries, and in the third example, companies.

\end{tcolorbox}

\begin{tcolorbox}[enhanced jigsaw, colframe=quarto-callout-tip-color-frame, opacityback=0, titlerule=0mm, bottomrule=.15mm, breakable, leftrule=.75mm, colbacktitle=quarto-callout-tip-color!10!white, title=\textcolor{quarto-callout-tip-color}{\faLightbulb}\hspace{0.5em}{Practice}, rightrule=.15mm, coltitle=black, opacitybacktitle=0.6, colback=white, left=2mm, arc=.35mm, toptitle=1mm, bottomtitle=1mm, toprule=.15mm]

Consider the mock dataset created below:

\begin{Shaded}
\begin{Highlighting}[]
\ImportTok{import}\NormalTok{ pandas }\ImportTok{as}\NormalTok{ pd}

\NormalTok{temperatures }\OperatorTok{=}\NormalTok{ pd.DataFrame(}
\NormalTok{    \{}
        \StringTok{"country"}\NormalTok{: [}\StringTok{"Sweden"}\NormalTok{, }\StringTok{"Denmark"}\NormalTok{, }\StringTok{"Norway"}\NormalTok{],}
        \StringTok{"avgtemp.1994"}\NormalTok{: [}\DecValTok{1}\NormalTok{, }\DecValTok{2}\NormalTok{, }\DecValTok{3}\NormalTok{],}
        \StringTok{"avgtemp.1995"}\NormalTok{: [}\DecValTok{3}\NormalTok{, }\DecValTok{4}\NormalTok{, }\DecValTok{5}\NormalTok{],}
        \StringTok{"avgtemp.1996"}\NormalTok{: [}\DecValTok{5}\NormalTok{, }\DecValTok{6}\NormalTok{, }\DecValTok{7}\NormalTok{],}
\NormalTok{    \}}
\NormalTok{)}
\NormalTok{temperatures}
\end{Highlighting}
\end{Shaded}

\begin{longtable}[]{@{}lllll@{}}
\toprule\noalign{}
& country & avgtemp.1994 & avgtemp.1995 & avgtemp.1996 \\
\midrule\noalign{}
\endhead
\bottomrule\noalign{}
\endlastfoot
0 & Sweden & 1 & 3 & 5 \\
1 & Denmark & 2 & 4 & 6 \\
2 & Norway & 3 & 5 & 7 \\
\end{longtable}

Is this data in a wide or long format?

\begin{Shaded}
\begin{Highlighting}[]
\CommentTok{\# Write your answer here}
\end{Highlighting}
\end{Shaded}

\end{tcolorbox}

\section{When should you use wide vs long
data?}\label{when-should-you-use-wide-vs-long-data}

The truth is: it really depends on what you want to do! The wide format
is great for \emph{displaying data} because it's easy to visually
compare values this way. Long data is best for some data analysis tasks,
like grouping and plotting.

It will therefore be essential for you to know how to switch from one
format to the other easily. Switching from the wide to the long format,
or the other way around, is called \textbf{pivoting}.

\section{Pivoting wide to long}\label{pivoting-wide-to-long}

To practice pivoting from a wide to a long format, we'll consider data
from Our World in Data on fossil fuel consumption per capita. You can
find the data
\href{https://ourworldindata.org/grapher/fossil-fuels-per-capita}{here}.

Below, we read in and view this data on fossil fuel consumption per
capita:

\begin{Shaded}
\begin{Highlighting}[]
\NormalTok{fuels\_wide }\OperatorTok{=}\NormalTok{ pd.read\_csv(}\StringTok{"data/oil\_per\_capita\_wide.csv"}\NormalTok{)}
\NormalTok{fuels\_wide}
\end{Highlighting}
\end{Shaded}

\begin{longtable}[]{@{}lllllllll@{}}
\toprule\noalign{}
& Entity & Code & y\_1970 & y\_1980 & y\_1990 & y\_2000 & y\_2010 &
y\_2020 \\
\midrule\noalign{}
\endhead
\bottomrule\noalign{}
\endlastfoot
0 & Algeria & DZA & 1764.8470 & 3532.7976 & 4381.6636 & 3351.2180 &
5064.9863 & 4877.2680 \\
1 & Argentina & ARG & 11677.9680 & 10598.3990 & 7046.2485 & 7146.8154 &
7966.7827 & 6399.2114 \\
2 & Australia & AUS & 23040.4550 & 25007.4380 & 23046.9510 & 23976.3550
& 23584.3070 & 20332.4100 \\
3 & Austria & AUT & 14338.8090 & 19064.0920 & 16595.1930 & 18189.0920 &
18424.1170 & 14934.0650 \\
4 & Azerbaijan & AZE & NaN & NaN & 13516.0190 & 9119.3470 & 4031.9407 &
5615.1157 \\
... & ... & ... & ... & ... & ... & ... & ... & ... \\
76 & United States & USA & 40813.9530 & 42365.6500 & 37525.5160 &
37730.1600 & 31791.3070 & 26895.4770 \\
77 & Uzbekistan & UZB & NaN & NaN & 6324.8677 & 3197.1330 & 1880.1338 &
1859.1548 \\
78 & Venezuela & VEN & 11138.2210 & 16234.0960 & 12404.5570 & 11239.9260
& 14948.3070 & 4742.6226 \\
79 & Vietnam & VNM & 1757.6117 & 439.9465 & 523.2565 & 1280.3065 &
2296.7590 & 2927.7446 \\
80 & World & OWID\_WRL & 7217.8340 & 8002.0854 & 7074.2583 & 6990.4272 &
6879.6110 & 6216.8060 \\
\end{longtable}

We observe that each observational unit (each country) occupies only one
row, with the repeated measurements of fossil fuel consumption (in
Kilowatt-hour equivalents) spread out across multiple columns. Hence
this dataset is in a wide format.

To convert to a long format, we can use the convenient \texttt{melt}
function. Within \texttt{melt} we define which columns we want to pivot:

\begin{Shaded}
\begin{Highlighting}[]
\NormalTok{(fuels\_wide}
\NormalTok{ .melt(id\_vars}\OperatorTok{=}\NormalTok{[}\StringTok{\textquotesingle{}Entity\textquotesingle{}}\NormalTok{, }\StringTok{\textquotesingle{}Code\textquotesingle{}}\NormalTok{], }
\NormalTok{       value\_vars}\OperatorTok{=}\NormalTok{fuels\_wide.columns[}\DecValTok{2}\NormalTok{:])}
\NormalTok{)}
\end{Highlighting}
\end{Shaded}

\begin{longtable}[]{@{}lllll@{}}
\toprule\noalign{}
& Entity & Code & variable & value \\
\midrule\noalign{}
\endhead
\bottomrule\noalign{}
\endlastfoot
0 & Algeria & DZA & y\_1970 & 1764.8470 \\
1 & Argentina & ARG & y\_1970 & 11677.9680 \\
2 & Australia & AUS & y\_1970 & 23040.4550 \\
3 & Austria & AUT & y\_1970 & 14338.8090 \\
4 & Azerbaijan & AZE & y\_1970 & NaN \\
... & ... & ... & ... & ... \\
481 & United States & USA & y\_2020 & 26895.4770 \\
482 & Uzbekistan & UZB & y\_2020 & 1859.1548 \\
483 & Venezuela & VEN & y\_2020 & 4742.6226 \\
484 & Vietnam & VNM & y\_2020 & 2927.7446 \\
485 & World & OWID\_WRL & y\_2020 & 6216.8060 \\
\end{longtable}

Very easy!

Let's break down the code:

\begin{itemize}
\tightlist
\item
  \texttt{id\_vars} refers to the column(s) that will not be pivoted. In
  this case, it's the country and the code.
\item
  \texttt{value\_vars} is the column(s) that will be pivoted. In this
  case, it's the years columns.
\end{itemize}

The years are now indicated in the variable \texttt{variable}, and all
the consumption values occupy a single variable, \texttt{value}.We may
wish to rename the \texttt{variable} column to \texttt{year}, and the
\texttt{value} column to \texttt{oil\_consumption}. This can be done
directly in the \texttt{melt} function:

\begin{Shaded}
\begin{Highlighting}[]
\NormalTok{(fuels\_wide}
\NormalTok{ .melt(id\_vars}\OperatorTok{=}\NormalTok{[}\StringTok{\textquotesingle{}Entity\textquotesingle{}}\NormalTok{, }\StringTok{\textquotesingle{}Code\textquotesingle{}}\NormalTok{], }
\NormalTok{       value\_vars}\OperatorTok{=}\NormalTok{fuels\_wide.columns[}\DecValTok{2}\NormalTok{:],}
\NormalTok{       var\_name}\OperatorTok{=}\StringTok{\textquotesingle{}year\textquotesingle{}}\NormalTok{, }
\NormalTok{       value\_name}\OperatorTok{=}\StringTok{\textquotesingle{}oil\_consumption\textquotesingle{}}\NormalTok{)}
\NormalTok{)}
\end{Highlighting}
\end{Shaded}

\begin{longtable}[]{@{}lllll@{}}
\toprule\noalign{}
& Entity & Code & year & oil\_consumption \\
\midrule\noalign{}
\endhead
\bottomrule\noalign{}
\endlastfoot
0 & Algeria & DZA & y\_1970 & 1764.8470 \\
1 & Argentina & ARG & y\_1970 & 11677.9680 \\
2 & Australia & AUS & y\_1970 & 23040.4550 \\
3 & Austria & AUT & y\_1970 & 14338.8090 \\
4 & Azerbaijan & AZE & y\_1970 & NaN \\
... & ... & ... & ... & ... \\
481 & United States & USA & y\_2020 & 26895.4770 \\
482 & Uzbekistan & UZB & y\_2020 & 1859.1548 \\
483 & Venezuela & VEN & y\_2020 & 4742.6226 \\
484 & Vietnam & VNM & y\_2020 & 2927.7446 \\
485 & World & OWID\_WRL & y\_2020 & 6216.8060 \\
\end{longtable}

You may also want to remove the \texttt{y\_} in front of each year. This
can be achieved with as string operation. We'll also arrange the data by
country and year:

\begin{Shaded}
\begin{Highlighting}[]
\NormalTok{(}
\NormalTok{    fuels\_wide.melt(}
\NormalTok{        id\_vars}\OperatorTok{=}\NormalTok{[}\StringTok{"Entity"}\NormalTok{, }\StringTok{"Code"}\NormalTok{],}
\NormalTok{        value\_vars}\OperatorTok{=}\NormalTok{fuels\_wide.columns[}\DecValTok{2}\NormalTok{:],}
\NormalTok{        var\_name}\OperatorTok{=}\StringTok{"year"}\NormalTok{,}
\NormalTok{        value\_name}\OperatorTok{=}\StringTok{"oil\_consumption"}\NormalTok{,}
\NormalTok{    )}
\NormalTok{    .assign(year}\OperatorTok{=}\KeywordTok{lambda}\NormalTok{ df: df[}\StringTok{"year"}\NormalTok{].}\BuiltInTok{str}\NormalTok{.replace(}\StringTok{"y\_"}\NormalTok{, }\StringTok{""}\NormalTok{).astype(}\BuiltInTok{int}\NormalTok{))}
\NormalTok{    .sort\_values(by}\OperatorTok{=}\NormalTok{[}\StringTok{"Entity"}\NormalTok{, }\StringTok{"year"}\NormalTok{])}
\NormalTok{)}
\end{Highlighting}
\end{Shaded}

\begin{longtable}[]{@{}lllll@{}}
\toprule\noalign{}
& Entity & Code & year & oil\_consumption \\
\midrule\noalign{}
\endhead
\bottomrule\noalign{}
\endlastfoot
0 & Algeria & DZA & 1970 & 1764.8470 \\
81 & Algeria & DZA & 1980 & 3532.7976 \\
162 & Algeria & DZA & 1990 & 4381.6636 \\
243 & Algeria & DZA & 2000 & 3351.2180 \\
324 & Algeria & DZA & 2010 & 5064.9863 \\
... & ... & ... & ... & ... \\
161 & World & OWID\_WRL & 1980 & 8002.0854 \\
242 & World & OWID\_WRL & 1990 & 7074.2583 \\
323 & World & OWID\_WRL & 2000 & 6990.4272 \\
404 & World & OWID\_WRL & 2010 & 6879.6110 \\
485 & World & OWID\_WRL & 2020 & 6216.8060 \\
\end{longtable}

Here's what we added above:

\begin{itemize}
\tightlist
\item
  In the \texttt{assign} function, we used a lambda function to replace
  the \texttt{y\_} in front of each year with an empty string, and then
  convert the year to an integer.
\item
  We used the \texttt{sort\_values} function to sort the data by country
  and year.
\end{itemize}

Now we have a clean, long dataset; great! For later use, let's now store
this data:

\begin{Shaded}
\begin{Highlighting}[]
\NormalTok{fuels\_long }\OperatorTok{=}\NormalTok{ (}
\NormalTok{    fuels\_wide.melt(}
\NormalTok{        id\_vars}\OperatorTok{=}\NormalTok{[}\StringTok{"Entity"}\NormalTok{, }\StringTok{"Code"}\NormalTok{],}
\NormalTok{        value\_vars}\OperatorTok{=}\NormalTok{fuels\_wide.columns[}\DecValTok{2}\NormalTok{:],}
\NormalTok{        var\_name}\OperatorTok{=}\StringTok{"year"}\NormalTok{,}
\NormalTok{        value\_name}\OperatorTok{=}\StringTok{"oil\_consumption"}\NormalTok{,}
\NormalTok{    )}
\NormalTok{    .assign(year}\OperatorTok{=}\KeywordTok{lambda}\NormalTok{ df: df[}\StringTok{"year"}\NormalTok{].}\BuiltInTok{str}\NormalTok{.replace(}\StringTok{"y\_"}\NormalTok{, }\StringTok{""}\NormalTok{).astype(}\BuiltInTok{int}\NormalTok{))}
\NormalTok{    .sort\_values(by}\OperatorTok{=}\NormalTok{[}\StringTok{"Entity"}\NormalTok{, }\StringTok{"year"}\NormalTok{])}
\NormalTok{)}
\end{Highlighting}
\end{Shaded}

\begin{tcolorbox}[enhanced jigsaw, colframe=quarto-callout-tip-color-frame, opacityback=0, titlerule=0mm, bottomrule=.15mm, breakable, leftrule=.75mm, colbacktitle=quarto-callout-tip-color!10!white, title=\textcolor{quarto-callout-tip-color}{\faLightbulb}\hspace{0.5em}{Practice}, rightrule=.15mm, coltitle=black, opacitybacktitle=0.6, colback=white, left=2mm, arc=.35mm, toptitle=1mm, bottomtitle=1mm, toprule=.15mm]

For this practice question, you will use the \texttt{euro\_births\_wide}
dataset from
\href{https://ec.europa.eu/eurostat/databrowser/view/tps00204/default/table}{Eurostat}.
It shows the annual number of births in 50 European countries:

\begin{Shaded}
\begin{Highlighting}[]
\NormalTok{euro\_births\_wide }\OperatorTok{=}\NormalTok{ pd.read\_csv(}\StringTok{"data/euro\_births\_wide.csv"}\NormalTok{)}
\NormalTok{euro\_births\_wide}
\end{Highlighting}
\end{Shaded}

\begin{longtable}[]{@{}lllllllll@{}}
\toprule\noalign{}
& country & x2015 & x2016 & x2017 & x2018 & x2019 & x2020 & x2021 \\
\midrule\noalign{}
\endhead
\bottomrule\noalign{}
\endlastfoot
0 & Belgium & 122274.0 & 121896.0 & 119690.0 & 118319.0 & 117695.0 &
114350.0 & 118349.0 \\
1 & Bulgaria & 65950.0 & 64984.0 & 63955.0 & 62197.0 & 61538.0 & 59086.0
& 58678.0 \\
2 & Czechia & 110764.0 & 112663.0 & 114405.0 & 114036.0 & 112231.0 &
110200.0 & 111793.0 \\
3 & Denmark & 58205.0 & 61614.0 & 61397.0 & 61476.0 & 61167.0 & 60937.0
& 63473.0 \\
4 & Germany & 737575.0 & 792141.0 & 784901.0 & 787523.0 & 778090.0 &
773144.0 & 795492.0 \\
... & ... & ... & ... & ... & ... & ... & ... & ... \\
45 & Ukraine & 411781.0 & 397037.0 & 363987.0 & 235.0 & 232.0 & NaN &
212.0 \\
46 & Armenia & 41763.0 & 40592.0 & 37700.0 & 335874.0 & 308817.0 &
293457.0 & 271983.0 \\
47 & Azerbaijan & 166210.0 & 159464.0 & 144041.0 & 36574.0 & 36041.0 &
36353.0 & NaN \\
48 & Georgia & 59249.0 & 56569.0 & 53293.0 & 138982.0 & 141179.0 &
126571.0 & 112284.0 \\
49 & NaN & NaN & NaN & NaN & 51138.0 & 48296.0 & 46520.0 & 45946.0 \\
\end{longtable}

The data is in a wide format. Convert it to a long format data frame
that has the following column names: ``country'', ``year'' and
``births\_count''

\begin{Shaded}
\begin{Highlighting}[]
\CommentTok{\# Your code here}
\end{Highlighting}
\end{Shaded}

\end{tcolorbox}

\section{Pivoting long to wide}\label{pivoting-long-to-wide}

Now you know how to pivot from wide to long with \texttt{melt}. How
about going the other way, from long to wide? For this, you can use the
\texttt{pivot} function.

But before we consider how to use this function to manipulate long data,
let's first consider \emph{where} you're likely to run into long data.

While wide data tends to come from external sources (as we have seen
above), long data on the other hand, is likely to be created by
\emph{you} while data wrangling, especially in the course of grouped
aggregations.

Let's see an example of this now.

We will use a dataset of contracts granted by the city of Chicago in the
years 2020 to 2023. You can find more information about the data
\href{https://data.cityofchicago.org/Administration-Finance/Contracts/rsxa-ify5/}{here}.
Below we import the data and do some data processing to prepare it for
analysis.

\begin{Shaded}
\begin{Highlighting}[]
\NormalTok{contracts\_raw }\OperatorTok{=}\NormalTok{ pd.read\_csv(}\StringTok{"data/chicago\_contracts\_20\_23.csv"}\NormalTok{)}

\NormalTok{contracts }\OperatorTok{=}\NormalTok{ (}
\NormalTok{    contracts\_raw}
\NormalTok{    .assign(year\_of\_contract}\OperatorTok{=}\KeywordTok{lambda}\NormalTok{ df: pd.to\_datetime(df[}\StringTok{"approval\_date"}\NormalTok{]).dt.year)}
\NormalTok{    .reindex(columns}\OperatorTok{=}\NormalTok{[}\StringTok{"year\_of\_contract"}\NormalTok{] }\OperatorTok{+} \BuiltInTok{list}\NormalTok{(contracts\_raw.columns.drop(}\StringTok{"approval\_date"}\NormalTok{)))}
\NormalTok{)}

\NormalTok{contracts}
\end{Highlighting}
\end{Shaded}

\begin{longtable}[]{@{}llllllllllllllllllll@{}}
\toprule\noalign{}
& year\_of\_contract & description & contract\_num & revision\_num &
specification\_num & contract\_type & start\_date & end\_date &
department & vendor\_name & vendor\_id & address\_1 & address\_2 & city
& state & zip & award\_amount & procurement\_type & contract\_pdf \\
\midrule\noalign{}
\endhead
\bottomrule\noalign{}
\endlastfoot
0 & 2020 & LEASE & 24406 & 32 & 96136 & PROPERTY LEASE & NaN & NaN & NaN
& 8700 BUILDING LLC & 89123305A & 7300 S NARRAGANSETT & NaN & BEDFORD
PARK & Illinois & 60638 & 321.1 & NaN & NaN \\
1 & 2020 & DFSS-HHS-CS-CEL: & 113798 & 0 & 1070196 & DELEGATE AGENCY &
12/01/2019 & 11/30/2022 & DEPT OF FAMILY AND SUPPORT SERVICES & CATHOLIC
CHARITIES OF THE ARCHDIOCESE OF CHICAGO & 102484615A & 1 E BANKS ST &
NaN & CHICAGO & Illinois & 60670 & 17692515.0 & NaN & NaN \\
2 & 2020 & DFSS-HHS-CS-CEL: & 113819 & 0 & 1070196 & DELEGATE AGENCY &
12/01/2019 & 11/30/2022 & DEPT OF FAMILY AND SUPPORT SERVICES & KIMBALL
DAYCARE CENTER \& KINDERGARTEN INC & 105458567Z & 1636-1638 N KIMBALL
AVE & NaN & CHICAGO & Illinois & 60647 & 11461500.0 & NaN &
http://ecm.cityofchicago.org/eSMARTContracts/s... \\
3 & 2020 & DFSS-HHS-CS-CEL: & 113818 & 0 & 1070196 & DELEGATE AGENCY &
12/01/2019 & 11/30/2022 & DEPT OF FAMILY AND SUPPORT SERVICES & JUDAH
INTERNATIONAL OUTREACH MINISTRIES, INC & 94219962X & 856 N PULASKI RD &
NaN & CHICAGO & Illinois & 60651 & 2356515.0 & NaN &
http://ecm.cityofchicago.org/eSMARTContracts/s... \\
4 & 2020 & DFSS-HHS-CS-CEL: & 113820 & 0 & 1070196 & DELEGATE AGENCY &
12/01/2019 & 11/30/2022 & DEPT OF FAMILY AND SUPPORT SERVICES & Marillac
St. Vincent Family Services Inc DBA S... & 97791861L & 212 S FRANCISCO
AVENUE EFT & NaN & CHICAGO & Illinois & 60612 & 3666015.0 & NaN &
http://ecm.cityofchicago.org/eSMARTContracts/s... \\
... & ... & ... & ... & ... & ... & ... & ... & ... & ... & ... & ... &
... & ... & ... & ... & ... & ... & ... & ... \\
28823 & 2023 & DFSS-CORP-HL-PSH: & 220413 & 3 & 1221503 & DELEGATE
AGENCY & NaN & NaN & DEPT OF FAMILY AND SUPPORT SERVICES & INNER VOICE
INC. & 6231926M & 1621 W WALNUT ST FL 1ST & NaN & CHICAGO & Illinois &
60612 & 0.0 & NaN & NaN \\
28824 & 2023 & DFSS-CORP-YS-OST: & 253846 & 0 & 1247493 & DELEGATE
AGENCY & NaN & NaN & DEPT OF FAMILY AND SUPPORT SERVICES & AFTER-SCHOOL
MATTERS, INC.\textbar CLEANED-UP & 72580818P & 66 E RANDOLPH ST FL 1ST &
NaN & CHICAGO & Illinois & 60601 & 32000.0 & NaN & NaN \\
28825 & 2023 & DFSS-IDHS-HL-INTHS: & 253843 & 0 & 1235949 & DELEGATE
AGENCY & NaN & NaN & DEPT OF FAMILY AND SUPPORT SERVICES & BREAKTHROUGH
URBAN MINISTRIES, INC. & 94722896V & 402 N ST LOUIS AVENUE EFT & NaN &
CHICAGO & Illinois & 60624 & 14400.0 & NaN & NaN \\
28826 & 2023 & CDPH-RW-PA: ESS-HRSA PO 116685 CHICAGO HOUSE A... &
192085 & 1 & 1095441 & DELEGATE AGENCY & NaN & NaN & DEPARTMENT OF
HEALTH & CHICAGO HOUSE \& SOCIAL SERVICE AGENCY & 105470138T & 2229 S
MICHIGAN AVE 304 EFT & NaN & CHICAGO & Illinois & 60616 & -32025.2 & NaN
& NaN \\
28827 & 2023 & DFSS-HHS-CS-CEL: & 222199 & 1 & 1070196 & DELEGATE AGENCY
& NaN & NaN & DEPT OF FAMILY AND SUPPORT SERVICES &
ALLISON\textquotesingle S INFANT \& TODDLER CENTER INC & 62751817Z & 234
E 115TH ST FL 1ST & NaN & CHICAGO & Illinois & 60628 & 141923.0 & NaN &
NaN \\
\end{longtable}

Each row corresponds to one contract, and we have each contract's id
number, the year in which they were granted, the amount of the contract,
and the vendor's name and address, among other variables.

Now, consider the following grouped summary of the \texttt{contracts}
dataset, which shows the number of contracts by state of the vendor in
each year:

\begin{Shaded}
\begin{Highlighting}[]
\NormalTok{contracts\_summary }\OperatorTok{=}\NormalTok{ (}
\NormalTok{    contracts.groupby([}\StringTok{"state"}\NormalTok{, }\StringTok{"year\_of\_contract"}\NormalTok{]).size().reset\_index(name}\OperatorTok{=}\StringTok{"n"}\NormalTok{)}
\NormalTok{)}

\NormalTok{contracts\_summary}
\end{Highlighting}
\end{Shaded}

\begin{longtable}[]{@{}llll@{}}
\toprule\noalign{}
& state & year\_of\_contract & n \\
\midrule\noalign{}
\endhead
\bottomrule\noalign{}
\endlastfoot
0 & Alabama & 2020 & 1 \\
1 & Alabama & 2021 & 2 \\
2 & Alabama & 2022 & 1 \\
3 & Alabama & 2023 & 7 \\
4 & Arizona & 2020 & 3 \\
... & ... & ... & ... \\
128 & Washington & 2021 & 1 \\
129 & Wisconsin & 2020 & 18 \\
130 & Wisconsin & 2021 & 15 \\
131 & Wisconsin & 2022 & 17 \\
132 & Wisconsin & 2023 & 25 \\
\end{longtable}

The output of this grouped operation is a quintessentially ``long''
dataset! Each observational unit (each contract type) occupies multiple
rows (four rows per state, to be exact), with one row for each
measurement (each year).

So, as you now see, long data often can arrive as an output of grouped
summaries, among other data manipulations.

Now, let's see how to convert such long data into a wide format with
\texttt{pivot}.

The code is quite straightforward:

\begin{Shaded}
\begin{Highlighting}[]
\NormalTok{(}
\NormalTok{    contracts\_summary.pivot(}
\NormalTok{        index}\OperatorTok{=}\StringTok{"state"}\NormalTok{, columns}\OperatorTok{=}\StringTok{"year\_of\_contract"}\NormalTok{, values}\OperatorTok{=}\StringTok{"n"}
\NormalTok{    ).reset\_index()}
\NormalTok{)}
\end{Highlighting}
\end{Shaded}

\begin{longtable}[]{@{}llllll@{}}
\toprule\noalign{}
year\_of\_contract & state & 2020 & 2021 & 2022 & 2023 \\
\midrule\noalign{}
\endhead
\bottomrule\noalign{}
\endlastfoot
0 & Alabama & 1.0 & 2.0 & 1.0 & 7.0 \\
1 & Arizona & 3.0 & 1.0 & 3.0 & 2.0 \\
2 & Arkansas & 1.0 & NaN & 1.0 & NaN \\
3 & British Columbia & NaN & 1.0 & NaN & NaN \\
4 & California & 36.0 & 42.0 & 43.0 & 38.0 \\
... & ... & ... & ... & ... & ... \\
38 & Texas & 25.0 & 24.0 & 37.0 & 28.0 \\
39 & Vermont & NaN & 1.0 & NaN & NaN \\
40 & Virginia & 4.0 & 4.0 & 7.0 & 9.0 \\
41 & Washington & NaN & 1.0 & NaN & NaN \\
42 & Wisconsin & 18.0 & 15.0 & 17.0 & 25.0 \\
\end{longtable}

As you can see, \texttt{pivot} has three important arguments:
\texttt{index}, \texttt{columns}, and \texttt{values}.

\begin{itemize}
\tightlist
\item
  \texttt{index} defines which column(s) to use as the new index. In our
  case, it's the state, since we want each row to represent one state.
\item
  \texttt{columns} identifies which variable to use to define column
  names in the wide format. In our case, it's the year of the contract.
  You can see that the years are now the column names.
\item
  \texttt{values} specifies which values will become the core of the
  wide data format. In our case, it's the number of contracts. data
  format.
\end{itemize}

You might also want to have the \emph{years} be your primary
observational/statistical unit, with each year occupying one row. This
can be carried out similarly to the above example, but with
\texttt{year\_of\_contract} as the index and \texttt{state} as the
columns:

\begin{Shaded}
\begin{Highlighting}[]
\NormalTok{(}
\NormalTok{    contracts\_summary.pivot(}
\NormalTok{        index}\OperatorTok{=}\StringTok{"year\_of\_contract"}\NormalTok{, columns}\OperatorTok{=}\StringTok{"state"}\NormalTok{, values}\OperatorTok{=}\StringTok{"n"}
\NormalTok{    ).reset\_index()}
\NormalTok{)}
\end{Highlighting}
\end{Shaded}

\begin{longtable}[]{@{}llllllllllllllllllllll@{}}
\toprule\noalign{}
state & year\_of\_contract & Alabama & Arizona & Arkansas & British
Columbia & California & Canada & Colorado & Connecticut & Delaware & ...
& Oregon & Pennsylvania & Rhode Island & South Carolina & Tennessee &
Texas & Vermont & Virginia & Washington & Wisconsin \\
\midrule\noalign{}
\endhead
\bottomrule\noalign{}
\endlastfoot
0 & 2020 & 1.0 & 3.0 & 1.0 & NaN & 36.0 & 1.0 & 6.0 & 1.0 & NaN & ... &
5.0 & 20.0 & 1.0 & 2.0 & 2.0 & 25.0 & NaN & 4.0 & NaN & 18.0 \\
1 & 2021 & 2.0 & 1.0 & NaN & 1.0 & 42.0 & 1.0 & 2.0 & 5.0 & NaN & ... &
NaN & 24.0 & NaN & 3.0 & 2.0 & 24.0 & 1.0 & 4.0 & 1.0 & 15.0 \\
2 & 2022 & 1.0 & 3.0 & 1.0 & NaN & 43.0 & 1.0 & 7.0 & 3.0 & 1.0 & ... &
NaN & 31.0 & NaN & 2.0 & 3.0 & 37.0 & NaN & 7.0 & NaN & 17.0 \\
3 & 2023 & 7.0 & 2.0 & NaN & NaN & 38.0 & NaN & 6.0 & 5.0 & 2.0 & ... &
NaN & 37.0 & NaN & 1.0 & 3.0 & 28.0 & NaN & 9.0 & NaN & 25.0 \\
\end{longtable}

Here the unique observation units (our rows) are now the years (2021,
2022, 2023).

\begin{tcolorbox}[enhanced jigsaw, colframe=quarto-callout-tip-color-frame, opacityback=0, titlerule=0mm, bottomrule=.15mm, breakable, leftrule=.75mm, colbacktitle=quarto-callout-tip-color!10!white, title=\textcolor{quarto-callout-tip-color}{\faLightbulb}\hspace{0.5em}{Practice}, rightrule=.15mm, coltitle=black, opacitybacktitle=0.6, colback=white, left=2mm, arc=.35mm, toptitle=1mm, bottomtitle=1mm, toprule=.15mm]

The \texttt{population} dataset shows the populations of 219 countries
over time.

Pivot this data into a wide format. Your answer should have 20 columns
and 219 rows.

\begin{Shaded}
\begin{Highlighting}[]
\NormalTok{population }\OperatorTok{=}\NormalTok{ pd.read\_csv(}\StringTok{"data/tidyr\_population.csv"}\NormalTok{)}
\NormalTok{population}
\end{Highlighting}
\end{Shaded}

\begin{longtable}[]{@{}llll@{}}
\toprule\noalign{}
& country & year & population \\
\midrule\noalign{}
\endhead
\bottomrule\noalign{}
\endlastfoot
0 & Afghanistan & 1995 & 17586073 \\
1 & Afghanistan & 1996 & 18415307 \\
2 & Afghanistan & 1997 & 19021226 \\
3 & Afghanistan & 1998 & 19496836 \\
4 & Afghanistan & 1999 & 19987071 \\
... & ... & ... & ... \\
4055 & Zimbabwe & 2009 & 12888918 \\
4056 & Zimbabwe & 2010 & 13076978 \\
4057 & Zimbabwe & 2011 & 13358738 \\
4058 & Zimbabwe & 2012 & 13724317 \\
4059 & Zimbabwe & 2013 & 14149648 \\
\end{longtable}

\begin{Shaded}
\begin{Highlighting}[]
\CommentTok{\# Your code here}
\end{Highlighting}
\end{Shaded}

\end{tcolorbox}

\section{Pivoting can be hard}\label{pivoting-can-be-hard}

We have mostly looked at very simple examples of pivoting here, but in
the wild, pivoting can be very difficult to do accurately.

When you run into such cases, we recommend looking at the
\href{https://pandas.pydata.org/docs/user_guide/reshaping.html}{official
documentation} of pivoting from the pandas team, as it is quite rich in
examples.

\section{Wrap up}\label{wrap-up-10}

Congratulations! You've mastered the art of reshaping data with pandas.

You now understand the differences between wide and long formats, and
can skillfully use melt() and pivot() to transform your data as needed.

Remember, there's no universally ``best'' format -- it depends on your
specific analysis or visualization needs. With these skills, you're now
equipped to handle data in any shape it comes in. Keep practicing with
different datasets to reinforce your learning!




\end{document}
